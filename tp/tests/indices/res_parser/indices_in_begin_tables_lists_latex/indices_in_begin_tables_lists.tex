\documentclass{book}
\usepackage{imakeidx}
\usepackage{amsfonts}
\usepackage{amsmath}
\usepackage[gen]{eurosym}
\usepackage[T1]{fontenc}
\usepackage{textcomp}
\usepackage{graphicx}
\usepackage{needspace}
\usepackage{etoolbox}
% a framemethod is needed for roundcorner
\usepackage[framemethod=TikZ]{mdframed}
\usepackage{fontsize}
\usepackage{enumitem}
\usepackage{geometry}
\usepackage{fancyhdr}
\usepackage{float}
\usepackage{babel}
% use hidelinks to remove boxes around links to be similar with Texinfo TeX
\usepackage[hidelinks]{hyperref}
\usepackage[utf8]{inputenc}

\makeindex[name=cp]%
\makeindex[name=fn]%
\makeindex[name=vr]%

% command used in \description format for indicateurl
\newcommand\GNUTexinfotablestyleindicateurl[1]{\ifstrempty{#1}{}{`\texttt{#1}'}}%

% command used in \description format for samp
\newcommand\GNUTexinfotablestylesamp[1]{\ifstrempty{#1}{}{`\texttt{#1}'}}%

% redefine the \mainmatter command such that it does not clear page
% as if in double page
\makeatletter
\renewcommand\mainmatter{\clearpage\@mainmattertrue\pagenumbering{arabic}}
\makeatother
% add command aliases to use the same command in book and report
\newcommand\GNUTexinfomainmatter{\mainmatter}
\newcommand\GNUTexinfofrontmatter{\frontmatter}

% this allows to select languages based on bcp47 codes.  bcp47 is a superset
% of the LL_CC ISO 639-2 LL ISO 3166 CC information of @documentlanguage
\babeladjust{
  autoload.bcp47 = on,
  autoload.bcp47.options = import
}

% set defaults for lists that match Texinfo TeX formatting
\setlist[description]{style=nextline, font=\normalfont}
\setlist[itemize]{label=\textbullet}
\setlist[enumerate]{label=\arabic*.}

% command that does nothing used to help with substitutions in commands
\newcommand{\GNUTexinfoplaceholder}[1]{}

% called when setting single headers
% use \nouppercase to match with Texinfo TeX style
\newcommand{\GNUTexinfosetsingleheader}{\pagestyle{fancy}
\fancyhf{}
\lhead{\nouppercase{\leftmark}}
\rhead{\thepage}
}

% called when setting double headers
\newcommand{\GNUTexinfosetdoubleheader}[1]{\pagestyle{fancy}
\fancyhf{}
\fancyhead[LE,RO]{\thepage}
\fancyhead[RE]{#1}
\fancyhead[LO]{\nouppercase{\leftmark}}
}

% for part and chapter, which call \thispagestyle{plain}
\fancypagestyle{plain}{ %
 \fancyhf{}
 \fancyhead[LE,RO]{\thepage}
}

% match Texinfo TeX style
\renewcommand{\headrulewidth}{0pt}%

% avoid pagebreak and headings setting for a sectionning command
\newcommand{\GNUTexinfonopagebreakheading}[2]{\let\clearpage\relax \let\cleardoublepage\relax \let\thispagestyle\GNUTexinfoplaceholder #1{#2}}

% the mdframed style for @cartouche
\mdfdefinestyle{GNUTexinfocartouche}{
innertopmargin=10pt, innerbottommargin=10pt,%
roundcorner=10pt}

\renewcommand{\includegraphics}[1]{\fbox{FIG #1}}

% set default for @setchapternewpage
\makeatletter
\patchcmd{\chapter}{\if@openright\cleardoublepage\else\clearpage\fi}{\GNUTexinfoplaceholder{setchapternewpage placeholder}\clearpage}{}{}
\makeatother
\GNUTexinfosetsingleheader{}%


\begin{document}
\chapter{chap}
\label{anchor:chapter}%

\begin{itemize}[label=-]
\item \index[cp]{also a cindex in itemize}%
e--mph item
\end{itemize}

\begin{itemize}[label=\textbullet{}]
\item \index[cp]{index entry within itemize}%
i--tem 1
\item \index[cp]{index entry right after "@item}%
i--tem 2
\end{itemize}

\begin{itemize}
\item T--ext before items.
\item \index[cp]{also a cindex in itemize}%
bullet item
\end{itemize}

\begin{enumerate}[start=1]

\item e--numerate
\end{enumerate}

\begin{enumerate}[start=1]
\item 
\index[cp]{index inter in enumerate between lines}%

\item enumerate item
\end{enumerate}

\begin{enumerate}[start=1]

\item \index[cp]{index inter in enumerate after line}%
enumerate item
\end{enumerate}

\begin{enumerate}[start=1]
\item \index[cp]{index inter in enumerate before line}%

\item enumerate item
\end{enumerate}

\begin{enumerate}[start=1]
\item Title
\item \index[cp]{cindex}%
enum
\end{enumerate}

\begin{enumerate}[start=1]
\item \index[cp]{first idx}%
\index[cp]{sedond idx}%
\index[cp]{another}%
enum
\end{enumerate}

\begin{description}[format=\texttt]
\item[acode{-}{-}b]
\index[vr]{acode--b@\texttt{acode{-}{-}b}}%
l--ine
\end{description}

\begin{description}
\item[aasis--b]
\index[vr]{aasis--b@\texttt{aasis{-}{-}b}}%
\item[b]
\index[vr]{b@\texttt{b}}%
l--ine
\end{description}

\begin{description}[format=\normalfont\emph]
\item[avar--b]
\index[fn]{avar--b@\texttt{avar{-}{-}b}}%
\index[cp]{index entry between item and itemx}%
\item[b]
\index[fn]{b@\texttt{b}}%
l--ine
\item[c]
\index[fn]{c@\texttt{c}}%


\item[d]
\index[fn]{d@\texttt{d}}%

\end{description}

\begin{description}[format=\texttt]
\item[] \index[cp]{cindex in table}%
\item[abb]
l--ine
\end{description}

\begin{description}[format=\texttt]
\item[] \index[cp]{cindex in table}%
Texte before first item.
\item[abb]
\end{description}

\begin{description}[format=\GNUTexinfotablestylesamp]
\item[] \index[cp]{samp cindex in table}%
\item[asamp{-}{-}bb]
l--ine samp
\end{description}

\begin{description}[format=\GNUTexinfotablestylesamp]
\item[] \index[cp]{samp cindex in table}%
Texte before first item samp.
\item[asamp{-}{-}bb]
\end{description}

\begin{description}[format=\GNUTexinfotablestylesamp]
\item[] 
\index[cp]{cindex between lines}%

\item[asamp{-}{-}bb1]
\end{description}

\begin{description}[format=\GNUTexinfotablestylesamp]
\item[] \index[cp]{cindex before line}%

\item[asamp{-}{-}bb2]
\end{description}

\begin{description}[format=\GNUTexinfotablestylesamp]
\item[] 
\index[cp]{cindex after line}%
\item[asamp{-}{-}bb2]
\end{description}

\begin{description}[format=\GNUTexinfotablestylesamp]
\item[] \index[cp]{cindex first}%
\index[cp]{second}%
\index[cp]{third}%
\item[asamp{-}{-}bb2]
\end{description}

\chapter{printindex}
\label{anchor:printindex}%

\printindex[cp]

\printindex[vr]

\printindex[fn]

\end{document}
