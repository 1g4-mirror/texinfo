\documentclass{book}
\usepackage{imakeidx}
\usepackage{amsfonts}
\usepackage{amsmath}
\usepackage[gen]{eurosym}
\usepackage[T1]{fontenc}
\usepackage{textcomp}
\usepackage{graphicx}
\usepackage{needspace}
\usepackage{etoolbox}
\usepackage{array}
\usepackage{embrac}
\usepackage{expl3}
\usepackage{tabularx}
\usepackage[framemethod=tikz]{mdframed}
\usepackage{enumitem}
\usepackage{titleps}
\usepackage{float}
% use hidelinks to remove boxes around links to be similar to Texinfo TeX
\usepackage[hidelinks]{hyperref}
\usepackage[utf8]{inputenc}

\makeatletter
\newcommand{\Texinfosettitle}{No Title}%

% no index headers or page break
\indexsetup{level=\relax,toclevel=section,noclearpage}%
\makeindex[name=codeidx,title=]%
\makeindex[name=cp,title=]%
\makeindex[name=fn,title=]%
\makeindex[name=tp,title=]%
\makeindex[name=truc,title=]%

% command used in \description format for samp
\newcommand\Texinfotablestylesamp[1]{\ifstrempty{#1}{}{`\texttt{#1}'}}%

% style command for cite in 'cmd_text' formatting context
\newcommand\Texinfocommandstyletextcite[1]{{\normalfont{}\textsl{#1}}}%

% style command for kbd in 'cmd_text' formatting context
\newcommand\Texinfocommandstyletextkbd[1]{{\ttfamily\textsl{#1}}}%

% style command for var in 'cmd_text' formatting context
\newcommand\Texinfocommandstyletextvar[1]{{\normalfont{}\textsl{#1}}}%

% redefine the \mainmatter command such that it does not clear page
% as if in double page
\renewcommand\mainmatter{\clearpage\@mainmattertrue\pagenumbering{arabic}}
\newenvironment{Texinfopreformatted}{%
  \par\GNUTobeylines\obeyspaces\frenchspacing\parskip=\z@\parindent=\z@}{}
{\catcode`\^^M=13 \gdef\GNUTobeylines{\catcode`\^^M=13 \def^^M{\null\par}}}
\newenvironment{Texinfoindented}{\begin{list}{}{}\item\relax}{\end{list}}


% set defaults for lists that match Texinfo TeX formatting
\setlist[description]{style=nextline, font=\normalfont}
\setlist[itemize]{label=\textbullet}
\setlist[enumerate]{label=\arabic*.}

% used for substitutions in commands
\newcommand{\Texinfoplaceholder}[1]{}

\newpagestyle{single}{\sethead[\chaptername{} \thechapter{} \chaptertitle{}][][\thepage]
                              {\chaptername{} \thechapter{} \chaptertitle{}}{}{\thepage}}

% avoid pagebreak and headings setting for a sectioning command
\newcommand{\Texinfonopagebreakheading}[2]{{\let\clearpage\relax \let\cleardoublepage\relax \let\thispagestyle\Texinfoplaceholder #1{#2}}}

% the mdframed style for @cartouche
\mdfdefinestyle{Texinfocartouche}{
innertopmargin=10pt, innerbottommargin=10pt,%
roundcorner=10pt}

% braces are upright in italic and slanted only in @def*
% so it is turned off here, and turned on @def* lines
\EmbracOff{}%

% allow line breaking at underscore
\let\Texinfounderscore\_
\renewcommand{\_}{\Texinfounderscore\discretionary{}{}{}}
\renewcommand{\includegraphics}[1]{\fbox{FIG \detokenize{#1}}}

\makeatother
% set default for @setchapternewpage
\makeatletter
\patchcmd{\chapter}{\if@openright\cleardoublepage\else\clearpage\fi}{\Texinfoplaceholder{setchapternewpage placeholder}\clearpage}{}{}
\makeatother

% no headings before titlepage
\pagestyle{empty}%



\begin{document}

\frontmatter
\begin{titlepage}
\begingroup
\newskip\titlepagetopglue \titlepagetopglue = 1.5in
\newskip\titlepagebottomglue \titlepagebottomglue = 2pc
\setlength{\parindent}{0pt}
% Leave some space at the very top of the page.
    \vglue\titlepagetopglue
{\raggedright {\huge \bfseries title --a}}
\vskip 4pt \hrule height 4pt width \hsize \vskip 4pt
\rightline{formatting subtitle --a}
\rightline{subtitle 2 --a}
\vskip 0pt plus 1filll
\leftline{\Large \bfseries author1 --a with accents in name T\'{e}\c{c}a}%
\leftline{\Large \bfseries author2 --a}%
In titlepage

<
>
"
\&
'
`

``simple-double--three---four----''\leavevmode{}\\
code: \texttt{{`}{`}simple-double{-}{-}three{-}{-}{-}four{-}{-}{-}-{'}{'}} \leavevmode{}\\
asis: ``simple-double--three---four----'' \leavevmode{}\\
strong: \textbf{``simple-double--three---four----''} \leavevmode{}\\
kbd: \Texinfocommandstyletextkbd{{`}{`}simple-double{-}{-}three{-}{-}{-}four{-}{-}{-}-{'}{'}} \leavevmode{}\\

`\hbox{}`simple-double-\hbox{}-three---four----'\hbox{}'\leavevmode{}\\

\index[cp]{-option@--option}%
\index[cp]{""@``}%
\index[fn]{``@\texttt{{`}{`}}}%
\index[fn]{--foption@\texttt{{-}{-}foption}}%

@"u \"{u} 
@"\{U\} \"{U} 
@\~{}n \~{n}
@\^{}a \^{a}
@'e \'{e}
@=o \={o}
@`i \`{i}
@'\{e\} \'{e}
@'\{@dotless\{i\}\} \'{\i{}} 
@dotless\{i\} \i{}
@dotless\{j\} \j{}
@`\{@=E\} \`{\={E}} 
@l\{\} \l{}
@,\{@'C\} \c{\'{C}}
@,c \c{c}
@,c@"u \c{c}\"{u} \leavevmode{}\\

@U\{0075\} u

@* \leavevmode{}\\
@ followed by a space
\ {}
@ followed by a tab
\ {}
@ followed by a new line
\ {}\texttt{@-} \-{}
\texttt{@:} \@
\texttt{@!} \@!
\texttt{@?} \@?
\texttt{@.} \@.
\texttt{@@} @
\texttt{@\}} \}
\texttt{@\{} \{
\texttt{@/} 

foo vs.\@ bar. 
colon :\@And something else.
semi colon ;\@.
And ? ?\@.
Now ! !\@@
but , ,\@

@TeX \TeX{}
@LaTeX \LaTeX{}
@bullet \textbullet{}
@copyright \copyright{}
@dots \dots{}\@
@enddots \dots{}
@equiv $\equiv{}$
@error \fbox{error}
@expansion $\mapsto{}$
@minus -
@point $\star{}$
@print $\dashv{}$
@result $\Rightarrow{}$
@today \today{}

@aa \aa{}
@AA \AA{}
@ae \ae{}
@oe \oe{}
@AE \AE{}
@OE \OE{}
@o \o{}
@O \O{}
@ss \ss{}
@l \l{}
@L \L{}
@DH \DH{}
@TH \TH{}
@dh \dh{}
@th \th{}

@exclamdown \textexclamdown{}
@questiondown \textquestiondown{}
@pounds \textsterling{}
@registeredsymbol \circledR{}
@ordf \textordfeminine{}
@ordm \textordmasculine{}
@comma ,
@quotedblleft \textquotedblleft{}
@quotedblright \textquotedblright{}
@quoteleft \textquoteleft{}
@quoteright \textquoteright{}
@quotedblbase \quotedblbase{}
@quotesinglbase \quotesinglbase{}
@guillemetleft \guillemotleft{}
@guillemetright \guillemotright{}
@guillemotleft \guillemotleft{}
@guillemotright \guillemotright{}
@guilsinglleft \guilsinglleft{}
@guilsinglright \guilsinglright{}

@textdegree \textdegree{}
@euro \euro{}
@arrow $\rightarrow{}$
@leq $\leq{}$
@geq $\geq{}$
@tie a~b

\texttt{@acronym\{{-}{-}a,an accronym\}} --a (an accronym)
\texttt{@acronym\{{-}{-}a\}} --a
\texttt{@abbr\{@'E{-}{-}.\ @comma\{\}A.,\ @'Etude Autonome \}} \'{E}--.\@ ,A.\@ (\'{E}tude Autonome)
\texttt{@abbr\{@'E{-}{-}.\ @comma\{\}A.\}} \'{E}--.\@ ,A.\@
\texttt{@asis\{{-}{-}a\}} --a
\texttt{@b\{{-}{-}a\}} \textbf{--a}
\texttt{@cite\{{-}{-}a\}} \Texinfocommandstyletextcite{--a}
\texttt{@code\{{-}{-}a\}} \texttt{{-}{-}a}
\texttt{@command\{{-}{-}a\}} \texttt{{-}{-}a}
\texttt{@dfn\{{-}{-}a\}} \textsl{--a}
\texttt{@dmn\{{-}{-}a\}} \thinspace --a
\texttt{@email\{{-}{-}a,{-}{-}b\}} \href{mailto:--a}{--b}
\texttt{@email\{,{-}{-}b\}} --b
\texttt{@email\{{-}{-}a\}} \href{mailto:--a}{\nolinkurl{--a}}
\texttt{@emph\{{-}{-}a\}} \emph{--a}
\texttt{@env\{{-}{-}a\}} \texttt{{-}{-}a}
\texttt{@file\{{-}{-}a\}} \texttt{{-}{-}a}
\texttt{@i\{{-}{-}a\}} \textit{--a}
\texttt{@kbd\{{-}{-}a\}} \Texinfocommandstyletextkbd{{-}{-}a}
\texttt{@key\{{-}{-}a\}} \texttt{{-}{-}a}
\texttt{@math\{{-}{-}a \{\textbackslash{}frac\{1\}\{2\}\}\ @minus\{\}\}} $--a {\frac{1}{2}} -$
\texttt{@option\{{-}{-}a\}} \texttt{{-}{-}a}
\texttt{@r\{{-}{-}a\}} \textnormal{--a}
\texttt{@samp\{{-}{-}a\}} `\texttt{{-}{-}a}'
\texttt{@sc\{{-}{-}a\}} \textsc{--a}
\texttt{@strong\{{-}{-}a\}} \textbf{--a}
\texttt{@t\{{-}{-}a\}} \texttt{{-}{-}a}
\texttt{@sansserif\{{-}{-}a\}} \textsf{--a}
\texttt{@slanted\{{-}{-}a\}} \textsl{--a}
\texttt{@titlefont\{{-}{-}a\}} {\huge \bfseries --a}
\texttt{@indicateurl\{{-}{-}a\}} `\texttt{{-}{-}a}'
\texttt{@uref\{{-}{-}a,{-}{-}b\}} \href{--a}{--b (\nolinkurl{--a})}
\texttt{@uref\{{-}{-}a\}} \url{--a}
\texttt{@uref\{,{-}{-}b\}} --b
\texttt{@uref\{{-}{-}a,{-}{-}b,{-}{-}c\}} --c
\texttt{@uref\{,{-}{-}b,{-}{-}c\}} --c
\texttt{@uref\{{-}{-}a{,}{,}{-}{-}c\}} --c
\texttt{@uref\{{,}{,}{-}{-}c\}} --c
\texttt{@url\{{-}{-}a,{-}{-}b\}} \href{--a}{--b (\nolinkurl{--a})}
\texttt{@url\{{-}{-}a,\}} \url{--a}
\texttt{@url\{,{-}{-}b\}} --b
\texttt{@var\{{-}{-}a\}} \Texinfocommandstyletextvar{--a}
\texttt{@verb\{:{-}{-}a:\}} \verb:--a:
\texttt{@verb\{:a  < \& @\ \% " {-}{-}    b:\}} \verb:a  < & @ % " --    b:
\texttt{@w\{a a a a a a a a a a a a a a a a a a a a a a a a a a a a a a a a a a a\}} \hbox{a a a a a a a a a a a a a a a a a a a a a a a a a a a a a a a a a a a}
\texttt{@H\{a\}} \H{a}
\texttt{@H\{{-}{-}a\}} \H{--a}
\texttt{@dotaccent\{a\}} \.{a}
\texttt{@dotaccent\{{-}{-}a\}} \.{--a}
\texttt{@ringaccent\{a\}} \r{a}
\texttt{@ringaccent\{{-}{-}a\}} \r{--a}
\texttt{@tieaccent\{a\}} \t{a}
\texttt{@tieaccent\{{-}{-}a\}} \t{--a}
\texttt{@u\{a\}} \u{a}
\texttt{@u\{{-}{-}a\}} \u{--a}
\texttt{@ubaraccent\{a\}} \b{a}
\texttt{@ubaraccent\{{-}{-}a\}} \b{--a}
\texttt{@udotaccent\{a\}} \d{a}
\texttt{@udotaccent\{{-}{-}a\}} \d{--a}
\texttt{@v\{a\}} \v{a}
\texttt{@v\{{-}{-}a\}} \v{--a}
\texttt{@,\{c\}} \c{c}
\texttt{@,\{{-}{-}c\}} \c{--c}
\texttt{@ogonek\{a\}} \k{a}
\texttt{@ogonek\{{-}{-}a\}} \k{--a}
\texttt{a@sup\{h\}@sub\{l\}} a\textsuperscript{h}\textsubscript{l}
\texttt{@footnote\{in footnote\}} \footnote{in footnote}
\texttt{@footnote\{in footnote2\}} \footnote{in footnote2}

\texttt{@sp 2}\leavevmode{}\\
\vskip 2\baselineskip %
\texttt{@page}\leavevmode{}\\
\vskip4pt \hrule height 2pt width \hsize
  \vskip\titlepagebottomglue
\endgroup
\newpage{}%
\phantom{blabla}%

\texttt{need 1002}
\needspace{1.002pt}%

\texttt{@clicksequence\{click @click\{\}\ A\}} click $\rightarrow{}$ A
After clickstyle $\Rightarrow{}$
\texttt{@clicksequence\{click @click\{\}\ A\}} click $\Rightarrow{}$ A


$$
disp--laymath
f(x) = {1 \over \sigma \sqrt{2\pi}}e^{-{1 \over 2}\left({x-\mu \over \sigma}\right)^2}
$$

$$
\mathbf{``simple-double--three---four----''} \hbox{aa}
`\hbox{}`simple-double-\hbox{}-three---four----'\hbox{}'
$$

$$
\imath{} \jmath{}
\mathord{\text{\l{}}} \textsl{\c{c}}
\textsl{\b{a}} \textsl{\d{a}} \textsl{\k{a}} a^{h}_{l}
 \ {}\ {} \ {}\-{}  ! @ \} \{ 
\today{}
$$

$$
\rightarrow{}
u
\bullet{} \copyright{} \dots{} \dots{} \equiv{}
\fbox{error} \mapsto{} - \dashv{} \Rightarrow{}
\mathord{\text{\AA{}}} \mathord{\text{\ae{}}} \mathord{\text{\oe{}}} \mathord{\text{\AE{}}} \mathord{\text{\OE{}}} \mathord{\text{\o{}}} \mathord{\text{\O{}}} \mathord{\text{\ss{}}} \mathord{\text{\l{}}} \mathord{\text{\L{}}} \mathord{\text{\DH{}}}
\mathord{\text{\TH{}}} \mathord{\text{\dh{}}} \mathord{\text{\th{}}} \mathord{\text{\textexclamdown{}}} \mathord{\text{\textquestiondown{}}} \mathsterling{}
\mathord{\text{\textordfeminine{}}} \mathord{\text{\textordmasculine{}}} , 
$$

$$
\mathord{\text{\textquotedblleft{}}} \mathord{\text{\textquotedblright{}}} 
\mathord{\text{\textquoteleft{}}} \mathord{\text{\textquoteright{}}} \mathord{\text{\quotedblbase{}}} \mathord{\text{\quotesinglbase{}}} \mathord{\text{\guillemotleft{}}}
\mathord{\text{\guillemotright{}}} \mathord{\text{\guillemotleft{}}} \mathord{\text{\guillemotright{}}} \mathord{\text{\guilsinglleft{}}}
\mathord{\text{\guilsinglright{}}} \euro{} \rightarrow{} \leq{} \geq{}
$$

$$
\mathbf{b} \mathit{i} \mathrm{r} sc \mathsf{sansserif} slanted
$$

\Texinfocommandstyletextkbd{default kbdinputstyle}
\begin{description}
\item[{\parbox[b]{\linewidth}{%
\Texinfocommandstyletextkbd{vtable i{-}{-}tem default kbdinputstyle}
\index[cp]{vtable i--tem default kbdinputstyle@\texttt{vtable i{-}{-}tem default kbdinputstyle}}%
}}]
\end{description}
\begin{Texinfoindented}
\begin{Texinfopreformatted}%
\ttfamily \Texinfocommandstyletextkbd{in example default kbdinputstyle}
\end{Texinfopreformatted}
\begin{description}
\item[{\parbox[b]{\linewidth}{%
\Texinfocommandstyletextkbd{vtable i{-}{-}tem in example default kbdinputstyle}
\index[cp]{vtable i--tem in example default kbdinputstyle@\texttt{vtable i{-}{-}tem in example default kbdinputstyle}}%
}}]
\end{description}
\end{Texinfoindented}

\texttt{code kbdinputstyle}
\begin{description}
\item[{\parbox[b]{\linewidth}{%
\texttt{vtable i{-}{-}tem code kbdinputstyle}
\index[cp]{vtable i--tem code kbdinputstyle@\texttt{vtable i{-}{-}tem code kbdinputstyle}}%
}}]
\end{description}
\begin{Texinfoindented}
\begin{Texinfopreformatted}%
\ttfamily \texttt{in example code kbdinputstyle}
\end{Texinfopreformatted}
\begin{description}
\item[{\parbox[b]{\linewidth}{%
\texttt{vtable i{-}{-}tem in example code kbdinputstyle}
\index[cp]{vtable i--tem in example code kbdinputstyle@\texttt{vtable i{-}{-}tem in example code kbdinputstyle}}%
}}]
\end{description}
\end{Texinfoindented}

\texttt{example kbdinputstyle}
\begin{description}
\item[{\parbox[b]{\linewidth}{%
\texttt{vtable i{-}{-}tem example kbdinputstyle}
\index[cp]{vtable i--tem example kbdinputstyle@\texttt{vtable i{-}{-}tem example kbdinputstyle}}%
}}]
\end{description}
\begin{Texinfoindented}
\begin{Texinfopreformatted}%
\ttfamily \Texinfocommandstyletextkbd{in example example kbdinputstyle}
\end{Texinfopreformatted}
\begin{description}
\item[{\parbox[b]{\linewidth}{%
\Texinfocommandstyletextkbd{vtable i{-}{-}tem in example example kbdinputstyle}
\index[cp]{vtable i--tem in example example kbdinputstyle@\texttt{vtable i{-}{-}tem in example example kbdinputstyle}}%
}}]
\end{description}
\end{Texinfoindented}

\Texinfocommandstyletextkbd{distinct kbdinputstyle}
\begin{description}
\item[{\parbox[b]{\linewidth}{%
\Texinfocommandstyletextkbd{vtable i{-}{-}tem distinct kbdinputstyle}
\index[cp]{vtable i--tem distinct kbdinputstyle@\texttt{vtable i{-}{-}tem distinct kbdinputstyle}}%
}}]
\end{description}
\begin{Texinfoindented}
\begin{Texinfopreformatted}%
\ttfamily \Texinfocommandstyletextkbd{in example distinct kbdinputstyle}
\end{Texinfopreformatted}
\begin{description}
\item[{\parbox[b]{\linewidth}{%
\Texinfocommandstyletextkbd{vtable i{-}{-}tem in example distinct kbdinputstyle}
\index[cp]{vtable i--tem in example distinct kbdinputstyle@\texttt{vtable i{-}{-}tem in example distinct kbdinputstyle}}%
}}]
\end{description}
\end{Texinfoindented}

\begin{quote}
A quot---ation
\end{quote}

\begin{quote}
\textbf{Note:} A Note
\end{quote}

\begin{quote}
\textbf{note:} A note
\end{quote}

\begin{quote}
\textbf{Caution:} Caution
\end{quote}

\begin{quote}
\textbf{Important:} Important
\end{quote}

\begin{quote}
\textbf{Tip:} a Tip
\end{quote}

\begin{quote}
\textbf{Warning:} a Warning.
\end{quote}

\begin{quote}
\textbf{something \'{e} \TeX{}:} The something \'{e} \TeX{} is here.
\end{quote}

\begin{quote}
\textbf{@ at the end of line \ {}:} A @ at the end of the @quotation line.
\end{quote}

\begin{quote}
\textbf{something, other thing:} something, other thing
\end{quote}

\begin{quote}
\textbf{Note, the note:} Note, the note
\end{quote}

\begin{quote}
\end{quote}

\begin{quote}
\textbf{Empty:} \end{quote}

\begin{quote}
\textbf{:} \end{quote}

\begin{quote}
\textbf{\leavevmode{}\\:} \end{quote}

\begin{quote}
aaa quotation
\end{quote}
\begin{center}
--- \emph{quotation author}
\end{center}

\begin{quote}
indent in quotation
\end{quote}

\begin{quote}
\leavevmode{}\\
\hbox{\kern -\leftmargin}%
exdented quotation line   and dash --- in quotation
\\
\end{quote}

\begin{quote}
Not exdented followed by exdented
\leavevmode{}\\
\hbox{\kern -\leftmargin}%
exdented quotation line
\\
\end{quote}

\begin{quote}
\leavevmode{}\\
\hbox{\kern -\leftmargin}%
exdented quotation line
\\
Followed by not exdented 
\end{quote}

\begin{quote}
quotation1
\leavevmode{}\\
\hbox{\kern -\leftmargin}%
in exdented protected eol \ {}
\\
following
\leavevmode{}\\
\hbox{\kern -\leftmargin}%
in exdented a @* \leavevmode{}\\ and following
\\
after exdented
\end{quote}

\begin{quote}
\begin{footnotesize}
A small quot---ation
\end{footnotesize}
\end{quote}

\begin{quote}
\begin{footnotesize}
\textbf{Note:} A small Note
\end{footnotesize}
\end{quote}

\begin{quote}
\begin{footnotesize}
\textbf{something, other thing:} something, other thing
\end{footnotesize}
\end{quote}

\begin{itemize}
\item i--temize
\end{itemize}

\begin{itemize}[label=+]
\item i--tem +
\end{itemize}

\begin{itemize}[label=\textbullet{}]
\item b--ullet
\end{itemize}

\begin{itemize}[label=-]
\item minu--s
\end{itemize}

\begin{itemize}[label=\emph{after emph}]
\item e--mph item
\end{itemize}

\begin{itemize}[label=\textbullet{} a--n itemize line]
\item \index[cp]{index entry within itemize@index entry within itemize}%
i--tem 1
\item i--tem 2
\end{itemize}

\begin{itemize}[label={}]
\item with w a--b
\item with w c--d
\end{itemize}

\begin{itemize}[label=\hbox{} on a line]
\item line w a--b
\item line with w c--d
\end{itemize}

\begin{enumerate}[start=1]
\item e--numerate
\end{enumerate}

\begin{enumerate}[start=3]
\item first third
\item second third
\end{enumerate}

\begin{enumerate}[label=\alph*.]
\item e--numerate
\end{enumerate}

\begin{enumerate}[label=\alph*.,start=3]
\item first c
\item second c
\end{enumerate}

\begin{tabular}{m{0.4\textwidth} m{0.6\textwidth}}%
mu--ltitable headitem &another tab\\
mu--ltitable item &multitable tab\\
mu--ltitable item 2 &multitable tab 2
\index[cp]{index entry within multitable@index entry within multitable}%
\\
lone mu--ltitable item&\\
\end{tabular}%

\begin{tabular}{m{0.4\textwidth} m{0.6\textwidth}}%
truc &bidule\\
\end{tabular}%

\begin{Texinfoindented}
\begin{Texinfopreformatted}%
\ttfamily e{-}{-}xample  some
\   text
\end{Texinfopreformatted}
\end{Texinfoindented}

\begin{Texinfoindented}
\begin{Texinfopreformatted}%
\ttfamily example one arg
\end{Texinfopreformatted}
\end{Texinfoindented}

\begin{Texinfoindented}
\begin{Texinfopreformatted}%
\ttfamily example two args
\end{Texinfopreformatted}
\end{Texinfoindented}

\begin{Texinfoindented}
\begin{Texinfopreformatted}%
\ttfamily example three args
\end{Texinfopreformatted}
\end{Texinfoindented}

\begin{Texinfoindented}
\begin{Texinfopreformatted}%
\ttfamily example four args
\end{Texinfopreformatted}
\end{Texinfoindented}

\begin{Texinfoindented}
\begin{Texinfopreformatted}%
\ttfamily example five args
\end{Texinfopreformatted}
\end{Texinfoindented}

\begin{Texinfoindented}
\begin{Texinfopreformatted}%
\ttfamily The something \'{e}\ \TeX{}\ is here.
\end{Texinfopreformatted}
\end{Texinfoindented}

\begin{Texinfoindented}
\begin{Texinfopreformatted}%
\ttfamily A @\ at the end of the @example line.
\end{Texinfopreformatted}
\end{Texinfoindented}

\begin{Texinfoindented}
\begin{Texinfopreformatted}%
\ttfamily example with empty args
\end{Texinfopreformatted}
\end{Texinfoindented}

\begin{Texinfoindented}
\begin{Texinfopreformatted}%
\ttfamily example with empty and non empty args mix
\end{Texinfopreformatted}
\end{Texinfoindented}

\begin{Texinfoindented}
\begin{Texinfopreformatted}%
\ttfamily Exam{-}{-}{-}ple

\end{Texinfopreformatted}
\leavevmode{}\\
\hbox{\kern -\leftmargin}%
Other li---ne
\\
\begin{Texinfopreformatted}%
\ttfamily not exdented
\end{Texinfopreformatted}
\end{Texinfoindented}

\begin{Texinfoindented}
\leavevmode{}\\
\hbox{\kern -\leftmargin}%
exdented  and dash --- in example
\\
\begin{Texinfopreformatted}%
\ttfamily Not exdented one
\end{Texinfopreformatted}
\leavevmode{}\\
\hbox{\kern -\leftmargin}%
exdented two
\\
\begin{Texinfopreformatted}%
\ttfamily Not exdented two
\end{Texinfopreformatted}
\end{Texinfoindented}

\begin{Texinfoindented}
\begin{Texinfopreformatted}%
\ttfamily Example   Hoho.
\end{Texinfopreformatted}
\begin{Texinfoindented}
\begin{Texinfopreformatted}%
\ttfamily Nested Other line
\end{Texinfopreformatted}
\leavevmode{}\\
\hbox{\kern -\leftmargin}%
exdented nested other line
\\
\end{Texinfoindented}
\end{Texinfoindented}

\begin{Texinfopreformatted}%
\ttfamily \footnotesize s{-}{-}mallexample
\end{Texinfopreformatted}

\texttt{@noindent} after smallexample.
\begin{Texinfopreformatted}%
\ttfamily \footnotesize \$ wget 'http://savannah.gnu.org/cgi-bin/viewcvs/config/config/config.guess?rev=HEAD\&content-type=text/plain'
\$ wget 'http://savannah.gnu.org/cgi-bin/viewcvs/config/config/config.sub?rev=HEAD\&content-type=text/plain'
\end{Texinfopreformatted}
\noindent{}Less recent versions are also present.

\begin{Texinfoindented}
\begin{Texinfopreformatted}%
d--isplay
\end{Texinfopreformatted}
\end{Texinfoindented}

\begin{Texinfopreformatted}%
\footnotesize s--malldisplay
\end{Texinfopreformatted}

\begin{Texinfoindented}
\begin{Texinfopreformatted}%
\ttfamily l{-}{-}isp
\end{Texinfopreformatted}
\end{Texinfoindented}

\begin{Texinfopreformatted}%
\ttfamily \footnotesize s{-}{-}malllisp
\end{Texinfopreformatted}

\begin{Texinfopreformatted}%
f--ormat
\end{Texinfopreformatted}

\begin{Texinfopreformatted}%
\footnotesize s--mallformat
\end{Texinfopreformatted}


\noindent\begin{tabularx}{\linewidth}{@{}Xr}
\rightskip=5em plus 1 fill \hangindent=2em \hyphenpenalty=10000
\texttt{d{-}{-}effn\_name \EmbracOn{}\textnormal{\textsl{a--rguments...}}\EmbracOff{}}& [c--ategory]
\end{tabularx}

\index[fn]{d--effn\_name@\texttt{d{-}{-}effn\_name}}%
\begin{quote}
\unskip{\parskip=0pt\noindent}%
d--effn
\end{quote}


\noindent\begin{tabularx}{\linewidth}{@{}Xr}
\rightskip=5em plus 1 fill \hangindent=2em \hyphenpenalty=10000
\texttt{de{-}{-}ffn\_name \EmbracOn{}\textnormal{\textsl{ar--guments    more args   even more so}}\EmbracOff{}}& [cate--gory]
\end{tabularx}

\index[fn]{de--ffn\_name@\texttt{de{-}{-}ffn\_name}}%
\begin{quote}
\unskip{\parskip=0pt\noindent}%
def--fn
\end{quote}


\noindent\begin{tabularx}{\linewidth}{@{}Xr}
\rightskip=5em plus 1 fill \hangindent=2em \hyphenpenalty=10000
\texttt{\Texinfocommandstyletextvar{i} \EmbracOn{}\textnormal{\textsl{a g}}\EmbracOff{}}& [fset]
\end{tabularx}

\index[fn]{i@\texttt{\Texinfocommandstyletextvar{i}}}%
\index[cp]{index entry within deffn@index entry within deffn}%

\noindent\begin{tabularx}{\linewidth}{@{}Xr}
\rightskip=5em plus 1 fill \hangindent=2em \hyphenpenalty=10000
\texttt{truc \EmbracOn{}\textnormal{\textsl{}}\EmbracOff{}}& [cmde]
\end{tabularx}

\index[fn]{truc@\texttt{truc}}%

\noindent\begin{tabularx}{\linewidth}{@{}Xr}
\rightskip=5em plus 1 fill \hangindent=2em \hyphenpenalty=10000
\texttt{log trap \EmbracOn{}\textnormal{\textsl{}}\EmbracOff{}}& [Command]
\end{tabularx}

\index[fn]{log trap@\texttt{log trap}}%

\noindent\begin{tabularx}{\linewidth}{@{}Xr}
\rightskip=5em plus 1 fill \hangindent=2em \hyphenpenalty=10000
\texttt{log trap1 \EmbracOn{}\textnormal{\textsl{}}\EmbracOff{}}& [Command]
\end{tabularx}

\index[fn]{log trap1@\texttt{log trap1}}%

\noindent\begin{tabularx}{\linewidth}{@{}Xr}
\rightskip=5em plus 1 fill \hangindent=2em \hyphenpenalty=10000
\texttt{log trap2 \EmbracOn{}\textnormal{\textsl{}}\EmbracOff{}}& [Command]
\end{tabularx}

\index[fn]{log trap2@\texttt{log trap2}}%

\noindent\begin{tabularx}{\linewidth}{@{}Xr}
\rightskip=5em plus 1 fill \hangindent=2em \hyphenpenalty=10000
\texttt{\textbf{id ule} \EmbracOn{}\textnormal{\textsl{truc}}\EmbracOff{}}& [cmde]
\end{tabularx}

\index[fn]{id ule@\texttt{\textbf{id ule}}}%

\noindent\begin{tabularx}{\linewidth}{@{}Xr}
\rightskip=5em plus 1 fill \hangindent=2em \hyphenpenalty=10000
\texttt{\textbf{id `\texttt{i}'\ ule} \EmbracOn{}\textnormal{\textsl{truc}}\EmbracOff{}}& [cmde2]
\end{tabularx}

\index[fn]{id i ule@\texttt{\textbf{id `\texttt{i}'\ ule}}}%

\noindent\begin{tabularx}{\linewidth}{@{}Xr}
\rightskip=5em plus 1 fill \hangindent=2em \hyphenpenalty=10000
\texttt{}& []
\end{tabularx}


\noindent\begin{tabularx}{\linewidth}{@{}Xr}
\rightskip=5em plus 1 fill \hangindent=2em \hyphenpenalty=10000
\texttt{machin}& []
\end{tabularx}

\index[fn]{machin@\texttt{machin}}%

\noindent\begin{tabularx}{\linewidth}{@{}Xr}
\rightskip=5em plus 1 fill \hangindent=2em \hyphenpenalty=10000
\texttt{bidule machin}& []
\end{tabularx}

\index[fn]{bidule machin@\texttt{bidule machin}}%

\noindent\begin{tabularx}{\linewidth}{@{}Xr}
\rightskip=5em plus 1 fill \hangindent=2em \hyphenpenalty=10000
\texttt{machin}& [truc]
\end{tabularx}

\index[fn]{machin@\texttt{machin}}%

\noindent\begin{tabularx}{\linewidth}{@{}Xr}
\rightskip=5em plus 1 fill \hangindent=2em \hyphenpenalty=10000
\texttt{}& [truc]
\end{tabularx}


\noindent\begin{tabularx}{\linewidth}{@{}Xr}
\rightskip=5em plus 1 fill \hangindent=2em \hyphenpenalty=10000
\texttt{followed \EmbracOn{}\textnormal{\textsl{by a comment}}\EmbracOff{}}& [truc]
\end{tabularx}

\index[fn]{followed@\texttt{followed}}%

\noindent\begin{tabularx}{\linewidth}{@{}Xr}
\rightskip=5em plus 1 fill \hangindent=2em \hyphenpenalty=10000
\texttt{}& []
\end{tabularx}


\noindent\begin{tabularx}{\linewidth}{@{}Xr}
\rightskip=5em plus 1 fill \hangindent=2em \hyphenpenalty=10000
\texttt{a \EmbracOn{}\textnormal{\textsl{b c d e \textbf{f g} h i}}\EmbracOff{}}& [truc]
\end{tabularx}

\index[fn]{a@\texttt{a}}%

\noindent\begin{tabularx}{\linewidth}{@{}Xr}
\rightskip=5em plus 1 fill \hangindent=2em \hyphenpenalty=10000
\texttt{deffnx \EmbracOn{}\textnormal{\textsl{before end deffn}}\EmbracOff{}}& [truc]
\end{tabularx}

\index[fn]{deffnx@\texttt{deffnx}}%



\noindent\begin{tabularx}{\linewidth}{@{}Xr}
\rightskip=5em plus 1 fill \hangindent=2em \hyphenpenalty=10000
\texttt{deffn}& [empty]
\end{tabularx}

\index[fn]{deffn@\texttt{deffn}}%


\noindent\begin{tabularx}{\linewidth}{@{}Xr}
\rightskip=5em plus 1 fill \hangindent=2em \hyphenpenalty=10000
\texttt{deffn \EmbracOn{}\textnormal{\textsl{with deffnx}}\EmbracOff{}}& [empty]
\end{tabularx}

\index[fn]{deffn@\texttt{deffn}}%

\noindent\begin{tabularx}{\linewidth}{@{}Xr}
\rightskip=5em plus 1 fill \hangindent=2em \hyphenpenalty=10000
\texttt{deffnx}& [empty]
\end{tabularx}

\index[fn]{deffnx@\texttt{deffnx}}%


\noindent\begin{tabularx}{\linewidth}{@{}Xr}
\rightskip=5em plus 1 fill \hangindent=2em \hyphenpenalty=10000
\texttt{\Texinfocommandstyletextvar{i} \EmbracOn{}\textnormal{\textsl{a g}}\EmbracOff{}}& [fset]
\end{tabularx}

\index[fn]{i@\texttt{\Texinfocommandstyletextvar{i}}}%

\noindent\begin{tabularx}{\linewidth}{@{}Xr}
\rightskip=5em plus 1 fill \hangindent=2em \hyphenpenalty=10000
\texttt{truc \EmbracOn{}\textnormal{\textsl{}}\EmbracOff{}}& [cmde]
\end{tabularx}

\index[fn]{truc@\texttt{truc}}%
\begin{quote}
\unskip{\parskip=0pt\noindent}%
text in def item for second def item
\end{quote}



\noindent\begin{tabularx}{\linewidth}{@{}Xr}
\rightskip=5em plus 1 fill \hangindent=2em \hyphenpenalty=10000
\texttt{d{-}{-}efvr\_name}& [c--ategory]
\end{tabularx}

\index[cp]{d--efvr\_name@\texttt{d{-}{-}efvr\_name}}%
\begin{quote}
\unskip{\parskip=0pt\noindent}%
d--efvr
\end{quote}


\noindent\begin{tabularx}{\linewidth}{@{}Xr}
\rightskip=5em plus 1 fill \hangindent=2em \hyphenpenalty=10000
\texttt{n{-}{-}ame \EmbracOn{}\textnormal{\textsl{a--rguments...}}\EmbracOff{}}& [c--ategory]
\end{tabularx}

\index[fn]{n--ame@\texttt{n{-}{-}ame}}%
\begin{quote}
\unskip{\parskip=0pt\noindent}%
d--effn
\end{quote}


\noindent\begin{tabularx}{\linewidth}{@{}Xr}
\rightskip=5em plus 1 fill \hangindent=2em \hyphenpenalty=10000
\texttt{n{-}{-}ame}& [c--ategory]
\end{tabularx}

\index[fn]{n--ame@\texttt{n{-}{-}ame}}%
\begin{quote}
\unskip{\parskip=0pt\noindent}%
d--effn no arg
\end{quote}


\noindent\begin{tabularx}{\linewidth}{@{}Xr}
\rightskip=5em plus 1 fill \hangindent=2em \hyphenpenalty=10000
\texttt{t{-}{-}ype d{-}{-}eftypefn\_name a{-}{-}rguments...}& [c--ategory]
\end{tabularx}

\index[fn]{d--eftypefn\_name@\texttt{d{-}{-}eftypefn\_name}}%
\begin{quote}
\unskip{\parskip=0pt\noindent}%
d--eftypefn
\end{quote}


\noindent\begin{tabularx}{\linewidth}{@{}Xr}
\rightskip=5em plus 1 fill \hangindent=2em \hyphenpenalty=10000
\texttt{t{-}{-}ype d{-}{-}eftypefn\_name}& [c--ategory]
\end{tabularx}

\index[fn]{d--eftypefn\_name@\texttt{d{-}{-}eftypefn\_name}}%
\begin{quote}
\unskip{\parskip=0pt\noindent}%
d--eftypefn no arg
\end{quote}


\noindent\begin{tabularx}{\linewidth}{@{}Xr}
\rightskip=5em plus 1 fill \hangindent=2em \hyphenpenalty=10000
\texttt{t{-}{-}ype d{-}{-}eftypeop\_name a{-}{-}rguments...}& [c--ategory on \texttt{c{-}{-}lass}]
\end{tabularx}

\index[fn]{d--eftypeop\_name on c--lass@\texttt{d{-}{-}eftypeop\_name\ on c{-}{-}lass}}%
\begin{quote}
\unskip{\parskip=0pt\noindent}%
d--eftypeop
\end{quote}


\noindent\begin{tabularx}{\linewidth}{@{}Xr}
\rightskip=5em plus 1 fill \hangindent=2em \hyphenpenalty=10000
\texttt{t{-}{-}ype d{-}{-}eftypeop\_name}& [c--ategory on \texttt{c{-}{-}lass}]
\end{tabularx}

\index[fn]{d--eftypeop\_name on c--lass@\texttt{d{-}{-}eftypeop\_name\ on c{-}{-}lass}}%
\begin{quote}
\unskip{\parskip=0pt\noindent}%
d--eftypeop no arg
\end{quote}


\noindent\begin{tabularx}{\linewidth}{@{}Xr}
\rightskip=5em plus 1 fill \hangindent=2em \hyphenpenalty=10000
\texttt{t{-}{-}ype d{-}{-}eftypevr\_name}& [c--ategory]
\end{tabularx}

\index[cp]{d--eftypevr\_name@\texttt{d{-}{-}eftypevr\_name}}%
\begin{quote}
\unskip{\parskip=0pt\noindent}%
d--eftypevr
\end{quote}


\noindent\begin{tabularx}{\linewidth}{@{}Xr}
\rightskip=5em plus 1 fill \hangindent=2em \hyphenpenalty=10000
\texttt{d{-}{-}efcv\_name}& [c--ategory of \texttt{c{-}{-}lass}]
\end{tabularx}

\index[cp]{d--efcv\_name@\texttt{d{-}{-}efcv\_name}}%
\begin{quote}
\unskip{\parskip=0pt\noindent}%
d--efcv
\end{quote}


\noindent\begin{tabularx}{\linewidth}{@{}Xr}
\rightskip=5em plus 1 fill \hangindent=2em \hyphenpenalty=10000
\texttt{d{-}{-}efcv\_name \EmbracOn{}\textnormal{\textsl{a--rguments...}}\EmbracOff{}}& [c--ategory of \texttt{c{-}{-}lass}]
\end{tabularx}

\index[cp]{d--efcv\_name@\texttt{d{-}{-}efcv\_name}}%
\begin{quote}
\unskip{\parskip=0pt\noindent}%
d--efcv with arguments
\end{quote}


\noindent\begin{tabularx}{\linewidth}{@{}Xr}
\rightskip=5em plus 1 fill \hangindent=2em \hyphenpenalty=10000
\texttt{t{-}{-}ype d{-}{-}eftypecv\_name}& [c--ategory of \texttt{c{-}{-}lass}]
\end{tabularx}

\index[cp]{d--eftypecv\_name of c--lass@\texttt{d{-}{-}eftypecv\_name\ of c{-}{-}lass}}%
\begin{quote}
\unskip{\parskip=0pt\noindent}%
d--eftypecv
\end{quote}


\noindent\begin{tabularx}{\linewidth}{@{}Xr}
\rightskip=5em plus 1 fill \hangindent=2em \hyphenpenalty=10000
\texttt{t{-}{-}ype d{-}{-}eftypecv\_name a{-}{-}rguments...}& [c--ategory of \texttt{c{-}{-}lass}]
\end{tabularx}

\index[cp]{d--eftypecv\_name of c--lass@\texttt{d{-}{-}eftypecv\_name\ of c{-}{-}lass}}%
\begin{quote}
\unskip{\parskip=0pt\noindent}%
d--eftypecv with arguments
\end{quote}


\noindent\begin{tabularx}{\linewidth}{@{}Xr}
\rightskip=5em plus 1 fill \hangindent=2em \hyphenpenalty=10000
\texttt{d{-}{-}efop\_name \EmbracOn{}\textnormal{\textsl{a--rguments...}}\EmbracOff{}}& [c--ategory on \texttt{c{-}{-}lass}]
\end{tabularx}

\index[fn]{d--efop\_name on c--lass@\texttt{d{-}{-}efop\_name\ on c{-}{-}lass}}%
\begin{quote}
\unskip{\parskip=0pt\noindent}%
d--efop
\end{quote}


\noindent\begin{tabularx}{\linewidth}{@{}Xr}
\rightskip=5em plus 1 fill \hangindent=2em \hyphenpenalty=10000
\texttt{d{-}{-}efop\_name}& [c--ategory on \texttt{c{-}{-}lass}]
\end{tabularx}

\index[fn]{d--efop\_name on c--lass@\texttt{d{-}{-}efop\_name\ on c{-}{-}lass}}%
\begin{quote}
\unskip{\parskip=0pt\noindent}%
d--efop no arg
\end{quote}


\noindent\begin{tabularx}{\linewidth}{@{}Xr}
\rightskip=5em plus 1 fill \hangindent=2em \hyphenpenalty=10000
\texttt{d{-}{-}eftp\_name \EmbracOn{}\textnormal{\textsl{a--ttributes...}}\EmbracOff{}}& [c--ategory]
\end{tabularx}

\index[tp]{d--eftp\_name@\texttt{d{-}{-}eftp\_name}}%
\begin{quote}
\unskip{\parskip=0pt\noindent}%
d--eftp
\end{quote}


\noindent\begin{tabularx}{\linewidth}{@{}Xr}
\rightskip=5em plus 1 fill \hangindent=2em \hyphenpenalty=10000
\texttt{d{-}{-}efun\_name \EmbracOn{}\textnormal{\textsl{a--rguments...}}\EmbracOff{}}& [Function]
\end{tabularx}

\index[fn]{d--efun\_name@\texttt{d{-}{-}efun\_name}}%
\begin{quote}
\unskip{\parskip=0pt\noindent}%
d--efun
\end{quote}


\noindent\begin{tabularx}{\linewidth}{@{}Xr}
\rightskip=5em plus 1 fill \hangindent=2em \hyphenpenalty=10000
\texttt{d{-}{-}efmac\_name \EmbracOn{}\textnormal{\textsl{a--rguments...}}\EmbracOff{}}& [Macro]
\end{tabularx}

\index[fn]{d--efmac\_name@\texttt{d{-}{-}efmac\_name}}%
\begin{quote}
\unskip{\parskip=0pt\noindent}%
d--efmac
\end{quote}


\noindent\begin{tabularx}{\linewidth}{@{}Xr}
\rightskip=5em plus 1 fill \hangindent=2em \hyphenpenalty=10000
\texttt{d{-}{-}efspec\_name \EmbracOn{}\textnormal{\textsl{a--rguments...}}\EmbracOff{}}& [Special Form]
\end{tabularx}

\index[fn]{d--efspec\_name@\texttt{d{-}{-}efspec\_name}}%
\begin{quote}
\unskip{\parskip=0pt\noindent}%
d--efspec
\end{quote}


\noindent\begin{tabularx}{\linewidth}{@{}Xr}
\rightskip=5em plus 1 fill \hangindent=2em \hyphenpenalty=10000
\texttt{d{-}{-}efvar\_name}& [Variable]
\end{tabularx}

\index[cp]{d--efvar\_name@\texttt{d{-}{-}efvar\_name}}%
\begin{quote}
\unskip{\parskip=0pt\noindent}%
d--efvar
\end{quote}


\noindent\begin{tabularx}{\linewidth}{@{}Xr}
\rightskip=5em plus 1 fill \hangindent=2em \hyphenpenalty=10000
\texttt{d{-}{-}efvar\_name \EmbracOn{}\textnormal{\textsl{arg--var arg--var1}}\EmbracOff{}}& [Variable]
\end{tabularx}

\index[cp]{d--efvar\_name@\texttt{d{-}{-}efvar\_name}}%
\begin{quote}
\unskip{\parskip=0pt\noindent}%
d--efvar with args
\end{quote}


\noindent\begin{tabularx}{\linewidth}{@{}Xr}
\rightskip=5em plus 1 fill \hangindent=2em \hyphenpenalty=10000
\texttt{d{-}{-}efopt\_name}& [User Option]
\end{tabularx}

\index[cp]{d--efopt\_name@\texttt{d{-}{-}efopt\_name}}%
\begin{quote}
\unskip{\parskip=0pt\noindent}%
d--efopt
\end{quote}


\noindent\begin{tabularx}{\linewidth}{@{}Xr}
\rightskip=5em plus 1 fill \hangindent=2em \hyphenpenalty=10000
\texttt{t{-}{-}ype d{-}{-}eftypefun\_name a{-}{-}rguments...}& [Function]
\end{tabularx}

\index[fn]{d--eftypefun\_name@\texttt{d{-}{-}eftypefun\_name}}%
\begin{quote}
\unskip{\parskip=0pt\noindent}%
d--eftypefun
\end{quote}


\noindent\begin{tabularx}{\linewidth}{@{}Xr}
\rightskip=5em plus 1 fill \hangindent=2em \hyphenpenalty=10000
\texttt{t{-}{-}ype d{-}{-}eftypevar\_name}& [Variable]
\end{tabularx}

\index[cp]{d--eftypevar\_name@\texttt{d{-}{-}eftypevar\_name}}%
\begin{quote}
\unskip{\parskip=0pt\noindent}%
d--eftypevar
\end{quote}


\noindent\begin{tabularx}{\linewidth}{@{}Xr}
\rightskip=5em plus 1 fill \hangindent=2em \hyphenpenalty=10000
\texttt{d{-}{-}efivar\_name}& [Instance Variable of \texttt{c{-}{-}lass}]
\end{tabularx}

\index[cp]{d--efivar\_name of c--lass@\texttt{d{-}{-}efivar\_name\ of c{-}{-}lass}}%
\begin{quote}
\unskip{\parskip=0pt\noindent}%
d--efivar
\end{quote}


\noindent\begin{tabularx}{\linewidth}{@{}Xr}
\rightskip=5em plus 1 fill \hangindent=2em \hyphenpenalty=10000
\texttt{t{-}{-}ype d{-}{-}eftypeivar\_name}& [Instance Variable of \texttt{c{-}{-}lass}]
\end{tabularx}

\index[cp]{d--eftypeivar\_name of c--lass@\texttt{d{-}{-}eftypeivar\_name\ of c{-}{-}lass}}%
\begin{quote}
\unskip{\parskip=0pt\noindent}%
d--eftypeivar
\end{quote}


\noindent\begin{tabularx}{\linewidth}{@{}Xr}
\rightskip=5em plus 1 fill \hangindent=2em \hyphenpenalty=10000
\texttt{d{-}{-}efmethod\_name \EmbracOn{}\textnormal{\textsl{a--rguments...}}\EmbracOff{}}& [Method on \texttt{c{-}{-}lass}]
\end{tabularx}

\index[fn]{d--efmethod\_name on c--lass@\texttt{d{-}{-}efmethod\_name\ on c{-}{-}lass}}%
\begin{quote}
\unskip{\parskip=0pt\noindent}%
d--efmethod
\end{quote}


\noindent\begin{tabularx}{\linewidth}{@{}Xr}
\rightskip=5em plus 1 fill \hangindent=2em \hyphenpenalty=10000
\texttt{t{-}{-}ype d{-}{-}eftypemethod\_name a{-}{-}rguments...}& [Method on \texttt{c{-}{-}lass}]
\end{tabularx}

\index[fn]{d--eftypemethod\_name on c--lass@\texttt{d{-}{-}eftypemethod\_name\ on c{-}{-}lass}}%
\begin{quote}
\unskip{\parskip=0pt\noindent}%
d--eftypemethod
\end{quote}



\noindent\begin{tabularx}{\linewidth}{@{}Xr}
\rightskip=5em plus 1 fill \hangindent=2em \hyphenpenalty=10000
\texttt{data-type2}& [Function]\\
\texttt{name2 arguments2...}\end{tabularx}

\index[fn]{name2@\texttt{name2}}%
\begin{quote}
\unskip{\parskip=0pt\noindent}%
aaa2
\end{quote}


\noindent\begin{tabularx}{\linewidth}{@{}Xr}
\rightskip=5em plus 1 fill \hangindent=2em \hyphenpenalty=10000
\texttt{t{-}{-}ype2}& [c--ategory2]\\
\texttt{d{-}{-}eftypefn\_name2}\end{tabularx}

\index[fn]{d--eftypefn\_name2@\texttt{d{-}{-}eftypefn\_name2}}%
\begin{quote}
\unskip{\parskip=0pt\noindent}%
d--eftypefn no arg2
\end{quote}


\noindent\begin{tabularx}{\linewidth}{@{}Xr}
\rightskip=5em plus 1 fill \hangindent=2em \hyphenpenalty=10000
\texttt{t{-}{-}ype2}& [c--ategory2 on \texttt{c{-}{-}lass2}]\\
\texttt{d{-}{-}eftypeop\_name2 a{-}{-}rguments2...}\end{tabularx}

\index[fn]{d--eftypeop\_name2 on c--lass2@\texttt{d{-}{-}eftypeop\_name2\ on c{-}{-}lass2}}%
\begin{quote}
\unskip{\parskip=0pt\noindent}%
d--eftypeop2
\end{quote}


\noindent\begin{tabularx}{\linewidth}{@{}Xr}
\rightskip=5em plus 1 fill \hangindent=2em \hyphenpenalty=10000
\texttt{t{-}{-}ype2}& [c--ategory2 on \texttt{c{-}{-}lass2}]\\
\texttt{d{-}{-}eftypeop\_name2}\end{tabularx}

\index[fn]{d--eftypeop\_name2 on c--lass2@\texttt{d{-}{-}eftypeop\_name2\ on c{-}{-}lass2}}%
\begin{quote}
\unskip{\parskip=0pt\noindent}%
d--eftypeop no arg2
\end{quote}


\noindent\begin{tabularx}{\linewidth}{@{}Xr}
\rightskip=5em plus 1 fill \hangindent=2em \hyphenpenalty=10000
\texttt{t{-}{-}ype2 d{-}{-}eftypecv\_name2}& [c--ategory2 of \texttt{c{-}{-}lass2}]
\end{tabularx}

\index[cp]{d--eftypecv\_name2 of c--lass2@\texttt{d{-}{-}eftypecv\_name2\ of c{-}{-}lass2}}%
\begin{quote}
\unskip{\parskip=0pt\noindent}%
d--eftypecv2
\end{quote}


\noindent\begin{tabularx}{\linewidth}{@{}Xr}
\rightskip=5em plus 1 fill \hangindent=2em \hyphenpenalty=10000
\texttt{t{-}{-}ype2 d{-}{-}eftypecv\_name2 a{-}{-}rguments2...}& [c--ategory2 of \texttt{c{-}{-}lass2}]
\end{tabularx}

\index[cp]{d--eftypecv\_name2 of c--lass2@\texttt{d{-}{-}eftypecv\_name2\ of c{-}{-}lass2}}%
\begin{quote}
\unskip{\parskip=0pt\noindent}%
d--eftypecv with arguments2
\end{quote}


\noindent\begin{tabularx}{\linewidth}{@{}Xr}
\rightskip=5em plus 1 fill \hangindent=2em \hyphenpenalty=10000
\texttt{arg2}& [fun2]
\end{tabularx}

\index[fn]{arg2@\texttt{arg2}}%
\begin{quote}
\unskip{\parskip=0pt\noindent}%
fff2
\end{quote}


\texttt{@xref\{c{-}{-}{-}hapter@@,\ cross r{-}{-}{-}ef name@@,\ t{-}{-}{-}itle@@,\ file n{-}{-}{-}ame@@,\ ma{-}{-}{-}nual@@\}} See Section ``t---itle@'' in \textsl{ma---nual@}.
\texttt{@ref\{chapter,\ cross ref name,\ title,\ file name,\ manual\}} Section ``title'' in \textsl{manual}
\texttt{@pxref\{chapter,\ cross ref name,\ title,\ file name,\ manual\}} see Section ``title'' in \textsl{manual}
\texttt{@inforef\{chapter,\ cross ref name,\ file name\}} Section ``chapter'' in \texttt{file name}

\texttt{@ref\{chapter\}} \hyperref[anchor:chapter]{\chaptername~\ref*{anchor:chapter} [chapter], page~\pageref*{anchor:chapter}}
\texttt{@xref\{chapter\}} See \hyperref[anchor:chapter]{\chaptername~\ref*{anchor:chapter} [chapter], page~\pageref*{anchor:chapter}}.
\texttt{@pxref\{chapter\}} see \hyperref[anchor:chapter]{\chaptername~\ref*{anchor:chapter} [chapter], page~\pageref*{anchor:chapter}}
\texttt{@ref\{s{-}{-}ect@comma\{\}ion\}} \hyperref[anchor:s_002d_002dect_002cion]{Section~\ref*{anchor:s_002d_002dect_002cion} [s--ect,ion], page~\pageref*{anchor:s_002d_002dect_002cion}}

\texttt{@ref\{s{-}{-}ect@comma\{\}ion,\ a @comma\{\}\ in cross
ref,\ a comma@comma\{\}\ in title,\ a comma@comma\{\}\ in file,\ a @comma\{\}\ in manual name \}}
Section ``a comma, in title'' in \textsl{a , in manual name}

\texttt{@ref\{chapter,cross ref name\}} \hyperref[anchor:chapter]{\chaptername~\ref*{anchor:chapter} [chapter], page~\pageref*{anchor:chapter}}
\texttt{@ref\{chapter{,}{,}title\}} \hyperref[anchor:chapter]{\chaptername~\ref*{anchor:chapter} [title], page~\pageref*{anchor:chapter}}
\texttt{@ref\{chapter{,}{,},file name\}} Section ``chapter'' in \texttt{file name}
\texttt{@ref\{chapter{,}{,}{,}{,}manual\}} Section ``chapter'' in \textsl{manual}
\texttt{@ref\{chapter,cross ref name,title,\}} \hyperref[anchor:chapter]{\chaptername~\ref*{anchor:chapter} [title], page~\pageref*{anchor:chapter}}
\texttt{@ref\{chapter,cross ref name{,}{,}file name\}} Section ``chapter'' in \texttt{file name}
\texttt{@ref\{chapter,cross ref name{,}{,},manual\}} Section ``chapter'' in \textsl{manual}
\texttt{@ref\{chapter,cross ref name,title,file name\}} Section ``title'' in \texttt{file name}
\texttt{@ref\{chapter,cross ref name,title{,}{,}manual\}} Section ``title'' in \textsl{manual}
\texttt{@ref\{chapter,cross ref name,title,\ file name,\ manual\}} Section ``title'' in \textsl{manual}
\texttt{@ref\{chapter{,}{,}title,file name\}} Section ``title'' in \texttt{file name}
\texttt{@ref\{chapter{,}{,}title{,}{,}manual\}} Section ``title'' in \textsl{manual}
\texttt{@ref\{chapter{,}{,}title,\ file name,\ manual\}} Section ``title'' in \textsl{manual}
\texttt{@ref\{chapter{,}{,},file name,manual\}} Section ``chapter'' in \textsl{manual}


\texttt{@ref\{(pman)anode,cross ref name\}} (pman)anode
\texttt{@ref\{(pman)anode{,}{,}title\}} title
\texttt{@ref\{(pman)anode{,}{,},file name\}} Section ``(pman)anode'' in \texttt{file name}
\texttt{@ref\{(pman)anode{,}{,}{,}{,}manual\}} Section ``(pman)anode'' in \textsl{manual}
\texttt{@ref\{(pman)anode,cross ref name,title,\}} title
\texttt{@ref\{(pman)anode,cross ref name{,}{,}file name\}} Section ``(pman)anode'' in \texttt{file name}
\texttt{@ref\{(pman)anode,cross ref name{,}{,},manual\}} Section ``(pman)anode'' in \textsl{manual}
\texttt{@ref\{(pman)anode,cross ref name,title,file name\}} Section ``title'' in \texttt{file name}
\texttt{@ref\{(pman)anode,cross ref name,title{,}{,}manual\}} Section ``title'' in \textsl{manual}
\texttt{@ref\{(pman)anode,cross ref name,title,\ file name,\ manual\}} Section ``title'' in \textsl{manual}
\texttt{@ref\{(pman)anode{,}{,}title,file name\}} Section ``title'' in \texttt{file name}
\texttt{@ref\{(pman)anode{,}{,}title{,}{,}manual\}} Section ``title'' in \textsl{manual}
\texttt{@ref\{(pman)anode{,}{,}title,\ file name,\ manual\}} Section ``title'' in \textsl{manual}
\texttt{@ref\{(pman)anode{,}{,},file name,manual\}} Section ``(pman)anode'' in \textsl{manual}


\texttt{@inforef\{chapter,\ cross ref name,\ file name\}} Section ``chapter'' in \texttt{file name}
\texttt{@inforef\{chapter\}} chapter
\texttt{@inforef\{chapter,\ cross ref name\}} chapter
\texttt{@inforef\{chapter{,}{,}file name\}} Section ``chapter'' in \texttt{file name}
\texttt{@inforef\{node,\ cross ref name,\ file name\}} Section ``node'' in \texttt{file name}
\texttt{@inforef\{node\}} node
\texttt{@inforef\{node,\ cross ref name\}} node
\texttt{@inforef\{node{,}{,}file name\}} Section ``node'' in \texttt{file name}
\texttt{@inforef\{chapter,\ cross ref name,\ file name,\ spurious arg\}} Section ``chapter'' in \texttt{file name,\ spurious arg}

\texttt{@inforef\{s{-}{-}ect@comma\{\}ion,\ a @comma\{\}\ in cross
ref,\ a comma@comma\{\}\ in file\}}
Section ``s--ect,ion'' in \texttt{a comma,\ in file}

`\texttt{\hyperref[anchor:chapter]{\chaptername~\ref*{anchor:chapter} [chapter], page~\pageref*{anchor:chapter}}}'.

Section ``title with uref2 \href{href://http/myhost.com/index2.html}{uref2 (\nolinkurl{href://http/myhost.com/index2.html})}'' in \textsl{printed manual with uref4 \href{href://http/myhost.com/index4.html}{uref4 (\nolinkurl{href://http/myhost.com/index4.html})}}
\hyperref[anchor:chapter]{\chaptername~\ref*{anchor:chapter} [title with uref2 \href{href://http/myhost.com/index2.html}{uref2 (\nolinkurl{href://http/myhost.com/index2.html})}], page~\pageref*{anchor:chapter}}

\begin{description}
\item[{\parbox[b]{\linewidth}{%
\textbf{a--strong}}}]
l--ine
\end{description}

\begin{description}
\item[{\parbox[b]{\linewidth}{%
a--asis\\
\index[cp]{a--asis@\texttt{a{-}{-}asis}}%
b
\index[cp]{b@\texttt{b}}%
}}]
l--ine
\end{description}

\begin{description}
\item[{\parbox[b]{\linewidth}{%
\emph{a}\\
\index[fn]{a@\texttt{a}}%
\index[cp]{index entry between item and itemx@index entry between item and itemx}%
\emph{b}
\index[fn]{b@\texttt{b}}%
}}]
l--ine
\end{description}

\begin{description}
\item[] Title
\item[{\parbox[b]{\linewidth}{%
\texttt{a{-}{-}code}}}]
Value--table code
\end{description}

\begin{description}
\item[] Title
\item[{\parbox[b]{\linewidth}{%
\Texinfotablestylesamp{a{-}{-}samp}\\
\Texinfotablestylesamp{a2{-}{-}samp}}}]
Value--table samp
\end{description}

\begin{mdframed}[style=Texinfocartouche]
c--artouche
\end{mdframed}

\begin{flushleft}
\begin{Texinfopreformatted}%
f--lushleft
more text
\end{Texinfopreformatted}
\end{flushleft}

\begin{flushright}
\begin{Texinfopreformatted}%
f--lushright
more text
\end{Texinfopreformatted}
\end{flushright}

\begin{center}
ce--ntered line
\end{center}

\begin{flushleft}
r--raggedright
more text
\end{flushleft}

\begin{verbatim}
\input texinfo @c -*-texinfo-*-

@c this file is used in tests in @verbatiminclude but not converted

@setfilename simplest.info

@node Top

This is a very simple texi manual @  <>.

@bye
\end{verbatim}

\begin{verbatim}
in verbatim ''
\end{verbatim}





$\frac{a < b \texttt{tex \hbox{ code }}}{b}$ ``

\Texinfonopagebreakheading{\chapter*}{{majorheading}}

\Texinfonopagebreakheading{\chapter*}{{chapheading}}

\section*{{heading}}

\subsection*{{subheading}}

\subsubsection*{{subsubheading}}


\texttt{@acronym\{{-}{-}a,an accronym @comma\{\}\ @enddots\{\}\}} --a (an accronym , \dots{})
\texttt{@abbr\{@'E{-}{-}.\ @comma\{\}A.,\ @'Etude{-}{-}@comma\{\}\ @b\{Autonome\}\ \}} \'{E}--.\@ ,A.\@ (\'{E}tude--, \textbf{Autonome})
\texttt{@abbr\{@'E{-}{-}.\ @comma\{\}A.\}} \'{E}--.\@ ,A.\@

\texttt{@math\{{-}{-}a@minus\{\}\ \{\textbackslash{}frac\{1\}\{2\}\}\}} $--a- {\frac{1}{2}}$




Somehow invalid use of @,:\leavevmode{}\\
@, \c{}\leavevmode{}\\
@,@"u \c{}\"{u}

Invalid use of @':\leavevmode{}\\
@' \'{}\leavevmode{}\\
@'@"u \'{}\"{u}

\texttt{@|} 

@dotless\{truc\} truc
@dotless\{ij\} ij
\texttt{@dotless\{{-}{-}a\}} --a
\texttt{@dotless\{a\}} a

@U, without braces @U\{\}, with empty arg 
@U\{z\}, with non-hex arg U+z
@U\{FFFFFFFFFFFFFF\}, value much too large U+FFFFFFFFFFFFFF
@U\{110000\}, value just beyond Unicode U+110000

@TeX, but without brace \TeX{}
\texttt{@\#} \#

\texttt{@w\{{-}{-}a\}} \hbox{--a}

\texttt{@image\{,1{-}{-}xt\}} 
\texttt{@image\{{,}{,}2{-}{-}xt\}} 
\texttt{@image\{{,}{,},3{-}{-}xt\}} 

\texttt{@image\{f-ile,aze{,}{,}a{-}{-}lt\}} \includegraphics[width=aze]{f-ile}
\texttt{@image\{f-ile{,}{,},alt@verb\{:jk \_" \%\@\}\}} \includegraphics{f-ile}

\texttt{@image\{f{-}{-}ile\}} \includegraphics{f--ile}
\texttt{@image\{f{-}{-}ile{,}{,},alt\}} \includegraphics{f--ile}
\texttt{@image\{f{-}{-}ile{,}{,}{,}{,}.e-d-xt\}} \includegraphics{f--ile}
\texttt{@image\{f{-}{-}ile,l{-}{-}i\}} \includegraphics[width=l--i]{f--ile}
\texttt{@image\{f{-}{-}ile{,}{,}l{-}{-}e\}} \includegraphics[height=l--e]{f--ile}
\texttt{@image\{f{-}{-}ile,aze,az,alt,.e{-}{-}xt\}} \includegraphics[width=aze,height=az]{f--ile}
\texttt{@image\{@file\{f{-}{-}ile\}@@@.,aze,az,alt,@file\{.file ext\}\ e{-}{-}xt@\}} \includegraphics[width=aze,height=az]{f--ile@.}

\texttt{@image\{f{-}{-}ile,aze,az,@verb\{:jk \_" \%@:\}\ @b\{in b "\},e{-}{-}xt\}} \includegraphics[width=aze,height=az]{f--ile}
\texttt{@image\{file@verb\{:jk \_" \%@:\}{,}{,},alt@verb\{:jk \_" \%@:\}\}} \includegraphics{filejk _" \%@}


{\bfseries author}%

$$
\ddot{u} \ddot{U} \tilde{n} \hat{a} \acute{e} \bar{o} \grave{i} \acute{e} \grave{\bar{E}}
\textsl{\c{\'{C}}} \textsl{\c{\'{C}}} \textsl{\H{a}} \dot{a} \mathring{a} \textsl{\t{a}}
\breve{a} \check{a}
 ? .
$$

$$
TeX LaTeX \star{} \mathord{\text{\aa{}}} \circledR{} ^{\circ{}} 
$$

$$
\mathtt{t} 
$$

\begin{itemize}[label=\emph{}]
\item e--mph item
\end{itemize}

\begin{itemize}[label=\emph{} after emph]
\item e--mph item
\end{itemize}

\begin{itemize}[label=\textbullet{} a--n itemize line]
\item i--tem 1
\item i--tem 2
\end{itemize}

\begin{itemize}[label={}]
\item without brace w a--b
\item without brace w c--d
\end{itemize}

\begin{description}
\item[{\parbox[b]{\linewidth}{%
a}}]
l--ine
\end{description}

\begin{description}
\item[{\parbox[b]{\linewidth}{%
a--missing style formatting}}]
l--ine
\end{description}

\begin{description}
\item[{\parbox[b]{\linewidth}{%
a\\
\index[fn]{a@\texttt{a}}%
\index[cp]{index entry between item and itemx@index entry between item and itemx}%
b
\index[fn]{b@\texttt{b}}%
}}]
l--ine
\end{description}


\noindent\begin{tabularx}{\linewidth}{@{}Xr}
\rightskip=5em plus 1 fill \hangindent=2em \hyphenpenalty=10000
\texttt{}& [fun]
\end{tabularx}


\noindent\begin{tabularx}{\linewidth}{@{}Xr}
\rightskip=5em plus 1 fill \hangindent=2em \hyphenpenalty=10000
\texttt{machin \EmbracOn{}\textnormal{\textsl{bidule chose and}}\EmbracOff{}}& [truc]
\end{tabularx}

\index[fn]{machin@\texttt{machin}}%

\noindent\begin{tabularx}{\linewidth}{@{}Xr}
\rightskip=5em plus 1 fill \hangindent=2em \hyphenpenalty=10000
\texttt{machin \EmbracOn{}\textnormal{\textsl{bidule chose and  after}}\EmbracOff{}}& [truc]
\end{tabularx}

\index[fn]{machin@\texttt{machin}}%

\noindent\begin{tabularx}{\linewidth}{@{}Xr}
\rightskip=5em plus 1 fill \hangindent=2em \hyphenpenalty=10000
\texttt{machin \EmbracOn{}\textnormal{\textsl{bidule chose and }}\EmbracOff{}}& [truc]
\end{tabularx}

\index[fn]{machin@\texttt{machin}}%

\noindent\begin{tabularx}{\linewidth}{@{}Xr}
\rightskip=5em plus 1 fill \hangindent=2em \hyphenpenalty=10000
\texttt{machin \EmbracOn{}\textnormal{\textsl{bidule chose and and after}}\EmbracOff{}}& [truc]
\end{tabularx}

\index[fn]{machin@\texttt{machin}}%

\noindent\begin{tabularx}{\linewidth}{@{}Xr}
\rightskip=5em plus 1 fill \hangindent=2em \hyphenpenalty=10000
\texttt{followed \EmbracOn{}\textnormal{\textsl{by a comment}}\EmbracOff{}}& [truc]
\end{tabularx}

\index[fn]{followed@\texttt{followed}}%
Various deff lines

\noindent\begin{tabularx}{\linewidth}{@{}Xr}
\rightskip=5em plus 1 fill \hangindent=2em \hyphenpenalty=10000
\texttt{after \EmbracOn{}\textnormal{\textsl{a deff item}}\EmbracOff{}}& [truc]
\end{tabularx}

\index[fn]{after@\texttt{after}}%


\noindent\begin{tabularx}{\linewidth}{@{}Xr}
\rightskip=5em plus 1 fill \hangindent=2em \hyphenpenalty=10000
\texttt{\Texinfocommandstyletextvar{invalid} \EmbracOn{}\textnormal{\textsl{a g}}\EmbracOff{}}& [fsetinv]
\end{tabularx}

\index[fn]{invalid@\texttt{\Texinfocommandstyletextvar{invalid}}}%

\noindent\begin{tabularx}{\linewidth}{@{}Xr}
\rightskip=5em plus 1 fill \hangindent=2em \hyphenpenalty=10000
\texttt{}& [\textbf{id `\texttt{i}' ule}]
\end{tabularx}



\noindent\begin{tabularx}{\linewidth}{@{}Xr}
\rightskip=5em plus 1 fill \hangindent=2em \hyphenpenalty=10000
\texttt{}& [aaa]
\end{tabularx}


\noindent\begin{tabularx}{\linewidth}{@{}Xr}
\rightskip=5em plus 1 fill \hangindent=2em \hyphenpenalty=10000
\texttt{}& []
\end{tabularx}


\noindent\begin{tabularx}{\linewidth}{@{}Xr}
\rightskip=5em plus 1 fill \hangindent=2em \hyphenpenalty=10000
\texttt{}& [truc]
\end{tabularx}


g--roupe

\texttt{@ref\{node\}} node

\texttt{@ref\{,cross ref name\}} 
\texttt{@ref\{{,}{,}title\}} title
\texttt{@ref\{{,}{,},file name\}} \texttt{file name}
\texttt{@ref\{{,}{,}{,}{,}manual\}} \textsl{manual}
\texttt{@ref\{node,cross ref name\}} node
\texttt{@ref\{node{,}{,}title\}} title
\texttt{@ref\{node{,}{,},file name\}} Section ``node'' in \texttt{file name}
\texttt{@ref\{node{,}{,}{,}{,}manual\}} Section ``node'' in \textsl{manual}
\texttt{@ref\{node,cross ref name,title,\}} title
\texttt{@ref\{node,cross ref name{,}{,}file name\}} Section ``node'' in \texttt{file name}
\texttt{@ref\{node,cross ref name{,}{,},manual\}} Section ``node'' in \textsl{manual}
\texttt{@ref\{node,cross ref name,title,file name\}} Section ``title'' in \texttt{file name}
\texttt{@ref\{node,cross ref name,title{,}{,}manual\}} Section ``title'' in \textsl{manual}
\texttt{@ref\{node,cross ref name,title,\ file name,\ manual\}} Section ``title'' in \textsl{manual}
\texttt{@ref\{node{,}{,}title,file name\}} Section ``title'' in \texttt{file name}
\texttt{@ref\{node{,}{,}title{,}{,}manual\}} Section ``title'' in \textsl{manual}
\texttt{@ref\{chapter{,}{,}title,\ file name,\ manual\}} Section ``title'' in \textsl{manual}
\texttt{@ref\{node{,}{,}title,\ file name,\ manual\}} Section ``title'' in \textsl{manual}
\texttt{@ref\{node{,}{,},file name,manual\}} Section ``node'' in \textsl{manual}
\texttt{@ref\{,cross ref name,title,\}} title
\texttt{@ref\{,cross ref name{,}{,}file name\}} \texttt{file name}
\texttt{@ref\{,cross ref name{,}{,},manual\}} \textsl{manual}
\texttt{@ref\{,cross ref name,title,file name\}} Section ``title'' in \texttt{file name}
\texttt{@ref\{,cross ref name,title{,}{,}manual\}} Section ``title'' in \textsl{manual}
\texttt{@ref\{,cross ref name,title,\ file name,\ manual\}} Section ``title'' in \textsl{manual}
\texttt{@ref\{{,}{,}title,file name\}} Section ``title'' in \texttt{file name}
\texttt{@ref\{{,}{,}title{,}{,}manual\}} Section ``title'' in \textsl{manual}
\texttt{@ref\{{,}{,}title,\ file name,\ manual\}} Section ``title'' in \textsl{manual}
\texttt{@ref\{{,}{,},file name,manual\}} \textsl{manual}

\texttt{@inforef\{,cross ref name \}} 
\texttt{@inforef\{{,}{,}file name\}} \texttt{file name}
\texttt{@inforef\{,cross ref name,\ file name\}} \texttt{file name}
\texttt{@inforef\{\}} 



Insercopying in titlepage
In copying

<
>
"
\&
'
`

``simple-double--three---four----''\leavevmode{}\\
code: \texttt{{`}{`}simple-double{-}{-}three{-}{-}{-}four{-}{-}{-}-{'}{'}} \leavevmode{}\\
asis: ``simple-double--three---four----'' \leavevmode{}\\
strong: \textbf{``simple-double--three---four----''} \leavevmode{}\\
kbd: \Texinfocommandstyletextkbd{{`}{`}simple-double{-}{-}three{-}{-}{-}four{-}{-}{-}-{'}{'}} \leavevmode{}\\

`\hbox{}`simple-double-\hbox{}-three---four----'\hbox{}'\leavevmode{}\\

\index[cp]{-option@--option}%
\index[cp]{""@``}%
\index[fn]{``@\texttt{{`}{`}}}%
\index[fn]{--foption@\texttt{{-}{-}foption}}%

@"u \"{u} 
@"\{U\} \"{U} 
@\~{}n \~{n}
@\^{}a \^{a}
@'e \'{e}
@=o \={o}
@`i \`{i}
@'\{e\} \'{e}
@'\{@dotless\{i\}\} \'{\i{}} 
@dotless\{i\} \i{}
@dotless\{j\} \j{}
@`\{@=E\} \`{\={E}} 
@l\{\} \l{}
@,\{@'C\} \c{\'{C}}
@,c \c{c}
@,c@"u \c{c}\"{u} \leavevmode{}\\

@U\{0075\} u

@* \leavevmode{}\\
@ followed by a space
\ {}
@ followed by a tab
\ {}
@ followed by a new line
\ {}\texttt{@-} \-{}
\texttt{@:} \@
\texttt{@!} \@!
\texttt{@?} \@?
\texttt{@.} \@.
\texttt{@@} @
\texttt{@\}} \}
\texttt{@\{} \{
\texttt{@/} 

foo vs.\@ bar. 
colon :\@And something else.
semi colon ;\@.
And ? ?\@.
Now ! !\@@
but , ,\@

@TeX \TeX{}
@LaTeX \LaTeX{}
@bullet \textbullet{}
@copyright \copyright{}
@dots \dots{}\@
@enddots \dots{}
@equiv $\equiv{}$
@error \fbox{error}
@expansion $\mapsto{}$
@minus -
@point $\star{}$
@print $\dashv{}$
@result $\Rightarrow{}$
@today \today{}

@aa \aa{}
@AA \AA{}
@ae \ae{}
@oe \oe{}
@AE \AE{}
@OE \OE{}
@o \o{}
@O \O{}
@ss \ss{}
@l \l{}
@L \L{}
@DH \DH{}
@TH \TH{}
@dh \dh{}
@th \th{}

@exclamdown \textexclamdown{}
@questiondown \textquestiondown{}
@pounds \textsterling{}
@registeredsymbol \circledR{}
@ordf \textordfeminine{}
@ordm \textordmasculine{}
@comma ,
@quotedblleft \textquotedblleft{}
@quotedblright \textquotedblright{}
@quoteleft \textquoteleft{}
@quoteright \textquoteright{}
@quotedblbase \quotedblbase{}
@quotesinglbase \quotesinglbase{}
@guillemetleft \guillemotleft{}
@guillemetright \guillemotright{}
@guillemotleft \guillemotleft{}
@guillemotright \guillemotright{}
@guilsinglleft \guilsinglleft{}
@guilsinglright \guilsinglright{}

@textdegree \textdegree{}
@euro \euro{}
@arrow $\rightarrow{}$
@leq $\leq{}$
@geq $\geq{}$
@tie a~b

\texttt{@acronym\{{-}{-}a,an accronym\}} --a (an accronym)
\texttt{@acronym\{{-}{-}a\}} --a
\texttt{@abbr\{@'E{-}{-}.\ @comma\{\}A.,\ @'Etude Autonome \}} \'{E}--.\@ ,A.\@ (\'{E}tude Autonome)
\texttt{@abbr\{@'E{-}{-}.\ @comma\{\}A.\}} \'{E}--.\@ ,A.\@
\texttt{@asis\{{-}{-}a\}} --a
\texttt{@b\{{-}{-}a\}} \textbf{--a}
\texttt{@cite\{{-}{-}a\}} \Texinfocommandstyletextcite{--a}
\texttt{@code\{{-}{-}a\}} \texttt{{-}{-}a}
\texttt{@command\{{-}{-}a\}} \texttt{{-}{-}a}
\texttt{@dfn\{{-}{-}a\}} \textsl{--a}
\texttt{@dmn\{{-}{-}a\}} \thinspace --a
\texttt{@email\{{-}{-}a,{-}{-}b\}} \href{mailto:--a}{--b}
\texttt{@email\{,{-}{-}b\}} --b
\texttt{@email\{{-}{-}a\}} \href{mailto:--a}{\nolinkurl{--a}}
\texttt{@emph\{{-}{-}a\}} \emph{--a}
\texttt{@env\{{-}{-}a\}} \texttt{{-}{-}a}
\texttt{@file\{{-}{-}a\}} \texttt{{-}{-}a}
\texttt{@i\{{-}{-}a\}} \textit{--a}
\texttt{@kbd\{{-}{-}a\}} \Texinfocommandstyletextkbd{{-}{-}a}
\texttt{@key\{{-}{-}a\}} \texttt{{-}{-}a}
\texttt{@math\{{-}{-}a \{\textbackslash{}frac\{1\}\{2\}\}\ @minus\{\}\}} $--a {\frac{1}{2}} -$
\texttt{@option\{{-}{-}a\}} \texttt{{-}{-}a}
\texttt{@r\{{-}{-}a\}} \textnormal{--a}
\texttt{@samp\{{-}{-}a\}} `\texttt{{-}{-}a}'
\texttt{@sc\{{-}{-}a\}} \textsc{--a}
\texttt{@strong\{{-}{-}a\}} \textbf{--a}
\texttt{@t\{{-}{-}a\}} \texttt{{-}{-}a}
\texttt{@sansserif\{{-}{-}a\}} \textsf{--a}
\texttt{@slanted\{{-}{-}a\}} \textsl{--a}
\texttt{@titlefont\{{-}{-}a\}} {\huge \bfseries --a}
\texttt{@indicateurl\{{-}{-}a\}} `\texttt{{-}{-}a}'
\texttt{@uref\{{-}{-}a,{-}{-}b\}} \href{--a}{--b (\nolinkurl{--a})}
\texttt{@uref\{{-}{-}a\}} \url{--a}
\texttt{@uref\{,{-}{-}b\}} --b
\texttt{@uref\{{-}{-}a,{-}{-}b,{-}{-}c\}} --c
\texttt{@uref\{,{-}{-}b,{-}{-}c\}} --c
\texttt{@uref\{{-}{-}a{,}{,}{-}{-}c\}} --c
\texttt{@uref\{{,}{,}{-}{-}c\}} --c
\texttt{@url\{{-}{-}a,{-}{-}b\}} \href{--a}{--b (\nolinkurl{--a})}
\texttt{@url\{{-}{-}a,\}} \url{--a}
\texttt{@url\{,{-}{-}b\}} --b
\texttt{@var\{{-}{-}a\}} \Texinfocommandstyletextvar{--a}
\texttt{@verb\{:{-}{-}a:\}} \verb:--a:
\texttt{@verb\{:a  < \& @\ \% " {-}{-}    b:\}} \verb:a  < & @ % " --    b:
\texttt{@w\{a a a a a a a a a a a a a a a a a a a a a a a a a a a a a a a a a a a\}} \hbox{a a a a a a a a a a a a a a a a a a a a a a a a a a a a a a a a a a a}
\texttt{@H\{a\}} \H{a}
\texttt{@H\{{-}{-}a\}} \H{--a}
\texttt{@dotaccent\{a\}} \.{a}
\texttt{@dotaccent\{{-}{-}a\}} \.{--a}
\texttt{@ringaccent\{a\}} \r{a}
\texttt{@ringaccent\{{-}{-}a\}} \r{--a}
\texttt{@tieaccent\{a\}} \t{a}
\texttt{@tieaccent\{{-}{-}a\}} \t{--a}
\texttt{@u\{a\}} \u{a}
\texttt{@u\{{-}{-}a\}} \u{--a}
\texttt{@ubaraccent\{a\}} \b{a}
\texttt{@ubaraccent\{{-}{-}a\}} \b{--a}
\texttt{@udotaccent\{a\}} \d{a}
\texttt{@udotaccent\{{-}{-}a\}} \d{--a}
\texttt{@v\{a\}} \v{a}
\texttt{@v\{{-}{-}a\}} \v{--a}
\texttt{@,\{c\}} \c{c}
\texttt{@,\{{-}{-}c\}} \c{--c}
\texttt{@ogonek\{a\}} \k{a}
\texttt{@ogonek\{{-}{-}a\}} \k{--a}
\texttt{a@sup\{h\}@sub\{l\}} a\textsuperscript{h}\textsubscript{l}
\texttt{@footnote\{in footnote\}} \footnote{in footnote}
\texttt{@footnote\{in footnote2\}} \footnote{in footnote2}

\texttt{@sp 2}\leavevmode{}\\
\vskip 2\baselineskip %
\texttt{@page}\leavevmode{}\\
\newpage{}%
\phantom{blabla}%

\texttt{need 1002}
\needspace{1.002pt}%

\texttt{@clicksequence\{click @click\{\}\ A\}} click $\rightarrow{}$ A
After clickstyle $\Rightarrow{}$
\texttt{@clicksequence\{click @click\{\}\ A\}} click $\Rightarrow{}$ A


$$
disp--laymath
f(x) = {1 \over \sigma \sqrt{2\pi}}e^{-{1 \over 2}\left({x-\mu \over \sigma}\right)^2}
$$

$$
\mathbf{``simple-double--three---four----''} \hbox{aa}
`\hbox{}`simple-double-\hbox{}-three---four----'\hbox{}'
$$

$$
\imath{} \jmath{}
\mathord{\text{\l{}}} \textsl{\c{c}}
\textsl{\b{a}} \textsl{\d{a}} \textsl{\k{a}} a^{h}_{l}
 \ {}\ {} \ {}\-{}  ! @ \} \{ 
\today{}
$$

$$
\rightarrow{}
u
\bullet{} \copyright{} \dots{} \dots{} \equiv{}
\fbox{error} \mapsto{} - \dashv{} \Rightarrow{}
\mathord{\text{\AA{}}} \mathord{\text{\ae{}}} \mathord{\text{\oe{}}} \mathord{\text{\AE{}}} \mathord{\text{\OE{}}} \mathord{\text{\o{}}} \mathord{\text{\O{}}} \mathord{\text{\ss{}}} \mathord{\text{\l{}}} \mathord{\text{\L{}}} \mathord{\text{\DH{}}}
\mathord{\text{\TH{}}} \mathord{\text{\dh{}}} \mathord{\text{\th{}}} \mathord{\text{\textexclamdown{}}} \mathord{\text{\textquestiondown{}}} \mathsterling{}
\mathord{\text{\textordfeminine{}}} \mathord{\text{\textordmasculine{}}} , 
$$

$$
\mathord{\text{\textquotedblleft{}}} \mathord{\text{\textquotedblright{}}} 
\mathord{\text{\textquoteleft{}}} \mathord{\text{\textquoteright{}}} \mathord{\text{\quotedblbase{}}} \mathord{\text{\quotesinglbase{}}} \mathord{\text{\guillemotleft{}}}
\mathord{\text{\guillemotright{}}} \mathord{\text{\guillemotleft{}}} \mathord{\text{\guillemotright{}}} \mathord{\text{\guilsinglleft{}}}
\mathord{\text{\guilsinglright{}}} \euro{} \rightarrow{} \leq{} \geq{}
$$

$$
\mathbf{b} \mathit{i} \mathrm{r} sc \mathsf{sansserif} slanted
$$

\Texinfocommandstyletextkbd{default kbdinputstyle}
\begin{description}
\item[{\parbox[b]{\linewidth}{%
\Texinfocommandstyletextkbd{vtable i{-}{-}tem default kbdinputstyle}
\index[cp]{vtable i--tem default kbdinputstyle@\texttt{vtable i{-}{-}tem default kbdinputstyle}}%
}}]
\end{description}
\begin{Texinfoindented}
\begin{Texinfopreformatted}%
\ttfamily \Texinfocommandstyletextkbd{in example default kbdinputstyle}
\end{Texinfopreformatted}
\begin{description}
\item[{\parbox[b]{\linewidth}{%
\Texinfocommandstyletextkbd{vtable i{-}{-}tem in example default kbdinputstyle}
\index[cp]{vtable i--tem in example default kbdinputstyle@\texttt{vtable i{-}{-}tem in example default kbdinputstyle}}%
}}]
\end{description}
\end{Texinfoindented}

\texttt{code kbdinputstyle}
\begin{description}
\item[{\parbox[b]{\linewidth}{%
\texttt{vtable i{-}{-}tem code kbdinputstyle}
\index[cp]{vtable i--tem code kbdinputstyle@\texttt{vtable i{-}{-}tem code kbdinputstyle}}%
}}]
\end{description}
\begin{Texinfoindented}
\begin{Texinfopreformatted}%
\ttfamily \texttt{in example code kbdinputstyle}
\end{Texinfopreformatted}
\begin{description}
\item[{\parbox[b]{\linewidth}{%
\texttt{vtable i{-}{-}tem in example code kbdinputstyle}
\index[cp]{vtable i--tem in example code kbdinputstyle@\texttt{vtable i{-}{-}tem in example code kbdinputstyle}}%
}}]
\end{description}
\end{Texinfoindented}

\texttt{example kbdinputstyle}
\begin{description}
\item[{\parbox[b]{\linewidth}{%
\texttt{vtable i{-}{-}tem example kbdinputstyle}
\index[cp]{vtable i--tem example kbdinputstyle@\texttt{vtable i{-}{-}tem example kbdinputstyle}}%
}}]
\end{description}
\begin{Texinfoindented}
\begin{Texinfopreformatted}%
\ttfamily \Texinfocommandstyletextkbd{in example example kbdinputstyle}
\end{Texinfopreformatted}
\begin{description}
\item[{\parbox[b]{\linewidth}{%
\Texinfocommandstyletextkbd{vtable i{-}{-}tem in example example kbdinputstyle}
\index[cp]{vtable i--tem in example example kbdinputstyle@\texttt{vtable i{-}{-}tem in example example kbdinputstyle}}%
}}]
\end{description}
\end{Texinfoindented}

\Texinfocommandstyletextkbd{distinct kbdinputstyle}
\begin{description}
\item[{\parbox[b]{\linewidth}{%
\Texinfocommandstyletextkbd{vtable i{-}{-}tem distinct kbdinputstyle}
\index[cp]{vtable i--tem distinct kbdinputstyle@\texttt{vtable i{-}{-}tem distinct kbdinputstyle}}%
}}]
\end{description}
\begin{Texinfoindented}
\begin{Texinfopreformatted}%
\ttfamily \Texinfocommandstyletextkbd{in example distinct kbdinputstyle}
\end{Texinfopreformatted}
\begin{description}
\item[{\parbox[b]{\linewidth}{%
\Texinfocommandstyletextkbd{vtable i{-}{-}tem in example distinct kbdinputstyle}
\index[cp]{vtable i--tem in example distinct kbdinputstyle@\texttt{vtable i{-}{-}tem in example distinct kbdinputstyle}}%
}}]
\end{description}
\end{Texinfoindented}

\begin{quote}
A quot---ation
\end{quote}

\begin{quote}
\textbf{Note:} A Note
\end{quote}

\begin{quote}
\textbf{note:} A note
\end{quote}

\begin{quote}
\textbf{Caution:} Caution
\end{quote}

\begin{quote}
\textbf{Important:} Important
\end{quote}

\begin{quote}
\textbf{Tip:} a Tip
\end{quote}

\begin{quote}
\textbf{Warning:} a Warning.
\end{quote}

\begin{quote}
\textbf{something \'{e} \TeX{}:} The something \'{e} \TeX{} is here.
\end{quote}

\begin{quote}
\textbf{@ at the end of line \ {}:} A @ at the end of the @quotation line.
\end{quote}

\begin{quote}
\textbf{something, other thing:} something, other thing
\end{quote}

\begin{quote}
\textbf{Note, the note:} Note, the note
\end{quote}

\begin{quote}
\end{quote}

\begin{quote}
\textbf{Empty:} \end{quote}

\begin{quote}
\textbf{:} \end{quote}

\begin{quote}
\textbf{\leavevmode{}\\:} \end{quote}

\begin{quote}
aaa quotation
\end{quote}
\begin{center}
--- \emph{quotation author}
\end{center}

\begin{quote}
indent in quotation
\end{quote}

\begin{quote}
\leavevmode{}\\
\hbox{\kern -\leftmargin}%
exdented quotation line   and dash --- in quotation
\\
\end{quote}

\begin{quote}
Not exdented followed by exdented
\leavevmode{}\\
\hbox{\kern -\leftmargin}%
exdented quotation line
\\
\end{quote}

\begin{quote}
\leavevmode{}\\
\hbox{\kern -\leftmargin}%
exdented quotation line
\\
Followed by not exdented 
\end{quote}

\begin{quote}
quotation1
\leavevmode{}\\
\hbox{\kern -\leftmargin}%
in exdented protected eol \ {}
\\
following
\leavevmode{}\\
\hbox{\kern -\leftmargin}%
in exdented a @* \leavevmode{}\\ and following
\\
after exdented
\end{quote}

\begin{quote}
\begin{footnotesize}
A small quot---ation
\end{footnotesize}
\end{quote}

\begin{quote}
\begin{footnotesize}
\textbf{Note:} A small Note
\end{footnotesize}
\end{quote}

\begin{quote}
\begin{footnotesize}
\textbf{something, other thing:} something, other thing
\end{footnotesize}
\end{quote}

\begin{itemize}
\item i--temize
\end{itemize}

\begin{itemize}[label=+]
\item i--tem +
\end{itemize}

\begin{itemize}[label=\textbullet{}]
\item b--ullet
\end{itemize}

\begin{itemize}[label=-]
\item minu--s
\end{itemize}

\begin{itemize}[label=\emph{after emph}]
\item e--mph item
\end{itemize}

\begin{itemize}[label=\textbullet{} a--n itemize line]
\item \index[cp]{index entry within itemize@index entry within itemize}%
i--tem 1
\item i--tem 2
\end{itemize}

\begin{itemize}[label={}]
\item with w a--b
\item with w c--d
\end{itemize}

\begin{itemize}[label=\hbox{} on a line]
\item line w a--b
\item line with w c--d
\end{itemize}

\begin{enumerate}[start=1]
\item e--numerate
\end{enumerate}

\begin{enumerate}[start=3]
\item first third
\item second third
\end{enumerate}

\begin{enumerate}[label=\alph*.]
\item e--numerate
\end{enumerate}

\begin{enumerate}[label=\alph*.,start=3]
\item first c
\item second c
\end{enumerate}

\begin{tabular}{m{0.4\textwidth} m{0.6\textwidth}}%
mu--ltitable headitem &another tab\\
mu--ltitable item &multitable tab\\
mu--ltitable item 2 &multitable tab 2
\index[cp]{index entry within multitable@index entry within multitable}%
\\
lone mu--ltitable item&\\
\end{tabular}%

\begin{tabular}{m{0.4\textwidth} m{0.6\textwidth}}%
truc &bidule\\
\end{tabular}%

\begin{Texinfoindented}
\begin{Texinfopreformatted}%
\ttfamily e{-}{-}xample  some
\   text
\end{Texinfopreformatted}
\end{Texinfoindented}

\begin{Texinfoindented}
\begin{Texinfopreformatted}%
\ttfamily example one arg
\end{Texinfopreformatted}
\end{Texinfoindented}

\begin{Texinfoindented}
\begin{Texinfopreformatted}%
\ttfamily example two args
\end{Texinfopreformatted}
\end{Texinfoindented}

\begin{Texinfoindented}
\begin{Texinfopreformatted}%
\ttfamily example three args
\end{Texinfopreformatted}
\end{Texinfoindented}

\begin{Texinfoindented}
\begin{Texinfopreformatted}%
\ttfamily example four args
\end{Texinfopreformatted}
\end{Texinfoindented}

\begin{Texinfoindented}
\begin{Texinfopreformatted}%
\ttfamily example five args
\end{Texinfopreformatted}
\end{Texinfoindented}

\begin{Texinfoindented}
\begin{Texinfopreformatted}%
\ttfamily The something \'{e}\ \TeX{}\ is here.
\end{Texinfopreformatted}
\end{Texinfoindented}

\begin{Texinfoindented}
\begin{Texinfopreformatted}%
\ttfamily A @\ at the end of the @example line.
\end{Texinfopreformatted}
\end{Texinfoindented}

\begin{Texinfoindented}
\begin{Texinfopreformatted}%
\ttfamily example with empty args
\end{Texinfopreformatted}
\end{Texinfoindented}

\begin{Texinfoindented}
\begin{Texinfopreformatted}%
\ttfamily example with empty and non empty args mix
\end{Texinfopreformatted}
\end{Texinfoindented}

\begin{Texinfoindented}
\begin{Texinfopreformatted}%
\ttfamily Exam{-}{-}{-}ple

\end{Texinfopreformatted}
\leavevmode{}\\
\hbox{\kern -\leftmargin}%
Other li---ne
\\
\begin{Texinfopreformatted}%
\ttfamily not exdented
\end{Texinfopreformatted}
\end{Texinfoindented}

\begin{Texinfoindented}
\leavevmode{}\\
\hbox{\kern -\leftmargin}%
exdented  and dash --- in example
\\
\begin{Texinfopreformatted}%
\ttfamily Not exdented one
\end{Texinfopreformatted}
\leavevmode{}\\
\hbox{\kern -\leftmargin}%
exdented two
\\
\begin{Texinfopreformatted}%
\ttfamily Not exdented two
\end{Texinfopreformatted}
\end{Texinfoindented}

\begin{Texinfoindented}
\begin{Texinfopreformatted}%
\ttfamily Example   Hoho.
\end{Texinfopreformatted}
\begin{Texinfoindented}
\begin{Texinfopreformatted}%
\ttfamily Nested Other line
\end{Texinfopreformatted}
\leavevmode{}\\
\hbox{\kern -\leftmargin}%
exdented nested other line
\\
\end{Texinfoindented}
\end{Texinfoindented}

\begin{Texinfopreformatted}%
\ttfamily \footnotesize s{-}{-}mallexample
\end{Texinfopreformatted}

\texttt{@noindent} after smallexample.
\begin{Texinfopreformatted}%
\ttfamily \footnotesize \$ wget 'http://savannah.gnu.org/cgi-bin/viewcvs/config/config/config.guess?rev=HEAD\&content-type=text/plain'
\$ wget 'http://savannah.gnu.org/cgi-bin/viewcvs/config/config/config.sub?rev=HEAD\&content-type=text/plain'
\end{Texinfopreformatted}
\noindent{}Less recent versions are also present.

\begin{Texinfoindented}
\begin{Texinfopreformatted}%
d--isplay
\end{Texinfopreformatted}
\end{Texinfoindented}

\begin{Texinfopreformatted}%
\footnotesize s--malldisplay
\end{Texinfopreformatted}

\begin{Texinfoindented}
\begin{Texinfopreformatted}%
\ttfamily l{-}{-}isp
\end{Texinfopreformatted}
\end{Texinfoindented}

\begin{Texinfopreformatted}%
\ttfamily \footnotesize s{-}{-}malllisp
\end{Texinfopreformatted}

\begin{Texinfopreformatted}%
f--ormat
\end{Texinfopreformatted}

\begin{Texinfopreformatted}%
\footnotesize s--mallformat
\end{Texinfopreformatted}


\noindent\begin{tabularx}{\linewidth}{@{}Xr}
\rightskip=5em plus 1 fill \hangindent=2em \hyphenpenalty=10000
\texttt{d{-}{-}effn\_name \EmbracOn{}\textnormal{\textsl{a--rguments...}}\EmbracOff{}}& [c--ategory]
\end{tabularx}

\index[fn]{d--effn\_name@\texttt{d{-}{-}effn\_name}}%
\begin{quote}
\unskip{\parskip=0pt\noindent}%
d--effn
\end{quote}


\noindent\begin{tabularx}{\linewidth}{@{}Xr}
\rightskip=5em plus 1 fill \hangindent=2em \hyphenpenalty=10000
\texttt{de{-}{-}ffn\_name \EmbracOn{}\textnormal{\textsl{ar--guments    more args   even more so}}\EmbracOff{}}& [cate--gory]
\end{tabularx}

\index[fn]{de--ffn\_name@\texttt{de{-}{-}ffn\_name}}%
\begin{quote}
\unskip{\parskip=0pt\noindent}%
def--fn
\end{quote}


\noindent\begin{tabularx}{\linewidth}{@{}Xr}
\rightskip=5em plus 1 fill \hangindent=2em \hyphenpenalty=10000
\texttt{\Texinfocommandstyletextvar{i} \EmbracOn{}\textnormal{\textsl{a g}}\EmbracOff{}}& [fset]
\end{tabularx}

\index[fn]{i@\texttt{\Texinfocommandstyletextvar{i}}}%
\index[cp]{index entry within deffn@index entry within deffn}%

\noindent\begin{tabularx}{\linewidth}{@{}Xr}
\rightskip=5em plus 1 fill \hangindent=2em \hyphenpenalty=10000
\texttt{truc \EmbracOn{}\textnormal{\textsl{}}\EmbracOff{}}& [cmde]
\end{tabularx}

\index[fn]{truc@\texttt{truc}}%

\noindent\begin{tabularx}{\linewidth}{@{}Xr}
\rightskip=5em plus 1 fill \hangindent=2em \hyphenpenalty=10000
\texttt{log trap \EmbracOn{}\textnormal{\textsl{}}\EmbracOff{}}& [Command]
\end{tabularx}

\index[fn]{log trap@\texttt{log trap}}%

\noindent\begin{tabularx}{\linewidth}{@{}Xr}
\rightskip=5em plus 1 fill \hangindent=2em \hyphenpenalty=10000
\texttt{log trap1 \EmbracOn{}\textnormal{\textsl{}}\EmbracOff{}}& [Command]
\end{tabularx}

\index[fn]{log trap1@\texttt{log trap1}}%

\noindent\begin{tabularx}{\linewidth}{@{}Xr}
\rightskip=5em plus 1 fill \hangindent=2em \hyphenpenalty=10000
\texttt{log trap2 \EmbracOn{}\textnormal{\textsl{}}\EmbracOff{}}& [Command]
\end{tabularx}

\index[fn]{log trap2@\texttt{log trap2}}%

\noindent\begin{tabularx}{\linewidth}{@{}Xr}
\rightskip=5em plus 1 fill \hangindent=2em \hyphenpenalty=10000
\texttt{\textbf{id ule} \EmbracOn{}\textnormal{\textsl{truc}}\EmbracOff{}}& [cmde]
\end{tabularx}

\index[fn]{id ule@\texttt{\textbf{id ule}}}%

\noindent\begin{tabularx}{\linewidth}{@{}Xr}
\rightskip=5em plus 1 fill \hangindent=2em \hyphenpenalty=10000
\texttt{\textbf{id `\texttt{i}'\ ule} \EmbracOn{}\textnormal{\textsl{truc}}\EmbracOff{}}& [cmde2]
\end{tabularx}

\index[fn]{id i ule@\texttt{\textbf{id `\texttt{i}'\ ule}}}%

\noindent\begin{tabularx}{\linewidth}{@{}Xr}
\rightskip=5em plus 1 fill \hangindent=2em \hyphenpenalty=10000
\texttt{}& []
\end{tabularx}


\noindent\begin{tabularx}{\linewidth}{@{}Xr}
\rightskip=5em plus 1 fill \hangindent=2em \hyphenpenalty=10000
\texttt{machin}& []
\end{tabularx}

\index[fn]{machin@\texttt{machin}}%

\noindent\begin{tabularx}{\linewidth}{@{}Xr}
\rightskip=5em plus 1 fill \hangindent=2em \hyphenpenalty=10000
\texttt{bidule machin}& []
\end{tabularx}

\index[fn]{bidule machin@\texttt{bidule machin}}%

\noindent\begin{tabularx}{\linewidth}{@{}Xr}
\rightskip=5em plus 1 fill \hangindent=2em \hyphenpenalty=10000
\texttt{machin}& [truc]
\end{tabularx}

\index[fn]{machin@\texttt{machin}}%

\noindent\begin{tabularx}{\linewidth}{@{}Xr}
\rightskip=5em plus 1 fill \hangindent=2em \hyphenpenalty=10000
\texttt{}& [truc]
\end{tabularx}


\noindent\begin{tabularx}{\linewidth}{@{}Xr}
\rightskip=5em plus 1 fill \hangindent=2em \hyphenpenalty=10000
\texttt{followed \EmbracOn{}\textnormal{\textsl{by a comment}}\EmbracOff{}}& [truc]
\end{tabularx}

\index[fn]{followed@\texttt{followed}}%

\noindent\begin{tabularx}{\linewidth}{@{}Xr}
\rightskip=5em plus 1 fill \hangindent=2em \hyphenpenalty=10000
\texttt{}& []
\end{tabularx}


\noindent\begin{tabularx}{\linewidth}{@{}Xr}
\rightskip=5em plus 1 fill \hangindent=2em \hyphenpenalty=10000
\texttt{a \EmbracOn{}\textnormal{\textsl{b c d e \textbf{f g} h i}}\EmbracOff{}}& [truc]
\end{tabularx}

\index[fn]{a@\texttt{a}}%

\noindent\begin{tabularx}{\linewidth}{@{}Xr}
\rightskip=5em plus 1 fill \hangindent=2em \hyphenpenalty=10000
\texttt{deffnx \EmbracOn{}\textnormal{\textsl{before end deffn}}\EmbracOff{}}& [truc]
\end{tabularx}

\index[fn]{deffnx@\texttt{deffnx}}%



\noindent\begin{tabularx}{\linewidth}{@{}Xr}
\rightskip=5em plus 1 fill \hangindent=2em \hyphenpenalty=10000
\texttt{deffn}& [empty]
\end{tabularx}

\index[fn]{deffn@\texttt{deffn}}%


\noindent\begin{tabularx}{\linewidth}{@{}Xr}
\rightskip=5em plus 1 fill \hangindent=2em \hyphenpenalty=10000
\texttt{deffn \EmbracOn{}\textnormal{\textsl{with deffnx}}\EmbracOff{}}& [empty]
\end{tabularx}

\index[fn]{deffn@\texttt{deffn}}%

\noindent\begin{tabularx}{\linewidth}{@{}Xr}
\rightskip=5em plus 1 fill \hangindent=2em \hyphenpenalty=10000
\texttt{deffnx}& [empty]
\end{tabularx}

\index[fn]{deffnx@\texttt{deffnx}}%


\noindent\begin{tabularx}{\linewidth}{@{}Xr}
\rightskip=5em plus 1 fill \hangindent=2em \hyphenpenalty=10000
\texttt{\Texinfocommandstyletextvar{i} \EmbracOn{}\textnormal{\textsl{a g}}\EmbracOff{}}& [fset]
\end{tabularx}

\index[fn]{i@\texttt{\Texinfocommandstyletextvar{i}}}%

\noindent\begin{tabularx}{\linewidth}{@{}Xr}
\rightskip=5em plus 1 fill \hangindent=2em \hyphenpenalty=10000
\texttt{truc \EmbracOn{}\textnormal{\textsl{}}\EmbracOff{}}& [cmde]
\end{tabularx}

\index[fn]{truc@\texttt{truc}}%
\begin{quote}
\unskip{\parskip=0pt\noindent}%
text in def item for second def item
\end{quote}



\noindent\begin{tabularx}{\linewidth}{@{}Xr}
\rightskip=5em plus 1 fill \hangindent=2em \hyphenpenalty=10000
\texttt{d{-}{-}efvr\_name}& [c--ategory]
\end{tabularx}

\index[cp]{d--efvr\_name@\texttt{d{-}{-}efvr\_name}}%
\begin{quote}
\unskip{\parskip=0pt\noindent}%
d--efvr
\end{quote}


\noindent\begin{tabularx}{\linewidth}{@{}Xr}
\rightskip=5em plus 1 fill \hangindent=2em \hyphenpenalty=10000
\texttt{n{-}{-}ame \EmbracOn{}\textnormal{\textsl{a--rguments...}}\EmbracOff{}}& [c--ategory]
\end{tabularx}

\index[fn]{n--ame@\texttt{n{-}{-}ame}}%
\begin{quote}
\unskip{\parskip=0pt\noindent}%
d--effn
\end{quote}


\noindent\begin{tabularx}{\linewidth}{@{}Xr}
\rightskip=5em plus 1 fill \hangindent=2em \hyphenpenalty=10000
\texttt{n{-}{-}ame}& [c--ategory]
\end{tabularx}

\index[fn]{n--ame@\texttt{n{-}{-}ame}}%
\begin{quote}
\unskip{\parskip=0pt\noindent}%
d--effn no arg
\end{quote}


\noindent\begin{tabularx}{\linewidth}{@{}Xr}
\rightskip=5em plus 1 fill \hangindent=2em \hyphenpenalty=10000
\texttt{t{-}{-}ype d{-}{-}eftypefn\_name a{-}{-}rguments...}& [c--ategory]
\end{tabularx}

\index[fn]{d--eftypefn\_name@\texttt{d{-}{-}eftypefn\_name}}%
\begin{quote}
\unskip{\parskip=0pt\noindent}%
d--eftypefn
\end{quote}


\noindent\begin{tabularx}{\linewidth}{@{}Xr}
\rightskip=5em plus 1 fill \hangindent=2em \hyphenpenalty=10000
\texttt{t{-}{-}ype d{-}{-}eftypefn\_name}& [c--ategory]
\end{tabularx}

\index[fn]{d--eftypefn\_name@\texttt{d{-}{-}eftypefn\_name}}%
\begin{quote}
\unskip{\parskip=0pt\noindent}%
d--eftypefn no arg
\end{quote}


\noindent\begin{tabularx}{\linewidth}{@{}Xr}
\rightskip=5em plus 1 fill \hangindent=2em \hyphenpenalty=10000
\texttt{t{-}{-}ype d{-}{-}eftypeop\_name a{-}{-}rguments...}& [c--ategory on \texttt{c{-}{-}lass}]
\end{tabularx}

\index[fn]{d--eftypeop\_name on c--lass@\texttt{d{-}{-}eftypeop\_name\ on c{-}{-}lass}}%
\begin{quote}
\unskip{\parskip=0pt\noindent}%
d--eftypeop
\end{quote}


\noindent\begin{tabularx}{\linewidth}{@{}Xr}
\rightskip=5em plus 1 fill \hangindent=2em \hyphenpenalty=10000
\texttt{t{-}{-}ype d{-}{-}eftypeop\_name}& [c--ategory on \texttt{c{-}{-}lass}]
\end{tabularx}

\index[fn]{d--eftypeop\_name on c--lass@\texttt{d{-}{-}eftypeop\_name\ on c{-}{-}lass}}%
\begin{quote}
\unskip{\parskip=0pt\noindent}%
d--eftypeop no arg
\end{quote}


\noindent\begin{tabularx}{\linewidth}{@{}Xr}
\rightskip=5em plus 1 fill \hangindent=2em \hyphenpenalty=10000
\texttt{t{-}{-}ype d{-}{-}eftypevr\_name}& [c--ategory]
\end{tabularx}

\index[cp]{d--eftypevr\_name@\texttt{d{-}{-}eftypevr\_name}}%
\begin{quote}
\unskip{\parskip=0pt\noindent}%
d--eftypevr
\end{quote}


\noindent\begin{tabularx}{\linewidth}{@{}Xr}
\rightskip=5em plus 1 fill \hangindent=2em \hyphenpenalty=10000
\texttt{d{-}{-}efcv\_name}& [c--ategory of \texttt{c{-}{-}lass}]
\end{tabularx}

\index[cp]{d--efcv\_name@\texttt{d{-}{-}efcv\_name}}%
\begin{quote}
\unskip{\parskip=0pt\noindent}%
d--efcv
\end{quote}


\noindent\begin{tabularx}{\linewidth}{@{}Xr}
\rightskip=5em plus 1 fill \hangindent=2em \hyphenpenalty=10000
\texttt{d{-}{-}efcv\_name \EmbracOn{}\textnormal{\textsl{a--rguments...}}\EmbracOff{}}& [c--ategory of \texttt{c{-}{-}lass}]
\end{tabularx}

\index[cp]{d--efcv\_name@\texttt{d{-}{-}efcv\_name}}%
\begin{quote}
\unskip{\parskip=0pt\noindent}%
d--efcv with arguments
\end{quote}


\noindent\begin{tabularx}{\linewidth}{@{}Xr}
\rightskip=5em plus 1 fill \hangindent=2em \hyphenpenalty=10000
\texttt{t{-}{-}ype d{-}{-}eftypecv\_name}& [c--ategory of \texttt{c{-}{-}lass}]
\end{tabularx}

\index[cp]{d--eftypecv\_name of c--lass@\texttt{d{-}{-}eftypecv\_name\ of c{-}{-}lass}}%
\begin{quote}
\unskip{\parskip=0pt\noindent}%
d--eftypecv
\end{quote}


\noindent\begin{tabularx}{\linewidth}{@{}Xr}
\rightskip=5em plus 1 fill \hangindent=2em \hyphenpenalty=10000
\texttt{t{-}{-}ype d{-}{-}eftypecv\_name a{-}{-}rguments...}& [c--ategory of \texttt{c{-}{-}lass}]
\end{tabularx}

\index[cp]{d--eftypecv\_name of c--lass@\texttt{d{-}{-}eftypecv\_name\ of c{-}{-}lass}}%
\begin{quote}
\unskip{\parskip=0pt\noindent}%
d--eftypecv with arguments
\end{quote}


\noindent\begin{tabularx}{\linewidth}{@{}Xr}
\rightskip=5em plus 1 fill \hangindent=2em \hyphenpenalty=10000
\texttt{d{-}{-}efop\_name \EmbracOn{}\textnormal{\textsl{a--rguments...}}\EmbracOff{}}& [c--ategory on \texttt{c{-}{-}lass}]
\end{tabularx}

\index[fn]{d--efop\_name on c--lass@\texttt{d{-}{-}efop\_name\ on c{-}{-}lass}}%
\begin{quote}
\unskip{\parskip=0pt\noindent}%
d--efop
\end{quote}


\noindent\begin{tabularx}{\linewidth}{@{}Xr}
\rightskip=5em plus 1 fill \hangindent=2em \hyphenpenalty=10000
\texttt{d{-}{-}efop\_name}& [c--ategory on \texttt{c{-}{-}lass}]
\end{tabularx}

\index[fn]{d--efop\_name on c--lass@\texttt{d{-}{-}efop\_name\ on c{-}{-}lass}}%
\begin{quote}
\unskip{\parskip=0pt\noindent}%
d--efop no arg
\end{quote}


\noindent\begin{tabularx}{\linewidth}{@{}Xr}
\rightskip=5em plus 1 fill \hangindent=2em \hyphenpenalty=10000
\texttt{d{-}{-}eftp\_name \EmbracOn{}\textnormal{\textsl{a--ttributes...}}\EmbracOff{}}& [c--ategory]
\end{tabularx}

\index[tp]{d--eftp\_name@\texttt{d{-}{-}eftp\_name}}%
\begin{quote}
\unskip{\parskip=0pt\noindent}%
d--eftp
\end{quote}


\noindent\begin{tabularx}{\linewidth}{@{}Xr}
\rightskip=5em plus 1 fill \hangindent=2em \hyphenpenalty=10000
\texttt{d{-}{-}efun\_name \EmbracOn{}\textnormal{\textsl{a--rguments...}}\EmbracOff{}}& [Function]
\end{tabularx}

\index[fn]{d--efun\_name@\texttt{d{-}{-}efun\_name}}%
\begin{quote}
\unskip{\parskip=0pt\noindent}%
d--efun
\end{quote}


\noindent\begin{tabularx}{\linewidth}{@{}Xr}
\rightskip=5em plus 1 fill \hangindent=2em \hyphenpenalty=10000
\texttt{d{-}{-}efmac\_name \EmbracOn{}\textnormal{\textsl{a--rguments...}}\EmbracOff{}}& [Macro]
\end{tabularx}

\index[fn]{d--efmac\_name@\texttt{d{-}{-}efmac\_name}}%
\begin{quote}
\unskip{\parskip=0pt\noindent}%
d--efmac
\end{quote}


\noindent\begin{tabularx}{\linewidth}{@{}Xr}
\rightskip=5em plus 1 fill \hangindent=2em \hyphenpenalty=10000
\texttt{d{-}{-}efspec\_name \EmbracOn{}\textnormal{\textsl{a--rguments...}}\EmbracOff{}}& [Special Form]
\end{tabularx}

\index[fn]{d--efspec\_name@\texttt{d{-}{-}efspec\_name}}%
\begin{quote}
\unskip{\parskip=0pt\noindent}%
d--efspec
\end{quote}


\noindent\begin{tabularx}{\linewidth}{@{}Xr}
\rightskip=5em plus 1 fill \hangindent=2em \hyphenpenalty=10000
\texttt{d{-}{-}efvar\_name}& [Variable]
\end{tabularx}

\index[cp]{d--efvar\_name@\texttt{d{-}{-}efvar\_name}}%
\begin{quote}
\unskip{\parskip=0pt\noindent}%
d--efvar
\end{quote}


\noindent\begin{tabularx}{\linewidth}{@{}Xr}
\rightskip=5em plus 1 fill \hangindent=2em \hyphenpenalty=10000
\texttt{d{-}{-}efvar\_name \EmbracOn{}\textnormal{\textsl{arg--var arg--var1}}\EmbracOff{}}& [Variable]
\end{tabularx}

\index[cp]{d--efvar\_name@\texttt{d{-}{-}efvar\_name}}%
\begin{quote}
\unskip{\parskip=0pt\noindent}%
d--efvar with args
\end{quote}


\noindent\begin{tabularx}{\linewidth}{@{}Xr}
\rightskip=5em plus 1 fill \hangindent=2em \hyphenpenalty=10000
\texttt{d{-}{-}efopt\_name}& [User Option]
\end{tabularx}

\index[cp]{d--efopt\_name@\texttt{d{-}{-}efopt\_name}}%
\begin{quote}
\unskip{\parskip=0pt\noindent}%
d--efopt
\end{quote}


\noindent\begin{tabularx}{\linewidth}{@{}Xr}
\rightskip=5em plus 1 fill \hangindent=2em \hyphenpenalty=10000
\texttt{t{-}{-}ype d{-}{-}eftypefun\_name a{-}{-}rguments...}& [Function]
\end{tabularx}

\index[fn]{d--eftypefun\_name@\texttt{d{-}{-}eftypefun\_name}}%
\begin{quote}
\unskip{\parskip=0pt\noindent}%
d--eftypefun
\end{quote}


\noindent\begin{tabularx}{\linewidth}{@{}Xr}
\rightskip=5em plus 1 fill \hangindent=2em \hyphenpenalty=10000
\texttt{t{-}{-}ype d{-}{-}eftypevar\_name}& [Variable]
\end{tabularx}

\index[cp]{d--eftypevar\_name@\texttt{d{-}{-}eftypevar\_name}}%
\begin{quote}
\unskip{\parskip=0pt\noindent}%
d--eftypevar
\end{quote}


\noindent\begin{tabularx}{\linewidth}{@{}Xr}
\rightskip=5em plus 1 fill \hangindent=2em \hyphenpenalty=10000
\texttt{d{-}{-}efivar\_name}& [Instance Variable of \texttt{c{-}{-}lass}]
\end{tabularx}

\index[cp]{d--efivar\_name of c--lass@\texttt{d{-}{-}efivar\_name\ of c{-}{-}lass}}%
\begin{quote}
\unskip{\parskip=0pt\noindent}%
d--efivar
\end{quote}


\noindent\begin{tabularx}{\linewidth}{@{}Xr}
\rightskip=5em plus 1 fill \hangindent=2em \hyphenpenalty=10000
\texttt{t{-}{-}ype d{-}{-}eftypeivar\_name}& [Instance Variable of \texttt{c{-}{-}lass}]
\end{tabularx}

\index[cp]{d--eftypeivar\_name of c--lass@\texttt{d{-}{-}eftypeivar\_name\ of c{-}{-}lass}}%
\begin{quote}
\unskip{\parskip=0pt\noindent}%
d--eftypeivar
\end{quote}


\noindent\begin{tabularx}{\linewidth}{@{}Xr}
\rightskip=5em plus 1 fill \hangindent=2em \hyphenpenalty=10000
\texttt{d{-}{-}efmethod\_name \EmbracOn{}\textnormal{\textsl{a--rguments...}}\EmbracOff{}}& [Method on \texttt{c{-}{-}lass}]
\end{tabularx}

\index[fn]{d--efmethod\_name on c--lass@\texttt{d{-}{-}efmethod\_name\ on c{-}{-}lass}}%
\begin{quote}
\unskip{\parskip=0pt\noindent}%
d--efmethod
\end{quote}


\noindent\begin{tabularx}{\linewidth}{@{}Xr}
\rightskip=5em plus 1 fill \hangindent=2em \hyphenpenalty=10000
\texttt{t{-}{-}ype d{-}{-}eftypemethod\_name a{-}{-}rguments...}& [Method on \texttt{c{-}{-}lass}]
\end{tabularx}

\index[fn]{d--eftypemethod\_name on c--lass@\texttt{d{-}{-}eftypemethod\_name\ on c{-}{-}lass}}%
\begin{quote}
\unskip{\parskip=0pt\noindent}%
d--eftypemethod
\end{quote}



\noindent\begin{tabularx}{\linewidth}{@{}Xr}
\rightskip=5em plus 1 fill \hangindent=2em \hyphenpenalty=10000
\texttt{data-type2}& [Function]\\
\texttt{name2 arguments2...}\end{tabularx}

\index[fn]{name2@\texttt{name2}}%
\begin{quote}
\unskip{\parskip=0pt\noindent}%
aaa2
\end{quote}


\noindent\begin{tabularx}{\linewidth}{@{}Xr}
\rightskip=5em plus 1 fill \hangindent=2em \hyphenpenalty=10000
\texttt{t{-}{-}ype2}& [c--ategory2]\\
\texttt{d{-}{-}eftypefn\_name2}\end{tabularx}

\index[fn]{d--eftypefn\_name2@\texttt{d{-}{-}eftypefn\_name2}}%
\begin{quote}
\unskip{\parskip=0pt\noindent}%
d--eftypefn no arg2
\end{quote}


\noindent\begin{tabularx}{\linewidth}{@{}Xr}
\rightskip=5em plus 1 fill \hangindent=2em \hyphenpenalty=10000
\texttt{t{-}{-}ype2}& [c--ategory2 on \texttt{c{-}{-}lass2}]\\
\texttt{d{-}{-}eftypeop\_name2 a{-}{-}rguments2...}\end{tabularx}

\index[fn]{d--eftypeop\_name2 on c--lass2@\texttt{d{-}{-}eftypeop\_name2\ on c{-}{-}lass2}}%
\begin{quote}
\unskip{\parskip=0pt\noindent}%
d--eftypeop2
\end{quote}


\noindent\begin{tabularx}{\linewidth}{@{}Xr}
\rightskip=5em plus 1 fill \hangindent=2em \hyphenpenalty=10000
\texttt{t{-}{-}ype2}& [c--ategory2 on \texttt{c{-}{-}lass2}]\\
\texttt{d{-}{-}eftypeop\_name2}\end{tabularx}

\index[fn]{d--eftypeop\_name2 on c--lass2@\texttt{d{-}{-}eftypeop\_name2\ on c{-}{-}lass2}}%
\begin{quote}
\unskip{\parskip=0pt\noindent}%
d--eftypeop no arg2
\end{quote}


\noindent\begin{tabularx}{\linewidth}{@{}Xr}
\rightskip=5em plus 1 fill \hangindent=2em \hyphenpenalty=10000
\texttt{t{-}{-}ype2 d{-}{-}eftypecv\_name2}& [c--ategory2 of \texttt{c{-}{-}lass2}]
\end{tabularx}

\index[cp]{d--eftypecv\_name2 of c--lass2@\texttt{d{-}{-}eftypecv\_name2\ of c{-}{-}lass2}}%
\begin{quote}
\unskip{\parskip=0pt\noindent}%
d--eftypecv2
\end{quote}


\noindent\begin{tabularx}{\linewidth}{@{}Xr}
\rightskip=5em plus 1 fill \hangindent=2em \hyphenpenalty=10000
\texttt{t{-}{-}ype2 d{-}{-}eftypecv\_name2 a{-}{-}rguments2...}& [c--ategory2 of \texttt{c{-}{-}lass2}]
\end{tabularx}

\index[cp]{d--eftypecv\_name2 of c--lass2@\texttt{d{-}{-}eftypecv\_name2\ of c{-}{-}lass2}}%
\begin{quote}
\unskip{\parskip=0pt\noindent}%
d--eftypecv with arguments2
\end{quote}


\noindent\begin{tabularx}{\linewidth}{@{}Xr}
\rightskip=5em plus 1 fill \hangindent=2em \hyphenpenalty=10000
\texttt{arg2}& [fun2]
\end{tabularx}

\index[fn]{arg2@\texttt{arg2}}%
\begin{quote}
\unskip{\parskip=0pt\noindent}%
fff2
\end{quote}


\texttt{@xref\{c{-}{-}{-}hapter@@,\ cross r{-}{-}{-}ef name@@,\ t{-}{-}{-}itle@@,\ file n{-}{-}{-}ame@@,\ ma{-}{-}{-}nual@@\}} See Section ``t---itle@'' in \textsl{ma---nual@}.
\texttt{@ref\{chapter,\ cross ref name,\ title,\ file name,\ manual\}} Section ``title'' in \textsl{manual}
\texttt{@pxref\{chapter,\ cross ref name,\ title,\ file name,\ manual\}} see Section ``title'' in \textsl{manual}
\texttt{@inforef\{chapter,\ cross ref name,\ file name\}} Section ``chapter'' in \texttt{file name}

\texttt{@ref\{chapter\}} \hyperref[anchor:chapter]{\chaptername~\ref*{anchor:chapter} [chapter], page~\pageref*{anchor:chapter}}
\texttt{@xref\{chapter\}} See \hyperref[anchor:chapter]{\chaptername~\ref*{anchor:chapter} [chapter], page~\pageref*{anchor:chapter}}.
\texttt{@pxref\{chapter\}} see \hyperref[anchor:chapter]{\chaptername~\ref*{anchor:chapter} [chapter], page~\pageref*{anchor:chapter}}
\texttt{@ref\{s{-}{-}ect@comma\{\}ion\}} \hyperref[anchor:s_002d_002dect_002cion]{Section~\ref*{anchor:s_002d_002dect_002cion} [s--ect,ion], page~\pageref*{anchor:s_002d_002dect_002cion}}

\texttt{@ref\{s{-}{-}ect@comma\{\}ion,\ a @comma\{\}\ in cross
ref,\ a comma@comma\{\}\ in title,\ a comma@comma\{\}\ in file,\ a @comma\{\}\ in manual name \}}
Section ``a comma, in title'' in \textsl{a , in manual name}

\texttt{@ref\{chapter,cross ref name\}} \hyperref[anchor:chapter]{\chaptername~\ref*{anchor:chapter} [chapter], page~\pageref*{anchor:chapter}}
\texttt{@ref\{chapter{,}{,}title\}} \hyperref[anchor:chapter]{\chaptername~\ref*{anchor:chapter} [title], page~\pageref*{anchor:chapter}}
\texttt{@ref\{chapter{,}{,},file name\}} Section ``chapter'' in \texttt{file name}
\texttt{@ref\{chapter{,}{,}{,}{,}manual\}} Section ``chapter'' in \textsl{manual}
\texttt{@ref\{chapter,cross ref name,title,\}} \hyperref[anchor:chapter]{\chaptername~\ref*{anchor:chapter} [title], page~\pageref*{anchor:chapter}}
\texttt{@ref\{chapter,cross ref name{,}{,}file name\}} Section ``chapter'' in \texttt{file name}
\texttt{@ref\{chapter,cross ref name{,}{,},manual\}} Section ``chapter'' in \textsl{manual}
\texttt{@ref\{chapter,cross ref name,title,file name\}} Section ``title'' in \texttt{file name}
\texttt{@ref\{chapter,cross ref name,title{,}{,}manual\}} Section ``title'' in \textsl{manual}
\texttt{@ref\{chapter,cross ref name,title,\ file name,\ manual\}} Section ``title'' in \textsl{manual}
\texttt{@ref\{chapter{,}{,}title,file name\}} Section ``title'' in \texttt{file name}
\texttt{@ref\{chapter{,}{,}title{,}{,}manual\}} Section ``title'' in \textsl{manual}
\texttt{@ref\{chapter{,}{,}title,\ file name,\ manual\}} Section ``title'' in \textsl{manual}
\texttt{@ref\{chapter{,}{,},file name,manual\}} Section ``chapter'' in \textsl{manual}


\texttt{@ref\{(pman)anode,cross ref name\}} (pman)anode
\texttt{@ref\{(pman)anode{,}{,}title\}} title
\texttt{@ref\{(pman)anode{,}{,},file name\}} Section ``(pman)anode'' in \texttt{file name}
\texttt{@ref\{(pman)anode{,}{,}{,}{,}manual\}} Section ``(pman)anode'' in \textsl{manual}
\texttt{@ref\{(pman)anode,cross ref name,title,\}} title
\texttt{@ref\{(pman)anode,cross ref name{,}{,}file name\}} Section ``(pman)anode'' in \texttt{file name}
\texttt{@ref\{(pman)anode,cross ref name{,}{,},manual\}} Section ``(pman)anode'' in \textsl{manual}
\texttt{@ref\{(pman)anode,cross ref name,title,file name\}} Section ``title'' in \texttt{file name}
\texttt{@ref\{(pman)anode,cross ref name,title{,}{,}manual\}} Section ``title'' in \textsl{manual}
\texttt{@ref\{(pman)anode,cross ref name,title,\ file name,\ manual\}} Section ``title'' in \textsl{manual}
\texttt{@ref\{(pman)anode{,}{,}title,file name\}} Section ``title'' in \texttt{file name}
\texttt{@ref\{(pman)anode{,}{,}title{,}{,}manual\}} Section ``title'' in \textsl{manual}
\texttt{@ref\{(pman)anode{,}{,}title,\ file name,\ manual\}} Section ``title'' in \textsl{manual}
\texttt{@ref\{(pman)anode{,}{,},file name,manual\}} Section ``(pman)anode'' in \textsl{manual}


\texttt{@inforef\{chapter,\ cross ref name,\ file name\}} Section ``chapter'' in \texttt{file name}
\texttt{@inforef\{chapter\}} chapter
\texttt{@inforef\{chapter,\ cross ref name\}} chapter
\texttt{@inforef\{chapter{,}{,}file name\}} Section ``chapter'' in \texttt{file name}
\texttt{@inforef\{node,\ cross ref name,\ file name\}} Section ``node'' in \texttt{file name}
\texttt{@inforef\{node\}} node
\texttt{@inforef\{node,\ cross ref name\}} node
\texttt{@inforef\{node{,}{,}file name\}} Section ``node'' in \texttt{file name}
\texttt{@inforef\{chapter,\ cross ref name,\ file name,\ spurious arg\}} Section ``chapter'' in \texttt{file name,\ spurious arg}

\texttt{@inforef\{s{-}{-}ect@comma\{\}ion,\ a @comma\{\}\ in cross
ref,\ a comma@comma\{\}\ in file\}}
Section ``s--ect,ion'' in \texttt{a comma,\ in file}

`\texttt{\hyperref[anchor:chapter]{\chaptername~\ref*{anchor:chapter} [chapter], page~\pageref*{anchor:chapter}}}'.

Section ``title with uref2 \href{href://http/myhost.com/index2.html}{uref2 (\nolinkurl{href://http/myhost.com/index2.html})}'' in \textsl{printed manual with uref4 \href{href://http/myhost.com/index4.html}{uref4 (\nolinkurl{href://http/myhost.com/index4.html})}}
\hyperref[anchor:chapter]{\chaptername~\ref*{anchor:chapter} [title with uref2 \href{href://http/myhost.com/index2.html}{uref2 (\nolinkurl{href://http/myhost.com/index2.html})}], page~\pageref*{anchor:chapter}}

\begin{description}
\item[{\parbox[b]{\linewidth}{%
\textbf{a--strong}}}]
l--ine
\end{description}

\begin{description}
\item[{\parbox[b]{\linewidth}{%
a--asis\\
\index[cp]{a--asis@\texttt{a{-}{-}asis}}%
b
\index[cp]{b@\texttt{b}}%
}}]
l--ine
\end{description}

\begin{description}
\item[{\parbox[b]{\linewidth}{%
\emph{a}\\
\index[fn]{a@\texttt{a}}%
\index[cp]{index entry between item and itemx@index entry between item and itemx}%
\emph{b}
\index[fn]{b@\texttt{b}}%
}}]
l--ine
\end{description}

\begin{description}
\item[] Title
\item[{\parbox[b]{\linewidth}{%
\texttt{a{-}{-}code}}}]
Value--table code
\end{description}

\begin{description}
\item[] Title
\item[{\parbox[b]{\linewidth}{%
\Texinfotablestylesamp{a{-}{-}samp}\\
\Texinfotablestylesamp{a2{-}{-}samp}}}]
Value--table samp
\end{description}

\begin{mdframed}[style=Texinfocartouche]
c--artouche
\end{mdframed}

\begin{flushleft}
\begin{Texinfopreformatted}%
f--lushleft
more text
\end{Texinfopreformatted}
\end{flushleft}

\begin{flushright}
\begin{Texinfopreformatted}%
f--lushright
more text
\end{Texinfopreformatted}
\end{flushright}

\begin{center}
ce--ntered line
\end{center}

\begin{flushleft}
r--raggedright
more text
\end{flushleft}

\begin{verbatim}
\input texinfo @c -*-texinfo-*-

@c this file is used in tests in @verbatiminclude but not converted

@setfilename simplest.info

@node Top

This is a very simple texi manual @  <>.

@bye
\end{verbatim}

\begin{verbatim}
in verbatim ''
\end{verbatim}





$\frac{a < b \texttt{tex \hbox{ code }}}{b}$ ``

\Texinfonopagebreakheading{\chapter*}{{majorheading}}

\Texinfonopagebreakheading{\chapter*}{{chapheading}}

\section*{{heading}}

\subsection*{{subheading}}

\subsubsection*{{subsubheading}}


\texttt{@acronym\{{-}{-}a,an accronym @comma\{\}\ @enddots\{\}\}} --a (an accronym , \dots{})
\texttt{@abbr\{@'E{-}{-}.\ @comma\{\}A.,\ @'Etude{-}{-}@comma\{\}\ @b\{Autonome\}\ \}} \'{E}--.\@ ,A.\@ (\'{E}tude--, \textbf{Autonome})
\texttt{@abbr\{@'E{-}{-}.\ @comma\{\}A.\}} \'{E}--.\@ ,A.\@

\texttt{@math\{{-}{-}a@minus\{\}\ \{\textbackslash{}frac\{1\}\{2\}\}\}} $--a- {\frac{1}{2}}$




Somehow invalid use of @,:\leavevmode{}\\
@, \c{}\leavevmode{}\\
@,@"u \c{}\"{u}

Invalid use of @':\leavevmode{}\\
@' \'{}\leavevmode{}\\
@'@"u \'{}\"{u}

\texttt{@|} 

@dotless\{truc\} truc
@dotless\{ij\} ij
\texttt{@dotless\{{-}{-}a\}} --a
\texttt{@dotless\{a\}} a

@U, without braces @U\{\}, with empty arg 
@U\{z\}, with non-hex arg U+z
@U\{FFFFFFFFFFFFFF\}, value much too large U+FFFFFFFFFFFFFF
@U\{110000\}, value just beyond Unicode U+110000

@TeX, but without brace \TeX{}
\texttt{@\#} \#

\texttt{@w\{{-}{-}a\}} \hbox{--a}

\texttt{@image\{,1{-}{-}xt\}} 
\texttt{@image\{{,}{,}2{-}{-}xt\}} 
\texttt{@image\{{,}{,},3{-}{-}xt\}} 

\texttt{@image\{f-ile,aze{,}{,}a{-}{-}lt\}} \includegraphics[width=aze]{f-ile}
\texttt{@image\{f-ile{,}{,},alt@verb\{:jk \_" \%\@\}\}} \includegraphics{f-ile}

\texttt{@image\{f{-}{-}ile\}} \includegraphics{f--ile}
\texttt{@image\{f{-}{-}ile{,}{,},alt\}} \includegraphics{f--ile}
\texttt{@image\{f{-}{-}ile{,}{,}{,}{,}.e-d-xt\}} \includegraphics{f--ile}
\texttt{@image\{f{-}{-}ile,l{-}{-}i\}} \includegraphics[width=l--i]{f--ile}
\texttt{@image\{f{-}{-}ile{,}{,}l{-}{-}e\}} \includegraphics[height=l--e]{f--ile}
\texttt{@image\{f{-}{-}ile,aze,az,alt,.e{-}{-}xt\}} \includegraphics[width=aze,height=az]{f--ile}
\texttt{@image\{@file\{f{-}{-}ile\}@@@.,aze,az,alt,@file\{.file ext\}\ e{-}{-}xt@\}} \includegraphics[width=aze,height=az]{f--ile@.}

\texttt{@image\{f{-}{-}ile,aze,az,@verb\{:jk \_" \%@:\}\ @b\{in b "\},e{-}{-}xt\}} \includegraphics[width=aze,height=az]{f--ile}
\texttt{@image\{file@verb\{:jk \_" \%@:\}{,}{,},alt@verb\{:jk \_" \%@:\}\}} \includegraphics{filejk _" \%@}


{\bfseries author}%

$$
\ddot{u} \ddot{U} \tilde{n} \hat{a} \acute{e} \bar{o} \grave{i} \acute{e} \grave{\bar{E}}
\textsl{\c{\'{C}}} \textsl{\c{\'{C}}} \textsl{\H{a}} \dot{a} \mathring{a} \textsl{\t{a}}
\breve{a} \check{a}
 ? .
$$

$$
TeX LaTeX \star{} \mathord{\text{\aa{}}} \circledR{} ^{\circ{}} 
$$

$$
\mathtt{t} 
$$

\begin{itemize}[label=\emph{}]
\item e--mph item
\end{itemize}

\begin{itemize}[label=\emph{} after emph]
\item e--mph item
\end{itemize}

\begin{itemize}[label=\textbullet{} a--n itemize line]
\item i--tem 1
\item i--tem 2
\end{itemize}

\begin{itemize}[label={}]
\item without brace w a--b
\item without brace w c--d
\end{itemize}

\begin{description}
\item[{\parbox[b]{\linewidth}{%
a}}]
l--ine
\end{description}

\begin{description}
\item[{\parbox[b]{\linewidth}{%
a--missing style formatting}}]
l--ine
\end{description}

\begin{description}
\item[{\parbox[b]{\linewidth}{%
a\\
\index[fn]{a@\texttt{a}}%
\index[cp]{index entry between item and itemx@index entry between item and itemx}%
b
\index[fn]{b@\texttt{b}}%
}}]
l--ine
\end{description}


\noindent\begin{tabularx}{\linewidth}{@{}Xr}
\rightskip=5em plus 1 fill \hangindent=2em \hyphenpenalty=10000
\texttt{}& [fun]
\end{tabularx}


\noindent\begin{tabularx}{\linewidth}{@{}Xr}
\rightskip=5em plus 1 fill \hangindent=2em \hyphenpenalty=10000
\texttt{machin \EmbracOn{}\textnormal{\textsl{bidule chose and}}\EmbracOff{}}& [truc]
\end{tabularx}

\index[fn]{machin@\texttt{machin}}%

\noindent\begin{tabularx}{\linewidth}{@{}Xr}
\rightskip=5em plus 1 fill \hangindent=2em \hyphenpenalty=10000
\texttt{machin \EmbracOn{}\textnormal{\textsl{bidule chose and  after}}\EmbracOff{}}& [truc]
\end{tabularx}

\index[fn]{machin@\texttt{machin}}%

\noindent\begin{tabularx}{\linewidth}{@{}Xr}
\rightskip=5em plus 1 fill \hangindent=2em \hyphenpenalty=10000
\texttt{machin \EmbracOn{}\textnormal{\textsl{bidule chose and }}\EmbracOff{}}& [truc]
\end{tabularx}

\index[fn]{machin@\texttt{machin}}%

\noindent\begin{tabularx}{\linewidth}{@{}Xr}
\rightskip=5em plus 1 fill \hangindent=2em \hyphenpenalty=10000
\texttt{machin \EmbracOn{}\textnormal{\textsl{bidule chose and and after}}\EmbracOff{}}& [truc]
\end{tabularx}

\index[fn]{machin@\texttt{machin}}%

\noindent\begin{tabularx}{\linewidth}{@{}Xr}
\rightskip=5em plus 1 fill \hangindent=2em \hyphenpenalty=10000
\texttt{followed \EmbracOn{}\textnormal{\textsl{by a comment}}\EmbracOff{}}& [truc]
\end{tabularx}

\index[fn]{followed@\texttt{followed}}%
Various deff lines

\noindent\begin{tabularx}{\linewidth}{@{}Xr}
\rightskip=5em plus 1 fill \hangindent=2em \hyphenpenalty=10000
\texttt{after \EmbracOn{}\textnormal{\textsl{a deff item}}\EmbracOff{}}& [truc]
\end{tabularx}

\index[fn]{after@\texttt{after}}%


\noindent\begin{tabularx}{\linewidth}{@{}Xr}
\rightskip=5em plus 1 fill \hangindent=2em \hyphenpenalty=10000
\texttt{\Texinfocommandstyletextvar{invalid} \EmbracOn{}\textnormal{\textsl{a g}}\EmbracOff{}}& [fsetinv]
\end{tabularx}

\index[fn]{invalid@\texttt{\Texinfocommandstyletextvar{invalid}}}%

\noindent\begin{tabularx}{\linewidth}{@{}Xr}
\rightskip=5em plus 1 fill \hangindent=2em \hyphenpenalty=10000
\texttt{}& [\textbf{id `\texttt{i}' ule}]
\end{tabularx}



\noindent\begin{tabularx}{\linewidth}{@{}Xr}
\rightskip=5em plus 1 fill \hangindent=2em \hyphenpenalty=10000
\texttt{}& [aaa]
\end{tabularx}


\noindent\begin{tabularx}{\linewidth}{@{}Xr}
\rightskip=5em plus 1 fill \hangindent=2em \hyphenpenalty=10000
\texttt{}& []
\end{tabularx}


\noindent\begin{tabularx}{\linewidth}{@{}Xr}
\rightskip=5em plus 1 fill \hangindent=2em \hyphenpenalty=10000
\texttt{}& [truc]
\end{tabularx}


g--roupe

\texttt{@ref\{node\}} node

\texttt{@ref\{,cross ref name\}} 
\texttt{@ref\{{,}{,}title\}} title
\texttt{@ref\{{,}{,},file name\}} \texttt{file name}
\texttt{@ref\{{,}{,}{,}{,}manual\}} \textsl{manual}
\texttt{@ref\{node,cross ref name\}} node
\texttt{@ref\{node{,}{,}title\}} title
\texttt{@ref\{node{,}{,},file name\}} Section ``node'' in \texttt{file name}
\texttt{@ref\{node{,}{,}{,}{,}manual\}} Section ``node'' in \textsl{manual}
\texttt{@ref\{node,cross ref name,title,\}} title
\texttt{@ref\{node,cross ref name{,}{,}file name\}} Section ``node'' in \texttt{file name}
\texttt{@ref\{node,cross ref name{,}{,},manual\}} Section ``node'' in \textsl{manual}
\texttt{@ref\{node,cross ref name,title,file name\}} Section ``title'' in \texttt{file name}
\texttt{@ref\{node,cross ref name,title{,}{,}manual\}} Section ``title'' in \textsl{manual}
\texttt{@ref\{node,cross ref name,title,\ file name,\ manual\}} Section ``title'' in \textsl{manual}
\texttt{@ref\{node{,}{,}title,file name\}} Section ``title'' in \texttt{file name}
\texttt{@ref\{node{,}{,}title{,}{,}manual\}} Section ``title'' in \textsl{manual}
\texttt{@ref\{chapter{,}{,}title,\ file name,\ manual\}} Section ``title'' in \textsl{manual}
\texttt{@ref\{node{,}{,}title,\ file name,\ manual\}} Section ``title'' in \textsl{manual}
\texttt{@ref\{node{,}{,},file name,manual\}} Section ``node'' in \textsl{manual}
\texttt{@ref\{,cross ref name,title,\}} title
\texttt{@ref\{,cross ref name{,}{,}file name\}} \texttt{file name}
\texttt{@ref\{,cross ref name{,}{,},manual\}} \textsl{manual}
\texttt{@ref\{,cross ref name,title,file name\}} Section ``title'' in \texttt{file name}
\texttt{@ref\{,cross ref name,title{,}{,}manual\}} Section ``title'' in \textsl{manual}
\texttt{@ref\{,cross ref name,title,\ file name,\ manual\}} Section ``title'' in \textsl{manual}
\texttt{@ref\{{,}{,}title,file name\}} Section ``title'' in \texttt{file name}
\texttt{@ref\{{,}{,}title{,}{,}manual\}} Section ``title'' in \textsl{manual}
\texttt{@ref\{{,}{,}title,\ file name,\ manual\}} Section ``title'' in \textsl{manual}
\texttt{@ref\{{,}{,},file name,manual\}} \textsl{manual}

\texttt{@inforef\{,cross ref name \}} 
\texttt{@inforef\{{,}{,}file name\}} \texttt{file name}
\texttt{@inforef\{,cross ref name,\ file name\}} \texttt{file name}
\texttt{@inforef\{\}} 


\end{titlepage}
\pagestyle{single}%
\mainmatter
\tableofcontents\newpage





\label{anchor:Top}%
\chapter{{chapter}}
\label{anchor:chapter}%

First para

\noindent{}qsddsqkdsqkkmljsqjsqodmmdsqdsmqj dqs sdq sqd sdq dsq sdq sqd sqd sdq sdq 
qsd dsq sdq dsq dssdq sdq sdq sdq dsq sdq dsq dsq sdq dsq sdqsd q

\noindent{}noindent in para.

unneeded indent

Insertcopying in normal text
In copying

<
>
"
\&
'
`

``simple-double--three---four----''\leavevmode{}\\
code: \texttt{{`}{`}simple-double{-}{-}three{-}{-}{-}four{-}{-}{-}-{'}{'}} \leavevmode{}\\
asis: ``simple-double--three---four----'' \leavevmode{}\\
strong: \textbf{``simple-double--three---four----''} \leavevmode{}\\
kbd: \Texinfocommandstyletextkbd{{`}{`}simple-double{-}{-}three{-}{-}{-}four{-}{-}{-}-{'}{'}} \leavevmode{}\\

`\hbox{}`simple-double-\hbox{}-three---four----'\hbox{}'\leavevmode{}\\

\index[cp]{-option@--option}%
\index[cp]{""@``}%
\index[fn]{``@\texttt{{`}{`}}}%
\index[fn]{--foption@\texttt{{-}{-}foption}}%

@"u \"{u} 
@"\{U\} \"{U} 
@\~{}n \~{n}
@\^{}a \^{a}
@'e \'{e}
@=o \={o}
@`i \`{i}
@'\{e\} \'{e}
@'\{@dotless\{i\}\} \'{\i{}} 
@dotless\{i\} \i{}
@dotless\{j\} \j{}
@`\{@=E\} \`{\={E}} 
@l\{\} \l{}
@,\{@'C\} \c{\'{C}}
@,c \c{c}
@,c@"u \c{c}\"{u} \leavevmode{}\\

@U\{0075\} u

@* \leavevmode{}\\
@ followed by a space
\ {}
@ followed by a tab
\ {}
@ followed by a new line
\ {}\texttt{@-} \-{}
\texttt{@:} \@
\texttt{@!} \@!
\texttt{@?} \@?
\texttt{@.} \@.
\texttt{@@} @
\texttt{@\}} \}
\texttt{@\{} \{
\texttt{@/} 

foo vs.\@ bar. 
colon :\@And something else.
semi colon ;\@.
And ? ?\@.
Now ! !\@@
but , ,\@

@TeX \TeX{}
@LaTeX \LaTeX{}
@bullet \textbullet{}
@copyright \copyright{}
@dots \dots{}\@
@enddots \dots{}
@equiv $\equiv{}$
@error \fbox{error}
@expansion $\mapsto{}$
@minus -
@point $\star{}$
@print $\dashv{}$
@result $\Rightarrow{}$
@today \today{}

@aa \aa{}
@AA \AA{}
@ae \ae{}
@oe \oe{}
@AE \AE{}
@OE \OE{}
@o \o{}
@O \O{}
@ss \ss{}
@l \l{}
@L \L{}
@DH \DH{}
@TH \TH{}
@dh \dh{}
@th \th{}

@exclamdown \textexclamdown{}
@questiondown \textquestiondown{}
@pounds \textsterling{}
@registeredsymbol \circledR{}
@ordf \textordfeminine{}
@ordm \textordmasculine{}
@comma ,
@quotedblleft \textquotedblleft{}
@quotedblright \textquotedblright{}
@quoteleft \textquoteleft{}
@quoteright \textquoteright{}
@quotedblbase \quotedblbase{}
@quotesinglbase \quotesinglbase{}
@guillemetleft \guillemotleft{}
@guillemetright \guillemotright{}
@guillemotleft \guillemotleft{}
@guillemotright \guillemotright{}
@guilsinglleft \guilsinglleft{}
@guilsinglright \guilsinglright{}

@textdegree \textdegree{}
@euro \euro{}
@arrow $\rightarrow{}$
@leq $\leq{}$
@geq $\geq{}$
@tie a~b

\texttt{@acronym\{{-}{-}a,an accronym\}} --a (an accronym)
\texttt{@acronym\{{-}{-}a\}} --a
\texttt{@abbr\{@'E{-}{-}.\ @comma\{\}A.,\ @'Etude Autonome \}} \'{E}--.\@ ,A.\@ (\'{E}tude Autonome)
\texttt{@abbr\{@'E{-}{-}.\ @comma\{\}A.\}} \'{E}--.\@ ,A.\@
\texttt{@asis\{{-}{-}a\}} --a
\texttt{@b\{{-}{-}a\}} \textbf{--a}
\texttt{@cite\{{-}{-}a\}} \Texinfocommandstyletextcite{--a}
\texttt{@code\{{-}{-}a\}} \texttt{{-}{-}a}
\texttt{@command\{{-}{-}a\}} \texttt{{-}{-}a}
\texttt{@dfn\{{-}{-}a\}} \textsl{--a}
\texttt{@dmn\{{-}{-}a\}} \thinspace --a
\texttt{@email\{{-}{-}a,{-}{-}b\}} \href{mailto:--a}{--b}
\texttt{@email\{,{-}{-}b\}} --b
\texttt{@email\{{-}{-}a\}} \href{mailto:--a}{\nolinkurl{--a}}
\texttt{@emph\{{-}{-}a\}} \emph{--a}
\texttt{@env\{{-}{-}a\}} \texttt{{-}{-}a}
\texttt{@file\{{-}{-}a\}} \texttt{{-}{-}a}
\texttt{@i\{{-}{-}a\}} \textit{--a}
\texttt{@kbd\{{-}{-}a\}} \Texinfocommandstyletextkbd{{-}{-}a}
\texttt{@key\{{-}{-}a\}} \texttt{{-}{-}a}
\texttt{@math\{{-}{-}a \{\textbackslash{}frac\{1\}\{2\}\}\ @minus\{\}\}} $--a {\frac{1}{2}} -$
\texttt{@option\{{-}{-}a\}} \texttt{{-}{-}a}
\texttt{@r\{{-}{-}a\}} \textnormal{--a}
\texttt{@samp\{{-}{-}a\}} `\texttt{{-}{-}a}'
\texttt{@sc\{{-}{-}a\}} \textsc{--a}
\texttt{@strong\{{-}{-}a\}} \textbf{--a}
\texttt{@t\{{-}{-}a\}} \texttt{{-}{-}a}
\texttt{@sansserif\{{-}{-}a\}} \textsf{--a}
\texttt{@slanted\{{-}{-}a\}} \textsl{--a}
\texttt{@titlefont\{{-}{-}a\}} {\huge \bfseries --a}
\texttt{@indicateurl\{{-}{-}a\}} `\texttt{{-}{-}a}'
\texttt{@uref\{{-}{-}a,{-}{-}b\}} \href{--a}{--b (\nolinkurl{--a})}
\texttt{@uref\{{-}{-}a\}} \url{--a}
\texttt{@uref\{,{-}{-}b\}} --b
\texttt{@uref\{{-}{-}a,{-}{-}b,{-}{-}c\}} --c
\texttt{@uref\{,{-}{-}b,{-}{-}c\}} --c
\texttt{@uref\{{-}{-}a{,}{,}{-}{-}c\}} --c
\texttt{@uref\{{,}{,}{-}{-}c\}} --c
\texttt{@url\{{-}{-}a,{-}{-}b\}} \href{--a}{--b (\nolinkurl{--a})}
\texttt{@url\{{-}{-}a,\}} \url{--a}
\texttt{@url\{,{-}{-}b\}} --b
\texttt{@var\{{-}{-}a\}} \Texinfocommandstyletextvar{--a}
\texttt{@verb\{:{-}{-}a:\}} \verb:--a:
\texttt{@verb\{:a  < \& @\ \% " {-}{-}    b:\}} \verb:a  < & @ % " --    b:
\texttt{@w\{a a a a a a a a a a a a a a a a a a a a a a a a a a a a a a a a a a a\}} \hbox{a a a a a a a a a a a a a a a a a a a a a a a a a a a a a a a a a a a}
\texttt{@H\{a\}} \H{a}
\texttt{@H\{{-}{-}a\}} \H{--a}
\texttt{@dotaccent\{a\}} \.{a}
\texttt{@dotaccent\{{-}{-}a\}} \.{--a}
\texttt{@ringaccent\{a\}} \r{a}
\texttt{@ringaccent\{{-}{-}a\}} \r{--a}
\texttt{@tieaccent\{a\}} \t{a}
\texttt{@tieaccent\{{-}{-}a\}} \t{--a}
\texttt{@u\{a\}} \u{a}
\texttt{@u\{{-}{-}a\}} \u{--a}
\texttt{@ubaraccent\{a\}} \b{a}
\texttt{@ubaraccent\{{-}{-}a\}} \b{--a}
\texttt{@udotaccent\{a\}} \d{a}
\texttt{@udotaccent\{{-}{-}a\}} \d{--a}
\texttt{@v\{a\}} \v{a}
\texttt{@v\{{-}{-}a\}} \v{--a}
\texttt{@,\{c\}} \c{c}
\texttt{@,\{{-}{-}c\}} \c{--c}
\texttt{@ogonek\{a\}} \k{a}
\texttt{@ogonek\{{-}{-}a\}} \k{--a}
\texttt{a@sup\{h\}@sub\{l\}} a\textsuperscript{h}\textsubscript{l}
\texttt{@footnote\{in footnote\}} \footnote{in footnote}
\texttt{@footnote\{in footnote2\}} \footnote{in footnote2}

\texttt{@sp 2}\leavevmode{}\\
\vskip 2\baselineskip %
\texttt{@page}\leavevmode{}\\
\newpage{}%
\phantom{blabla}%

\texttt{need 1002}
\needspace{1.002pt}%

\texttt{@clicksequence\{click @click\{\}\ A\}} click $\rightarrow{}$ A
After clickstyle $\Rightarrow{}$
\texttt{@clicksequence\{click @click\{\}\ A\}} click $\Rightarrow{}$ A


$$
disp--laymath
f(x) = {1 \over \sigma \sqrt{2\pi}}e^{-{1 \over 2}\left({x-\mu \over \sigma}\right)^2}
$$

$$
\mathbf{``simple-double--three---four----''} \hbox{aa}
`\hbox{}`simple-double-\hbox{}-three---four----'\hbox{}'
$$

$$
\imath{} \jmath{}
\mathord{\text{\l{}}} \textsl{\c{c}}
\textsl{\b{a}} \textsl{\d{a}} \textsl{\k{a}} a^{h}_{l}
 \ {}\ {} \ {}\-{}  ! @ \} \{ 
\today{}
$$

$$
\rightarrow{}
u
\bullet{} \copyright{} \dots{} \dots{} \equiv{}
\fbox{error} \mapsto{} - \dashv{} \Rightarrow{}
\mathord{\text{\AA{}}} \mathord{\text{\ae{}}} \mathord{\text{\oe{}}} \mathord{\text{\AE{}}} \mathord{\text{\OE{}}} \mathord{\text{\o{}}} \mathord{\text{\O{}}} \mathord{\text{\ss{}}} \mathord{\text{\l{}}} \mathord{\text{\L{}}} \mathord{\text{\DH{}}}
\mathord{\text{\TH{}}} \mathord{\text{\dh{}}} \mathord{\text{\th{}}} \mathord{\text{\textexclamdown{}}} \mathord{\text{\textquestiondown{}}} \mathsterling{}
\mathord{\text{\textordfeminine{}}} \mathord{\text{\textordmasculine{}}} , 
$$

$$
\mathord{\text{\textquotedblleft{}}} \mathord{\text{\textquotedblright{}}} 
\mathord{\text{\textquoteleft{}}} \mathord{\text{\textquoteright{}}} \mathord{\text{\quotedblbase{}}} \mathord{\text{\quotesinglbase{}}} \mathord{\text{\guillemotleft{}}}
\mathord{\text{\guillemotright{}}} \mathord{\text{\guillemotleft{}}} \mathord{\text{\guillemotright{}}} \mathord{\text{\guilsinglleft{}}}
\mathord{\text{\guilsinglright{}}} \euro{} \rightarrow{} \leq{} \geq{}
$$

$$
\mathbf{b} \mathit{i} \mathrm{r} sc \mathsf{sansserif} slanted
$$

\Texinfocommandstyletextkbd{default kbdinputstyle}
\begin{description}
\item[{\parbox[b]{\linewidth}{%
\Texinfocommandstyletextkbd{vtable i{-}{-}tem default kbdinputstyle}
\index[cp]{vtable i--tem default kbdinputstyle@\texttt{vtable i{-}{-}tem default kbdinputstyle}}%
}}]
\end{description}
\begin{Texinfoindented}
\begin{Texinfopreformatted}%
\ttfamily \Texinfocommandstyletextkbd{in example default kbdinputstyle}
\end{Texinfopreformatted}
\begin{description}
\item[{\parbox[b]{\linewidth}{%
\Texinfocommandstyletextkbd{vtable i{-}{-}tem in example default kbdinputstyle}
\index[cp]{vtable i--tem in example default kbdinputstyle@\texttt{vtable i{-}{-}tem in example default kbdinputstyle}}%
}}]
\end{description}
\end{Texinfoindented}

\texttt{code kbdinputstyle}
\begin{description}
\item[{\parbox[b]{\linewidth}{%
\texttt{vtable i{-}{-}tem code kbdinputstyle}
\index[cp]{vtable i--tem code kbdinputstyle@\texttt{vtable i{-}{-}tem code kbdinputstyle}}%
}}]
\end{description}
\begin{Texinfoindented}
\begin{Texinfopreformatted}%
\ttfamily \texttt{in example code kbdinputstyle}
\end{Texinfopreformatted}
\begin{description}
\item[{\parbox[b]{\linewidth}{%
\texttt{vtable i{-}{-}tem in example code kbdinputstyle}
\index[cp]{vtable i--tem in example code kbdinputstyle@\texttt{vtable i{-}{-}tem in example code kbdinputstyle}}%
}}]
\end{description}
\end{Texinfoindented}

\texttt{example kbdinputstyle}
\begin{description}
\item[{\parbox[b]{\linewidth}{%
\texttt{vtable i{-}{-}tem example kbdinputstyle}
\index[cp]{vtable i--tem example kbdinputstyle@\texttt{vtable i{-}{-}tem example kbdinputstyle}}%
}}]
\end{description}
\begin{Texinfoindented}
\begin{Texinfopreformatted}%
\ttfamily \Texinfocommandstyletextkbd{in example example kbdinputstyle}
\end{Texinfopreformatted}
\begin{description}
\item[{\parbox[b]{\linewidth}{%
\Texinfocommandstyletextkbd{vtable i{-}{-}tem in example example kbdinputstyle}
\index[cp]{vtable i--tem in example example kbdinputstyle@\texttt{vtable i{-}{-}tem in example example kbdinputstyle}}%
}}]
\end{description}
\end{Texinfoindented}

\Texinfocommandstyletextkbd{distinct kbdinputstyle}
\begin{description}
\item[{\parbox[b]{\linewidth}{%
\Texinfocommandstyletextkbd{vtable i{-}{-}tem distinct kbdinputstyle}
\index[cp]{vtable i--tem distinct kbdinputstyle@\texttt{vtable i{-}{-}tem distinct kbdinputstyle}}%
}}]
\end{description}
\begin{Texinfoindented}
\begin{Texinfopreformatted}%
\ttfamily \Texinfocommandstyletextkbd{in example distinct kbdinputstyle}
\end{Texinfopreformatted}
\begin{description}
\item[{\parbox[b]{\linewidth}{%
\Texinfocommandstyletextkbd{vtable i{-}{-}tem in example distinct kbdinputstyle}
\index[cp]{vtable i--tem in example distinct kbdinputstyle@\texttt{vtable i{-}{-}tem in example distinct kbdinputstyle}}%
}}]
\end{description}
\end{Texinfoindented}

\begin{quote}
A quot---ation
\end{quote}

\begin{quote}
\textbf{Note:} A Note
\end{quote}

\begin{quote}
\textbf{note:} A note
\end{quote}

\begin{quote}
\textbf{Caution:} Caution
\end{quote}

\begin{quote}
\textbf{Important:} Important
\end{quote}

\begin{quote}
\textbf{Tip:} a Tip
\end{quote}

\begin{quote}
\textbf{Warning:} a Warning.
\end{quote}

\begin{quote}
\textbf{something \'{e} \TeX{}:} The something \'{e} \TeX{} is here.
\end{quote}

\begin{quote}
\textbf{@ at the end of line \ {}:} A @ at the end of the @quotation line.
\end{quote}

\begin{quote}
\textbf{something, other thing:} something, other thing
\end{quote}

\begin{quote}
\textbf{Note, the note:} Note, the note
\end{quote}

\begin{quote}
\end{quote}

\begin{quote}
\textbf{Empty:} \end{quote}

\begin{quote}
\textbf{:} \end{quote}

\begin{quote}
\textbf{\leavevmode{}\\:} \end{quote}

\begin{quote}
aaa quotation
\end{quote}
\begin{center}
--- \emph{quotation author}
\end{center}

\begin{quote}
indent in quotation
\end{quote}

\begin{quote}
\leavevmode{}\\
\hbox{\kern -\leftmargin}%
exdented quotation line   and dash --- in quotation
\\
\end{quote}

\begin{quote}
Not exdented followed by exdented
\leavevmode{}\\
\hbox{\kern -\leftmargin}%
exdented quotation line
\\
\end{quote}

\begin{quote}
\leavevmode{}\\
\hbox{\kern -\leftmargin}%
exdented quotation line
\\
Followed by not exdented 
\end{quote}

\begin{quote}
quotation1
\leavevmode{}\\
\hbox{\kern -\leftmargin}%
in exdented protected eol \ {}
\\
following
\leavevmode{}\\
\hbox{\kern -\leftmargin}%
in exdented a @* \leavevmode{}\\ and following
\\
after exdented
\end{quote}

\begin{quote}
\begin{footnotesize}
A small quot---ation
\end{footnotesize}
\end{quote}

\begin{quote}
\begin{footnotesize}
\textbf{Note:} A small Note
\end{footnotesize}
\end{quote}

\begin{quote}
\begin{footnotesize}
\textbf{something, other thing:} something, other thing
\end{footnotesize}
\end{quote}

\begin{itemize}
\item i--temize
\end{itemize}

\begin{itemize}[label=+]
\item i--tem +
\end{itemize}

\begin{itemize}[label=\textbullet{}]
\item b--ullet
\end{itemize}

\begin{itemize}[label=-]
\item minu--s
\end{itemize}

\begin{itemize}[label=\emph{after emph}]
\item e--mph item
\end{itemize}

\begin{itemize}[label=\textbullet{} a--n itemize line]
\item \index[cp]{index entry within itemize@index entry within itemize}%
i--tem 1
\item i--tem 2
\end{itemize}

\begin{itemize}[label={}]
\item with w a--b
\item with w c--d
\end{itemize}

\begin{itemize}[label=\hbox{} on a line]
\item line w a--b
\item line with w c--d
\end{itemize}

\begin{enumerate}[start=1]
\item e--numerate
\end{enumerate}

\begin{enumerate}[start=3]
\item first third
\item second third
\end{enumerate}

\begin{enumerate}[label=\alph*.]
\item e--numerate
\end{enumerate}

\begin{enumerate}[label=\alph*.,start=3]
\item first c
\item second c
\end{enumerate}

\begin{tabular}{m{0.4\textwidth} m{0.6\textwidth}}%
mu--ltitable headitem &another tab\\
mu--ltitable item &multitable tab\\
mu--ltitable item 2 &multitable tab 2
\index[cp]{index entry within multitable@index entry within multitable}%
\\
lone mu--ltitable item&\\
\end{tabular}%

\begin{tabular}{m{0.4\textwidth} m{0.6\textwidth}}%
truc &bidule\\
\end{tabular}%

\begin{Texinfoindented}
\begin{Texinfopreformatted}%
\ttfamily e{-}{-}xample  some
\   text
\end{Texinfopreformatted}
\end{Texinfoindented}

\begin{Texinfoindented}
\begin{Texinfopreformatted}%
\ttfamily example one arg
\end{Texinfopreformatted}
\end{Texinfoindented}

\begin{Texinfoindented}
\begin{Texinfopreformatted}%
\ttfamily example two args
\end{Texinfopreformatted}
\end{Texinfoindented}

\begin{Texinfoindented}
\begin{Texinfopreformatted}%
\ttfamily example three args
\end{Texinfopreformatted}
\end{Texinfoindented}

\begin{Texinfoindented}
\begin{Texinfopreformatted}%
\ttfamily example four args
\end{Texinfopreformatted}
\end{Texinfoindented}

\begin{Texinfoindented}
\begin{Texinfopreformatted}%
\ttfamily example five args
\end{Texinfopreformatted}
\end{Texinfoindented}

\begin{Texinfoindented}
\begin{Texinfopreformatted}%
\ttfamily The something \'{e}\ \TeX{}\ is here.
\end{Texinfopreformatted}
\end{Texinfoindented}

\begin{Texinfoindented}
\begin{Texinfopreformatted}%
\ttfamily A @\ at the end of the @example line.
\end{Texinfopreformatted}
\end{Texinfoindented}

\begin{Texinfoindented}
\begin{Texinfopreformatted}%
\ttfamily example with empty args
\end{Texinfopreformatted}
\end{Texinfoindented}

\begin{Texinfoindented}
\begin{Texinfopreformatted}%
\ttfamily example with empty and non empty args mix
\end{Texinfopreformatted}
\end{Texinfoindented}

\begin{Texinfoindented}
\begin{Texinfopreformatted}%
\ttfamily Exam{-}{-}{-}ple

\end{Texinfopreformatted}
\leavevmode{}\\
\hbox{\kern -\leftmargin}%
Other li---ne
\\
\begin{Texinfopreformatted}%
\ttfamily not exdented
\end{Texinfopreformatted}
\end{Texinfoindented}

\begin{Texinfoindented}
\leavevmode{}\\
\hbox{\kern -\leftmargin}%
exdented  and dash --- in example
\\
\begin{Texinfopreformatted}%
\ttfamily Not exdented one
\end{Texinfopreformatted}
\leavevmode{}\\
\hbox{\kern -\leftmargin}%
exdented two
\\
\begin{Texinfopreformatted}%
\ttfamily Not exdented two
\end{Texinfopreformatted}
\end{Texinfoindented}

\begin{Texinfoindented}
\begin{Texinfopreformatted}%
\ttfamily Example   Hoho.
\end{Texinfopreformatted}
\begin{Texinfoindented}
\begin{Texinfopreformatted}%
\ttfamily Nested Other line
\end{Texinfopreformatted}
\leavevmode{}\\
\hbox{\kern -\leftmargin}%
exdented nested other line
\\
\end{Texinfoindented}
\end{Texinfoindented}

\begin{Texinfopreformatted}%
\ttfamily \footnotesize s{-}{-}mallexample
\end{Texinfopreformatted}

\texttt{@noindent} after smallexample.
\begin{Texinfopreformatted}%
\ttfamily \footnotesize \$ wget 'http://savannah.gnu.org/cgi-bin/viewcvs/config/config/config.guess?rev=HEAD\&content-type=text/plain'
\$ wget 'http://savannah.gnu.org/cgi-bin/viewcvs/config/config/config.sub?rev=HEAD\&content-type=text/plain'
\end{Texinfopreformatted}
\noindent{}Less recent versions are also present.

\begin{Texinfoindented}
\begin{Texinfopreformatted}%
d--isplay
\end{Texinfopreformatted}
\end{Texinfoindented}

\begin{Texinfopreformatted}%
\footnotesize s--malldisplay
\end{Texinfopreformatted}

\begin{Texinfoindented}
\begin{Texinfopreformatted}%
\ttfamily l{-}{-}isp
\end{Texinfopreformatted}
\end{Texinfoindented}

\begin{Texinfopreformatted}%
\ttfamily \footnotesize s{-}{-}malllisp
\end{Texinfopreformatted}

\begin{Texinfopreformatted}%
f--ormat
\end{Texinfopreformatted}

\begin{Texinfopreformatted}%
\footnotesize s--mallformat
\end{Texinfopreformatted}


\noindent\begin{tabularx}{\linewidth}{@{}Xr}
\rightskip=5em plus 1 fill \hangindent=2em \hyphenpenalty=10000
\texttt{d{-}{-}effn\_name \EmbracOn{}\textnormal{\textsl{a--rguments...}}\EmbracOff{}}& [c--ategory]
\end{tabularx}

\index[fn]{d--effn\_name@\texttt{d{-}{-}effn\_name}}%
\begin{quote}
\unskip{\parskip=0pt\noindent}%
d--effn
\end{quote}


\noindent\begin{tabularx}{\linewidth}{@{}Xr}
\rightskip=5em plus 1 fill \hangindent=2em \hyphenpenalty=10000
\texttt{de{-}{-}ffn\_name \EmbracOn{}\textnormal{\textsl{ar--guments    more args   even more so}}\EmbracOff{}}& [cate--gory]
\end{tabularx}

\index[fn]{de--ffn\_name@\texttt{de{-}{-}ffn\_name}}%
\begin{quote}
\unskip{\parskip=0pt\noindent}%
def--fn
\end{quote}


\noindent\begin{tabularx}{\linewidth}{@{}Xr}
\rightskip=5em plus 1 fill \hangindent=2em \hyphenpenalty=10000
\texttt{\Texinfocommandstyletextvar{i} \EmbracOn{}\textnormal{\textsl{a g}}\EmbracOff{}}& [fset]
\end{tabularx}

\index[fn]{i@\texttt{\Texinfocommandstyletextvar{i}}}%
\index[cp]{index entry within deffn@index entry within deffn}%

\noindent\begin{tabularx}{\linewidth}{@{}Xr}
\rightskip=5em plus 1 fill \hangindent=2em \hyphenpenalty=10000
\texttt{truc \EmbracOn{}\textnormal{\textsl{}}\EmbracOff{}}& [cmde]
\end{tabularx}

\index[fn]{truc@\texttt{truc}}%

\noindent\begin{tabularx}{\linewidth}{@{}Xr}
\rightskip=5em plus 1 fill \hangindent=2em \hyphenpenalty=10000
\texttt{log trap \EmbracOn{}\textnormal{\textsl{}}\EmbracOff{}}& [Command]
\end{tabularx}

\index[fn]{log trap@\texttt{log trap}}%

\noindent\begin{tabularx}{\linewidth}{@{}Xr}
\rightskip=5em plus 1 fill \hangindent=2em \hyphenpenalty=10000
\texttt{log trap1 \EmbracOn{}\textnormal{\textsl{}}\EmbracOff{}}& [Command]
\end{tabularx}

\index[fn]{log trap1@\texttt{log trap1}}%

\noindent\begin{tabularx}{\linewidth}{@{}Xr}
\rightskip=5em plus 1 fill \hangindent=2em \hyphenpenalty=10000
\texttt{log trap2 \EmbracOn{}\textnormal{\textsl{}}\EmbracOff{}}& [Command]
\end{tabularx}

\index[fn]{log trap2@\texttt{log trap2}}%

\noindent\begin{tabularx}{\linewidth}{@{}Xr}
\rightskip=5em plus 1 fill \hangindent=2em \hyphenpenalty=10000
\texttt{\textbf{id ule} \EmbracOn{}\textnormal{\textsl{truc}}\EmbracOff{}}& [cmde]
\end{tabularx}

\index[fn]{id ule@\texttt{\textbf{id ule}}}%

\noindent\begin{tabularx}{\linewidth}{@{}Xr}
\rightskip=5em plus 1 fill \hangindent=2em \hyphenpenalty=10000
\texttt{\textbf{id `\texttt{i}'\ ule} \EmbracOn{}\textnormal{\textsl{truc}}\EmbracOff{}}& [cmde2]
\end{tabularx}

\index[fn]{id i ule@\texttt{\textbf{id `\texttt{i}'\ ule}}}%

\noindent\begin{tabularx}{\linewidth}{@{}Xr}
\rightskip=5em plus 1 fill \hangindent=2em \hyphenpenalty=10000
\texttt{}& []
\end{tabularx}


\noindent\begin{tabularx}{\linewidth}{@{}Xr}
\rightskip=5em plus 1 fill \hangindent=2em \hyphenpenalty=10000
\texttt{machin}& []
\end{tabularx}

\index[fn]{machin@\texttt{machin}}%

\noindent\begin{tabularx}{\linewidth}{@{}Xr}
\rightskip=5em plus 1 fill \hangindent=2em \hyphenpenalty=10000
\texttt{bidule machin}& []
\end{tabularx}

\index[fn]{bidule machin@\texttt{bidule machin}}%

\noindent\begin{tabularx}{\linewidth}{@{}Xr}
\rightskip=5em plus 1 fill \hangindent=2em \hyphenpenalty=10000
\texttt{machin}& [truc]
\end{tabularx}

\index[fn]{machin@\texttt{machin}}%

\noindent\begin{tabularx}{\linewidth}{@{}Xr}
\rightskip=5em plus 1 fill \hangindent=2em \hyphenpenalty=10000
\texttt{}& [truc]
\end{tabularx}


\noindent\begin{tabularx}{\linewidth}{@{}Xr}
\rightskip=5em plus 1 fill \hangindent=2em \hyphenpenalty=10000
\texttt{followed \EmbracOn{}\textnormal{\textsl{by a comment}}\EmbracOff{}}& [truc]
\end{tabularx}

\index[fn]{followed@\texttt{followed}}%

\noindent\begin{tabularx}{\linewidth}{@{}Xr}
\rightskip=5em plus 1 fill \hangindent=2em \hyphenpenalty=10000
\texttt{}& []
\end{tabularx}


\noindent\begin{tabularx}{\linewidth}{@{}Xr}
\rightskip=5em plus 1 fill \hangindent=2em \hyphenpenalty=10000
\texttt{a \EmbracOn{}\textnormal{\textsl{b c d e \textbf{f g} h i}}\EmbracOff{}}& [truc]
\end{tabularx}

\index[fn]{a@\texttt{a}}%

\noindent\begin{tabularx}{\linewidth}{@{}Xr}
\rightskip=5em plus 1 fill \hangindent=2em \hyphenpenalty=10000
\texttt{deffnx \EmbracOn{}\textnormal{\textsl{before end deffn}}\EmbracOff{}}& [truc]
\end{tabularx}

\index[fn]{deffnx@\texttt{deffnx}}%



\noindent\begin{tabularx}{\linewidth}{@{}Xr}
\rightskip=5em plus 1 fill \hangindent=2em \hyphenpenalty=10000
\texttt{deffn}& [empty]
\end{tabularx}

\index[fn]{deffn@\texttt{deffn}}%


\noindent\begin{tabularx}{\linewidth}{@{}Xr}
\rightskip=5em plus 1 fill \hangindent=2em \hyphenpenalty=10000
\texttt{deffn \EmbracOn{}\textnormal{\textsl{with deffnx}}\EmbracOff{}}& [empty]
\end{tabularx}

\index[fn]{deffn@\texttt{deffn}}%

\noindent\begin{tabularx}{\linewidth}{@{}Xr}
\rightskip=5em plus 1 fill \hangindent=2em \hyphenpenalty=10000
\texttt{deffnx}& [empty]
\end{tabularx}

\index[fn]{deffnx@\texttt{deffnx}}%


\noindent\begin{tabularx}{\linewidth}{@{}Xr}
\rightskip=5em plus 1 fill \hangindent=2em \hyphenpenalty=10000
\texttt{\Texinfocommandstyletextvar{i} \EmbracOn{}\textnormal{\textsl{a g}}\EmbracOff{}}& [fset]
\end{tabularx}

\index[fn]{i@\texttt{\Texinfocommandstyletextvar{i}}}%

\noindent\begin{tabularx}{\linewidth}{@{}Xr}
\rightskip=5em plus 1 fill \hangindent=2em \hyphenpenalty=10000
\texttt{truc \EmbracOn{}\textnormal{\textsl{}}\EmbracOff{}}& [cmde]
\end{tabularx}

\index[fn]{truc@\texttt{truc}}%
\begin{quote}
\unskip{\parskip=0pt\noindent}%
text in def item for second def item
\end{quote}



\noindent\begin{tabularx}{\linewidth}{@{}Xr}
\rightskip=5em plus 1 fill \hangindent=2em \hyphenpenalty=10000
\texttt{d{-}{-}efvr\_name}& [c--ategory]
\end{tabularx}

\index[cp]{d--efvr\_name@\texttt{d{-}{-}efvr\_name}}%
\begin{quote}
\unskip{\parskip=0pt\noindent}%
d--efvr
\end{quote}


\noindent\begin{tabularx}{\linewidth}{@{}Xr}
\rightskip=5em plus 1 fill \hangindent=2em \hyphenpenalty=10000
\texttt{n{-}{-}ame \EmbracOn{}\textnormal{\textsl{a--rguments...}}\EmbracOff{}}& [c--ategory]
\end{tabularx}

\index[fn]{n--ame@\texttt{n{-}{-}ame}}%
\begin{quote}
\unskip{\parskip=0pt\noindent}%
d--effn
\end{quote}


\noindent\begin{tabularx}{\linewidth}{@{}Xr}
\rightskip=5em plus 1 fill \hangindent=2em \hyphenpenalty=10000
\texttt{n{-}{-}ame}& [c--ategory]
\end{tabularx}

\index[fn]{n--ame@\texttt{n{-}{-}ame}}%
\begin{quote}
\unskip{\parskip=0pt\noindent}%
d--effn no arg
\end{quote}


\noindent\begin{tabularx}{\linewidth}{@{}Xr}
\rightskip=5em plus 1 fill \hangindent=2em \hyphenpenalty=10000
\texttt{t{-}{-}ype d{-}{-}eftypefn\_name a{-}{-}rguments...}& [c--ategory]
\end{tabularx}

\index[fn]{d--eftypefn\_name@\texttt{d{-}{-}eftypefn\_name}}%
\begin{quote}
\unskip{\parskip=0pt\noindent}%
d--eftypefn
\end{quote}


\noindent\begin{tabularx}{\linewidth}{@{}Xr}
\rightskip=5em plus 1 fill \hangindent=2em \hyphenpenalty=10000
\texttt{t{-}{-}ype d{-}{-}eftypefn\_name}& [c--ategory]
\end{tabularx}

\index[fn]{d--eftypefn\_name@\texttt{d{-}{-}eftypefn\_name}}%
\begin{quote}
\unskip{\parskip=0pt\noindent}%
d--eftypefn no arg
\end{quote}


\noindent\begin{tabularx}{\linewidth}{@{}Xr}
\rightskip=5em plus 1 fill \hangindent=2em \hyphenpenalty=10000
\texttt{t{-}{-}ype d{-}{-}eftypeop\_name a{-}{-}rguments...}& [c--ategory on \texttt{c{-}{-}lass}]
\end{tabularx}

\index[fn]{d--eftypeop\_name on c--lass@\texttt{d{-}{-}eftypeop\_name\ on c{-}{-}lass}}%
\begin{quote}
\unskip{\parskip=0pt\noindent}%
d--eftypeop
\end{quote}


\noindent\begin{tabularx}{\linewidth}{@{}Xr}
\rightskip=5em plus 1 fill \hangindent=2em \hyphenpenalty=10000
\texttt{t{-}{-}ype d{-}{-}eftypeop\_name}& [c--ategory on \texttt{c{-}{-}lass}]
\end{tabularx}

\index[fn]{d--eftypeop\_name on c--lass@\texttt{d{-}{-}eftypeop\_name\ on c{-}{-}lass}}%
\begin{quote}
\unskip{\parskip=0pt\noindent}%
d--eftypeop no arg
\end{quote}


\noindent\begin{tabularx}{\linewidth}{@{}Xr}
\rightskip=5em plus 1 fill \hangindent=2em \hyphenpenalty=10000
\texttt{t{-}{-}ype d{-}{-}eftypevr\_name}& [c--ategory]
\end{tabularx}

\index[cp]{d--eftypevr\_name@\texttt{d{-}{-}eftypevr\_name}}%
\begin{quote}
\unskip{\parskip=0pt\noindent}%
d--eftypevr
\end{quote}


\noindent\begin{tabularx}{\linewidth}{@{}Xr}
\rightskip=5em plus 1 fill \hangindent=2em \hyphenpenalty=10000
\texttt{d{-}{-}efcv\_name}& [c--ategory of \texttt{c{-}{-}lass}]
\end{tabularx}

\index[cp]{d--efcv\_name@\texttt{d{-}{-}efcv\_name}}%
\begin{quote}
\unskip{\parskip=0pt\noindent}%
d--efcv
\end{quote}


\noindent\begin{tabularx}{\linewidth}{@{}Xr}
\rightskip=5em plus 1 fill \hangindent=2em \hyphenpenalty=10000
\texttt{d{-}{-}efcv\_name \EmbracOn{}\textnormal{\textsl{a--rguments...}}\EmbracOff{}}& [c--ategory of \texttt{c{-}{-}lass}]
\end{tabularx}

\index[cp]{d--efcv\_name@\texttt{d{-}{-}efcv\_name}}%
\begin{quote}
\unskip{\parskip=0pt\noindent}%
d--efcv with arguments
\end{quote}


\noindent\begin{tabularx}{\linewidth}{@{}Xr}
\rightskip=5em plus 1 fill \hangindent=2em \hyphenpenalty=10000
\texttt{t{-}{-}ype d{-}{-}eftypecv\_name}& [c--ategory of \texttt{c{-}{-}lass}]
\end{tabularx}

\index[cp]{d--eftypecv\_name of c--lass@\texttt{d{-}{-}eftypecv\_name\ of c{-}{-}lass}}%
\begin{quote}
\unskip{\parskip=0pt\noindent}%
d--eftypecv
\end{quote}


\noindent\begin{tabularx}{\linewidth}{@{}Xr}
\rightskip=5em plus 1 fill \hangindent=2em \hyphenpenalty=10000
\texttt{t{-}{-}ype d{-}{-}eftypecv\_name a{-}{-}rguments...}& [c--ategory of \texttt{c{-}{-}lass}]
\end{tabularx}

\index[cp]{d--eftypecv\_name of c--lass@\texttt{d{-}{-}eftypecv\_name\ of c{-}{-}lass}}%
\begin{quote}
\unskip{\parskip=0pt\noindent}%
d--eftypecv with arguments
\end{quote}


\noindent\begin{tabularx}{\linewidth}{@{}Xr}
\rightskip=5em plus 1 fill \hangindent=2em \hyphenpenalty=10000
\texttt{d{-}{-}efop\_name \EmbracOn{}\textnormal{\textsl{a--rguments...}}\EmbracOff{}}& [c--ategory on \texttt{c{-}{-}lass}]
\end{tabularx}

\index[fn]{d--efop\_name on c--lass@\texttt{d{-}{-}efop\_name\ on c{-}{-}lass}}%
\begin{quote}
\unskip{\parskip=0pt\noindent}%
d--efop
\end{quote}


\noindent\begin{tabularx}{\linewidth}{@{}Xr}
\rightskip=5em plus 1 fill \hangindent=2em \hyphenpenalty=10000
\texttt{d{-}{-}efop\_name}& [c--ategory on \texttt{c{-}{-}lass}]
\end{tabularx}

\index[fn]{d--efop\_name on c--lass@\texttt{d{-}{-}efop\_name\ on c{-}{-}lass}}%
\begin{quote}
\unskip{\parskip=0pt\noindent}%
d--efop no arg
\end{quote}


\noindent\begin{tabularx}{\linewidth}{@{}Xr}
\rightskip=5em plus 1 fill \hangindent=2em \hyphenpenalty=10000
\texttt{d{-}{-}eftp\_name \EmbracOn{}\textnormal{\textsl{a--ttributes...}}\EmbracOff{}}& [c--ategory]
\end{tabularx}

\index[tp]{d--eftp\_name@\texttt{d{-}{-}eftp\_name}}%
\begin{quote}
\unskip{\parskip=0pt\noindent}%
d--eftp
\end{quote}


\noindent\begin{tabularx}{\linewidth}{@{}Xr}
\rightskip=5em plus 1 fill \hangindent=2em \hyphenpenalty=10000
\texttt{d{-}{-}efun\_name \EmbracOn{}\textnormal{\textsl{a--rguments...}}\EmbracOff{}}& [Function]
\end{tabularx}

\index[fn]{d--efun\_name@\texttt{d{-}{-}efun\_name}}%
\begin{quote}
\unskip{\parskip=0pt\noindent}%
d--efun
\end{quote}


\noindent\begin{tabularx}{\linewidth}{@{}Xr}
\rightskip=5em plus 1 fill \hangindent=2em \hyphenpenalty=10000
\texttt{d{-}{-}efmac\_name \EmbracOn{}\textnormal{\textsl{a--rguments...}}\EmbracOff{}}& [Macro]
\end{tabularx}

\index[fn]{d--efmac\_name@\texttt{d{-}{-}efmac\_name}}%
\begin{quote}
\unskip{\parskip=0pt\noindent}%
d--efmac
\end{quote}


\noindent\begin{tabularx}{\linewidth}{@{}Xr}
\rightskip=5em plus 1 fill \hangindent=2em \hyphenpenalty=10000
\texttt{d{-}{-}efspec\_name \EmbracOn{}\textnormal{\textsl{a--rguments...}}\EmbracOff{}}& [Special Form]
\end{tabularx}

\index[fn]{d--efspec\_name@\texttt{d{-}{-}efspec\_name}}%
\begin{quote}
\unskip{\parskip=0pt\noindent}%
d--efspec
\end{quote}


\noindent\begin{tabularx}{\linewidth}{@{}Xr}
\rightskip=5em plus 1 fill \hangindent=2em \hyphenpenalty=10000
\texttt{d{-}{-}efvar\_name}& [Variable]
\end{tabularx}

\index[cp]{d--efvar\_name@\texttt{d{-}{-}efvar\_name}}%
\begin{quote}
\unskip{\parskip=0pt\noindent}%
d--efvar
\end{quote}


\noindent\begin{tabularx}{\linewidth}{@{}Xr}
\rightskip=5em plus 1 fill \hangindent=2em \hyphenpenalty=10000
\texttt{d{-}{-}efvar\_name \EmbracOn{}\textnormal{\textsl{arg--var arg--var1}}\EmbracOff{}}& [Variable]
\end{tabularx}

\index[cp]{d--efvar\_name@\texttt{d{-}{-}efvar\_name}}%
\begin{quote}
\unskip{\parskip=0pt\noindent}%
d--efvar with args
\end{quote}


\noindent\begin{tabularx}{\linewidth}{@{}Xr}
\rightskip=5em plus 1 fill \hangindent=2em \hyphenpenalty=10000
\texttt{d{-}{-}efopt\_name}& [User Option]
\end{tabularx}

\index[cp]{d--efopt\_name@\texttt{d{-}{-}efopt\_name}}%
\begin{quote}
\unskip{\parskip=0pt\noindent}%
d--efopt
\end{quote}


\noindent\begin{tabularx}{\linewidth}{@{}Xr}
\rightskip=5em plus 1 fill \hangindent=2em \hyphenpenalty=10000
\texttt{t{-}{-}ype d{-}{-}eftypefun\_name a{-}{-}rguments...}& [Function]
\end{tabularx}

\index[fn]{d--eftypefun\_name@\texttt{d{-}{-}eftypefun\_name}}%
\begin{quote}
\unskip{\parskip=0pt\noindent}%
d--eftypefun
\end{quote}


\noindent\begin{tabularx}{\linewidth}{@{}Xr}
\rightskip=5em plus 1 fill \hangindent=2em \hyphenpenalty=10000
\texttt{t{-}{-}ype d{-}{-}eftypevar\_name}& [Variable]
\end{tabularx}

\index[cp]{d--eftypevar\_name@\texttt{d{-}{-}eftypevar\_name}}%
\begin{quote}
\unskip{\parskip=0pt\noindent}%
d--eftypevar
\end{quote}


\noindent\begin{tabularx}{\linewidth}{@{}Xr}
\rightskip=5em plus 1 fill \hangindent=2em \hyphenpenalty=10000
\texttt{d{-}{-}efivar\_name}& [Instance Variable of \texttt{c{-}{-}lass}]
\end{tabularx}

\index[cp]{d--efivar\_name of c--lass@\texttt{d{-}{-}efivar\_name\ of c{-}{-}lass}}%
\begin{quote}
\unskip{\parskip=0pt\noindent}%
d--efivar
\end{quote}


\noindent\begin{tabularx}{\linewidth}{@{}Xr}
\rightskip=5em plus 1 fill \hangindent=2em \hyphenpenalty=10000
\texttt{t{-}{-}ype d{-}{-}eftypeivar\_name}& [Instance Variable of \texttt{c{-}{-}lass}]
\end{tabularx}

\index[cp]{d--eftypeivar\_name of c--lass@\texttt{d{-}{-}eftypeivar\_name\ of c{-}{-}lass}}%
\begin{quote}
\unskip{\parskip=0pt\noindent}%
d--eftypeivar
\end{quote}


\noindent\begin{tabularx}{\linewidth}{@{}Xr}
\rightskip=5em plus 1 fill \hangindent=2em \hyphenpenalty=10000
\texttt{d{-}{-}efmethod\_name \EmbracOn{}\textnormal{\textsl{a--rguments...}}\EmbracOff{}}& [Method on \texttt{c{-}{-}lass}]
\end{tabularx}

\index[fn]{d--efmethod\_name on c--lass@\texttt{d{-}{-}efmethod\_name\ on c{-}{-}lass}}%
\begin{quote}
\unskip{\parskip=0pt\noindent}%
d--efmethod
\end{quote}


\noindent\begin{tabularx}{\linewidth}{@{}Xr}
\rightskip=5em plus 1 fill \hangindent=2em \hyphenpenalty=10000
\texttt{t{-}{-}ype d{-}{-}eftypemethod\_name a{-}{-}rguments...}& [Method on \texttt{c{-}{-}lass}]
\end{tabularx}

\index[fn]{d--eftypemethod\_name on c--lass@\texttt{d{-}{-}eftypemethod\_name\ on c{-}{-}lass}}%
\begin{quote}
\unskip{\parskip=0pt\noindent}%
d--eftypemethod
\end{quote}



\noindent\begin{tabularx}{\linewidth}{@{}Xr}
\rightskip=5em plus 1 fill \hangindent=2em \hyphenpenalty=10000
\texttt{data-type2}& [Function]\\
\texttt{name2 arguments2...}\end{tabularx}

\index[fn]{name2@\texttt{name2}}%
\begin{quote}
\unskip{\parskip=0pt\noindent}%
aaa2
\end{quote}


\noindent\begin{tabularx}{\linewidth}{@{}Xr}
\rightskip=5em plus 1 fill \hangindent=2em \hyphenpenalty=10000
\texttt{t{-}{-}ype2}& [c--ategory2]\\
\texttt{d{-}{-}eftypefn\_name2}\end{tabularx}

\index[fn]{d--eftypefn\_name2@\texttt{d{-}{-}eftypefn\_name2}}%
\begin{quote}
\unskip{\parskip=0pt\noindent}%
d--eftypefn no arg2
\end{quote}


\noindent\begin{tabularx}{\linewidth}{@{}Xr}
\rightskip=5em plus 1 fill \hangindent=2em \hyphenpenalty=10000
\texttt{t{-}{-}ype2}& [c--ategory2 on \texttt{c{-}{-}lass2}]\\
\texttt{d{-}{-}eftypeop\_name2 a{-}{-}rguments2...}\end{tabularx}

\index[fn]{d--eftypeop\_name2 on c--lass2@\texttt{d{-}{-}eftypeop\_name2\ on c{-}{-}lass2}}%
\begin{quote}
\unskip{\parskip=0pt\noindent}%
d--eftypeop2
\end{quote}


\noindent\begin{tabularx}{\linewidth}{@{}Xr}
\rightskip=5em plus 1 fill \hangindent=2em \hyphenpenalty=10000
\texttt{t{-}{-}ype2}& [c--ategory2 on \texttt{c{-}{-}lass2}]\\
\texttt{d{-}{-}eftypeop\_name2}\end{tabularx}

\index[fn]{d--eftypeop\_name2 on c--lass2@\texttt{d{-}{-}eftypeop\_name2\ on c{-}{-}lass2}}%
\begin{quote}
\unskip{\parskip=0pt\noindent}%
d--eftypeop no arg2
\end{quote}


\noindent\begin{tabularx}{\linewidth}{@{}Xr}
\rightskip=5em plus 1 fill \hangindent=2em \hyphenpenalty=10000
\texttt{t{-}{-}ype2 d{-}{-}eftypecv\_name2}& [c--ategory2 of \texttt{c{-}{-}lass2}]
\end{tabularx}

\index[cp]{d--eftypecv\_name2 of c--lass2@\texttt{d{-}{-}eftypecv\_name2\ of c{-}{-}lass2}}%
\begin{quote}
\unskip{\parskip=0pt\noindent}%
d--eftypecv2
\end{quote}


\noindent\begin{tabularx}{\linewidth}{@{}Xr}
\rightskip=5em plus 1 fill \hangindent=2em \hyphenpenalty=10000
\texttt{t{-}{-}ype2 d{-}{-}eftypecv\_name2 a{-}{-}rguments2...}& [c--ategory2 of \texttt{c{-}{-}lass2}]
\end{tabularx}

\index[cp]{d--eftypecv\_name2 of c--lass2@\texttt{d{-}{-}eftypecv\_name2\ of c{-}{-}lass2}}%
\begin{quote}
\unskip{\parskip=0pt\noindent}%
d--eftypecv with arguments2
\end{quote}


\noindent\begin{tabularx}{\linewidth}{@{}Xr}
\rightskip=5em plus 1 fill \hangindent=2em \hyphenpenalty=10000
\texttt{arg2}& [fun2]
\end{tabularx}

\index[fn]{arg2@\texttt{arg2}}%
\begin{quote}
\unskip{\parskip=0pt\noindent}%
fff2
\end{quote}


\texttt{@xref\{c{-}{-}{-}hapter@@,\ cross r{-}{-}{-}ef name@@,\ t{-}{-}{-}itle@@,\ file n{-}{-}{-}ame@@,\ ma{-}{-}{-}nual@@\}} See Section ``t---itle@'' in \textsl{ma---nual@}.
\texttt{@ref\{chapter,\ cross ref name,\ title,\ file name,\ manual\}} Section ``title'' in \textsl{manual}
\texttt{@pxref\{chapter,\ cross ref name,\ title,\ file name,\ manual\}} see Section ``title'' in \textsl{manual}
\texttt{@inforef\{chapter,\ cross ref name,\ file name\}} Section ``chapter'' in \texttt{file name}

\texttt{@ref\{chapter\}} \hyperref[anchor:chapter]{\chaptername~\ref*{anchor:chapter} [chapter], page~\pageref*{anchor:chapter}}
\texttt{@xref\{chapter\}} See \hyperref[anchor:chapter]{\chaptername~\ref*{anchor:chapter} [chapter], page~\pageref*{anchor:chapter}}.
\texttt{@pxref\{chapter\}} see \hyperref[anchor:chapter]{\chaptername~\ref*{anchor:chapter} [chapter], page~\pageref*{anchor:chapter}}
\texttt{@ref\{s{-}{-}ect@comma\{\}ion\}} \hyperref[anchor:s_002d_002dect_002cion]{Section~\ref*{anchor:s_002d_002dect_002cion} [s--ect,ion], page~\pageref*{anchor:s_002d_002dect_002cion}}

\texttt{@ref\{s{-}{-}ect@comma\{\}ion,\ a @comma\{\}\ in cross
ref,\ a comma@comma\{\}\ in title,\ a comma@comma\{\}\ in file,\ a @comma\{\}\ in manual name \}}
Section ``a comma, in title'' in \textsl{a , in manual name}

\texttt{@ref\{chapter,cross ref name\}} \hyperref[anchor:chapter]{\chaptername~\ref*{anchor:chapter} [chapter], page~\pageref*{anchor:chapter}}
\texttt{@ref\{chapter{,}{,}title\}} \hyperref[anchor:chapter]{\chaptername~\ref*{anchor:chapter} [title], page~\pageref*{anchor:chapter}}
\texttt{@ref\{chapter{,}{,},file name\}} Section ``chapter'' in \texttt{file name}
\texttt{@ref\{chapter{,}{,}{,}{,}manual\}} Section ``chapter'' in \textsl{manual}
\texttt{@ref\{chapter,cross ref name,title,\}} \hyperref[anchor:chapter]{\chaptername~\ref*{anchor:chapter} [title], page~\pageref*{anchor:chapter}}
\texttt{@ref\{chapter,cross ref name{,}{,}file name\}} Section ``chapter'' in \texttt{file name}
\texttt{@ref\{chapter,cross ref name{,}{,},manual\}} Section ``chapter'' in \textsl{manual}
\texttt{@ref\{chapter,cross ref name,title,file name\}} Section ``title'' in \texttt{file name}
\texttt{@ref\{chapter,cross ref name,title{,}{,}manual\}} Section ``title'' in \textsl{manual}
\texttt{@ref\{chapter,cross ref name,title,\ file name,\ manual\}} Section ``title'' in \textsl{manual}
\texttt{@ref\{chapter{,}{,}title,file name\}} Section ``title'' in \texttt{file name}
\texttt{@ref\{chapter{,}{,}title{,}{,}manual\}} Section ``title'' in \textsl{manual}
\texttt{@ref\{chapter{,}{,}title,\ file name,\ manual\}} Section ``title'' in \textsl{manual}
\texttt{@ref\{chapter{,}{,},file name,manual\}} Section ``chapter'' in \textsl{manual}


\texttt{@ref\{(pman)anode,cross ref name\}} (pman)anode
\texttt{@ref\{(pman)anode{,}{,}title\}} title
\texttt{@ref\{(pman)anode{,}{,},file name\}} Section ``(pman)anode'' in \texttt{file name}
\texttt{@ref\{(pman)anode{,}{,}{,}{,}manual\}} Section ``(pman)anode'' in \textsl{manual}
\texttt{@ref\{(pman)anode,cross ref name,title,\}} title
\texttt{@ref\{(pman)anode,cross ref name{,}{,}file name\}} Section ``(pman)anode'' in \texttt{file name}
\texttt{@ref\{(pman)anode,cross ref name{,}{,},manual\}} Section ``(pman)anode'' in \textsl{manual}
\texttt{@ref\{(pman)anode,cross ref name,title,file name\}} Section ``title'' in \texttt{file name}
\texttt{@ref\{(pman)anode,cross ref name,title{,}{,}manual\}} Section ``title'' in \textsl{manual}
\texttt{@ref\{(pman)anode,cross ref name,title,\ file name,\ manual\}} Section ``title'' in \textsl{manual}
\texttt{@ref\{(pman)anode{,}{,}title,file name\}} Section ``title'' in \texttt{file name}
\texttt{@ref\{(pman)anode{,}{,}title{,}{,}manual\}} Section ``title'' in \textsl{manual}
\texttt{@ref\{(pman)anode{,}{,}title,\ file name,\ manual\}} Section ``title'' in \textsl{manual}
\texttt{@ref\{(pman)anode{,}{,},file name,manual\}} Section ``(pman)anode'' in \textsl{manual}


\texttt{@inforef\{chapter,\ cross ref name,\ file name\}} Section ``chapter'' in \texttt{file name}
\texttt{@inforef\{chapter\}} chapter
\texttt{@inforef\{chapter,\ cross ref name\}} chapter
\texttt{@inforef\{chapter{,}{,}file name\}} Section ``chapter'' in \texttt{file name}
\texttt{@inforef\{node,\ cross ref name,\ file name\}} Section ``node'' in \texttt{file name}
\texttt{@inforef\{node\}} node
\texttt{@inforef\{node,\ cross ref name\}} node
\texttt{@inforef\{node{,}{,}file name\}} Section ``node'' in \texttt{file name}
\texttt{@inforef\{chapter,\ cross ref name,\ file name,\ spurious arg\}} Section ``chapter'' in \texttt{file name,\ spurious arg}

\texttt{@inforef\{s{-}{-}ect@comma\{\}ion,\ a @comma\{\}\ in cross
ref,\ a comma@comma\{\}\ in file\}}
Section ``s--ect,ion'' in \texttt{a comma,\ in file}

`\texttt{\hyperref[anchor:chapter]{\chaptername~\ref*{anchor:chapter} [chapter], page~\pageref*{anchor:chapter}}}'.

Section ``title with uref2 \href{href://http/myhost.com/index2.html}{uref2 (\nolinkurl{href://http/myhost.com/index2.html})}'' in \textsl{printed manual with uref4 \href{href://http/myhost.com/index4.html}{uref4 (\nolinkurl{href://http/myhost.com/index4.html})}}
\hyperref[anchor:chapter]{\chaptername~\ref*{anchor:chapter} [title with uref2 \href{href://http/myhost.com/index2.html}{uref2 (\nolinkurl{href://http/myhost.com/index2.html})}], page~\pageref*{anchor:chapter}}

\begin{description}
\item[{\parbox[b]{\linewidth}{%
\textbf{a--strong}}}]
l--ine
\end{description}

\begin{description}
\item[{\parbox[b]{\linewidth}{%
a--asis\\
\index[cp]{a--asis@\texttt{a{-}{-}asis}}%
b
\index[cp]{b@\texttt{b}}%
}}]
l--ine
\end{description}

\begin{description}
\item[{\parbox[b]{\linewidth}{%
\emph{a}\\
\index[fn]{a@\texttt{a}}%
\index[cp]{index entry between item and itemx@index entry between item and itemx}%
\emph{b}
\index[fn]{b@\texttt{b}}%
}}]
l--ine
\end{description}

\begin{description}
\item[] Title
\item[{\parbox[b]{\linewidth}{%
\texttt{a{-}{-}code}}}]
Value--table code
\end{description}

\begin{description}
\item[] Title
\item[{\parbox[b]{\linewidth}{%
\Texinfotablestylesamp{a{-}{-}samp}\\
\Texinfotablestylesamp{a2{-}{-}samp}}}]
Value--table samp
\end{description}

\begin{mdframed}[style=Texinfocartouche]
c--artouche
\end{mdframed}

\begin{flushleft}
\begin{Texinfopreformatted}%
f--lushleft
more text
\end{Texinfopreformatted}
\end{flushleft}

\begin{flushright}
\begin{Texinfopreformatted}%
f--lushright
more text
\end{Texinfopreformatted}
\end{flushright}

\begin{center}
ce--ntered line
\end{center}

\begin{flushleft}
r--raggedright
more text
\end{flushleft}

\begin{verbatim}
\input texinfo @c -*-texinfo-*-

@c this file is used in tests in @verbatiminclude but not converted

@setfilename simplest.info

@node Top

This is a very simple texi manual @  <>.

@bye
\end{verbatim}

\begin{verbatim}
in verbatim ''
\end{verbatim}





$\frac{a < b \texttt{tex \hbox{ code }}}{b}$ ``

\Texinfonopagebreakheading{\chapter*}{{majorheading}}

\Texinfonopagebreakheading{\chapter*}{{chapheading}}

\section*{{heading}}

\subsection*{{subheading}}

\subsubsection*{{subsubheading}}


\texttt{@acronym\{{-}{-}a,an accronym @comma\{\}\ @enddots\{\}\}} --a (an accronym , \dots{})
\texttt{@abbr\{@'E{-}{-}.\ @comma\{\}A.,\ @'Etude{-}{-}@comma\{\}\ @b\{Autonome\}\ \}} \'{E}--.\@ ,A.\@ (\'{E}tude--, \textbf{Autonome})
\texttt{@abbr\{@'E{-}{-}.\ @comma\{\}A.\}} \'{E}--.\@ ,A.\@

\texttt{@math\{{-}{-}a@minus\{\}\ \{\textbackslash{}frac\{1\}\{2\}\}\}} $--a- {\frac{1}{2}}$




Somehow invalid use of @,:\leavevmode{}\\
@, \c{}\leavevmode{}\\
@,@"u \c{}\"{u}

Invalid use of @':\leavevmode{}\\
@' \'{}\leavevmode{}\\
@'@"u \'{}\"{u}

\texttt{@|} 

@dotless\{truc\} truc
@dotless\{ij\} ij
\texttt{@dotless\{{-}{-}a\}} --a
\texttt{@dotless\{a\}} a

@U, without braces @U\{\}, with empty arg 
@U\{z\}, with non-hex arg U+z
@U\{FFFFFFFFFFFFFF\}, value much too large U+FFFFFFFFFFFFFF
@U\{110000\}, value just beyond Unicode U+110000

@TeX, but without brace \TeX{}
\texttt{@\#} \#

\texttt{@w\{{-}{-}a\}} \hbox{--a}

\texttt{@image\{,1{-}{-}xt\}} 
\texttt{@image\{{,}{,}2{-}{-}xt\}} 
\texttt{@image\{{,}{,},3{-}{-}xt\}} 

\texttt{@image\{f-ile,aze{,}{,}a{-}{-}lt\}} \includegraphics[width=aze]{f-ile}
\texttt{@image\{f-ile{,}{,},alt@verb\{:jk \_" \%\@\}\}} \includegraphics{f-ile}

\texttt{@image\{f{-}{-}ile\}} \includegraphics{f--ile}
\texttt{@image\{f{-}{-}ile{,}{,},alt\}} \includegraphics{f--ile}
\texttt{@image\{f{-}{-}ile{,}{,}{,}{,}.e-d-xt\}} \includegraphics{f--ile}
\texttt{@image\{f{-}{-}ile,l{-}{-}i\}} \includegraphics[width=l--i]{f--ile}
\texttt{@image\{f{-}{-}ile{,}{,}l{-}{-}e\}} \includegraphics[height=l--e]{f--ile}
\texttt{@image\{f{-}{-}ile,aze,az,alt,.e{-}{-}xt\}} \includegraphics[width=aze,height=az]{f--ile}
\texttt{@image\{@file\{f{-}{-}ile\}@@@.,aze,az,alt,@file\{.file ext\}\ e{-}{-}xt@\}} \includegraphics[width=aze,height=az]{f--ile@.}

\texttt{@image\{f{-}{-}ile,aze,az,@verb\{:jk \_" \%@:\}\ @b\{in b "\},e{-}{-}xt\}} \includegraphics[width=aze,height=az]{f--ile}
\texttt{@image\{file@verb\{:jk \_" \%@:\}{,}{,},alt@verb\{:jk \_" \%@:\}\}} \includegraphics{filejk _" \%@}


{\bfseries author}%

$$
\ddot{u} \ddot{U} \tilde{n} \hat{a} \acute{e} \bar{o} \grave{i} \acute{e} \grave{\bar{E}}
\textsl{\c{\'{C}}} \textsl{\c{\'{C}}} \textsl{\H{a}} \dot{a} \mathring{a} \textsl{\t{a}}
\breve{a} \check{a}
 ? .
$$

$$
TeX LaTeX \star{} \mathord{\text{\aa{}}} \circledR{} ^{\circ{}} 
$$

$$
\mathtt{t} 
$$

\begin{itemize}[label=\emph{}]
\item e--mph item
\end{itemize}

\begin{itemize}[label=\emph{} after emph]
\item e--mph item
\end{itemize}

\begin{itemize}[label=\textbullet{} a--n itemize line]
\item i--tem 1
\item i--tem 2
\end{itemize}

\begin{itemize}[label={}]
\item without brace w a--b
\item without brace w c--d
\end{itemize}

\begin{description}
\item[{\parbox[b]{\linewidth}{%
a}}]
l--ine
\end{description}

\begin{description}
\item[{\parbox[b]{\linewidth}{%
a--missing style formatting}}]
l--ine
\end{description}

\begin{description}
\item[{\parbox[b]{\linewidth}{%
a\\
\index[fn]{a@\texttt{a}}%
\index[cp]{index entry between item and itemx@index entry between item and itemx}%
b
\index[fn]{b@\texttt{b}}%
}}]
l--ine
\end{description}


\noindent\begin{tabularx}{\linewidth}{@{}Xr}
\rightskip=5em plus 1 fill \hangindent=2em \hyphenpenalty=10000
\texttt{}& [fun]
\end{tabularx}


\noindent\begin{tabularx}{\linewidth}{@{}Xr}
\rightskip=5em plus 1 fill \hangindent=2em \hyphenpenalty=10000
\texttt{machin \EmbracOn{}\textnormal{\textsl{bidule chose and}}\EmbracOff{}}& [truc]
\end{tabularx}

\index[fn]{machin@\texttt{machin}}%

\noindent\begin{tabularx}{\linewidth}{@{}Xr}
\rightskip=5em plus 1 fill \hangindent=2em \hyphenpenalty=10000
\texttt{machin \EmbracOn{}\textnormal{\textsl{bidule chose and  after}}\EmbracOff{}}& [truc]
\end{tabularx}

\index[fn]{machin@\texttt{machin}}%

\noindent\begin{tabularx}{\linewidth}{@{}Xr}
\rightskip=5em plus 1 fill \hangindent=2em \hyphenpenalty=10000
\texttt{machin \EmbracOn{}\textnormal{\textsl{bidule chose and }}\EmbracOff{}}& [truc]
\end{tabularx}

\index[fn]{machin@\texttt{machin}}%

\noindent\begin{tabularx}{\linewidth}{@{}Xr}
\rightskip=5em plus 1 fill \hangindent=2em \hyphenpenalty=10000
\texttt{machin \EmbracOn{}\textnormal{\textsl{bidule chose and and after}}\EmbracOff{}}& [truc]
\end{tabularx}

\index[fn]{machin@\texttt{machin}}%

\noindent\begin{tabularx}{\linewidth}{@{}Xr}
\rightskip=5em plus 1 fill \hangindent=2em \hyphenpenalty=10000
\texttt{followed \EmbracOn{}\textnormal{\textsl{by a comment}}\EmbracOff{}}& [truc]
\end{tabularx}

\index[fn]{followed@\texttt{followed}}%
Various deff lines

\noindent\begin{tabularx}{\linewidth}{@{}Xr}
\rightskip=5em plus 1 fill \hangindent=2em \hyphenpenalty=10000
\texttt{after \EmbracOn{}\textnormal{\textsl{a deff item}}\EmbracOff{}}& [truc]
\end{tabularx}

\index[fn]{after@\texttt{after}}%


\noindent\begin{tabularx}{\linewidth}{@{}Xr}
\rightskip=5em plus 1 fill \hangindent=2em \hyphenpenalty=10000
\texttt{\Texinfocommandstyletextvar{invalid} \EmbracOn{}\textnormal{\textsl{a g}}\EmbracOff{}}& [fsetinv]
\end{tabularx}

\index[fn]{invalid@\texttt{\Texinfocommandstyletextvar{invalid}}}%

\noindent\begin{tabularx}{\linewidth}{@{}Xr}
\rightskip=5em plus 1 fill \hangindent=2em \hyphenpenalty=10000
\texttt{}& [\textbf{id `\texttt{i}' ule}]
\end{tabularx}



\noindent\begin{tabularx}{\linewidth}{@{}Xr}
\rightskip=5em plus 1 fill \hangindent=2em \hyphenpenalty=10000
\texttt{}& [aaa]
\end{tabularx}


\noindent\begin{tabularx}{\linewidth}{@{}Xr}
\rightskip=5em plus 1 fill \hangindent=2em \hyphenpenalty=10000
\texttt{}& []
\end{tabularx}


\noindent\begin{tabularx}{\linewidth}{@{}Xr}
\rightskip=5em plus 1 fill \hangindent=2em \hyphenpenalty=10000
\texttt{}& [truc]
\end{tabularx}


g--roupe

\texttt{@ref\{node\}} node

\texttt{@ref\{,cross ref name\}} 
\texttt{@ref\{{,}{,}title\}} title
\texttt{@ref\{{,}{,},file name\}} \texttt{file name}
\texttt{@ref\{{,}{,}{,}{,}manual\}} \textsl{manual}
\texttt{@ref\{node,cross ref name\}} node
\texttt{@ref\{node{,}{,}title\}} title
\texttt{@ref\{node{,}{,},file name\}} Section ``node'' in \texttt{file name}
\texttt{@ref\{node{,}{,}{,}{,}manual\}} Section ``node'' in \textsl{manual}
\texttt{@ref\{node,cross ref name,title,\}} title
\texttt{@ref\{node,cross ref name{,}{,}file name\}} Section ``node'' in \texttt{file name}
\texttt{@ref\{node,cross ref name{,}{,},manual\}} Section ``node'' in \textsl{manual}
\texttt{@ref\{node,cross ref name,title,file name\}} Section ``title'' in \texttt{file name}
\texttt{@ref\{node,cross ref name,title{,}{,}manual\}} Section ``title'' in \textsl{manual}
\texttt{@ref\{node,cross ref name,title,\ file name,\ manual\}} Section ``title'' in \textsl{manual}
\texttt{@ref\{node{,}{,}title,file name\}} Section ``title'' in \texttt{file name}
\texttt{@ref\{node{,}{,}title{,}{,}manual\}} Section ``title'' in \textsl{manual}
\texttt{@ref\{chapter{,}{,}title,\ file name,\ manual\}} Section ``title'' in \textsl{manual}
\texttt{@ref\{node{,}{,}title,\ file name,\ manual\}} Section ``title'' in \textsl{manual}
\texttt{@ref\{node{,}{,},file name,manual\}} Section ``node'' in \textsl{manual}
\texttt{@ref\{,cross ref name,title,\}} title
\texttt{@ref\{,cross ref name{,}{,}file name\}} \texttt{file name}
\texttt{@ref\{,cross ref name{,}{,},manual\}} \textsl{manual}
\texttt{@ref\{,cross ref name,title,file name\}} Section ``title'' in \texttt{file name}
\texttt{@ref\{,cross ref name,title{,}{,}manual\}} Section ``title'' in \textsl{manual}
\texttt{@ref\{,cross ref name,title,\ file name,\ manual\}} Section ``title'' in \textsl{manual}
\texttt{@ref\{{,}{,}title,file name\}} Section ``title'' in \texttt{file name}
\texttt{@ref\{{,}{,}title{,}{,}manual\}} Section ``title'' in \textsl{manual}
\texttt{@ref\{{,}{,}title,\ file name,\ manual\}} Section ``title'' in \textsl{manual}
\texttt{@ref\{{,}{,},file name,manual\}} \textsl{manual}

\texttt{@inforef\{,cross ref name \}} 
\texttt{@inforef\{{,}{,}file name\}} \texttt{file name}
\texttt{@inforef\{,cross ref name,\ file name\}} \texttt{file name}
\texttt{@inforef\{\}} 



Normal text

<
>
"
\&
'
`

``simple-double--three---four----''\leavevmode{}\\
code: \texttt{{`}{`}simple-double{-}{-}three{-}{-}{-}four{-}{-}{-}-{'}{'}} \leavevmode{}\\
asis: ``simple-double--three---four----'' \leavevmode{}\\
strong: \textbf{``simple-double--three---four----''} \leavevmode{}\\
kbd: \Texinfocommandstyletextkbd{{`}{`}simple-double{-}{-}three{-}{-}{-}four{-}{-}{-}-{'}{'}} \leavevmode{}\\

`\hbox{}`simple-double-\hbox{}-three---four----'\hbox{}'\leavevmode{}\\

\index[cp]{-option@--option}%
\index[cp]{""@``}%
\index[fn]{``@\texttt{{`}{`}}}%
\index[fn]{--foption@\texttt{{-}{-}foption}}%

@"u \"{u} 
@"\{U\} \"{U} 
@\~{}n \~{n}
@\^{}a \^{a}
@'e \'{e}
@=o \={o}
@`i \`{i}
@'\{e\} \'{e}
@'\{@dotless\{i\}\} \'{\i{}} 
@dotless\{i\} \i{}
@dotless\{j\} \j{}
@`\{@=E\} \`{\={E}} 
@l\{\} \l{}
@,\{@'C\} \c{\'{C}}
@,c \c{c}
@,c@"u \c{c}\"{u} \leavevmode{}\\

@U\{0075\} u

@* \leavevmode{}\\
@ followed by a space
\ {}
@ followed by a tab
\ {}
@ followed by a new line
\ {}\texttt{@-} \-{}
\texttt{@:} \@
\texttt{@!} \@!
\texttt{@?} \@?
\texttt{@.} \@.
\texttt{@@} @
\texttt{@\}} \}
\texttt{@\{} \{
\texttt{@/} 

foo vs.\@ bar. 
colon :\@And something else.
semi colon ;\@.
And ? ?\@.
Now ! !\@@
but , ,\@

@TeX \TeX{}
@LaTeX \LaTeX{}
@bullet \textbullet{}
@copyright \copyright{}
@dots \dots{}\@
@enddots \dots{}
@equiv $\equiv{}$
@error \fbox{error}
@expansion $\mapsto{}$
@minus -
@point $\star{}$
@print $\dashv{}$
@result $\Rightarrow{}$
@today \today{}

@aa \aa{}
@AA \AA{}
@ae \ae{}
@oe \oe{}
@AE \AE{}
@OE \OE{}
@o \o{}
@O \O{}
@ss \ss{}
@l \l{}
@L \L{}
@DH \DH{}
@TH \TH{}
@dh \dh{}
@th \th{}

@exclamdown \textexclamdown{}
@questiondown \textquestiondown{}
@pounds \textsterling{}
@registeredsymbol \circledR{}
@ordf \textordfeminine{}
@ordm \textordmasculine{}
@comma ,
@quotedblleft \textquotedblleft{}
@quotedblright \textquotedblright{}
@quoteleft \textquoteleft{}
@quoteright \textquoteright{}
@quotedblbase \quotedblbase{}
@quotesinglbase \quotesinglbase{}
@guillemetleft \guillemotleft{}
@guillemetright \guillemotright{}
@guillemotleft \guillemotleft{}
@guillemotright \guillemotright{}
@guilsinglleft \guilsinglleft{}
@guilsinglright \guilsinglright{}

@textdegree \textdegree{}
@euro \euro{}
@arrow $\rightarrow{}$
@leq $\leq{}$
@geq $\geq{}$
@tie a~b

\texttt{@acronym\{{-}{-}a,an accronym\}} --a (an accronym)
\texttt{@acronym\{{-}{-}a\}} --a
\texttt{@abbr\{@'E{-}{-}.\ @comma\{\}A.,\ @'Etude Autonome \}} \'{E}--.\@ ,A.\@ (\'{E}tude Autonome)
\texttt{@abbr\{@'E{-}{-}.\ @comma\{\}A.\}} \'{E}--.\@ ,A.\@
\texttt{@asis\{{-}{-}a\}} --a
\texttt{@b\{{-}{-}a\}} \textbf{--a}
\texttt{@cite\{{-}{-}a\}} \Texinfocommandstyletextcite{--a}
\texttt{@code\{{-}{-}a\}} \texttt{{-}{-}a}
\texttt{@command\{{-}{-}a\}} \texttt{{-}{-}a}
\texttt{@dfn\{{-}{-}a\}} \textsl{--a}
\texttt{@dmn\{{-}{-}a\}} \thinspace --a
\texttt{@email\{{-}{-}a,{-}{-}b\}} \href{mailto:--a}{--b}
\texttt{@email\{,{-}{-}b\}} --b
\texttt{@email\{{-}{-}a\}} \href{mailto:--a}{\nolinkurl{--a}}
\texttt{@emph\{{-}{-}a\}} \emph{--a}
\texttt{@env\{{-}{-}a\}} \texttt{{-}{-}a}
\texttt{@file\{{-}{-}a\}} \texttt{{-}{-}a}
\texttt{@i\{{-}{-}a\}} \textit{--a}
\texttt{@kbd\{{-}{-}a\}} \Texinfocommandstyletextkbd{{-}{-}a}
\texttt{@key\{{-}{-}a\}} \texttt{{-}{-}a}
\texttt{@math\{{-}{-}a \{\textbackslash{}frac\{1\}\{2\}\}\ @minus\{\}\}} $--a {\frac{1}{2}} -$
\texttt{@option\{{-}{-}a\}} \texttt{{-}{-}a}
\texttt{@r\{{-}{-}a\}} \textnormal{--a}
\texttt{@samp\{{-}{-}a\}} `\texttt{{-}{-}a}'
\texttt{@sc\{{-}{-}a\}} \textsc{--a}
\texttt{@strong\{{-}{-}a\}} \textbf{--a}
\texttt{@t\{{-}{-}a\}} \texttt{{-}{-}a}
\texttt{@sansserif\{{-}{-}a\}} \textsf{--a}
\texttt{@slanted\{{-}{-}a\}} \textsl{--a}
\texttt{@titlefont\{{-}{-}a\}} {\huge \bfseries --a}
\texttt{@indicateurl\{{-}{-}a\}} `\texttt{{-}{-}a}'
\texttt{@uref\{{-}{-}a,{-}{-}b\}} \href{--a}{--b (\nolinkurl{--a})}
\texttt{@uref\{{-}{-}a\}} \url{--a}
\texttt{@uref\{,{-}{-}b\}} --b
\texttt{@uref\{{-}{-}a,{-}{-}b,{-}{-}c\}} --c
\texttt{@uref\{,{-}{-}b,{-}{-}c\}} --c
\texttt{@uref\{{-}{-}a{,}{,}{-}{-}c\}} --c
\texttt{@uref\{{,}{,}{-}{-}c\}} --c
\texttt{@url\{{-}{-}a,{-}{-}b\}} \href{--a}{--b (\nolinkurl{--a})}
\texttt{@url\{{-}{-}a,\}} \url{--a}
\texttt{@url\{,{-}{-}b\}} --b
\texttt{@var\{{-}{-}a\}} \Texinfocommandstyletextvar{--a}
\texttt{@verb\{:{-}{-}a:\}} \verb:--a:
\texttt{@verb\{:a  < \& @\ \% " {-}{-}    b:\}} \verb:a  < & @ % " --    b:
\texttt{@w\{a a a a a a a a a a a a a a a a a a a a a a a a a a a a a a a a a a a\}} \hbox{a a a a a a a a a a a a a a a a a a a a a a a a a a a a a a a a a a a}
\texttt{@H\{a\}} \H{a}
\texttt{@H\{{-}{-}a\}} \H{--a}
\texttt{@dotaccent\{a\}} \.{a}
\texttt{@dotaccent\{{-}{-}a\}} \.{--a}
\texttt{@ringaccent\{a\}} \r{a}
\texttt{@ringaccent\{{-}{-}a\}} \r{--a}
\texttt{@tieaccent\{a\}} \t{a}
\texttt{@tieaccent\{{-}{-}a\}} \t{--a}
\texttt{@u\{a\}} \u{a}
\texttt{@u\{{-}{-}a\}} \u{--a}
\texttt{@ubaraccent\{a\}} \b{a}
\texttt{@ubaraccent\{{-}{-}a\}} \b{--a}
\texttt{@udotaccent\{a\}} \d{a}
\texttt{@udotaccent\{{-}{-}a\}} \d{--a}
\texttt{@v\{a\}} \v{a}
\texttt{@v\{{-}{-}a\}} \v{--a}
\texttt{@,\{c\}} \c{c}
\texttt{@,\{{-}{-}c\}} \c{--c}
\texttt{@ogonek\{a\}} \k{a}
\texttt{@ogonek\{{-}{-}a\}} \k{--a}
\texttt{a@sup\{h\}@sub\{l\}} a\textsuperscript{h}\textsubscript{l}
\texttt{@footnote\{in footnote\}} \footnote{in footnote}
\texttt{@footnote\{in footnote2\}} \footnote{in footnote2}

\texttt{@sp 2}\leavevmode{}\\
\vskip 2\baselineskip %
\texttt{@page}\leavevmode{}\\
\newpage{}%
\phantom{blabla}%

\texttt{need 1002}
\needspace{1.002pt}%

\texttt{@clicksequence\{click @click\{\}\ A\}} click $\rightarrow{}$ A
After clickstyle $\Rightarrow{}$
\texttt{@clicksequence\{click @click\{\}\ A\}} click $\Rightarrow{}$ A


$$
disp--laymath
f(x) = {1 \over \sigma \sqrt{2\pi}}e^{-{1 \over 2}\left({x-\mu \over \sigma}\right)^2}
$$

$$
\mathbf{``simple-double--three---four----''} \hbox{aa}
`\hbox{}`simple-double-\hbox{}-three---four----'\hbox{}'
$$

$$
\imath{} \jmath{}
\mathord{\text{\l{}}} \textsl{\c{c}}
\textsl{\b{a}} \textsl{\d{a}} \textsl{\k{a}} a^{h}_{l}
 \ {}\ {} \ {}\-{}  ! @ \} \{ 
\today{}
$$

$$
\rightarrow{}
u
\bullet{} \copyright{} \dots{} \dots{} \equiv{}
\fbox{error} \mapsto{} - \dashv{} \Rightarrow{}
\mathord{\text{\AA{}}} \mathord{\text{\ae{}}} \mathord{\text{\oe{}}} \mathord{\text{\AE{}}} \mathord{\text{\OE{}}} \mathord{\text{\o{}}} \mathord{\text{\O{}}} \mathord{\text{\ss{}}} \mathord{\text{\l{}}} \mathord{\text{\L{}}} \mathord{\text{\DH{}}}
\mathord{\text{\TH{}}} \mathord{\text{\dh{}}} \mathord{\text{\th{}}} \mathord{\text{\textexclamdown{}}} \mathord{\text{\textquestiondown{}}} \mathsterling{}
\mathord{\text{\textordfeminine{}}} \mathord{\text{\textordmasculine{}}} , 
$$

$$
\mathord{\text{\textquotedblleft{}}} \mathord{\text{\textquotedblright{}}} 
\mathord{\text{\textquoteleft{}}} \mathord{\text{\textquoteright{}}} \mathord{\text{\quotedblbase{}}} \mathord{\text{\quotesinglbase{}}} \mathord{\text{\guillemotleft{}}}
\mathord{\text{\guillemotright{}}} \mathord{\text{\guillemotleft{}}} \mathord{\text{\guillemotright{}}} \mathord{\text{\guilsinglleft{}}}
\mathord{\text{\guilsinglright{}}} \euro{} \rightarrow{} \leq{} \geq{}
$$

$$
\mathbf{b} \mathit{i} \mathrm{r} sc \mathsf{sansserif} slanted
$$

\Texinfocommandstyletextkbd{default kbdinputstyle}
\begin{description}
\item[{\parbox[b]{\linewidth}{%
\Texinfocommandstyletextkbd{vtable i{-}{-}tem default kbdinputstyle}
\index[cp]{vtable i--tem default kbdinputstyle@\texttt{vtable i{-}{-}tem default kbdinputstyle}}%
}}]
\end{description}
\begin{Texinfoindented}
\begin{Texinfopreformatted}%
\ttfamily \Texinfocommandstyletextkbd{in example default kbdinputstyle}
\end{Texinfopreformatted}
\begin{description}
\item[{\parbox[b]{\linewidth}{%
\Texinfocommandstyletextkbd{vtable i{-}{-}tem in example default kbdinputstyle}
\index[cp]{vtable i--tem in example default kbdinputstyle@\texttt{vtable i{-}{-}tem in example default kbdinputstyle}}%
}}]
\end{description}
\end{Texinfoindented}

\texttt{code kbdinputstyle}
\begin{description}
\item[{\parbox[b]{\linewidth}{%
\texttt{vtable i{-}{-}tem code kbdinputstyle}
\index[cp]{vtable i--tem code kbdinputstyle@\texttt{vtable i{-}{-}tem code kbdinputstyle}}%
}}]
\end{description}
\begin{Texinfoindented}
\begin{Texinfopreformatted}%
\ttfamily \texttt{in example code kbdinputstyle}
\end{Texinfopreformatted}
\begin{description}
\item[{\parbox[b]{\linewidth}{%
\texttt{vtable i{-}{-}tem in example code kbdinputstyle}
\index[cp]{vtable i--tem in example code kbdinputstyle@\texttt{vtable i{-}{-}tem in example code kbdinputstyle}}%
}}]
\end{description}
\end{Texinfoindented}

\texttt{example kbdinputstyle}
\begin{description}
\item[{\parbox[b]{\linewidth}{%
\texttt{vtable i{-}{-}tem example kbdinputstyle}
\index[cp]{vtable i--tem example kbdinputstyle@\texttt{vtable i{-}{-}tem example kbdinputstyle}}%
}}]
\end{description}
\begin{Texinfoindented}
\begin{Texinfopreformatted}%
\ttfamily \Texinfocommandstyletextkbd{in example example kbdinputstyle}
\end{Texinfopreformatted}
\begin{description}
\item[{\parbox[b]{\linewidth}{%
\Texinfocommandstyletextkbd{vtable i{-}{-}tem in example example kbdinputstyle}
\index[cp]{vtable i--tem in example example kbdinputstyle@\texttt{vtable i{-}{-}tem in example example kbdinputstyle}}%
}}]
\end{description}
\end{Texinfoindented}

\Texinfocommandstyletextkbd{distinct kbdinputstyle}
\begin{description}
\item[{\parbox[b]{\linewidth}{%
\Texinfocommandstyletextkbd{vtable i{-}{-}tem distinct kbdinputstyle}
\index[cp]{vtable i--tem distinct kbdinputstyle@\texttt{vtable i{-}{-}tem distinct kbdinputstyle}}%
}}]
\end{description}
\begin{Texinfoindented}
\begin{Texinfopreformatted}%
\ttfamily \Texinfocommandstyletextkbd{in example distinct kbdinputstyle}
\end{Texinfopreformatted}
\begin{description}
\item[{\parbox[b]{\linewidth}{%
\Texinfocommandstyletextkbd{vtable i{-}{-}tem in example distinct kbdinputstyle}
\index[cp]{vtable i--tem in example distinct kbdinputstyle@\texttt{vtable i{-}{-}tem in example distinct kbdinputstyle}}%
}}]
\end{description}
\end{Texinfoindented}

\begin{quote}
A quot---ation
\end{quote}

\begin{quote}
\textbf{Note:} A Note
\end{quote}

\begin{quote}
\textbf{note:} A note
\end{quote}

\begin{quote}
\textbf{Caution:} Caution
\end{quote}

\begin{quote}
\textbf{Important:} Important
\end{quote}

\begin{quote}
\textbf{Tip:} a Tip
\end{quote}

\begin{quote}
\textbf{Warning:} a Warning.
\end{quote}

\begin{quote}
\textbf{something \'{e} \TeX{}:} The something \'{e} \TeX{} is here.
\end{quote}

\begin{quote}
\textbf{@ at the end of line \ {}:} A @ at the end of the @quotation line.
\end{quote}

\begin{quote}
\textbf{something, other thing:} something, other thing
\end{quote}

\begin{quote}
\textbf{Note, the note:} Note, the note
\end{quote}

\begin{quote}
\end{quote}

\begin{quote}
\textbf{Empty:} \end{quote}

\begin{quote}
\textbf{:} \end{quote}

\begin{quote}
\textbf{\leavevmode{}\\:} \end{quote}

\begin{quote}
aaa quotation
\end{quote}
\begin{center}
--- \emph{quotation author}
\end{center}

\begin{quote}
indent in quotation
\end{quote}

\begin{quote}
\leavevmode{}\\
\hbox{\kern -\leftmargin}%
exdented quotation line   and dash --- in quotation
\\
\end{quote}

\begin{quote}
Not exdented followed by exdented
\leavevmode{}\\
\hbox{\kern -\leftmargin}%
exdented quotation line
\\
\end{quote}

\begin{quote}
\leavevmode{}\\
\hbox{\kern -\leftmargin}%
exdented quotation line
\\
Followed by not exdented 
\end{quote}

\begin{quote}
quotation1
\leavevmode{}\\
\hbox{\kern -\leftmargin}%
in exdented protected eol \ {}
\\
following
\leavevmode{}\\
\hbox{\kern -\leftmargin}%
in exdented a @* \leavevmode{}\\ and following
\\
after exdented
\end{quote}

\begin{quote}
\begin{footnotesize}
A small quot---ation
\end{footnotesize}
\end{quote}

\begin{quote}
\begin{footnotesize}
\textbf{Note:} A small Note
\end{footnotesize}
\end{quote}

\begin{quote}
\begin{footnotesize}
\textbf{something, other thing:} something, other thing
\end{footnotesize}
\end{quote}

\begin{itemize}
\item i--temize
\end{itemize}

\begin{itemize}[label=+]
\item i--tem +
\end{itemize}

\begin{itemize}[label=\textbullet{}]
\item b--ullet
\end{itemize}

\begin{itemize}[label=-]
\item minu--s
\end{itemize}

\begin{itemize}[label=\emph{after emph}]
\item e--mph item
\end{itemize}

\begin{itemize}[label=\textbullet{} a--n itemize line]
\item \index[cp]{index entry within itemize@index entry within itemize}%
i--tem 1
\item i--tem 2
\end{itemize}

\begin{itemize}[label={}]
\item with w a--b
\item with w c--d
\end{itemize}

\begin{itemize}[label=\hbox{} on a line]
\item line w a--b
\item line with w c--d
\end{itemize}

\begin{enumerate}[start=1]
\item e--numerate
\end{enumerate}

\begin{enumerate}[start=3]
\item first third
\item second third
\end{enumerate}

\begin{enumerate}[label=\alph*.]
\item e--numerate
\end{enumerate}

\begin{enumerate}[label=\alph*.,start=3]
\item first c
\item second c
\end{enumerate}

\begin{tabular}{m{0.4\textwidth} m{0.6\textwidth}}%
mu--ltitable headitem &another tab\\
mu--ltitable item &multitable tab\\
mu--ltitable item 2 &multitable tab 2
\index[cp]{index entry within multitable@index entry within multitable}%
\\
lone mu--ltitable item&\\
\end{tabular}%

\begin{tabular}{m{0.4\textwidth} m{0.6\textwidth}}%
truc &bidule\\
\end{tabular}%

\begin{Texinfoindented}
\begin{Texinfopreformatted}%
\ttfamily e{-}{-}xample  some
\   text
\end{Texinfopreformatted}
\end{Texinfoindented}

\begin{Texinfoindented}
\begin{Texinfopreformatted}%
\ttfamily example one arg
\end{Texinfopreformatted}
\end{Texinfoindented}

\begin{Texinfoindented}
\begin{Texinfopreformatted}%
\ttfamily example two args
\end{Texinfopreformatted}
\end{Texinfoindented}

\begin{Texinfoindented}
\begin{Texinfopreformatted}%
\ttfamily example three args
\end{Texinfopreformatted}
\end{Texinfoindented}

\begin{Texinfoindented}
\begin{Texinfopreformatted}%
\ttfamily example four args
\end{Texinfopreformatted}
\end{Texinfoindented}

\begin{Texinfoindented}
\begin{Texinfopreformatted}%
\ttfamily example five args
\end{Texinfopreformatted}
\end{Texinfoindented}

\begin{Texinfoindented}
\begin{Texinfopreformatted}%
\ttfamily The something \'{e}\ \TeX{}\ is here.
\end{Texinfopreformatted}
\end{Texinfoindented}

\begin{Texinfoindented}
\begin{Texinfopreformatted}%
\ttfamily A @\ at the end of the @example line.
\end{Texinfopreformatted}
\end{Texinfoindented}

\begin{Texinfoindented}
\begin{Texinfopreformatted}%
\ttfamily example with empty args
\end{Texinfopreformatted}
\end{Texinfoindented}

\begin{Texinfoindented}
\begin{Texinfopreformatted}%
\ttfamily example with empty and non empty args mix
\end{Texinfopreformatted}
\end{Texinfoindented}

\begin{Texinfoindented}
\begin{Texinfopreformatted}%
\ttfamily Exam{-}{-}{-}ple

\end{Texinfopreformatted}
\leavevmode{}\\
\hbox{\kern -\leftmargin}%
Other li---ne
\\
\begin{Texinfopreformatted}%
\ttfamily not exdented
\end{Texinfopreformatted}
\end{Texinfoindented}

\begin{Texinfoindented}
\leavevmode{}\\
\hbox{\kern -\leftmargin}%
exdented  and dash --- in example
\\
\begin{Texinfopreformatted}%
\ttfamily Not exdented one
\end{Texinfopreformatted}
\leavevmode{}\\
\hbox{\kern -\leftmargin}%
exdented two
\\
\begin{Texinfopreformatted}%
\ttfamily Not exdented two
\end{Texinfopreformatted}
\end{Texinfoindented}

\begin{Texinfoindented}
\begin{Texinfopreformatted}%
\ttfamily Example   Hoho.
\end{Texinfopreformatted}
\begin{Texinfoindented}
\begin{Texinfopreformatted}%
\ttfamily Nested Other line
\end{Texinfopreformatted}
\leavevmode{}\\
\hbox{\kern -\leftmargin}%
exdented nested other line
\\
\end{Texinfoindented}
\end{Texinfoindented}

\begin{Texinfopreformatted}%
\ttfamily \footnotesize s{-}{-}mallexample
\end{Texinfopreformatted}

\texttt{@noindent} after smallexample.
\begin{Texinfopreformatted}%
\ttfamily \footnotesize \$ wget 'http://savannah.gnu.org/cgi-bin/viewcvs/config/config/config.guess?rev=HEAD\&content-type=text/plain'
\$ wget 'http://savannah.gnu.org/cgi-bin/viewcvs/config/config/config.sub?rev=HEAD\&content-type=text/plain'
\end{Texinfopreformatted}
\noindent{}Less recent versions are also present.

\begin{Texinfoindented}
\begin{Texinfopreformatted}%
d--isplay
\end{Texinfopreformatted}
\end{Texinfoindented}

\begin{Texinfopreformatted}%
\footnotesize s--malldisplay
\end{Texinfopreformatted}

\begin{Texinfoindented}
\begin{Texinfopreformatted}%
\ttfamily l{-}{-}isp
\end{Texinfopreformatted}
\end{Texinfoindented}

\begin{Texinfopreformatted}%
\ttfamily \footnotesize s{-}{-}malllisp
\end{Texinfopreformatted}

\begin{Texinfopreformatted}%
f--ormat
\end{Texinfopreformatted}

\begin{Texinfopreformatted}%
\footnotesize s--mallformat
\end{Texinfopreformatted}


\noindent\begin{tabularx}{\linewidth}{@{}Xr}
\rightskip=5em plus 1 fill \hangindent=2em \hyphenpenalty=10000
\texttt{d{-}{-}effn\_name \EmbracOn{}\textnormal{\textsl{a--rguments...}}\EmbracOff{}}& [c--ategory]
\end{tabularx}

\index[fn]{d--effn\_name@\texttt{d{-}{-}effn\_name}}%
\begin{quote}
\unskip{\parskip=0pt\noindent}%
d--effn
\end{quote}


\noindent\begin{tabularx}{\linewidth}{@{}Xr}
\rightskip=5em plus 1 fill \hangindent=2em \hyphenpenalty=10000
\texttt{de{-}{-}ffn\_name \EmbracOn{}\textnormal{\textsl{ar--guments    more args   even more so}}\EmbracOff{}}& [cate--gory]
\end{tabularx}

\index[fn]{de--ffn\_name@\texttt{de{-}{-}ffn\_name}}%
\begin{quote}
\unskip{\parskip=0pt\noindent}%
def--fn
\end{quote}


\noindent\begin{tabularx}{\linewidth}{@{}Xr}
\rightskip=5em plus 1 fill \hangindent=2em \hyphenpenalty=10000
\texttt{\Texinfocommandstyletextvar{i} \EmbracOn{}\textnormal{\textsl{a g}}\EmbracOff{}}& [fset]
\end{tabularx}

\index[fn]{i@\texttt{\Texinfocommandstyletextvar{i}}}%
\index[cp]{index entry within deffn@index entry within deffn}%

\noindent\begin{tabularx}{\linewidth}{@{}Xr}
\rightskip=5em plus 1 fill \hangindent=2em \hyphenpenalty=10000
\texttt{truc \EmbracOn{}\textnormal{\textsl{}}\EmbracOff{}}& [cmde]
\end{tabularx}

\index[fn]{truc@\texttt{truc}}%

\noindent\begin{tabularx}{\linewidth}{@{}Xr}
\rightskip=5em plus 1 fill \hangindent=2em \hyphenpenalty=10000
\texttt{log trap \EmbracOn{}\textnormal{\textsl{}}\EmbracOff{}}& [Command]
\end{tabularx}

\index[fn]{log trap@\texttt{log trap}}%

\noindent\begin{tabularx}{\linewidth}{@{}Xr}
\rightskip=5em plus 1 fill \hangindent=2em \hyphenpenalty=10000
\texttt{log trap1 \EmbracOn{}\textnormal{\textsl{}}\EmbracOff{}}& [Command]
\end{tabularx}

\index[fn]{log trap1@\texttt{log trap1}}%

\noindent\begin{tabularx}{\linewidth}{@{}Xr}
\rightskip=5em plus 1 fill \hangindent=2em \hyphenpenalty=10000
\texttt{log trap2 \EmbracOn{}\textnormal{\textsl{}}\EmbracOff{}}& [Command]
\end{tabularx}

\index[fn]{log trap2@\texttt{log trap2}}%

\noindent\begin{tabularx}{\linewidth}{@{}Xr}
\rightskip=5em plus 1 fill \hangindent=2em \hyphenpenalty=10000
\texttt{\textbf{id ule} \EmbracOn{}\textnormal{\textsl{truc}}\EmbracOff{}}& [cmde]
\end{tabularx}

\index[fn]{id ule@\texttt{\textbf{id ule}}}%

\noindent\begin{tabularx}{\linewidth}{@{}Xr}
\rightskip=5em plus 1 fill \hangindent=2em \hyphenpenalty=10000
\texttt{\textbf{id `\texttt{i}'\ ule} \EmbracOn{}\textnormal{\textsl{truc}}\EmbracOff{}}& [cmde2]
\end{tabularx}

\index[fn]{id i ule@\texttt{\textbf{id `\texttt{i}'\ ule}}}%

\noindent\begin{tabularx}{\linewidth}{@{}Xr}
\rightskip=5em plus 1 fill \hangindent=2em \hyphenpenalty=10000
\texttt{}& []
\end{tabularx}


\noindent\begin{tabularx}{\linewidth}{@{}Xr}
\rightskip=5em plus 1 fill \hangindent=2em \hyphenpenalty=10000
\texttt{machin}& []
\end{tabularx}

\index[fn]{machin@\texttt{machin}}%

\noindent\begin{tabularx}{\linewidth}{@{}Xr}
\rightskip=5em plus 1 fill \hangindent=2em \hyphenpenalty=10000
\texttt{bidule machin}& []
\end{tabularx}

\index[fn]{bidule machin@\texttt{bidule machin}}%

\noindent\begin{tabularx}{\linewidth}{@{}Xr}
\rightskip=5em plus 1 fill \hangindent=2em \hyphenpenalty=10000
\texttt{machin}& [truc]
\end{tabularx}

\index[fn]{machin@\texttt{machin}}%

\noindent\begin{tabularx}{\linewidth}{@{}Xr}
\rightskip=5em plus 1 fill \hangindent=2em \hyphenpenalty=10000
\texttt{}& [truc]
\end{tabularx}


\noindent\begin{tabularx}{\linewidth}{@{}Xr}
\rightskip=5em plus 1 fill \hangindent=2em \hyphenpenalty=10000
\texttt{followed \EmbracOn{}\textnormal{\textsl{by a comment}}\EmbracOff{}}& [truc]
\end{tabularx}

\index[fn]{followed@\texttt{followed}}%

\noindent\begin{tabularx}{\linewidth}{@{}Xr}
\rightskip=5em plus 1 fill \hangindent=2em \hyphenpenalty=10000
\texttt{}& []
\end{tabularx}


\noindent\begin{tabularx}{\linewidth}{@{}Xr}
\rightskip=5em plus 1 fill \hangindent=2em \hyphenpenalty=10000
\texttt{a \EmbracOn{}\textnormal{\textsl{b c d e \textbf{f g} h i}}\EmbracOff{}}& [truc]
\end{tabularx}

\index[fn]{a@\texttt{a}}%

\noindent\begin{tabularx}{\linewidth}{@{}Xr}
\rightskip=5em plus 1 fill \hangindent=2em \hyphenpenalty=10000
\texttt{deffnx \EmbracOn{}\textnormal{\textsl{before end deffn}}\EmbracOff{}}& [truc]
\end{tabularx}

\index[fn]{deffnx@\texttt{deffnx}}%



\noindent\begin{tabularx}{\linewidth}{@{}Xr}
\rightskip=5em plus 1 fill \hangindent=2em \hyphenpenalty=10000
\texttt{deffn}& [empty]
\end{tabularx}

\index[fn]{deffn@\texttt{deffn}}%


\noindent\begin{tabularx}{\linewidth}{@{}Xr}
\rightskip=5em plus 1 fill \hangindent=2em \hyphenpenalty=10000
\texttt{deffn \EmbracOn{}\textnormal{\textsl{with deffnx}}\EmbracOff{}}& [empty]
\end{tabularx}

\index[fn]{deffn@\texttt{deffn}}%

\noindent\begin{tabularx}{\linewidth}{@{}Xr}
\rightskip=5em plus 1 fill \hangindent=2em \hyphenpenalty=10000
\texttt{deffnx}& [empty]
\end{tabularx}

\index[fn]{deffnx@\texttt{deffnx}}%


\noindent\begin{tabularx}{\linewidth}{@{}Xr}
\rightskip=5em plus 1 fill \hangindent=2em \hyphenpenalty=10000
\texttt{\Texinfocommandstyletextvar{i} \EmbracOn{}\textnormal{\textsl{a g}}\EmbracOff{}}& [fset]
\end{tabularx}

\index[fn]{i@\texttt{\Texinfocommandstyletextvar{i}}}%

\noindent\begin{tabularx}{\linewidth}{@{}Xr}
\rightskip=5em plus 1 fill \hangindent=2em \hyphenpenalty=10000
\texttt{truc \EmbracOn{}\textnormal{\textsl{}}\EmbracOff{}}& [cmde]
\end{tabularx}

\index[fn]{truc@\texttt{truc}}%
\begin{quote}
\unskip{\parskip=0pt\noindent}%
text in def item for second def item
\end{quote}



\noindent\begin{tabularx}{\linewidth}{@{}Xr}
\rightskip=5em plus 1 fill \hangindent=2em \hyphenpenalty=10000
\texttt{d{-}{-}efvr\_name}& [c--ategory]
\end{tabularx}

\index[cp]{d--efvr\_name@\texttt{d{-}{-}efvr\_name}}%
\begin{quote}
\unskip{\parskip=0pt\noindent}%
d--efvr
\end{quote}


\noindent\begin{tabularx}{\linewidth}{@{}Xr}
\rightskip=5em plus 1 fill \hangindent=2em \hyphenpenalty=10000
\texttt{n{-}{-}ame \EmbracOn{}\textnormal{\textsl{a--rguments...}}\EmbracOff{}}& [c--ategory]
\end{tabularx}

\index[fn]{n--ame@\texttt{n{-}{-}ame}}%
\begin{quote}
\unskip{\parskip=0pt\noindent}%
d--effn
\end{quote}


\noindent\begin{tabularx}{\linewidth}{@{}Xr}
\rightskip=5em plus 1 fill \hangindent=2em \hyphenpenalty=10000
\texttt{n{-}{-}ame}& [c--ategory]
\end{tabularx}

\index[fn]{n--ame@\texttt{n{-}{-}ame}}%
\begin{quote}
\unskip{\parskip=0pt\noindent}%
d--effn no arg
\end{quote}


\noindent\begin{tabularx}{\linewidth}{@{}Xr}
\rightskip=5em plus 1 fill \hangindent=2em \hyphenpenalty=10000
\texttt{t{-}{-}ype d{-}{-}eftypefn\_name a{-}{-}rguments...}& [c--ategory]
\end{tabularx}

\index[fn]{d--eftypefn\_name@\texttt{d{-}{-}eftypefn\_name}}%
\begin{quote}
\unskip{\parskip=0pt\noindent}%
d--eftypefn
\end{quote}


\noindent\begin{tabularx}{\linewidth}{@{}Xr}
\rightskip=5em plus 1 fill \hangindent=2em \hyphenpenalty=10000
\texttt{t{-}{-}ype d{-}{-}eftypefn\_name}& [c--ategory]
\end{tabularx}

\index[fn]{d--eftypefn\_name@\texttt{d{-}{-}eftypefn\_name}}%
\begin{quote}
\unskip{\parskip=0pt\noindent}%
d--eftypefn no arg
\end{quote}


\noindent\begin{tabularx}{\linewidth}{@{}Xr}
\rightskip=5em plus 1 fill \hangindent=2em \hyphenpenalty=10000
\texttt{t{-}{-}ype d{-}{-}eftypeop\_name a{-}{-}rguments...}& [c--ategory on \texttt{c{-}{-}lass}]
\end{tabularx}

\index[fn]{d--eftypeop\_name on c--lass@\texttt{d{-}{-}eftypeop\_name\ on c{-}{-}lass}}%
\begin{quote}
\unskip{\parskip=0pt\noindent}%
d--eftypeop
\end{quote}


\noindent\begin{tabularx}{\linewidth}{@{}Xr}
\rightskip=5em plus 1 fill \hangindent=2em \hyphenpenalty=10000
\texttt{t{-}{-}ype d{-}{-}eftypeop\_name}& [c--ategory on \texttt{c{-}{-}lass}]
\end{tabularx}

\index[fn]{d--eftypeop\_name on c--lass@\texttt{d{-}{-}eftypeop\_name\ on c{-}{-}lass}}%
\begin{quote}
\unskip{\parskip=0pt\noindent}%
d--eftypeop no arg
\end{quote}


\noindent\begin{tabularx}{\linewidth}{@{}Xr}
\rightskip=5em plus 1 fill \hangindent=2em \hyphenpenalty=10000
\texttt{t{-}{-}ype d{-}{-}eftypevr\_name}& [c--ategory]
\end{tabularx}

\index[cp]{d--eftypevr\_name@\texttt{d{-}{-}eftypevr\_name}}%
\begin{quote}
\unskip{\parskip=0pt\noindent}%
d--eftypevr
\end{quote}


\noindent\begin{tabularx}{\linewidth}{@{}Xr}
\rightskip=5em plus 1 fill \hangindent=2em \hyphenpenalty=10000
\texttt{d{-}{-}efcv\_name}& [c--ategory of \texttt{c{-}{-}lass}]
\end{tabularx}

\index[cp]{d--efcv\_name@\texttt{d{-}{-}efcv\_name}}%
\begin{quote}
\unskip{\parskip=0pt\noindent}%
d--efcv
\end{quote}


\noindent\begin{tabularx}{\linewidth}{@{}Xr}
\rightskip=5em plus 1 fill \hangindent=2em \hyphenpenalty=10000
\texttt{d{-}{-}efcv\_name \EmbracOn{}\textnormal{\textsl{a--rguments...}}\EmbracOff{}}& [c--ategory of \texttt{c{-}{-}lass}]
\end{tabularx}

\index[cp]{d--efcv\_name@\texttt{d{-}{-}efcv\_name}}%
\begin{quote}
\unskip{\parskip=0pt\noindent}%
d--efcv with arguments
\end{quote}


\noindent\begin{tabularx}{\linewidth}{@{}Xr}
\rightskip=5em plus 1 fill \hangindent=2em \hyphenpenalty=10000
\texttt{t{-}{-}ype d{-}{-}eftypecv\_name}& [c--ategory of \texttt{c{-}{-}lass}]
\end{tabularx}

\index[cp]{d--eftypecv\_name of c--lass@\texttt{d{-}{-}eftypecv\_name\ of c{-}{-}lass}}%
\begin{quote}
\unskip{\parskip=0pt\noindent}%
d--eftypecv
\end{quote}


\noindent\begin{tabularx}{\linewidth}{@{}Xr}
\rightskip=5em plus 1 fill \hangindent=2em \hyphenpenalty=10000
\texttt{t{-}{-}ype d{-}{-}eftypecv\_name a{-}{-}rguments...}& [c--ategory of \texttt{c{-}{-}lass}]
\end{tabularx}

\index[cp]{d--eftypecv\_name of c--lass@\texttt{d{-}{-}eftypecv\_name\ of c{-}{-}lass}}%
\begin{quote}
\unskip{\parskip=0pt\noindent}%
d--eftypecv with arguments
\end{quote}


\noindent\begin{tabularx}{\linewidth}{@{}Xr}
\rightskip=5em plus 1 fill \hangindent=2em \hyphenpenalty=10000
\texttt{d{-}{-}efop\_name \EmbracOn{}\textnormal{\textsl{a--rguments...}}\EmbracOff{}}& [c--ategory on \texttt{c{-}{-}lass}]
\end{tabularx}

\index[fn]{d--efop\_name on c--lass@\texttt{d{-}{-}efop\_name\ on c{-}{-}lass}}%
\begin{quote}
\unskip{\parskip=0pt\noindent}%
d--efop
\end{quote}


\noindent\begin{tabularx}{\linewidth}{@{}Xr}
\rightskip=5em plus 1 fill \hangindent=2em \hyphenpenalty=10000
\texttt{d{-}{-}efop\_name}& [c--ategory on \texttt{c{-}{-}lass}]
\end{tabularx}

\index[fn]{d--efop\_name on c--lass@\texttt{d{-}{-}efop\_name\ on c{-}{-}lass}}%
\begin{quote}
\unskip{\parskip=0pt\noindent}%
d--efop no arg
\end{quote}


\noindent\begin{tabularx}{\linewidth}{@{}Xr}
\rightskip=5em plus 1 fill \hangindent=2em \hyphenpenalty=10000
\texttt{d{-}{-}eftp\_name \EmbracOn{}\textnormal{\textsl{a--ttributes...}}\EmbracOff{}}& [c--ategory]
\end{tabularx}

\index[tp]{d--eftp\_name@\texttt{d{-}{-}eftp\_name}}%
\begin{quote}
\unskip{\parskip=0pt\noindent}%
d--eftp
\end{quote}


\noindent\begin{tabularx}{\linewidth}{@{}Xr}
\rightskip=5em plus 1 fill \hangindent=2em \hyphenpenalty=10000
\texttt{d{-}{-}efun\_name \EmbracOn{}\textnormal{\textsl{a--rguments...}}\EmbracOff{}}& [Function]
\end{tabularx}

\index[fn]{d--efun\_name@\texttt{d{-}{-}efun\_name}}%
\begin{quote}
\unskip{\parskip=0pt\noindent}%
d--efun
\end{quote}


\noindent\begin{tabularx}{\linewidth}{@{}Xr}
\rightskip=5em plus 1 fill \hangindent=2em \hyphenpenalty=10000
\texttt{d{-}{-}efmac\_name \EmbracOn{}\textnormal{\textsl{a--rguments...}}\EmbracOff{}}& [Macro]
\end{tabularx}

\index[fn]{d--efmac\_name@\texttt{d{-}{-}efmac\_name}}%
\begin{quote}
\unskip{\parskip=0pt\noindent}%
d--efmac
\end{quote}


\noindent\begin{tabularx}{\linewidth}{@{}Xr}
\rightskip=5em plus 1 fill \hangindent=2em \hyphenpenalty=10000
\texttt{d{-}{-}efspec\_name \EmbracOn{}\textnormal{\textsl{a--rguments...}}\EmbracOff{}}& [Special Form]
\end{tabularx}

\index[fn]{d--efspec\_name@\texttt{d{-}{-}efspec\_name}}%
\begin{quote}
\unskip{\parskip=0pt\noindent}%
d--efspec
\end{quote}


\noindent\begin{tabularx}{\linewidth}{@{}Xr}
\rightskip=5em plus 1 fill \hangindent=2em \hyphenpenalty=10000
\texttt{d{-}{-}efvar\_name}& [Variable]
\end{tabularx}

\index[cp]{d--efvar\_name@\texttt{d{-}{-}efvar\_name}}%
\begin{quote}
\unskip{\parskip=0pt\noindent}%
d--efvar
\end{quote}


\noindent\begin{tabularx}{\linewidth}{@{}Xr}
\rightskip=5em plus 1 fill \hangindent=2em \hyphenpenalty=10000
\texttt{d{-}{-}efvar\_name \EmbracOn{}\textnormal{\textsl{arg--var arg--var1}}\EmbracOff{}}& [Variable]
\end{tabularx}

\index[cp]{d--efvar\_name@\texttt{d{-}{-}efvar\_name}}%
\begin{quote}
\unskip{\parskip=0pt\noindent}%
d--efvar with args
\end{quote}


\noindent\begin{tabularx}{\linewidth}{@{}Xr}
\rightskip=5em plus 1 fill \hangindent=2em \hyphenpenalty=10000
\texttt{d{-}{-}efopt\_name}& [User Option]
\end{tabularx}

\index[cp]{d--efopt\_name@\texttt{d{-}{-}efopt\_name}}%
\begin{quote}
\unskip{\parskip=0pt\noindent}%
d--efopt
\end{quote}


\noindent\begin{tabularx}{\linewidth}{@{}Xr}
\rightskip=5em plus 1 fill \hangindent=2em \hyphenpenalty=10000
\texttt{t{-}{-}ype d{-}{-}eftypefun\_name a{-}{-}rguments...}& [Function]
\end{tabularx}

\index[fn]{d--eftypefun\_name@\texttt{d{-}{-}eftypefun\_name}}%
\begin{quote}
\unskip{\parskip=0pt\noindent}%
d--eftypefun
\end{quote}


\noindent\begin{tabularx}{\linewidth}{@{}Xr}
\rightskip=5em plus 1 fill \hangindent=2em \hyphenpenalty=10000
\texttt{t{-}{-}ype d{-}{-}eftypevar\_name}& [Variable]
\end{tabularx}

\index[cp]{d--eftypevar\_name@\texttt{d{-}{-}eftypevar\_name}}%
\begin{quote}
\unskip{\parskip=0pt\noindent}%
d--eftypevar
\end{quote}


\noindent\begin{tabularx}{\linewidth}{@{}Xr}
\rightskip=5em plus 1 fill \hangindent=2em \hyphenpenalty=10000
\texttt{d{-}{-}efivar\_name}& [Instance Variable of \texttt{c{-}{-}lass}]
\end{tabularx}

\index[cp]{d--efivar\_name of c--lass@\texttt{d{-}{-}efivar\_name\ of c{-}{-}lass}}%
\begin{quote}
\unskip{\parskip=0pt\noindent}%
d--efivar
\end{quote}


\noindent\begin{tabularx}{\linewidth}{@{}Xr}
\rightskip=5em plus 1 fill \hangindent=2em \hyphenpenalty=10000
\texttt{t{-}{-}ype d{-}{-}eftypeivar\_name}& [Instance Variable of \texttt{c{-}{-}lass}]
\end{tabularx}

\index[cp]{d--eftypeivar\_name of c--lass@\texttt{d{-}{-}eftypeivar\_name\ of c{-}{-}lass}}%
\begin{quote}
\unskip{\parskip=0pt\noindent}%
d--eftypeivar
\end{quote}


\noindent\begin{tabularx}{\linewidth}{@{}Xr}
\rightskip=5em plus 1 fill \hangindent=2em \hyphenpenalty=10000
\texttt{d{-}{-}efmethod\_name \EmbracOn{}\textnormal{\textsl{a--rguments...}}\EmbracOff{}}& [Method on \texttt{c{-}{-}lass}]
\end{tabularx}

\index[fn]{d--efmethod\_name on c--lass@\texttt{d{-}{-}efmethod\_name\ on c{-}{-}lass}}%
\begin{quote}
\unskip{\parskip=0pt\noindent}%
d--efmethod
\end{quote}


\noindent\begin{tabularx}{\linewidth}{@{}Xr}
\rightskip=5em plus 1 fill \hangindent=2em \hyphenpenalty=10000
\texttt{t{-}{-}ype d{-}{-}eftypemethod\_name a{-}{-}rguments...}& [Method on \texttt{c{-}{-}lass}]
\end{tabularx}

\index[fn]{d--eftypemethod\_name on c--lass@\texttt{d{-}{-}eftypemethod\_name\ on c{-}{-}lass}}%
\begin{quote}
\unskip{\parskip=0pt\noindent}%
d--eftypemethod
\end{quote}



\noindent\begin{tabularx}{\linewidth}{@{}Xr}
\rightskip=5em plus 1 fill \hangindent=2em \hyphenpenalty=10000
\texttt{data-type2}& [Function]\\
\texttt{name2 arguments2...}\end{tabularx}

\index[fn]{name2@\texttt{name2}}%
\begin{quote}
\unskip{\parskip=0pt\noindent}%
aaa2
\end{quote}


\noindent\begin{tabularx}{\linewidth}{@{}Xr}
\rightskip=5em plus 1 fill \hangindent=2em \hyphenpenalty=10000
\texttt{t{-}{-}ype2}& [c--ategory2]\\
\texttt{d{-}{-}eftypefn\_name2}\end{tabularx}

\index[fn]{d--eftypefn\_name2@\texttt{d{-}{-}eftypefn\_name2}}%
\begin{quote}
\unskip{\parskip=0pt\noindent}%
d--eftypefn no arg2
\end{quote}


\noindent\begin{tabularx}{\linewidth}{@{}Xr}
\rightskip=5em plus 1 fill \hangindent=2em \hyphenpenalty=10000
\texttt{t{-}{-}ype2}& [c--ategory2 on \texttt{c{-}{-}lass2}]\\
\texttt{d{-}{-}eftypeop\_name2 a{-}{-}rguments2...}\end{tabularx}

\index[fn]{d--eftypeop\_name2 on c--lass2@\texttt{d{-}{-}eftypeop\_name2\ on c{-}{-}lass2}}%
\begin{quote}
\unskip{\parskip=0pt\noindent}%
d--eftypeop2
\end{quote}


\noindent\begin{tabularx}{\linewidth}{@{}Xr}
\rightskip=5em plus 1 fill \hangindent=2em \hyphenpenalty=10000
\texttt{t{-}{-}ype2}& [c--ategory2 on \texttt{c{-}{-}lass2}]\\
\texttt{d{-}{-}eftypeop\_name2}\end{tabularx}

\index[fn]{d--eftypeop\_name2 on c--lass2@\texttt{d{-}{-}eftypeop\_name2\ on c{-}{-}lass2}}%
\begin{quote}
\unskip{\parskip=0pt\noindent}%
d--eftypeop no arg2
\end{quote}


\noindent\begin{tabularx}{\linewidth}{@{}Xr}
\rightskip=5em plus 1 fill \hangindent=2em \hyphenpenalty=10000
\texttt{t{-}{-}ype2 d{-}{-}eftypecv\_name2}& [c--ategory2 of \texttt{c{-}{-}lass2}]
\end{tabularx}

\index[cp]{d--eftypecv\_name2 of c--lass2@\texttt{d{-}{-}eftypecv\_name2\ of c{-}{-}lass2}}%
\begin{quote}
\unskip{\parskip=0pt\noindent}%
d--eftypecv2
\end{quote}


\noindent\begin{tabularx}{\linewidth}{@{}Xr}
\rightskip=5em plus 1 fill \hangindent=2em \hyphenpenalty=10000
\texttt{t{-}{-}ype2 d{-}{-}eftypecv\_name2 a{-}{-}rguments2...}& [c--ategory2 of \texttt{c{-}{-}lass2}]
\end{tabularx}

\index[cp]{d--eftypecv\_name2 of c--lass2@\texttt{d{-}{-}eftypecv\_name2\ of c{-}{-}lass2}}%
\begin{quote}
\unskip{\parskip=0pt\noindent}%
d--eftypecv with arguments2
\end{quote}


\noindent\begin{tabularx}{\linewidth}{@{}Xr}
\rightskip=5em plus 1 fill \hangindent=2em \hyphenpenalty=10000
\texttt{arg2}& [fun2]
\end{tabularx}

\index[fn]{arg2@\texttt{arg2}}%
\begin{quote}
\unskip{\parskip=0pt\noindent}%
fff2
\end{quote}


\texttt{@xref\{c{-}{-}{-}hapter@@,\ cross r{-}{-}{-}ef name@@,\ t{-}{-}{-}itle@@,\ file n{-}{-}{-}ame@@,\ ma{-}{-}{-}nual@@\}} See Section ``t---itle@'' in \textsl{ma---nual@}.
\texttt{@ref\{chapter,\ cross ref name,\ title,\ file name,\ manual\}} Section ``title'' in \textsl{manual}
\texttt{@pxref\{chapter,\ cross ref name,\ title,\ file name,\ manual\}} see Section ``title'' in \textsl{manual}
\texttt{@inforef\{chapter,\ cross ref name,\ file name\}} Section ``chapter'' in \texttt{file name}

\texttt{@ref\{chapter\}} \hyperref[anchor:chapter]{\chaptername~\ref*{anchor:chapter} [chapter], page~\pageref*{anchor:chapter}}
\texttt{@xref\{chapter\}} See \hyperref[anchor:chapter]{\chaptername~\ref*{anchor:chapter} [chapter], page~\pageref*{anchor:chapter}}.
\texttt{@pxref\{chapter\}} see \hyperref[anchor:chapter]{\chaptername~\ref*{anchor:chapter} [chapter], page~\pageref*{anchor:chapter}}
\texttt{@ref\{s{-}{-}ect@comma\{\}ion\}} \hyperref[anchor:s_002d_002dect_002cion]{Section~\ref*{anchor:s_002d_002dect_002cion} [s--ect,ion], page~\pageref*{anchor:s_002d_002dect_002cion}}

\texttt{@ref\{s{-}{-}ect@comma\{\}ion,\ a @comma\{\}\ in cross
ref,\ a comma@comma\{\}\ in title,\ a comma@comma\{\}\ in file,\ a @comma\{\}\ in manual name \}}
Section ``a comma, in title'' in \textsl{a , in manual name}

\texttt{@ref\{chapter,cross ref name\}} \hyperref[anchor:chapter]{\chaptername~\ref*{anchor:chapter} [chapter], page~\pageref*{anchor:chapter}}
\texttt{@ref\{chapter{,}{,}title\}} \hyperref[anchor:chapter]{\chaptername~\ref*{anchor:chapter} [title], page~\pageref*{anchor:chapter}}
\texttt{@ref\{chapter{,}{,},file name\}} Section ``chapter'' in \texttt{file name}
\texttt{@ref\{chapter{,}{,}{,}{,}manual\}} Section ``chapter'' in \textsl{manual}
\texttt{@ref\{chapter,cross ref name,title,\}} \hyperref[anchor:chapter]{\chaptername~\ref*{anchor:chapter} [title], page~\pageref*{anchor:chapter}}
\texttt{@ref\{chapter,cross ref name{,}{,}file name\}} Section ``chapter'' in \texttt{file name}
\texttt{@ref\{chapter,cross ref name{,}{,},manual\}} Section ``chapter'' in \textsl{manual}
\texttt{@ref\{chapter,cross ref name,title,file name\}} Section ``title'' in \texttt{file name}
\texttt{@ref\{chapter,cross ref name,title{,}{,}manual\}} Section ``title'' in \textsl{manual}
\texttt{@ref\{chapter,cross ref name,title,\ file name,\ manual\}} Section ``title'' in \textsl{manual}
\texttt{@ref\{chapter{,}{,}title,file name\}} Section ``title'' in \texttt{file name}
\texttt{@ref\{chapter{,}{,}title{,}{,}manual\}} Section ``title'' in \textsl{manual}
\texttt{@ref\{chapter{,}{,}title,\ file name,\ manual\}} Section ``title'' in \textsl{manual}
\texttt{@ref\{chapter{,}{,},file name,manual\}} Section ``chapter'' in \textsl{manual}


\texttt{@ref\{(pman)anode,cross ref name\}} (pman)anode
\texttt{@ref\{(pman)anode{,}{,}title\}} title
\texttt{@ref\{(pman)anode{,}{,},file name\}} Section ``(pman)anode'' in \texttt{file name}
\texttt{@ref\{(pman)anode{,}{,}{,}{,}manual\}} Section ``(pman)anode'' in \textsl{manual}
\texttt{@ref\{(pman)anode,cross ref name,title,\}} title
\texttt{@ref\{(pman)anode,cross ref name{,}{,}file name\}} Section ``(pman)anode'' in \texttt{file name}
\texttt{@ref\{(pman)anode,cross ref name{,}{,},manual\}} Section ``(pman)anode'' in \textsl{manual}
\texttt{@ref\{(pman)anode,cross ref name,title,file name\}} Section ``title'' in \texttt{file name}
\texttt{@ref\{(pman)anode,cross ref name,title{,}{,}manual\}} Section ``title'' in \textsl{manual}
\texttt{@ref\{(pman)anode,cross ref name,title,\ file name,\ manual\}} Section ``title'' in \textsl{manual}
\texttt{@ref\{(pman)anode{,}{,}title,file name\}} Section ``title'' in \texttt{file name}
\texttt{@ref\{(pman)anode{,}{,}title{,}{,}manual\}} Section ``title'' in \textsl{manual}
\texttt{@ref\{(pman)anode{,}{,}title,\ file name,\ manual\}} Section ``title'' in \textsl{manual}
\texttt{@ref\{(pman)anode{,}{,},file name,manual\}} Section ``(pman)anode'' in \textsl{manual}


\texttt{@inforef\{chapter,\ cross ref name,\ file name\}} Section ``chapter'' in \texttt{file name}
\texttt{@inforef\{chapter\}} chapter
\texttt{@inforef\{chapter,\ cross ref name\}} chapter
\texttt{@inforef\{chapter{,}{,}file name\}} Section ``chapter'' in \texttt{file name}
\texttt{@inforef\{node,\ cross ref name,\ file name\}} Section ``node'' in \texttt{file name}
\texttt{@inforef\{node\}} node
\texttt{@inforef\{node,\ cross ref name\}} node
\texttt{@inforef\{node{,}{,}file name\}} Section ``node'' in \texttt{file name}
\texttt{@inforef\{chapter,\ cross ref name,\ file name,\ spurious arg\}} Section ``chapter'' in \texttt{file name,\ spurious arg}

\texttt{@inforef\{s{-}{-}ect@comma\{\}ion,\ a @comma\{\}\ in cross
ref,\ a comma@comma\{\}\ in file\}}
Section ``s--ect,ion'' in \texttt{a comma,\ in file}

`\texttt{\hyperref[anchor:chapter]{\chaptername~\ref*{anchor:chapter} [chapter], page~\pageref*{anchor:chapter}}}'.

Section ``title with uref2 \href{href://http/myhost.com/index2.html}{uref2 (\nolinkurl{href://http/myhost.com/index2.html})}'' in \textsl{printed manual with uref4 \href{href://http/myhost.com/index4.html}{uref4 (\nolinkurl{href://http/myhost.com/index4.html})}}
\hyperref[anchor:chapter]{\chaptername~\ref*{anchor:chapter} [title with uref2 \href{href://http/myhost.com/index2.html}{uref2 (\nolinkurl{href://http/myhost.com/index2.html})}], page~\pageref*{anchor:chapter}}

\begin{description}
\item[{\parbox[b]{\linewidth}{%
\textbf{a--strong}}}]
l--ine
\end{description}

\begin{description}
\item[{\parbox[b]{\linewidth}{%
a--asis\\
\index[cp]{a--asis@\texttt{a{-}{-}asis}}%
b
\index[cp]{b@\texttt{b}}%
}}]
l--ine
\end{description}

\begin{description}
\item[{\parbox[b]{\linewidth}{%
\emph{a}\\
\index[fn]{a@\texttt{a}}%
\index[cp]{index entry between item and itemx@index entry between item and itemx}%
\emph{b}
\index[fn]{b@\texttt{b}}%
}}]
l--ine
\end{description}

\begin{description}
\item[] Title
\item[{\parbox[b]{\linewidth}{%
\texttt{a{-}{-}code}}}]
Value--table code
\end{description}

\begin{description}
\item[] Title
\item[{\parbox[b]{\linewidth}{%
\Texinfotablestylesamp{a{-}{-}samp}\\
\Texinfotablestylesamp{a2{-}{-}samp}}}]
Value--table samp
\end{description}

\begin{mdframed}[style=Texinfocartouche]
c--artouche
\end{mdframed}

\begin{flushleft}
\begin{Texinfopreformatted}%
f--lushleft
more text
\end{Texinfopreformatted}
\end{flushleft}

\begin{flushright}
\begin{Texinfopreformatted}%
f--lushright
more text
\end{Texinfopreformatted}
\end{flushright}

\begin{center}
ce--ntered line
\end{center}

\begin{flushleft}
r--raggedright
more text
\end{flushleft}

\begin{verbatim}
\input texinfo @c -*-texinfo-*-

@c this file is used in tests in @verbatiminclude but not converted

@setfilename simplest.info

@node Top

This is a very simple texi manual @  <>.

@bye
\end{verbatim}

\begin{verbatim}
in verbatim ''
\end{verbatim}





$\frac{a < b \texttt{tex \hbox{ code }}}{b}$ ``

\Texinfonopagebreakheading{\chapter*}{{majorheading}}

\Texinfonopagebreakheading{\chapter*}{{chapheading}}

\section*{{heading}}

\subsection*{{subheading}}

\subsubsection*{{subsubheading}}


\texttt{@acronym\{{-}{-}a,an accronym @comma\{\}\ @enddots\{\}\}} --a (an accronym , \dots{})
\texttt{@abbr\{@'E{-}{-}.\ @comma\{\}A.,\ @'Etude{-}{-}@comma\{\}\ @b\{Autonome\}\ \}} \'{E}--.\@ ,A.\@ (\'{E}tude--, \textbf{Autonome})
\texttt{@abbr\{@'E{-}{-}.\ @comma\{\}A.\}} \'{E}--.\@ ,A.\@

\texttt{@math\{{-}{-}a@minus\{\}\ \{\textbackslash{}frac\{1\}\{2\}\}\}} $--a- {\frac{1}{2}}$




Somehow invalid use of @,:\leavevmode{}\\
@, \c{}\leavevmode{}\\
@,@"u \c{}\"{u}

Invalid use of @':\leavevmode{}\\
@' \'{}\leavevmode{}\\
@'@"u \'{}\"{u}

\texttt{@|} 

@dotless\{truc\} truc
@dotless\{ij\} ij
\texttt{@dotless\{{-}{-}a\}} --a
\texttt{@dotless\{a\}} a

@U, without braces @U\{\}, with empty arg 
@U\{z\}, with non-hex arg U+z
@U\{FFFFFFFFFFFFFF\}, value much too large U+FFFFFFFFFFFFFF
@U\{110000\}, value just beyond Unicode U+110000

@TeX, but without brace \TeX{}
\texttt{@\#} \#

\texttt{@w\{{-}{-}a\}} \hbox{--a}

\texttt{@image\{,1{-}{-}xt\}} 
\texttt{@image\{{,}{,}2{-}{-}xt\}} 
\texttt{@image\{{,}{,},3{-}{-}xt\}} 

\texttt{@image\{f-ile,aze{,}{,}a{-}{-}lt\}} \includegraphics[width=aze]{f-ile}
\texttt{@image\{f-ile{,}{,},alt@verb\{:jk \_" \%\@\}\}} \includegraphics{f-ile}

\texttt{@image\{f{-}{-}ile\}} \includegraphics{f--ile}
\texttt{@image\{f{-}{-}ile{,}{,},alt\}} \includegraphics{f--ile}
\texttt{@image\{f{-}{-}ile{,}{,}{,}{,}.e-d-xt\}} \includegraphics{f--ile}
\texttt{@image\{f{-}{-}ile,l{-}{-}i\}} \includegraphics[width=l--i]{f--ile}
\texttt{@image\{f{-}{-}ile{,}{,}l{-}{-}e\}} \includegraphics[height=l--e]{f--ile}
\texttt{@image\{f{-}{-}ile,aze,az,alt,.e{-}{-}xt\}} \includegraphics[width=aze,height=az]{f--ile}
\texttt{@image\{@file\{f{-}{-}ile\}@@@.,aze,az,alt,@file\{.file ext\}\ e{-}{-}xt@\}} \includegraphics[width=aze,height=az]{f--ile@.}

\texttt{@image\{f{-}{-}ile,aze,az,@verb\{:jk \_" \%@:\}\ @b\{in b "\},e{-}{-}xt\}} \includegraphics[width=aze,height=az]{f--ile}
\texttt{@image\{file@verb\{:jk \_" \%@:\}{,}{,},alt@verb\{:jk \_" \%@:\}\}} \includegraphics{filejk _" \%@}


{\bfseries author}%

$$
\ddot{u} \ddot{U} \tilde{n} \hat{a} \acute{e} \bar{o} \grave{i} \acute{e} \grave{\bar{E}}
\textsl{\c{\'{C}}} \textsl{\c{\'{C}}} \textsl{\H{a}} \dot{a} \mathring{a} \textsl{\t{a}}
\breve{a} \check{a}
 ? .
$$

$$
TeX LaTeX \star{} \mathord{\text{\aa{}}} \circledR{} ^{\circ{}} 
$$

$$
\mathtt{t} 
$$

\begin{itemize}[label=\emph{}]
\item e--mph item
\end{itemize}

\begin{itemize}[label=\emph{} after emph]
\item e--mph item
\end{itemize}

\begin{itemize}[label=\textbullet{} a--n itemize line]
\item i--tem 1
\item i--tem 2
\end{itemize}

\begin{itemize}[label={}]
\item without brace w a--b
\item without brace w c--d
\end{itemize}

\begin{description}
\item[{\parbox[b]{\linewidth}{%
a}}]
l--ine
\end{description}

\begin{description}
\item[{\parbox[b]{\linewidth}{%
a--missing style formatting}}]
l--ine
\end{description}

\begin{description}
\item[{\parbox[b]{\linewidth}{%
a\\
\index[fn]{a@\texttt{a}}%
\index[cp]{index entry between item and itemx@index entry between item and itemx}%
b
\index[fn]{b@\texttt{b}}%
}}]
l--ine
\end{description}


\noindent\begin{tabularx}{\linewidth}{@{}Xr}
\rightskip=5em plus 1 fill \hangindent=2em \hyphenpenalty=10000
\texttt{}& [fun]
\end{tabularx}


\noindent\begin{tabularx}{\linewidth}{@{}Xr}
\rightskip=5em plus 1 fill \hangindent=2em \hyphenpenalty=10000
\texttt{machin \EmbracOn{}\textnormal{\textsl{bidule chose and}}\EmbracOff{}}& [truc]
\end{tabularx}

\index[fn]{machin@\texttt{machin}}%

\noindent\begin{tabularx}{\linewidth}{@{}Xr}
\rightskip=5em plus 1 fill \hangindent=2em \hyphenpenalty=10000
\texttt{machin \EmbracOn{}\textnormal{\textsl{bidule chose and  after}}\EmbracOff{}}& [truc]
\end{tabularx}

\index[fn]{machin@\texttt{machin}}%

\noindent\begin{tabularx}{\linewidth}{@{}Xr}
\rightskip=5em plus 1 fill \hangindent=2em \hyphenpenalty=10000
\texttt{machin \EmbracOn{}\textnormal{\textsl{bidule chose and }}\EmbracOff{}}& [truc]
\end{tabularx}

\index[fn]{machin@\texttt{machin}}%

\noindent\begin{tabularx}{\linewidth}{@{}Xr}
\rightskip=5em plus 1 fill \hangindent=2em \hyphenpenalty=10000
\texttt{machin \EmbracOn{}\textnormal{\textsl{bidule chose and and after}}\EmbracOff{}}& [truc]
\end{tabularx}

\index[fn]{machin@\texttt{machin}}%

\noindent\begin{tabularx}{\linewidth}{@{}Xr}
\rightskip=5em plus 1 fill \hangindent=2em \hyphenpenalty=10000
\texttt{followed \EmbracOn{}\textnormal{\textsl{by a comment}}\EmbracOff{}}& [truc]
\end{tabularx}

\index[fn]{followed@\texttt{followed}}%
Various deff lines

\noindent\begin{tabularx}{\linewidth}{@{}Xr}
\rightskip=5em plus 1 fill \hangindent=2em \hyphenpenalty=10000
\texttt{after \EmbracOn{}\textnormal{\textsl{a deff item}}\EmbracOff{}}& [truc]
\end{tabularx}

\index[fn]{after@\texttt{after}}%


\noindent\begin{tabularx}{\linewidth}{@{}Xr}
\rightskip=5em plus 1 fill \hangindent=2em \hyphenpenalty=10000
\texttt{\Texinfocommandstyletextvar{invalid} \EmbracOn{}\textnormal{\textsl{a g}}\EmbracOff{}}& [fsetinv]
\end{tabularx}

\index[fn]{invalid@\texttt{\Texinfocommandstyletextvar{invalid}}}%

\noindent\begin{tabularx}{\linewidth}{@{}Xr}
\rightskip=5em plus 1 fill \hangindent=2em \hyphenpenalty=10000
\texttt{}& [\textbf{id `\texttt{i}' ule}]
\end{tabularx}



\noindent\begin{tabularx}{\linewidth}{@{}Xr}
\rightskip=5em plus 1 fill \hangindent=2em \hyphenpenalty=10000
\texttt{}& [aaa]
\end{tabularx}


\noindent\begin{tabularx}{\linewidth}{@{}Xr}
\rightskip=5em plus 1 fill \hangindent=2em \hyphenpenalty=10000
\texttt{}& []
\end{tabularx}


\noindent\begin{tabularx}{\linewidth}{@{}Xr}
\rightskip=5em plus 1 fill \hangindent=2em \hyphenpenalty=10000
\texttt{}& [truc]
\end{tabularx}


g--roupe

\texttt{@ref\{node\}} node

\texttt{@ref\{,cross ref name\}} 
\texttt{@ref\{{,}{,}title\}} title
\texttt{@ref\{{,}{,},file name\}} \texttt{file name}
\texttt{@ref\{{,}{,}{,}{,}manual\}} \textsl{manual}
\texttt{@ref\{node,cross ref name\}} node
\texttt{@ref\{node{,}{,}title\}} title
\texttt{@ref\{node{,}{,},file name\}} Section ``node'' in \texttt{file name}
\texttt{@ref\{node{,}{,}{,}{,}manual\}} Section ``node'' in \textsl{manual}
\texttt{@ref\{node,cross ref name,title,\}} title
\texttt{@ref\{node,cross ref name{,}{,}file name\}} Section ``node'' in \texttt{file name}
\texttt{@ref\{node,cross ref name{,}{,},manual\}} Section ``node'' in \textsl{manual}
\texttt{@ref\{node,cross ref name,title,file name\}} Section ``title'' in \texttt{file name}
\texttt{@ref\{node,cross ref name,title{,}{,}manual\}} Section ``title'' in \textsl{manual}
\texttt{@ref\{node,cross ref name,title,\ file name,\ manual\}} Section ``title'' in \textsl{manual}
\texttt{@ref\{node{,}{,}title,file name\}} Section ``title'' in \texttt{file name}
\texttt{@ref\{node{,}{,}title{,}{,}manual\}} Section ``title'' in \textsl{manual}
\texttt{@ref\{chapter{,}{,}title,\ file name,\ manual\}} Section ``title'' in \textsl{manual}
\texttt{@ref\{node{,}{,}title,\ file name,\ manual\}} Section ``title'' in \textsl{manual}
\texttt{@ref\{node{,}{,},file name,manual\}} Section ``node'' in \textsl{manual}
\texttt{@ref\{,cross ref name,title,\}} title
\texttt{@ref\{,cross ref name{,}{,}file name\}} \texttt{file name}
\texttt{@ref\{,cross ref name{,}{,},manual\}} \textsl{manual}
\texttt{@ref\{,cross ref name,title,file name\}} Section ``title'' in \texttt{file name}
\texttt{@ref\{,cross ref name,title{,}{,}manual\}} Section ``title'' in \textsl{manual}
\texttt{@ref\{,cross ref name,title,\ file name,\ manual\}} Section ``title'' in \textsl{manual}
\texttt{@ref\{{,}{,}title,file name\}} Section ``title'' in \texttt{file name}
\texttt{@ref\{{,}{,}title{,}{,}manual\}} Section ``title'' in \textsl{manual}
\texttt{@ref\{{,}{,}title,\ file name,\ manual\}} Section ``title'' in \textsl{manual}
\texttt{@ref\{{,}{,},file name,manual\}} \textsl{manual}

\texttt{@inforef\{,cross ref name \}} 
\texttt{@inforef\{{,}{,}file name\}} \texttt{file name}
\texttt{@inforef\{,cross ref name,\ file name\}} \texttt{file name}
\texttt{@inforef\{\}} 



In example.
\begin{Texinfoindented}
\begin{Texinfopreformatted}%
\ttfamily 
<
>
"
\&
'
`

{`}{`}simple-double{-}{-}three{-}{-}{-}four{-}{-}{-}-{'}{'}\leavevmode{}\\
code:\ \texttt{{`}{`}simple-double{-}{-}three{-}{-}{-}four{-}{-}{-}-{'}{'}}\ \leavevmode{}\\
asis:\ {`}{`}simple-double{-}{-}three{-}{-}{-}four{-}{-}{-}-{'}{'}\ \leavevmode{}\\
strong:\ \textbf{{`}{`}simple-double{-}{-}three{-}{-}{-}four{-}{-}{-}-{'}{'}}\ \leavevmode{}\\
kbd:\ \Texinfocommandstyletextkbd{{`}{`}simple-double{-}{-}three{-}{-}{-}four{-}{-}{-}-{'}{'}}\ \leavevmode{}\\

`\hbox{}`simple-double-\hbox{}-three{-}{-}{-}four{-}{-}{-}-'\hbox{}'\leavevmode{}\\

\index[cp]{-option@--option}%
\index[cp]{""@``}%
\index[fn]{``@\texttt{{`}{`}}}%
\index[fn]{--foption@\texttt{{-}{-}foption}}%

@"u \"{u}\ 
@"\{U\}\ \"{U}\ 
@\~{}n \~{n}
@\^{}a \^{a}
@'e \'{e}
@=o \={o}
@`i \`{i}
@'\{e\}\ \'{e}
@'\{@dotless\{i\}\}\ \'{\i{}}\ 
@dotless\{i\}\ \i{}
@dotless\{j\}\ \j{}
@`\{@=E\}\ \`{\={E}}\ 
@l\{\}\ \l{}
@,\{@'C\}\ \c{\'{C}}
@,c \c{c}
@,c@"u \c{c}\"{u}\ \leavevmode{}\\

@U\{0075\}\ u

@* \leavevmode{}\\
@\ followed by a space
\ {}
@\ followed by a tab
\ {}
@\ followed by a new line
\ {}\texttt{@-}\ \-{}
\texttt{@:}\ \@
\texttt{@!}\ \@!
\texttt{@?}\ \@?
\texttt{@.}\ \@.
\texttt{@@}\ @
\texttt{@\}}\ \}
\texttt{@\{}\ \{
\texttt{@/}\ 

foo vs.\@\ bar.\ 
colon :\@And something else.
semi colon ;\@.
And ?\ ?\@.
Now !\ !\@@
but ,\ ,\@

@TeX \TeX{}
@LaTeX \LaTeX{}
@bullet \textbullet{}
@copyright \copyright{}
@dots \dots{}\@
@enddots \dots{}
@equiv $\equiv{}$
@error \fbox{error}
@expansion $\mapsto{}$
@minus -
@point $\star{}$
@print $\dashv{}$
@result $\Rightarrow{}$
@today \today{}

@aa \aa{}
@AA \AA{}
@ae \ae{}
@oe \oe{}
@AE \AE{}
@OE \OE{}
@o \o{}
@O \O{}
@ss \ss{}
@l \l{}
@L \L{}
@DH \DH{}
@TH \TH{}
@dh \dh{}
@th \th{}

@exclamdown \textexclamdown{}
@questiondown \textquestiondown{}
@pounds \textsterling{}
@registeredsymbol \circledR{}
@ordf \textordfeminine{}
@ordm \textordmasculine{}
@comma ,
@quotedblleft \textquotedblleft{}
@quotedblright \textquotedblright{}
@quoteleft \textquoteleft{}
@quoteright \textquoteright{}
@quotedblbase \quotedblbase{}
@quotesinglbase \quotesinglbase{}
@guillemetleft \guillemotleft{}
@guillemetright \guillemotright{}
@guillemotleft \guillemotleft{}
@guillemotright \guillemotright{}
@guilsinglleft \guilsinglleft{}
@guilsinglright \guilsinglright{}

@textdegree \textdegree{}
@euro \euro{}
@arrow $\rightarrow{}$
@leq $\leq{}$
@geq $\geq{}$
@tie a~b

\texttt{@acronym\{{-}{-}a,an accronym\}}\ {-}{-}a\ (an accronym)
\texttt{@acronym\{{-}{-}a\}}\ {-}{-}a
\texttt{@abbr\{@'E{-}{-}.\ @comma\{\}A.,\ @'Etude Autonome \}}\ \'{E}{-}{-}.\@\ ,A.\@\ (\'{E}tude Autonome)
\texttt{@abbr\{@'E{-}{-}.\ @comma\{\}A.\}}\ \'{E}{-}{-}.\@\ ,A.\@
\texttt{@asis\{{-}{-}a\}}\ {-}{-}a
\texttt{@b\{{-}{-}a\}}\ \textbf{{-}{-}a}
\texttt{@cite\{{-}{-}a\}}\ \Texinfocommandstyletextcite{--a}
\texttt{@code\{{-}{-}a\}}\ \texttt{{-}{-}a}
\texttt{@command\{{-}{-}a\}}\ \texttt{{-}{-}a}
\texttt{@dfn\{{-}{-}a\}}\ \textsl{{-}{-}a}
\texttt{@dmn\{{-}{-}a\}}\ \thinspace {-}{-}a
\texttt{@email\{{-}{-}a,{-}{-}b\}}\ \href{mailto:--a}{{-}{-}b}
\texttt{@email\{,{-}{-}b\}}\ {-}{-}b
\texttt{@email\{{-}{-}a\}}\ \href{mailto:--a}{\nolinkurl{--a}}
\texttt{@emph\{{-}{-}a\}}\ \emph{{-}{-}a}
\texttt{@env\{{-}{-}a\}}\ \texttt{{-}{-}a}
\texttt{@file\{{-}{-}a\}}\ \texttt{{-}{-}a}
\texttt{@i\{{-}{-}a\}}\ \textit{{-}{-}a}
\texttt{@kbd\{{-}{-}a\}}\ \Texinfocommandstyletextkbd{{-}{-}a}
\texttt{@key\{{-}{-}a\}}\ \texttt{{-}{-}a}
\texttt{@math\{{-}{-}a \{\textbackslash{}frac\{1\}\{2\}\}\ @minus\{\}\}}\ $--a {\frac{1}{2}} -$
\texttt{@option\{{-}{-}a\}}\ \texttt{{-}{-}a}
\texttt{@r\{{-}{-}a\}}\ \textnormal{--a}
\texttt{@samp\{{-}{-}a\}}\ `\texttt{{-}{-}a}'
\texttt{@sc\{{-}{-}a\}}\ \textsc{{-}{-}a}
\texttt{@strong\{{-}{-}a\}}\ \textbf{{-}{-}a}
\texttt{@t\{{-}{-}a\}}\ \texttt{{-}{-}a}
\texttt{@sansserif\{{-}{-}a\}}\ \textsf{{-}{-}a}
\texttt{@slanted\{{-}{-}a\}}\ \textsl{{-}{-}a}
\texttt{@titlefont\{{-}{-}a\}}\ \end{Texinfopreformatted}
{\huge \bfseries --a}\begin{Texinfopreformatted}%
\ttfamily 
\texttt{@indicateurl\{{-}{-}a\}}\ `\texttt{{-}{-}a}'
\texttt{@uref\{{-}{-}a,{-}{-}b\}}\ \href{--a}{{-}{-}b (\nolinkurl{--a})}
\texttt{@uref\{{-}{-}a\}}\ \url{--a}
\texttt{@uref\{,{-}{-}b\}}\ {-}{-}b
\texttt{@uref\{{-}{-}a,{-}{-}b,{-}{-}c\}}\ {-}{-}c
\texttt{@uref\{,{-}{-}b,{-}{-}c\}}\ {-}{-}c
\texttt{@uref\{{-}{-}a{,}{,}{-}{-}c\}}\ {-}{-}c
\texttt{@uref\{{,}{,}{-}{-}c\}}\ {-}{-}c
\texttt{@url\{{-}{-}a,{-}{-}b\}}\ \href{--a}{{-}{-}b (\nolinkurl{--a})}
\texttt{@url\{{-}{-}a,\}}\ \url{--a}
\texttt{@url\{,{-}{-}b\}}\ {-}{-}b
\texttt{@var\{{-}{-}a\}}\ \Texinfocommandstyletextvar{--a}
\texttt{@verb\{:{-}{-}a:\}}\ \verb:--a:
\texttt{@verb\{:a  < \& @\ \% " {-}{-}    b:\}}\ \verb:a  < & @ % " --    b:
\texttt{@w\{a a a a a a a a a a a a a a a a a a a a a a a a a a a a a a a a a a a\}}\ \hbox{a a a a a a a a a a a a a a a a a a a a a a a a a a a a a a a a a a a}
\texttt{@H\{a\}}\ \H{a}
\texttt{@H\{{-}{-}a\}}\ \H{{-}{-}a}
\texttt{@dotaccent\{a\}}\ \.{a}
\texttt{@dotaccent\{{-}{-}a\}}\ \.{{-}{-}a}
\texttt{@ringaccent\{a\}}\ \r{a}
\texttt{@ringaccent\{{-}{-}a\}}\ \r{{-}{-}a}
\texttt{@tieaccent\{a\}}\ \t{a}
\texttt{@tieaccent\{{-}{-}a\}}\ \t{{-}{-}a}
\texttt{@u\{a\}}\ \u{a}
\texttt{@u\{{-}{-}a\}}\ \u{{-}{-}a}
\texttt{@ubaraccent\{a\}}\ \b{a}
\texttt{@ubaraccent\{{-}{-}a\}}\ \b{{-}{-}a}
\texttt{@udotaccent\{a\}}\ \d{a}
\texttt{@udotaccent\{{-}{-}a\}}\ \d{{-}{-}a}
\texttt{@v\{a\}}\ \v{a}
\texttt{@v\{{-}{-}a\}}\ \v{{-}{-}a}
\texttt{@,\{c\}}\ \c{c}
\texttt{@,\{{-}{-}c\}}\ \c{{-}{-}c}
\texttt{@ogonek\{a\}}\ \k{a}
\texttt{@ogonek\{{-}{-}a\}}\ \k{{-}{-}a}
\texttt{a@sup\{h\}@sub\{l\}}\ a\textsuperscript{h}\textsubscript{l}
\texttt{@footnote\{in footnote\}}\ \footnote{in footnote}
\texttt{@footnote\{in footnote2\}}\ \footnote{in footnote2}

\texttt{@sp 2}\leavevmode{}\\
\vskip 2\baselineskip %
\texttt{@page}\leavevmode{}\\
\end{Texinfopreformatted}
\newpage{}%
\phantom{blabla}%
\begin{Texinfopreformatted}%
\ttfamily 
\texttt{need 1002}
\end{Texinfopreformatted}
\needspace{1.002pt}%
\begin{Texinfopreformatted}%
\ttfamily 
\texttt{@clicksequence\{click @click\{\}\ A\}}\ click $\rightarrow{}$\ A
After clickstyle $\Rightarrow{}$
\texttt{@clicksequence\{click @click\{\}\ A\}}\ click $\Rightarrow{}$\ A


\end{Texinfopreformatted}
$$
disp--laymath
f(x) = {1 \over \sigma \sqrt{2\pi}}e^{-{1 \over 2}\left({x-\mu \over \sigma}\right)^2}
$$
\begin{Texinfopreformatted}%
\ttfamily 
\end{Texinfopreformatted}
$$
\mathbf{``simple-double--three---four----''} \hbox{aa}
`\hbox{}`simple-double-\hbox{}-three---four----'\hbox{}'
$$
\begin{Texinfopreformatted}%
\ttfamily 
\end{Texinfopreformatted}
$$
\imath{} \jmath{}
\mathord{\text{\l{}}} \textsl{\c{c}}
\textsl{\b{a}} \textsl{\d{a}} \textsl{\k{a}} a^{h}_{l}
 \ {}\ {} \ {}\-{}  ! @ \} \{ 
\today{}
$$
\begin{Texinfopreformatted}%
\ttfamily 
\end{Texinfopreformatted}
$$
\rightarrow{}
u
\bullet{} \copyright{} \dots{} \dots{} \equiv{}
\fbox{error} \mapsto{} - \dashv{} \Rightarrow{}
\mathord{\text{\AA{}}} \mathord{\text{\ae{}}} \mathord{\text{\oe{}}} \mathord{\text{\AE{}}} \mathord{\text{\OE{}}} \mathord{\text{\o{}}} \mathord{\text{\O{}}} \mathord{\text{\ss{}}} \mathord{\text{\l{}}} \mathord{\text{\L{}}} \mathord{\text{\DH{}}}
\mathord{\text{\TH{}}} \mathord{\text{\dh{}}} \mathord{\text{\th{}}} \mathord{\text{\textexclamdown{}}} \mathord{\text{\textquestiondown{}}} \mathsterling{}
\mathord{\text{\textordfeminine{}}} \mathord{\text{\textordmasculine{}}} , 
$$
\begin{Texinfopreformatted}%
\ttfamily 
\end{Texinfopreformatted}
$$
\mathord{\text{\textquotedblleft{}}} \mathord{\text{\textquotedblright{}}} 
\mathord{\text{\textquoteleft{}}} \mathord{\text{\textquoteright{}}} \mathord{\text{\quotedblbase{}}} \mathord{\text{\quotesinglbase{}}} \mathord{\text{\guillemotleft{}}}
\mathord{\text{\guillemotright{}}} \mathord{\text{\guillemotleft{}}} \mathord{\text{\guillemotright{}}} \mathord{\text{\guilsinglleft{}}}
\mathord{\text{\guilsinglright{}}} \euro{} \rightarrow{} \leq{} \geq{}
$$
\begin{Texinfopreformatted}%
\ttfamily 
\end{Texinfopreformatted}
$$
\mathbf{b} \mathit{i} \mathrm{r} sc \mathsf{sansserif} slanted
$$
\begin{Texinfopreformatted}%
\ttfamily 
\Texinfocommandstyletextkbd{default kbdinputstyle}
\end{Texinfopreformatted}
\begin{description}
\item[{\parbox[b]{\linewidth}{%
\Texinfocommandstyletextkbd{vtable i{-}{-}tem default kbdinputstyle}
\index[cp]{vtable i--tem default kbdinputstyle@\texttt{vtable i{-}{-}tem default kbdinputstyle}}%
}}]
\end{description}
\begin{Texinfopreformatted}%
\ttfamily \end{Texinfopreformatted}
\begin{Texinfoindented}
\begin{Texinfopreformatted}%
\ttfamily \Texinfocommandstyletextkbd{in example default kbdinputstyle}
\end{Texinfopreformatted}
\begin{description}
\item[{\parbox[b]{\linewidth}{%
\Texinfocommandstyletextkbd{vtable i{-}{-}tem in example default kbdinputstyle}
\index[cp]{vtable i--tem in example default kbdinputstyle@\texttt{vtable i{-}{-}tem in example default kbdinputstyle}}%
}}]
\end{description}
\end{Texinfoindented}
\begin{Texinfopreformatted}%
\ttfamily 
\texttt{code kbdinputstyle}
\end{Texinfopreformatted}
\begin{description}
\item[{\parbox[b]{\linewidth}{%
\texttt{vtable i{-}{-}tem code kbdinputstyle}
\index[cp]{vtable i--tem code kbdinputstyle@\texttt{vtable i{-}{-}tem code kbdinputstyle}}%
}}]
\end{description}
\begin{Texinfopreformatted}%
\ttfamily \end{Texinfopreformatted}
\begin{Texinfoindented}
\begin{Texinfopreformatted}%
\ttfamily \texttt{in example code kbdinputstyle}
\end{Texinfopreformatted}
\begin{description}
\item[{\parbox[b]{\linewidth}{%
\texttt{vtable i{-}{-}tem in example code kbdinputstyle}
\index[cp]{vtable i--tem in example code kbdinputstyle@\texttt{vtable i{-}{-}tem in example code kbdinputstyle}}%
}}]
\end{description}
\end{Texinfoindented}
\begin{Texinfopreformatted}%
\ttfamily 
\Texinfocommandstyletextkbd{example kbdinputstyle}
\end{Texinfopreformatted}
\begin{description}
\item[{\parbox[b]{\linewidth}{%
\Texinfocommandstyletextkbd{vtable i{-}{-}tem example kbdinputstyle}
\index[cp]{vtable i--tem example kbdinputstyle@\texttt{vtable i{-}{-}tem example kbdinputstyle}}%
}}]
\end{description}
\begin{Texinfopreformatted}%
\ttfamily \end{Texinfopreformatted}
\begin{Texinfoindented}
\begin{Texinfopreformatted}%
\ttfamily \Texinfocommandstyletextkbd{in example example kbdinputstyle}
\end{Texinfopreformatted}
\begin{description}
\item[{\parbox[b]{\linewidth}{%
\Texinfocommandstyletextkbd{vtable i{-}{-}tem in example example kbdinputstyle}
\index[cp]{vtable i--tem in example example kbdinputstyle@\texttt{vtable i{-}{-}tem in example example kbdinputstyle}}%
}}]
\end{description}
\end{Texinfoindented}
\begin{Texinfopreformatted}%
\ttfamily 
\Texinfocommandstyletextkbd{distinct kbdinputstyle}
\end{Texinfopreformatted}
\begin{description}
\item[{\parbox[b]{\linewidth}{%
\Texinfocommandstyletextkbd{vtable i{-}{-}tem distinct kbdinputstyle}
\index[cp]{vtable i--tem distinct kbdinputstyle@\texttt{vtable i{-}{-}tem distinct kbdinputstyle}}%
}}]
\end{description}
\begin{Texinfopreformatted}%
\ttfamily \end{Texinfopreformatted}
\begin{Texinfoindented}
\begin{Texinfopreformatted}%
\ttfamily \Texinfocommandstyletextkbd{in example distinct kbdinputstyle}
\end{Texinfopreformatted}
\begin{description}
\item[{\parbox[b]{\linewidth}{%
\Texinfocommandstyletextkbd{vtable i{-}{-}tem in example distinct kbdinputstyle}
\index[cp]{vtable i--tem in example distinct kbdinputstyle@\texttt{vtable i{-}{-}tem in example distinct kbdinputstyle}}%
}}]
\end{description}
\end{Texinfoindented}
\begin{Texinfopreformatted}%
\ttfamily 
\end{Texinfopreformatted}
\begin{quote}
\begin{Texinfopreformatted}%
\ttfamily A quot{-}{-}{-}ation
\end{Texinfopreformatted}
\end{quote}
\begin{Texinfopreformatted}%
\ttfamily 
\end{Texinfopreformatted}
\begin{quote}
\textbf{Note:} \begin{Texinfopreformatted}%
\ttfamily A Note
\end{Texinfopreformatted}
\end{quote}
\begin{Texinfopreformatted}%
\ttfamily 
\end{Texinfopreformatted}
\begin{quote}
\textbf{note:} \begin{Texinfopreformatted}%
\ttfamily A note
\end{Texinfopreformatted}
\end{quote}
\begin{Texinfopreformatted}%
\ttfamily 
\end{Texinfopreformatted}
\begin{quote}
\textbf{Caution:} \begin{Texinfopreformatted}%
\ttfamily Caution
\end{Texinfopreformatted}
\end{quote}
\begin{Texinfopreformatted}%
\ttfamily 
\end{Texinfopreformatted}
\begin{quote}
\textbf{Important:} \begin{Texinfopreformatted}%
\ttfamily Important
\end{Texinfopreformatted}
\end{quote}
\begin{Texinfopreformatted}%
\ttfamily 
\end{Texinfopreformatted}
\begin{quote}
\textbf{Tip:} \begin{Texinfopreformatted}%
\ttfamily a Tip
\end{Texinfopreformatted}
\end{quote}
\begin{Texinfopreformatted}%
\ttfamily 
\end{Texinfopreformatted}
\begin{quote}
\textbf{Warning:} \begin{Texinfopreformatted}%
\ttfamily a Warning.
\end{Texinfopreformatted}
\end{quote}
\begin{Texinfopreformatted}%
\ttfamily 
\end{Texinfopreformatted}
\begin{quote}
\textbf{something \'{e} \TeX{}:} \begin{Texinfopreformatted}%
\ttfamily The something \'{e}\ \TeX{}\ is here.
\end{Texinfopreformatted}
\end{quote}
\begin{Texinfopreformatted}%
\ttfamily 
\end{Texinfopreformatted}
\begin{quote}
\textbf{@ at the end of line \ {}:} \begin{Texinfopreformatted}%
\ttfamily A @\ at the end of the @quotation line.
\end{Texinfopreformatted}
\end{quote}
\begin{Texinfopreformatted}%
\ttfamily 
\end{Texinfopreformatted}
\begin{quote}
\textbf{something, other thing:} \begin{Texinfopreformatted}%
\ttfamily something,\ other thing
\end{Texinfopreformatted}
\end{quote}
\begin{Texinfopreformatted}%
\ttfamily 
\end{Texinfopreformatted}
\begin{quote}
\textbf{Note, the note:} \begin{Texinfopreformatted}%
\ttfamily Note,\ the note
\end{Texinfopreformatted}
\end{quote}
\begin{Texinfopreformatted}%
\ttfamily 
\end{Texinfopreformatted}
\begin{quote}
\end{quote}
\begin{Texinfopreformatted}%
\ttfamily 
\end{Texinfopreformatted}
\begin{quote}
\textbf{Empty:} \end{quote}
\begin{Texinfopreformatted}%
\ttfamily 
\end{Texinfopreformatted}
\begin{quote}
\textbf{:} \end{quote}
\begin{Texinfopreformatted}%
\ttfamily 
\end{Texinfopreformatted}
\begin{quote}
\textbf{\leavevmode{}\\:} \end{quote}
\begin{Texinfopreformatted}%
\ttfamily 
\end{Texinfopreformatted}
\begin{quote}
\begin{Texinfopreformatted}%
\ttfamily aaa quotation
\end{Texinfopreformatted}
\end{quote}
\begin{center}
--- \emph{quotation author}
\end{center}
\begin{Texinfopreformatted}%
\ttfamily 
\end{Texinfopreformatted}
\begin{quote}
\begin{Texinfopreformatted}%
\ttfamily indent in quotation
\end{Texinfopreformatted}
\end{quote}
\begin{Texinfopreformatted}%
\ttfamily 
\end{Texinfopreformatted}
\begin{quote}
\leavevmode{}\\
\hbox{\kern -\leftmargin}%
exdented quotation line   and dash --- in quotation
\\
\end{quote}
\begin{Texinfopreformatted}%
\ttfamily 
\end{Texinfopreformatted}
\begin{quote}
\begin{Texinfopreformatted}%
\ttfamily Not exdented followed by exdented
\end{Texinfopreformatted}
\leavevmode{}\\
\hbox{\kern -\leftmargin}%
exdented quotation line
\\
\end{quote}
\begin{Texinfopreformatted}%
\ttfamily 
\end{Texinfopreformatted}
\begin{quote}
\leavevmode{}\\
\hbox{\kern -\leftmargin}%
exdented quotation line
\\
\begin{Texinfopreformatted}%
\ttfamily Followed by not exdented 
\end{Texinfopreformatted}
\end{quote}
\begin{Texinfopreformatted}%
\ttfamily 
\end{Texinfopreformatted}
\begin{quote}
\begin{Texinfopreformatted}%
\ttfamily quotation1
\end{Texinfopreformatted}
\leavevmode{}\\
\hbox{\kern -\leftmargin}%
in exdented protected eol \ {}
\\
\begin{Texinfopreformatted}%
\ttfamily following
\end{Texinfopreformatted}
\leavevmode{}\\
\hbox{\kern -\leftmargin}%
in exdented a @* \leavevmode{}\\ and following
\\
\begin{Texinfopreformatted}%
\ttfamily after exdented
\end{Texinfopreformatted}
\end{quote}
\begin{Texinfopreformatted}%
\ttfamily 
\end{Texinfopreformatted}
\begin{quote}
\begin{footnotesize}
\begin{Texinfopreformatted}%
\ttfamily A small quot{-}{-}{-}ation
\end{Texinfopreformatted}
\end{footnotesize}
\end{quote}
\begin{Texinfopreformatted}%
\ttfamily 
\end{Texinfopreformatted}
\begin{quote}
\begin{footnotesize}
\textbf{Note:} \begin{Texinfopreformatted}%
\ttfamily A small Note
\end{Texinfopreformatted}
\end{footnotesize}
\end{quote}
\begin{Texinfopreformatted}%
\ttfamily 
\end{Texinfopreformatted}
\begin{quote}
\begin{footnotesize}
\textbf{something, other thing:} \begin{Texinfopreformatted}%
\ttfamily something,\ other thing
\end{Texinfopreformatted}
\end{footnotesize}
\end{quote}
\begin{Texinfopreformatted}%
\ttfamily 
\end{Texinfopreformatted}
\begin{itemize}
\item \begin{Texinfopreformatted}%
\ttfamily i{-}{-}temize
\end{Texinfopreformatted}
\end{itemize}
\begin{Texinfopreformatted}%
\ttfamily 
\end{Texinfopreformatted}
\begin{itemize}[label=+]
\item \begin{Texinfopreformatted}%
\ttfamily i{-}{-}tem +
\end{Texinfopreformatted}
\end{itemize}
\begin{Texinfopreformatted}%
\ttfamily 
\end{Texinfopreformatted}
\begin{itemize}[label=\textbullet{}]
\item \begin{Texinfopreformatted}%
\ttfamily b{-}{-}ullet
\end{Texinfopreformatted}
\end{itemize}
\begin{Texinfopreformatted}%
\ttfamily 
\end{Texinfopreformatted}
\begin{itemize}[label=-]
\item \begin{Texinfopreformatted}%
\ttfamily minu{-}{-}s
\end{Texinfopreformatted}
\end{itemize}
\begin{Texinfopreformatted}%
\ttfamily 
\end{Texinfopreformatted}
\begin{itemize}[label=\emph{after emph}]
\item \begin{Texinfopreformatted}%
\ttfamily \end{Texinfopreformatted}
\item \begin{Texinfopreformatted}%
\ttfamily e{-}{-}mph item
\end{Texinfopreformatted}
\end{itemize}
\begin{Texinfopreformatted}%
\ttfamily 
\end{Texinfopreformatted}
\begin{itemize}[label=\textbullet{} a--n itemize line]
\item \begin{Texinfopreformatted}%
\ttfamily \end{Texinfopreformatted}
\item \begin{Texinfopreformatted}%
\ttfamily \index[cp]{index entry within itemize@index entry within itemize}%
i{-}{-}tem 1
\end{Texinfopreformatted}
\item \begin{Texinfopreformatted}%
\ttfamily i{-}{-}tem 2
\end{Texinfopreformatted}
\end{itemize}
\begin{Texinfopreformatted}%
\ttfamily 
\end{Texinfopreformatted}
\begin{itemize}[label={}]
\item \begin{Texinfopreformatted}%
\ttfamily with w a{-}{-}b
\end{Texinfopreformatted}
\item \begin{Texinfopreformatted}%
\ttfamily with w c{-}{-}d
\end{Texinfopreformatted}
\end{itemize}
\begin{Texinfopreformatted}%
\ttfamily 
\end{Texinfopreformatted}
\begin{itemize}[label=\hbox{} on a line]
\item \begin{Texinfopreformatted}%
\ttfamily line w a{-}{-}b
\end{Texinfopreformatted}
\item \begin{Texinfopreformatted}%
\ttfamily line with w c{-}{-}d
\end{Texinfopreformatted}
\end{itemize}
\begin{Texinfopreformatted}%
\ttfamily 
\end{Texinfopreformatted}
\begin{enumerate}[start=1]
\item \begin{Texinfopreformatted}%
\ttfamily e{-}{-}numerate
\end{Texinfopreformatted}
\end{enumerate}
\begin{Texinfopreformatted}%
\ttfamily 
\end{Texinfopreformatted}
\begin{enumerate}[start=3]
\item \begin{Texinfopreformatted}%
\ttfamily first third
\end{Texinfopreformatted}
\item \begin{Texinfopreformatted}%
\ttfamily second third
\end{Texinfopreformatted}
\end{enumerate}
\begin{Texinfopreformatted}%
\ttfamily 
\end{Texinfopreformatted}
\begin{enumerate}[label=\alph*.]
\item \begin{Texinfopreformatted}%
\ttfamily e{-}{-}numerate
\end{Texinfopreformatted}
\end{enumerate}
\begin{Texinfopreformatted}%
\ttfamily 
\end{Texinfopreformatted}
\begin{enumerate}[label=\alph*.,start=3]
\item \begin{Texinfopreformatted}%
\ttfamily first c
\end{Texinfopreformatted}
\item \begin{Texinfopreformatted}%
\ttfamily second c
\end{Texinfopreformatted}
\end{enumerate}
\begin{Texinfopreformatted}%
\ttfamily 
\end{Texinfopreformatted}
\begin{tabular}{m{0.4\textwidth} m{0.6\textwidth}}%
\begin{Texinfopreformatted}%
\ttfamily mu{-}{-}ltitable headitem \end{Texinfopreformatted}&
\begin{Texinfopreformatted}%
\ttfamily another tab
\end{Texinfopreformatted}\\
\begin{Texinfopreformatted}%
\ttfamily mu{-}{-}ltitable item \end{Texinfopreformatted}&
\begin{Texinfopreformatted}%
\ttfamily multitable tab
\end{Texinfopreformatted}\\
\begin{Texinfopreformatted}%
\ttfamily mu{-}{-}ltitable item 2 \end{Texinfopreformatted}&
\begin{Texinfopreformatted}%
\ttfamily multitable tab 2
\index[cp]{index entry within multitable@index entry within multitable}%
\end{Texinfopreformatted}\\
\begin{Texinfopreformatted}%
\ttfamily lone mu{-}{-}ltitable item
\end{Texinfopreformatted}&\\
\end{tabular}%
\begin{Texinfopreformatted}%
\ttfamily 
\end{Texinfopreformatted}
\begin{tabular}{m{0.4\textwidth} m{0.6\textwidth}}%
\begin{Texinfopreformatted}%
\ttfamily truc \end{Texinfopreformatted}&
\begin{Texinfopreformatted}%
\ttfamily bidule
\end{Texinfopreformatted}\\
\end{tabular}%
\begin{Texinfopreformatted}%
\ttfamily 
\end{Texinfopreformatted}
\begin{Texinfoindented}
\begin{Texinfopreformatted}%
\ttfamily e{-}{-}xample  some
\   text
\end{Texinfopreformatted}
\end{Texinfoindented}
\begin{Texinfopreformatted}%
\ttfamily 
\end{Texinfopreformatted}
\begin{Texinfoindented}
\begin{Texinfopreformatted}%
\ttfamily example one arg
\end{Texinfopreformatted}
\end{Texinfoindented}
\begin{Texinfopreformatted}%
\ttfamily 
\end{Texinfopreformatted}
\begin{Texinfoindented}
\begin{Texinfopreformatted}%
\ttfamily example two args
\end{Texinfopreformatted}
\end{Texinfoindented}
\begin{Texinfopreformatted}%
\ttfamily 
\end{Texinfopreformatted}
\begin{Texinfoindented}
\begin{Texinfopreformatted}%
\ttfamily example three args
\end{Texinfopreformatted}
\end{Texinfoindented}
\begin{Texinfopreformatted}%
\ttfamily 
\end{Texinfopreformatted}
\begin{Texinfoindented}
\begin{Texinfopreformatted}%
\ttfamily example four args
\end{Texinfopreformatted}
\end{Texinfoindented}
\begin{Texinfopreformatted}%
\ttfamily 
\end{Texinfopreformatted}
\begin{Texinfoindented}
\begin{Texinfopreformatted}%
\ttfamily example five args
\end{Texinfopreformatted}
\end{Texinfoindented}
\begin{Texinfopreformatted}%
\ttfamily 
\end{Texinfopreformatted}
\begin{Texinfoindented}
\begin{Texinfopreformatted}%
\ttfamily The something \'{e}\ \TeX{}\ is here.
\end{Texinfopreformatted}
\end{Texinfoindented}
\begin{Texinfopreformatted}%
\ttfamily 
\end{Texinfopreformatted}
\begin{Texinfoindented}
\begin{Texinfopreformatted}%
\ttfamily A @\ at the end of the @example line.
\end{Texinfopreformatted}
\end{Texinfoindented}
\begin{Texinfopreformatted}%
\ttfamily 
\end{Texinfopreformatted}
\begin{Texinfoindented}
\begin{Texinfopreformatted}%
\ttfamily example with empty args
\end{Texinfopreformatted}
\end{Texinfoindented}
\begin{Texinfopreformatted}%
\ttfamily 
\end{Texinfopreformatted}
\begin{Texinfoindented}
\begin{Texinfopreformatted}%
\ttfamily example with empty and non empty args mix
\end{Texinfopreformatted}
\end{Texinfoindented}
\begin{Texinfopreformatted}%
\ttfamily 
\end{Texinfopreformatted}
\begin{Texinfoindented}
\begin{Texinfopreformatted}%
\ttfamily Exam{-}{-}{-}ple

\end{Texinfopreformatted}
\leavevmode{}\\
\hbox{\kern -\leftmargin}%
Other li---ne
\\
\begin{Texinfopreformatted}%
\ttfamily not exdented
\end{Texinfopreformatted}
\end{Texinfoindented}
\begin{Texinfopreformatted}%
\ttfamily 
\end{Texinfopreformatted}
\begin{Texinfoindented}
\leavevmode{}\\
\hbox{\kern -\leftmargin}%
exdented  and dash --- in example
\\
\begin{Texinfopreformatted}%
\ttfamily Not exdented one
\end{Texinfopreformatted}
\leavevmode{}\\
\hbox{\kern -\leftmargin}%
exdented two
\\
\begin{Texinfopreformatted}%
\ttfamily Not exdented two
\end{Texinfopreformatted}
\end{Texinfoindented}
\begin{Texinfopreformatted}%
\ttfamily 
\end{Texinfopreformatted}
\begin{Texinfoindented}
\begin{Texinfopreformatted}%
\ttfamily Example   Hoho.
\end{Texinfopreformatted}
\begin{Texinfoindented}
\begin{Texinfopreformatted}%
\ttfamily Nested Other line
\end{Texinfopreformatted}
\leavevmode{}\\
\hbox{\kern -\leftmargin}%
exdented nested other line
\\
\end{Texinfoindented}
\end{Texinfoindented}
\begin{Texinfopreformatted}%
\ttfamily 
\end{Texinfopreformatted}
\begin{Texinfopreformatted}%
\ttfamily \footnotesize s{-}{-}mallexample
\end{Texinfopreformatted}
\begin{Texinfopreformatted}%
\ttfamily 
\texttt{@noindent}\ after smallexample.
\end{Texinfopreformatted}
\begin{Texinfopreformatted}%
\ttfamily \footnotesize \$ wget 'http://savannah.gnu.org/cgi-bin/viewcvs/config/config/config.guess?rev=HEAD\&content-type=text/plain'
\$ wget 'http://savannah.gnu.org/cgi-bin/viewcvs/config/config/config.sub?rev=HEAD\&content-type=text/plain'
\end{Texinfopreformatted}
\begin{Texinfopreformatted}%
\ttfamily \noindent{}Less recent versions are also present.

\end{Texinfopreformatted}
\begin{Texinfoindented}
\begin{Texinfopreformatted}%
d--isplay
\end{Texinfopreformatted}
\end{Texinfoindented}
\begin{Texinfopreformatted}%
\ttfamily 
\end{Texinfopreformatted}
\begin{Texinfopreformatted}%
\footnotesize s--malldisplay
\end{Texinfopreformatted}
\begin{Texinfopreformatted}%
\ttfamily 
\end{Texinfopreformatted}
\begin{Texinfoindented}
\begin{Texinfopreformatted}%
\ttfamily l{-}{-}isp
\end{Texinfopreformatted}
\end{Texinfoindented}
\begin{Texinfopreformatted}%
\ttfamily 
\end{Texinfopreformatted}
\begin{Texinfopreformatted}%
\ttfamily \footnotesize s{-}{-}malllisp
\end{Texinfopreformatted}
\begin{Texinfopreformatted}%
\ttfamily 
\end{Texinfopreformatted}
\begin{Texinfopreformatted}%
f--ormat
\end{Texinfopreformatted}
\begin{Texinfopreformatted}%
\ttfamily 
\end{Texinfopreformatted}
\begin{Texinfopreformatted}%
\footnotesize s--mallformat
\end{Texinfopreformatted}
\begin{Texinfopreformatted}%
\ttfamily 
\end{Texinfopreformatted}

\noindent\begin{tabularx}{\linewidth}{@{}Xr}
\rightskip=5em plus 1 fill \hangindent=2em \hyphenpenalty=10000
\texttt{d{-}{-}effn\_name \EmbracOn{}\textnormal{\textsl{a--rguments...}}\EmbracOff{}}& [c--ategory]
\end{tabularx}

\index[fn]{d--effn\_name@\texttt{d{-}{-}effn\_name}}%
\begin{quote}
\unskip{\parskip=0pt\noindent}%
\begin{Texinfopreformatted}%
\ttfamily d{-}{-}effn
\end{Texinfopreformatted}
\end{quote}
\begin{Texinfopreformatted}%
\ttfamily 
\end{Texinfopreformatted}

\noindent\begin{tabularx}{\linewidth}{@{}Xr}
\rightskip=5em plus 1 fill \hangindent=2em \hyphenpenalty=10000
\texttt{de{-}{-}ffn\_name \EmbracOn{}\textnormal{\textsl{ar--guments    more args   even more so}}\EmbracOff{}}& [cate--gory]
\end{tabularx}

\index[fn]{de--ffn\_name@\texttt{de{-}{-}ffn\_name}}%
\begin{quote}
\unskip{\parskip=0pt\noindent}%
\begin{Texinfopreformatted}%
\ttfamily def{-}{-}fn
\end{Texinfopreformatted}
\end{quote}
\begin{Texinfopreformatted}%
\ttfamily 
\end{Texinfopreformatted}

\noindent\begin{tabularx}{\linewidth}{@{}Xr}
\rightskip=5em plus 1 fill \hangindent=2em \hyphenpenalty=10000
\texttt{\Texinfocommandstyletextvar{i} \EmbracOn{}\textnormal{\textsl{a g}}\EmbracOff{}}& [fset]
\end{tabularx}

\index[fn]{i@\texttt{\Texinfocommandstyletextvar{i}}}%
\begin{Texinfopreformatted}%
\ttfamily \index[cp]{index entry within deffn@index entry within deffn}%
\end{Texinfopreformatted}

\noindent\begin{tabularx}{\linewidth}{@{}Xr}
\rightskip=5em plus 1 fill \hangindent=2em \hyphenpenalty=10000
\texttt{truc \EmbracOn{}\textnormal{\textsl{}}\EmbracOff{}}& [cmde]
\end{tabularx}

\index[fn]{truc@\texttt{truc}}%

\noindent\begin{tabularx}{\linewidth}{@{}Xr}
\rightskip=5em plus 1 fill \hangindent=2em \hyphenpenalty=10000
\texttt{log trap \EmbracOn{}\textnormal{\textsl{}}\EmbracOff{}}& [Command]
\end{tabularx}

\index[fn]{log trap@\texttt{log trap}}%

\noindent\begin{tabularx}{\linewidth}{@{}Xr}
\rightskip=5em plus 1 fill \hangindent=2em \hyphenpenalty=10000
\texttt{log trap1 \EmbracOn{}\textnormal{\textsl{}}\EmbracOff{}}& [Command]
\end{tabularx}

\index[fn]{log trap1@\texttt{log trap1}}%

\noindent\begin{tabularx}{\linewidth}{@{}Xr}
\rightskip=5em plus 1 fill \hangindent=2em \hyphenpenalty=10000
\texttt{log trap2 \EmbracOn{}\textnormal{\textsl{}}\EmbracOff{}}& [Command]
\end{tabularx}

\index[fn]{log trap2@\texttt{log trap2}}%

\noindent\begin{tabularx}{\linewidth}{@{}Xr}
\rightskip=5em plus 1 fill \hangindent=2em \hyphenpenalty=10000
\texttt{\textbf{id ule} \EmbracOn{}\textnormal{\textsl{truc}}\EmbracOff{}}& [cmde]
\end{tabularx}

\index[fn]{id ule@\texttt{\textbf{id ule}}}%

\noindent\begin{tabularx}{\linewidth}{@{}Xr}
\rightskip=5em plus 1 fill \hangindent=2em \hyphenpenalty=10000
\texttt{\textbf{id `\texttt{i}'\ ule} \EmbracOn{}\textnormal{\textsl{truc}}\EmbracOff{}}& [cmde2]
\end{tabularx}

\index[fn]{id i ule@\texttt{\textbf{id `\texttt{i}'\ ule}}}%

\noindent\begin{tabularx}{\linewidth}{@{}Xr}
\rightskip=5em plus 1 fill \hangindent=2em \hyphenpenalty=10000
\texttt{}& []
\end{tabularx}


\noindent\begin{tabularx}{\linewidth}{@{}Xr}
\rightskip=5em plus 1 fill \hangindent=2em \hyphenpenalty=10000
\texttt{machin}& []
\end{tabularx}

\index[fn]{machin@\texttt{machin}}%

\noindent\begin{tabularx}{\linewidth}{@{}Xr}
\rightskip=5em plus 1 fill \hangindent=2em \hyphenpenalty=10000
\texttt{bidule machin}& []
\end{tabularx}

\index[fn]{bidule machin@\texttt{bidule machin}}%

\noindent\begin{tabularx}{\linewidth}{@{}Xr}
\rightskip=5em plus 1 fill \hangindent=2em \hyphenpenalty=10000
\texttt{machin}& [truc]
\end{tabularx}

\index[fn]{machin@\texttt{machin}}%

\noindent\begin{tabularx}{\linewidth}{@{}Xr}
\rightskip=5em plus 1 fill \hangindent=2em \hyphenpenalty=10000
\texttt{}& [truc]
\end{tabularx}


\noindent\begin{tabularx}{\linewidth}{@{}Xr}
\rightskip=5em plus 1 fill \hangindent=2em \hyphenpenalty=10000
\texttt{followed \EmbracOn{}\textnormal{\textsl{by a comment}}\EmbracOff{}}& [truc]
\end{tabularx}

\index[fn]{followed@\texttt{followed}}%
\begin{Texinfopreformatted}%
\ttfamily \end{Texinfopreformatted}

\noindent\begin{tabularx}{\linewidth}{@{}Xr}
\rightskip=5em plus 1 fill \hangindent=2em \hyphenpenalty=10000
\texttt{}& []
\end{tabularx}


\noindent\begin{tabularx}{\linewidth}{@{}Xr}
\rightskip=5em plus 1 fill \hangindent=2em \hyphenpenalty=10000
\texttt{a \EmbracOn{}\textnormal{\textsl{b c d e \textbf{f g} h i}}\EmbracOff{}}& [truc]
\end{tabularx}

\index[fn]{a@\texttt{a}}%

\noindent\begin{tabularx}{\linewidth}{@{}Xr}
\rightskip=5em plus 1 fill \hangindent=2em \hyphenpenalty=10000
\texttt{deffnx \EmbracOn{}\textnormal{\textsl{before end deffn}}\EmbracOff{}}& [truc]
\end{tabularx}

\index[fn]{deffnx@\texttt{deffnx}}%
\begin{Texinfopreformatted}%
\ttfamily 

\end{Texinfopreformatted}

\noindent\begin{tabularx}{\linewidth}{@{}Xr}
\rightskip=5em plus 1 fill \hangindent=2em \hyphenpenalty=10000
\texttt{deffn}& [empty]
\end{tabularx}

\index[fn]{deffn@\texttt{deffn}}%
\begin{Texinfopreformatted}%
\ttfamily 
\end{Texinfopreformatted}

\noindent\begin{tabularx}{\linewidth}{@{}Xr}
\rightskip=5em plus 1 fill \hangindent=2em \hyphenpenalty=10000
\texttt{deffn \EmbracOn{}\textnormal{\textsl{with deffnx}}\EmbracOff{}}& [empty]
\end{tabularx}

\index[fn]{deffn@\texttt{deffn}}%
\begin{Texinfopreformatted}%
\ttfamily \end{Texinfopreformatted}

\noindent\begin{tabularx}{\linewidth}{@{}Xr}
\rightskip=5em plus 1 fill \hangindent=2em \hyphenpenalty=10000
\texttt{deffnx}& [empty]
\end{tabularx}

\index[fn]{deffnx@\texttt{deffnx}}%
\begin{Texinfopreformatted}%
\ttfamily 
\end{Texinfopreformatted}

\noindent\begin{tabularx}{\linewidth}{@{}Xr}
\rightskip=5em plus 1 fill \hangindent=2em \hyphenpenalty=10000
\texttt{\Texinfocommandstyletextvar{i} \EmbracOn{}\textnormal{\textsl{a g}}\EmbracOff{}}& [fset]
\end{tabularx}

\index[fn]{i@\texttt{\Texinfocommandstyletextvar{i}}}%

\noindent\begin{tabularx}{\linewidth}{@{}Xr}
\rightskip=5em plus 1 fill \hangindent=2em \hyphenpenalty=10000
\texttt{truc \EmbracOn{}\textnormal{\textsl{}}\EmbracOff{}}& [cmde]
\end{tabularx}

\index[fn]{truc@\texttt{truc}}%
\begin{quote}
\unskip{\parskip=0pt\noindent}%
\begin{Texinfopreformatted}%
\ttfamily text in def item for second def item
\end{Texinfopreformatted}
\end{quote}
\begin{Texinfopreformatted}%
\ttfamily 

\end{Texinfopreformatted}

\noindent\begin{tabularx}{\linewidth}{@{}Xr}
\rightskip=5em plus 1 fill \hangindent=2em \hyphenpenalty=10000
\texttt{d{-}{-}efvr\_name}& [c--ategory]
\end{tabularx}

\index[cp]{d--efvr\_name@\texttt{d{-}{-}efvr\_name}}%
\begin{quote}
\unskip{\parskip=0pt\noindent}%
\begin{Texinfopreformatted}%
\ttfamily d{-}{-}efvr
\end{Texinfopreformatted}
\end{quote}
\begin{Texinfopreformatted}%
\ttfamily 
\end{Texinfopreformatted}

\noindent\begin{tabularx}{\linewidth}{@{}Xr}
\rightskip=5em plus 1 fill \hangindent=2em \hyphenpenalty=10000
\texttt{n{-}{-}ame \EmbracOn{}\textnormal{\textsl{a--rguments...}}\EmbracOff{}}& [c--ategory]
\end{tabularx}

\index[fn]{n--ame@\texttt{n{-}{-}ame}}%
\begin{quote}
\unskip{\parskip=0pt\noindent}%
\begin{Texinfopreformatted}%
\ttfamily d{-}{-}effn
\end{Texinfopreformatted}
\end{quote}
\begin{Texinfopreformatted}%
\ttfamily 
\end{Texinfopreformatted}

\noindent\begin{tabularx}{\linewidth}{@{}Xr}
\rightskip=5em plus 1 fill \hangindent=2em \hyphenpenalty=10000
\texttt{n{-}{-}ame}& [c--ategory]
\end{tabularx}

\index[fn]{n--ame@\texttt{n{-}{-}ame}}%
\begin{quote}
\unskip{\parskip=0pt\noindent}%
\begin{Texinfopreformatted}%
\ttfamily d{-}{-}effn no arg
\end{Texinfopreformatted}
\end{quote}
\begin{Texinfopreformatted}%
\ttfamily 
\end{Texinfopreformatted}

\noindent\begin{tabularx}{\linewidth}{@{}Xr}
\rightskip=5em plus 1 fill \hangindent=2em \hyphenpenalty=10000
\texttt{t{-}{-}ype d{-}{-}eftypefn\_name a{-}{-}rguments...}& [c--ategory]
\end{tabularx}

\index[fn]{d--eftypefn\_name@\texttt{d{-}{-}eftypefn\_name}}%
\begin{quote}
\unskip{\parskip=0pt\noindent}%
\begin{Texinfopreformatted}%
\ttfamily d{-}{-}eftypefn
\end{Texinfopreformatted}
\end{quote}
\begin{Texinfopreformatted}%
\ttfamily 
\end{Texinfopreformatted}

\noindent\begin{tabularx}{\linewidth}{@{}Xr}
\rightskip=5em plus 1 fill \hangindent=2em \hyphenpenalty=10000
\texttt{t{-}{-}ype d{-}{-}eftypefn\_name}& [c--ategory]
\end{tabularx}

\index[fn]{d--eftypefn\_name@\texttt{d{-}{-}eftypefn\_name}}%
\begin{quote}
\unskip{\parskip=0pt\noindent}%
\begin{Texinfopreformatted}%
\ttfamily d{-}{-}eftypefn no arg
\end{Texinfopreformatted}
\end{quote}
\begin{Texinfopreformatted}%
\ttfamily 
\end{Texinfopreformatted}

\noindent\begin{tabularx}{\linewidth}{@{}Xr}
\rightskip=5em plus 1 fill \hangindent=2em \hyphenpenalty=10000
\texttt{t{-}{-}ype d{-}{-}eftypeop\_name a{-}{-}rguments...}& [c--ategory on \texttt{c{-}{-}lass}]
\end{tabularx}

\index[fn]{d--eftypeop\_name on c--lass@\texttt{d{-}{-}eftypeop\_name\ on c{-}{-}lass}}%
\begin{quote}
\unskip{\parskip=0pt\noindent}%
\begin{Texinfopreformatted}%
\ttfamily d{-}{-}eftypeop
\end{Texinfopreformatted}
\end{quote}
\begin{Texinfopreformatted}%
\ttfamily 
\end{Texinfopreformatted}

\noindent\begin{tabularx}{\linewidth}{@{}Xr}
\rightskip=5em plus 1 fill \hangindent=2em \hyphenpenalty=10000
\texttt{t{-}{-}ype d{-}{-}eftypeop\_name}& [c--ategory on \texttt{c{-}{-}lass}]
\end{tabularx}

\index[fn]{d--eftypeop\_name on c--lass@\texttt{d{-}{-}eftypeop\_name\ on c{-}{-}lass}}%
\begin{quote}
\unskip{\parskip=0pt\noindent}%
\begin{Texinfopreformatted}%
\ttfamily d{-}{-}eftypeop no arg
\end{Texinfopreformatted}
\end{quote}
\begin{Texinfopreformatted}%
\ttfamily 
\end{Texinfopreformatted}

\noindent\begin{tabularx}{\linewidth}{@{}Xr}
\rightskip=5em plus 1 fill \hangindent=2em \hyphenpenalty=10000
\texttt{t{-}{-}ype d{-}{-}eftypevr\_name}& [c--ategory]
\end{tabularx}

\index[cp]{d--eftypevr\_name@\texttt{d{-}{-}eftypevr\_name}}%
\begin{quote}
\unskip{\parskip=0pt\noindent}%
\begin{Texinfopreformatted}%
\ttfamily d{-}{-}eftypevr
\end{Texinfopreformatted}
\end{quote}
\begin{Texinfopreformatted}%
\ttfamily 
\end{Texinfopreformatted}

\noindent\begin{tabularx}{\linewidth}{@{}Xr}
\rightskip=5em plus 1 fill \hangindent=2em \hyphenpenalty=10000
\texttt{d{-}{-}efcv\_name}& [c--ategory of \texttt{c{-}{-}lass}]
\end{tabularx}

\index[cp]{d--efcv\_name@\texttt{d{-}{-}efcv\_name}}%
\begin{quote}
\unskip{\parskip=0pt\noindent}%
\begin{Texinfopreformatted}%
\ttfamily d{-}{-}efcv
\end{Texinfopreformatted}
\end{quote}
\begin{Texinfopreformatted}%
\ttfamily 
\end{Texinfopreformatted}

\noindent\begin{tabularx}{\linewidth}{@{}Xr}
\rightskip=5em plus 1 fill \hangindent=2em \hyphenpenalty=10000
\texttt{d{-}{-}efcv\_name \EmbracOn{}\textnormal{\textsl{a--rguments...}}\EmbracOff{}}& [c--ategory of \texttt{c{-}{-}lass}]
\end{tabularx}

\index[cp]{d--efcv\_name@\texttt{d{-}{-}efcv\_name}}%
\begin{quote}
\unskip{\parskip=0pt\noindent}%
\begin{Texinfopreformatted}%
\ttfamily d{-}{-}efcv with arguments
\end{Texinfopreformatted}
\end{quote}
\begin{Texinfopreformatted}%
\ttfamily 
\end{Texinfopreformatted}

\noindent\begin{tabularx}{\linewidth}{@{}Xr}
\rightskip=5em plus 1 fill \hangindent=2em \hyphenpenalty=10000
\texttt{t{-}{-}ype d{-}{-}eftypecv\_name}& [c--ategory of \texttt{c{-}{-}lass}]
\end{tabularx}

\index[cp]{d--eftypecv\_name of c--lass@\texttt{d{-}{-}eftypecv\_name\ of c{-}{-}lass}}%
\begin{quote}
\unskip{\parskip=0pt\noindent}%
\begin{Texinfopreformatted}%
\ttfamily d{-}{-}eftypecv
\end{Texinfopreformatted}
\end{quote}
\begin{Texinfopreformatted}%
\ttfamily 
\end{Texinfopreformatted}

\noindent\begin{tabularx}{\linewidth}{@{}Xr}
\rightskip=5em plus 1 fill \hangindent=2em \hyphenpenalty=10000
\texttt{t{-}{-}ype d{-}{-}eftypecv\_name a{-}{-}rguments...}& [c--ategory of \texttt{c{-}{-}lass}]
\end{tabularx}

\index[cp]{d--eftypecv\_name of c--lass@\texttt{d{-}{-}eftypecv\_name\ of c{-}{-}lass}}%
\begin{quote}
\unskip{\parskip=0pt\noindent}%
\begin{Texinfopreformatted}%
\ttfamily d{-}{-}eftypecv with arguments
\end{Texinfopreformatted}
\end{quote}
\begin{Texinfopreformatted}%
\ttfamily 
\end{Texinfopreformatted}

\noindent\begin{tabularx}{\linewidth}{@{}Xr}
\rightskip=5em plus 1 fill \hangindent=2em \hyphenpenalty=10000
\texttt{d{-}{-}efop\_name \EmbracOn{}\textnormal{\textsl{a--rguments...}}\EmbracOff{}}& [c--ategory on \texttt{c{-}{-}lass}]
\end{tabularx}

\index[fn]{d--efop\_name on c--lass@\texttt{d{-}{-}efop\_name\ on c{-}{-}lass}}%
\begin{quote}
\unskip{\parskip=0pt\noindent}%
\begin{Texinfopreformatted}%
\ttfamily d{-}{-}efop
\end{Texinfopreformatted}
\end{quote}
\begin{Texinfopreformatted}%
\ttfamily 
\end{Texinfopreformatted}

\noindent\begin{tabularx}{\linewidth}{@{}Xr}
\rightskip=5em plus 1 fill \hangindent=2em \hyphenpenalty=10000
\texttt{d{-}{-}efop\_name}& [c--ategory on \texttt{c{-}{-}lass}]
\end{tabularx}

\index[fn]{d--efop\_name on c--lass@\texttt{d{-}{-}efop\_name\ on c{-}{-}lass}}%
\begin{quote}
\unskip{\parskip=0pt\noindent}%
\begin{Texinfopreformatted}%
\ttfamily d{-}{-}efop no arg
\end{Texinfopreformatted}
\end{quote}
\begin{Texinfopreformatted}%
\ttfamily 
\end{Texinfopreformatted}

\noindent\begin{tabularx}{\linewidth}{@{}Xr}
\rightskip=5em plus 1 fill \hangindent=2em \hyphenpenalty=10000
\texttt{d{-}{-}eftp\_name \EmbracOn{}\textnormal{\textsl{a--ttributes...}}\EmbracOff{}}& [c--ategory]
\end{tabularx}

\index[tp]{d--eftp\_name@\texttt{d{-}{-}eftp\_name}}%
\begin{quote}
\unskip{\parskip=0pt\noindent}%
\begin{Texinfopreformatted}%
\ttfamily d{-}{-}eftp
\end{Texinfopreformatted}
\end{quote}
\begin{Texinfopreformatted}%
\ttfamily 
\end{Texinfopreformatted}

\noindent\begin{tabularx}{\linewidth}{@{}Xr}
\rightskip=5em plus 1 fill \hangindent=2em \hyphenpenalty=10000
\texttt{d{-}{-}efun\_name \EmbracOn{}\textnormal{\textsl{a--rguments...}}\EmbracOff{}}& [Function]
\end{tabularx}

\index[fn]{d--efun\_name@\texttt{d{-}{-}efun\_name}}%
\begin{quote}
\unskip{\parskip=0pt\noindent}%
\begin{Texinfopreformatted}%
\ttfamily d{-}{-}efun
\end{Texinfopreformatted}
\end{quote}
\begin{Texinfopreformatted}%
\ttfamily 
\end{Texinfopreformatted}

\noindent\begin{tabularx}{\linewidth}{@{}Xr}
\rightskip=5em plus 1 fill \hangindent=2em \hyphenpenalty=10000
\texttt{d{-}{-}efmac\_name \EmbracOn{}\textnormal{\textsl{a--rguments...}}\EmbracOff{}}& [Macro]
\end{tabularx}

\index[fn]{d--efmac\_name@\texttt{d{-}{-}efmac\_name}}%
\begin{quote}
\unskip{\parskip=0pt\noindent}%
\begin{Texinfopreformatted}%
\ttfamily d{-}{-}efmac
\end{Texinfopreformatted}
\end{quote}
\begin{Texinfopreformatted}%
\ttfamily 
\end{Texinfopreformatted}

\noindent\begin{tabularx}{\linewidth}{@{}Xr}
\rightskip=5em plus 1 fill \hangindent=2em \hyphenpenalty=10000
\texttt{d{-}{-}efspec\_name \EmbracOn{}\textnormal{\textsl{a--rguments...}}\EmbracOff{}}& [Special Form]
\end{tabularx}

\index[fn]{d--efspec\_name@\texttt{d{-}{-}efspec\_name}}%
\begin{quote}
\unskip{\parskip=0pt\noindent}%
\begin{Texinfopreformatted}%
\ttfamily d{-}{-}efspec
\end{Texinfopreformatted}
\end{quote}
\begin{Texinfopreformatted}%
\ttfamily 
\end{Texinfopreformatted}

\noindent\begin{tabularx}{\linewidth}{@{}Xr}
\rightskip=5em plus 1 fill \hangindent=2em \hyphenpenalty=10000
\texttt{d{-}{-}efvar\_name}& [Variable]
\end{tabularx}

\index[cp]{d--efvar\_name@\texttt{d{-}{-}efvar\_name}}%
\begin{quote}
\unskip{\parskip=0pt\noindent}%
\begin{Texinfopreformatted}%
\ttfamily d{-}{-}efvar
\end{Texinfopreformatted}
\end{quote}
\begin{Texinfopreformatted}%
\ttfamily 
\end{Texinfopreformatted}

\noindent\begin{tabularx}{\linewidth}{@{}Xr}
\rightskip=5em plus 1 fill \hangindent=2em \hyphenpenalty=10000
\texttt{d{-}{-}efvar\_name \EmbracOn{}\textnormal{\textsl{arg--var arg--var1}}\EmbracOff{}}& [Variable]
\end{tabularx}

\index[cp]{d--efvar\_name@\texttt{d{-}{-}efvar\_name}}%
\begin{quote}
\unskip{\parskip=0pt\noindent}%
\begin{Texinfopreformatted}%
\ttfamily d{-}{-}efvar with args
\end{Texinfopreformatted}
\end{quote}
\begin{Texinfopreformatted}%
\ttfamily 
\end{Texinfopreformatted}

\noindent\begin{tabularx}{\linewidth}{@{}Xr}
\rightskip=5em plus 1 fill \hangindent=2em \hyphenpenalty=10000
\texttt{d{-}{-}efopt\_name}& [User Option]
\end{tabularx}

\index[cp]{d--efopt\_name@\texttt{d{-}{-}efopt\_name}}%
\begin{quote}
\unskip{\parskip=0pt\noindent}%
\begin{Texinfopreformatted}%
\ttfamily d{-}{-}efopt
\end{Texinfopreformatted}
\end{quote}
\begin{Texinfopreformatted}%
\ttfamily 
\end{Texinfopreformatted}

\noindent\begin{tabularx}{\linewidth}{@{}Xr}
\rightskip=5em plus 1 fill \hangindent=2em \hyphenpenalty=10000
\texttt{t{-}{-}ype d{-}{-}eftypefun\_name a{-}{-}rguments...}& [Function]
\end{tabularx}

\index[fn]{d--eftypefun\_name@\texttt{d{-}{-}eftypefun\_name}}%
\begin{quote}
\unskip{\parskip=0pt\noindent}%
\begin{Texinfopreformatted}%
\ttfamily d{-}{-}eftypefun
\end{Texinfopreformatted}
\end{quote}
\begin{Texinfopreformatted}%
\ttfamily 
\end{Texinfopreformatted}

\noindent\begin{tabularx}{\linewidth}{@{}Xr}
\rightskip=5em plus 1 fill \hangindent=2em \hyphenpenalty=10000
\texttt{t{-}{-}ype d{-}{-}eftypevar\_name}& [Variable]
\end{tabularx}

\index[cp]{d--eftypevar\_name@\texttt{d{-}{-}eftypevar\_name}}%
\begin{quote}
\unskip{\parskip=0pt\noindent}%
\begin{Texinfopreformatted}%
\ttfamily d{-}{-}eftypevar
\end{Texinfopreformatted}
\end{quote}
\begin{Texinfopreformatted}%
\ttfamily 
\end{Texinfopreformatted}

\noindent\begin{tabularx}{\linewidth}{@{}Xr}
\rightskip=5em plus 1 fill \hangindent=2em \hyphenpenalty=10000
\texttt{d{-}{-}efivar\_name}& [Instance Variable of \texttt{c{-}{-}lass}]
\end{tabularx}

\index[cp]{d--efivar\_name of c--lass@\texttt{d{-}{-}efivar\_name\ of c{-}{-}lass}}%
\begin{quote}
\unskip{\parskip=0pt\noindent}%
\begin{Texinfopreformatted}%
\ttfamily d{-}{-}efivar
\end{Texinfopreformatted}
\end{quote}
\begin{Texinfopreformatted}%
\ttfamily 
\end{Texinfopreformatted}

\noindent\begin{tabularx}{\linewidth}{@{}Xr}
\rightskip=5em plus 1 fill \hangindent=2em \hyphenpenalty=10000
\texttt{t{-}{-}ype d{-}{-}eftypeivar\_name}& [Instance Variable of \texttt{c{-}{-}lass}]
\end{tabularx}

\index[cp]{d--eftypeivar\_name of c--lass@\texttt{d{-}{-}eftypeivar\_name\ of c{-}{-}lass}}%
\begin{quote}
\unskip{\parskip=0pt\noindent}%
\begin{Texinfopreformatted}%
\ttfamily d{-}{-}eftypeivar
\end{Texinfopreformatted}
\end{quote}
\begin{Texinfopreformatted}%
\ttfamily 
\end{Texinfopreformatted}

\noindent\begin{tabularx}{\linewidth}{@{}Xr}
\rightskip=5em plus 1 fill \hangindent=2em \hyphenpenalty=10000
\texttt{d{-}{-}efmethod\_name \EmbracOn{}\textnormal{\textsl{a--rguments...}}\EmbracOff{}}& [Method on \texttt{c{-}{-}lass}]
\end{tabularx}

\index[fn]{d--efmethod\_name on c--lass@\texttt{d{-}{-}efmethod\_name\ on c{-}{-}lass}}%
\begin{quote}
\unskip{\parskip=0pt\noindent}%
\begin{Texinfopreformatted}%
\ttfamily d{-}{-}efmethod
\end{Texinfopreformatted}
\end{quote}
\begin{Texinfopreformatted}%
\ttfamily 
\end{Texinfopreformatted}

\noindent\begin{tabularx}{\linewidth}{@{}Xr}
\rightskip=5em plus 1 fill \hangindent=2em \hyphenpenalty=10000
\texttt{t{-}{-}ype d{-}{-}eftypemethod\_name a{-}{-}rguments...}& [Method on \texttt{c{-}{-}lass}]
\end{tabularx}

\index[fn]{d--eftypemethod\_name on c--lass@\texttt{d{-}{-}eftypemethod\_name\ on c{-}{-}lass}}%
\begin{quote}
\unskip{\parskip=0pt\noindent}%
\begin{Texinfopreformatted}%
\ttfamily d{-}{-}eftypemethod
\end{Texinfopreformatted}
\end{quote}
\begin{Texinfopreformatted}%
\ttfamily 

\end{Texinfopreformatted}

\noindent\begin{tabularx}{\linewidth}{@{}Xr}
\rightskip=5em plus 1 fill \hangindent=2em \hyphenpenalty=10000
\texttt{data-type2}& [Function]\\
\texttt{name2 arguments2...}\end{tabularx}

\index[fn]{name2@\texttt{name2}}%
\begin{quote}
\unskip{\parskip=0pt\noindent}%
\begin{Texinfopreformatted}%
\ttfamily aaa2
\end{Texinfopreformatted}
\end{quote}
\begin{Texinfopreformatted}%
\ttfamily 
\end{Texinfopreformatted}

\noindent\begin{tabularx}{\linewidth}{@{}Xr}
\rightskip=5em plus 1 fill \hangindent=2em \hyphenpenalty=10000
\texttt{t{-}{-}ype2}& [c--ategory2]\\
\texttt{d{-}{-}eftypefn\_name2}\end{tabularx}

\index[fn]{d--eftypefn\_name2@\texttt{d{-}{-}eftypefn\_name2}}%
\begin{quote}
\unskip{\parskip=0pt\noindent}%
\begin{Texinfopreformatted}%
\ttfamily d{-}{-}eftypefn no arg2
\end{Texinfopreformatted}
\end{quote}
\begin{Texinfopreformatted}%
\ttfamily 
\end{Texinfopreformatted}

\noindent\begin{tabularx}{\linewidth}{@{}Xr}
\rightskip=5em plus 1 fill \hangindent=2em \hyphenpenalty=10000
\texttt{t{-}{-}ype2}& [c--ategory2 on \texttt{c{-}{-}lass2}]\\
\texttt{d{-}{-}eftypeop\_name2 a{-}{-}rguments2...}\end{tabularx}

\index[fn]{d--eftypeop\_name2 on c--lass2@\texttt{d{-}{-}eftypeop\_name2\ on c{-}{-}lass2}}%
\begin{quote}
\unskip{\parskip=0pt\noindent}%
\begin{Texinfopreformatted}%
\ttfamily d{-}{-}eftypeop2
\end{Texinfopreformatted}
\end{quote}
\begin{Texinfopreformatted}%
\ttfamily 
\end{Texinfopreformatted}

\noindent\begin{tabularx}{\linewidth}{@{}Xr}
\rightskip=5em plus 1 fill \hangindent=2em \hyphenpenalty=10000
\texttt{t{-}{-}ype2}& [c--ategory2 on \texttt{c{-}{-}lass2}]\\
\texttt{d{-}{-}eftypeop\_name2}\end{tabularx}

\index[fn]{d--eftypeop\_name2 on c--lass2@\texttt{d{-}{-}eftypeop\_name2\ on c{-}{-}lass2}}%
\begin{quote}
\unskip{\parskip=0pt\noindent}%
\begin{Texinfopreformatted}%
\ttfamily d{-}{-}eftypeop no arg2
\end{Texinfopreformatted}
\end{quote}
\begin{Texinfopreformatted}%
\ttfamily 
\end{Texinfopreformatted}

\noindent\begin{tabularx}{\linewidth}{@{}Xr}
\rightskip=5em plus 1 fill \hangindent=2em \hyphenpenalty=10000
\texttt{t{-}{-}ype2 d{-}{-}eftypecv\_name2}& [c--ategory2 of \texttt{c{-}{-}lass2}]
\end{tabularx}

\index[cp]{d--eftypecv\_name2 of c--lass2@\texttt{d{-}{-}eftypecv\_name2\ of c{-}{-}lass2}}%
\begin{quote}
\unskip{\parskip=0pt\noindent}%
\begin{Texinfopreformatted}%
\ttfamily d{-}{-}eftypecv2
\end{Texinfopreformatted}
\end{quote}
\begin{Texinfopreformatted}%
\ttfamily 
\end{Texinfopreformatted}

\noindent\begin{tabularx}{\linewidth}{@{}Xr}
\rightskip=5em plus 1 fill \hangindent=2em \hyphenpenalty=10000
\texttt{t{-}{-}ype2 d{-}{-}eftypecv\_name2 a{-}{-}rguments2...}& [c--ategory2 of \texttt{c{-}{-}lass2}]
\end{tabularx}

\index[cp]{d--eftypecv\_name2 of c--lass2@\texttt{d{-}{-}eftypecv\_name2\ of c{-}{-}lass2}}%
\begin{quote}
\unskip{\parskip=0pt\noindent}%
\begin{Texinfopreformatted}%
\ttfamily d{-}{-}eftypecv with arguments2
\end{Texinfopreformatted}
\end{quote}
\begin{Texinfopreformatted}%
\ttfamily 
\end{Texinfopreformatted}

\noindent\begin{tabularx}{\linewidth}{@{}Xr}
\rightskip=5em plus 1 fill \hangindent=2em \hyphenpenalty=10000
\texttt{arg2}& [fun2]
\end{tabularx}

\index[fn]{arg2@\texttt{arg2}}%
\begin{quote}
\unskip{\parskip=0pt\noindent}%
\begin{Texinfopreformatted}%
\ttfamily fff2
\end{Texinfopreformatted}
\end{quote}
\begin{Texinfopreformatted}%
\ttfamily 

\texttt{@xref\{c{-}{-}{-}hapter@@,\ cross r{-}{-}{-}ef name@@,\ t{-}{-}{-}itle@@,\ file n{-}{-}{-}ame@@,\ ma{-}{-}{-}nual@@\}}\ See Section ``t{-}{-}{-}itle@'' in \textsl{ma{-}{-}{-}nual@}.
\texttt{@ref\{chapter,\ cross ref name,\ title,\ file name,\ manual\}}\ Section ``title'' in \textsl{manual}
\texttt{@pxref\{chapter,\ cross ref name,\ title,\ file name,\ manual\}}\ see Section ``title'' in \textsl{manual}
\texttt{@inforef\{chapter,\ cross ref name,\ file name\}}\ Section ``chapter'' in \texttt{file name}

\texttt{@ref\{chapter\}}\ \hyperref[anchor:chapter]{\chaptername~\ref*{anchor:chapter} [chapter], page~\pageref*{anchor:chapter}}
\texttt{@xref\{chapter\}}\ See \hyperref[anchor:chapter]{\chaptername~\ref*{anchor:chapter} [chapter], page~\pageref*{anchor:chapter}}.
\texttt{@pxref\{chapter\}}\ see \hyperref[anchor:chapter]{\chaptername~\ref*{anchor:chapter} [chapter], page~\pageref*{anchor:chapter}}
\texttt{@ref\{s{-}{-}ect@comma\{\}ion\}}\ \hyperref[anchor:s_002d_002dect_002cion]{Section~\ref*{anchor:s_002d_002dect_002cion} [s{-}{-}ect,ion], page~\pageref*{anchor:s_002d_002dect_002cion}}

\texttt{@ref\{s{-}{-}ect@comma\{\}ion,\ a @comma\{\}\ in cross
ref,\ a comma@comma\{\}\ in title,\ a comma@comma\{\}\ in file,\ a @comma\{\}\ in manual name \}}
Section ``a comma,\ in title'' in \textsl{a ,\ in manual name}

\texttt{@ref\{chapter,cross ref name\}}\ \hyperref[anchor:chapter]{\chaptername~\ref*{anchor:chapter} [chapter], page~\pageref*{anchor:chapter}}
\texttt{@ref\{chapter{,}{,}title\}}\ \hyperref[anchor:chapter]{\chaptername~\ref*{anchor:chapter} [title], page~\pageref*{anchor:chapter}}
\texttt{@ref\{chapter{,}{,},file name\}}\ Section ``chapter'' in \texttt{file name}
\texttt{@ref\{chapter{,}{,}{,}{,}manual\}}\ Section ``chapter'' in \textsl{manual}
\texttt{@ref\{chapter,cross ref name,title,\}}\ \hyperref[anchor:chapter]{\chaptername~\ref*{anchor:chapter} [title], page~\pageref*{anchor:chapter}}
\texttt{@ref\{chapter,cross ref name{,}{,}file name\}}\ Section ``chapter'' in \texttt{file name}
\texttt{@ref\{chapter,cross ref name{,}{,},manual\}}\ Section ``chapter'' in \textsl{manual}
\texttt{@ref\{chapter,cross ref name,title,file name\}}\ Section ``title'' in \texttt{file name}
\texttt{@ref\{chapter,cross ref name,title{,}{,}manual\}}\ Section ``title'' in \textsl{manual}
\texttt{@ref\{chapter,cross ref name,title,\ file name,\ manual\}}\ Section ``title'' in \textsl{manual}
\texttt{@ref\{chapter{,}{,}title,file name\}}\ Section ``title'' in \texttt{file name}
\texttt{@ref\{chapter{,}{,}title{,}{,}manual\}}\ Section ``title'' in \textsl{manual}
\texttt{@ref\{chapter{,}{,}title,\ file name,\ manual\}}\ Section ``title'' in \textsl{manual}
\texttt{@ref\{chapter{,}{,},file name,manual\}}\ Section ``chapter'' in \textsl{manual}


\texttt{@ref\{(pman)anode,cross ref name\}}\ (pman)anode
\texttt{@ref\{(pman)anode{,}{,}title\}}\ title
\texttt{@ref\{(pman)anode{,}{,},file name\}}\ Section ``(pman)anode'' in \texttt{file name}
\texttt{@ref\{(pman)anode{,}{,}{,}{,}manual\}}\ Section ``(pman)anode'' in \textsl{manual}
\texttt{@ref\{(pman)anode,cross ref name,title,\}}\ title
\texttt{@ref\{(pman)anode,cross ref name{,}{,}file name\}}\ Section ``(pman)anode'' in \texttt{file name}
\texttt{@ref\{(pman)anode,cross ref name{,}{,},manual\}}\ Section ``(pman)anode'' in \textsl{manual}
\texttt{@ref\{(pman)anode,cross ref name,title,file name\}}\ Section ``title'' in \texttt{file name}
\texttt{@ref\{(pman)anode,cross ref name,title{,}{,}manual\}}\ Section ``title'' in \textsl{manual}
\texttt{@ref\{(pman)anode,cross ref name,title,\ file name,\ manual\}}\ Section ``title'' in \textsl{manual}
\texttt{@ref\{(pman)anode{,}{,}title,file name\}}\ Section ``title'' in \texttt{file name}
\texttt{@ref\{(pman)anode{,}{,}title{,}{,}manual\}}\ Section ``title'' in \textsl{manual}
\texttt{@ref\{(pman)anode{,}{,}title,\ file name,\ manual\}}\ Section ``title'' in \textsl{manual}
\texttt{@ref\{(pman)anode{,}{,},file name,manual\}}\ Section ``(pman)anode'' in \textsl{manual}


\texttt{@inforef\{chapter,\ cross ref name,\ file name\}}\ Section ``chapter'' in \texttt{file name}
\texttt{@inforef\{chapter\}}\ chapter
\texttt{@inforef\{chapter,\ cross ref name\}}\ chapter
\texttt{@inforef\{chapter{,}{,}file name\}}\ Section ``chapter'' in \texttt{file name}
\texttt{@inforef\{node,\ cross ref name,\ file name\}}\ Section ``node'' in \texttt{file name}
\texttt{@inforef\{node\}}\ node
\texttt{@inforef\{node,\ cross ref name\}}\ node
\texttt{@inforef\{node{,}{,}file name\}}\ Section ``node'' in \texttt{file name}
\texttt{@inforef\{chapter,\ cross ref name,\ file name,\ spurious arg\}}\ Section ``chapter'' in \texttt{file name,\ spurious arg}

\texttt{@inforef\{s{-}{-}ect@comma\{\}ion,\ a @comma\{\}\ in cross
ref,\ a comma@comma\{\}\ in file\}}
Section ``s{-}{-}ect,ion'' in \texttt{a comma,\ in file}

`\texttt{\hyperref[anchor:chapter]{\chaptername~\ref*{anchor:chapter} [chapter], page~\pageref*{anchor:chapter}}}'.

Section ``title with uref2 \href{href://http/myhost.com/index2.html}{uref2 (\nolinkurl{href://http/myhost.com/index2.html})}'' in \textsl{printed manual with uref4 \href{href://http/myhost.com/index4.html}{uref4 (\nolinkurl{href://http/myhost.com/index4.html})}}
\hyperref[anchor:chapter]{\chaptername~\ref*{anchor:chapter} [title with uref2 \href{href://http/myhost.com/index2.html}{uref2 (\nolinkurl{href://http/myhost.com/index2.html})}], page~\pageref*{anchor:chapter}}

\end{Texinfopreformatted}
\begin{description}
\item[] \begin{Texinfopreformatted}%
\ttfamily \end{Texinfopreformatted}
\item[{\parbox[b]{\linewidth}{%
\textbf{a--strong}}}]
\begin{Texinfopreformatted}%
\ttfamily l{-}{-}ine
\end{Texinfopreformatted}
\end{description}
\begin{Texinfopreformatted}%
\ttfamily 
\end{Texinfopreformatted}
\begin{description}
\item[{\parbox[b]{\linewidth}{%
a--asis\\
\index[cp]{a--asis@\texttt{a{-}{-}asis}}%
b
\index[cp]{b@\texttt{b}}%
}}]
\begin{Texinfopreformatted}%
\ttfamily l{-}{-}ine
\end{Texinfopreformatted}
\end{description}
\begin{Texinfopreformatted}%
\ttfamily 
\end{Texinfopreformatted}
\begin{description}
\item[{\parbox[b]{\linewidth}{%
\emph{a}\\
\index[fn]{a@\texttt{a}}%
\index[cp]{index entry between item and itemx@index entry between item and itemx}%
\emph{b}
\index[fn]{b@\texttt{b}}%
}}]
\begin{Texinfopreformatted}%
\ttfamily l{-}{-}ine
\end{Texinfopreformatted}
\end{description}
\begin{Texinfopreformatted}%
\ttfamily 
\end{Texinfopreformatted}
\begin{description}
\item[] \begin{Texinfopreformatted}%
\ttfamily Title
\end{Texinfopreformatted}
\item[{\parbox[b]{\linewidth}{%
\texttt{a{-}{-}code}}}]
\begin{Texinfopreformatted}%
\ttfamily Value{-}{-}table code
\end{Texinfopreformatted}
\end{description}
\begin{Texinfopreformatted}%
\ttfamily 
\end{Texinfopreformatted}
\begin{description}
\item[] \begin{Texinfopreformatted}%
\ttfamily Title
\end{Texinfopreformatted}
\item[{\parbox[b]{\linewidth}{%
\Texinfotablestylesamp{a{-}{-}samp}\\
\Texinfotablestylesamp{a2{-}{-}samp}}}]
\begin{Texinfopreformatted}%
\ttfamily Value{-}{-}table samp
\end{Texinfopreformatted}
\end{description}
\begin{Texinfopreformatted}%
\ttfamily 
\end{Texinfopreformatted}
\begin{mdframed}[style=Texinfocartouche]
\begin{Texinfopreformatted}%
\ttfamily c{-}{-}artouche
\end{Texinfopreformatted}
\end{mdframed}
\begin{Texinfopreformatted}%
\ttfamily 
\end{Texinfopreformatted}
\begin{flushleft}
\begin{Texinfopreformatted}%
\begin{Texinfopreformatted}%
\ttfamily f{-}{-}lushleft
more text
\end{Texinfopreformatted}
\end{Texinfopreformatted}
\end{flushleft}
\begin{Texinfopreformatted}%
\ttfamily 
\end{Texinfopreformatted}
\begin{flushright}
\begin{Texinfopreformatted}%
\begin{Texinfopreformatted}%
\ttfamily f{-}{-}lushright
more text
\end{Texinfopreformatted}
\end{Texinfopreformatted}
\end{flushright}
\begin{Texinfopreformatted}%
\ttfamily 
\end{Texinfopreformatted}
\begin{center}
ce--ntered line
\end{center}
\begin{Texinfopreformatted}%
\ttfamily 
\end{Texinfopreformatted}
\begin{flushleft}
\begin{Texinfopreformatted}%
\ttfamily r{-}{-}raggedright
more text
\end{Texinfopreformatted}
\end{flushleft}
\begin{Texinfopreformatted}%
\ttfamily 
\end{Texinfopreformatted}
\begin{verbatim}
\input texinfo @c -*-texinfo-*-

@c this file is used in tests in @verbatiminclude but not converted

@setfilename simplest.info

@node Top

This is a very simple texi manual @  <>.

@bye
\end{verbatim}
\begin{Texinfopreformatted}%
\ttfamily 
\end{Texinfopreformatted}
\begin{verbatim}
in verbatim ''
\end{verbatim}
\begin{Texinfopreformatted}%
\ttfamily 




$\frac{a < b \texttt{tex \hbox{ code }}}{b}$ ``

\end{Texinfopreformatted}
\Texinfonopagebreakheading{\chapter*}{{majorheading}}
\begin{Texinfopreformatted}%
\ttfamily 
\end{Texinfopreformatted}
\Texinfonopagebreakheading{\chapter*}{{chapheading}}
\begin{Texinfopreformatted}%
\ttfamily 
\end{Texinfopreformatted}
\section*{{heading}}
\begin{Texinfopreformatted}%
\ttfamily 
\end{Texinfopreformatted}
\subsection*{{subheading}}
\begin{Texinfopreformatted}%
\ttfamily 
\end{Texinfopreformatted}
\subsubsection*{{subsubheading}}
\begin{Texinfopreformatted}%
\ttfamily 

\texttt{@acronym\{{-}{-}a,an accronym @comma\{\}\ @enddots\{\}\}}\ {-}{-}a\ (an accronym ,\ \dots{})
\texttt{@abbr\{@'E{-}{-}.\ @comma\{\}A.,\ @'Etude{-}{-}@comma\{\}\ @b\{Autonome\}\ \}}\ \'{E}{-}{-}.\@\ ,A.\@\ (\'{E}tude{-}{-},\ \textbf{Autonome})
\texttt{@abbr\{@'E{-}{-}.\ @comma\{\}A.\}}\ \'{E}{-}{-}.\@\ ,A.\@

\texttt{@math\{{-}{-}a@minus\{\}\ \{\textbackslash{}frac\{1\}\{2\}\}\}}\ $--a- {\frac{1}{2}}$




Somehow invalid use of @,:\leavevmode{}\\
@,\ \c{}\leavevmode{}\\
@,@"u \c{}\"{u}

Invalid use of @':\leavevmode{}\\
@' \'{}\leavevmode{}\\
@'@"u \'{}\"{u}

\texttt{@|}\ 

@dotless\{truc\}\ truc
@dotless\{ij\}\ ij
\texttt{@dotless\{{-}{-}a\}}\ \{-\}\{-\}a
\texttt{@dotless\{a\}}\ a

@U,\ without braces @U\{\},\ with empty arg 
@U\{z\},\ with non-hex arg U+z
@U\{FFFFFFFFFFFFFF\},\ value much too large U+FFFFFFFFFFFFFF
@U\{110000\},\ value just beyond Unicode U+110000

@TeX,\ but without brace \TeX{}
\texttt{@\#}\ \#

\texttt{@w\{{-}{-}a\}}\ \hbox{{-}{-}a}

\texttt{@image\{,1{-}{-}xt\}}\ 
\texttt{@image\{{,}{,}2{-}{-}xt\}}\ 
\texttt{@image\{{,}{,},3{-}{-}xt\}}\ 

\texttt{@image\{f-ile,aze{,}{,}a{-}{-}lt\}}\ \includegraphics[width=aze]{f-ile}
\texttt{@image\{f-ile{,}{,},alt@verb\{:jk \_" \%\@\}\}}\ \includegraphics{f-ile}

\texttt{@image\{f{-}{-}ile\}}\ \includegraphics{f--ile}
\texttt{@image\{f{-}{-}ile{,}{,},alt\}}\ \includegraphics{f--ile}
\texttt{@image\{f{-}{-}ile{,}{,}{,}{,}.e-d-xt\}}\ \includegraphics{f--ile}
\texttt{@image\{f{-}{-}ile,l{-}{-}i\}}\ \includegraphics[width=l--i]{f--ile}
\texttt{@image\{f{-}{-}ile{,}{,}l{-}{-}e\}}\ \includegraphics[height=l--e]{f--ile}
\texttt{@image\{f{-}{-}ile,aze,az,alt,.e{-}{-}xt\}}\ \includegraphics[width=aze,height=az]{f--ile}
\texttt{@image\{@file\{f{-}{-}ile\}@@@.,aze,az,alt,@file\{.file ext\}\ e{-}{-}xt@\}}\ \includegraphics[width=aze,height=az]{f--ile@.}

\texttt{@image\{f{-}{-}ile,aze,az,@verb\{:jk \_" \%@:\}\ @b\{in b "\},e{-}{-}xt\}}\ \includegraphics[width=aze,height=az]{f--ile}
\texttt{@image\{file@verb\{:jk \_" \%@:\}{,}{,},alt@verb\{:jk \_" \%@:\}\}}\ \includegraphics{filejk _" \%@}


{\bfseries author}%

\end{Texinfopreformatted}
$$
\ddot{u} \ddot{U} \tilde{n} \hat{a} \acute{e} \bar{o} \grave{i} \acute{e} \grave{\bar{E}}
\textsl{\c{\'{C}}} \textsl{\c{\'{C}}} \textsl{\H{a}} \dot{a} \mathring{a} \textsl{\t{a}}
\breve{a} \check{a}
 ? .
$$
\begin{Texinfopreformatted}%
\ttfamily 
\end{Texinfopreformatted}
$$
TeX LaTeX \star{} \mathord{\text{\aa{}}} \circledR{} ^{\circ{}} 
$$
\begin{Texinfopreformatted}%
\ttfamily 
\end{Texinfopreformatted}
$$
\mathtt{t} 
$$
\begin{Texinfopreformatted}%
\ttfamily 
\end{Texinfopreformatted}
\begin{itemize}[label=\emph{}]
\item \begin{Texinfopreformatted}%
\ttfamily e{-}{-}mph item
\end{Texinfopreformatted}
\end{itemize}
\begin{Texinfopreformatted}%
\ttfamily 
\end{Texinfopreformatted}
\begin{itemize}[label=\emph{} after emph]
\item \begin{Texinfopreformatted}%
\ttfamily e{-}{-}mph item
\end{Texinfopreformatted}
\end{itemize}
\begin{Texinfopreformatted}%
\ttfamily 
\end{Texinfopreformatted}
\begin{itemize}[label=\textbullet{} a--n itemize line]
\item \begin{Texinfopreformatted}%
\ttfamily i{-}{-}tem 1
\end{Texinfopreformatted}
\item \begin{Texinfopreformatted}%
\ttfamily i{-}{-}tem 2
\end{Texinfopreformatted}
\end{itemize}
\begin{Texinfopreformatted}%
\ttfamily 
\end{Texinfopreformatted}
\begin{itemize}[label={}]
\item \begin{Texinfopreformatted}%
\ttfamily without brace w a{-}{-}b
\end{Texinfopreformatted}
\item \begin{Texinfopreformatted}%
\ttfamily without brace w c{-}{-}d
\end{Texinfopreformatted}
\end{itemize}
\begin{Texinfopreformatted}%
\ttfamily 
\end{Texinfopreformatted}
\begin{description}
\item[] \begin{Texinfopreformatted}%
\ttfamily \end{Texinfopreformatted}
\item[{\parbox[b]{\linewidth}{%
a}}]
\begin{Texinfopreformatted}%
\ttfamily l{-}{-}ine
\end{Texinfopreformatted}
\end{description}
\begin{Texinfopreformatted}%
\ttfamily 
\end{Texinfopreformatted}
\begin{description}
\item[] \begin{Texinfopreformatted}%
\ttfamily \end{Texinfopreformatted}
\item[{\parbox[b]{\linewidth}{%
a--missing style formatting}}]
\begin{Texinfopreformatted}%
\ttfamily l{-}{-}ine
\end{Texinfopreformatted}
\end{description}
\begin{Texinfopreformatted}%
\ttfamily 
\end{Texinfopreformatted}
\begin{description}
\item[{\parbox[b]{\linewidth}{%
a\\
\index[fn]{a@\texttt{a}}%
\index[cp]{index entry between item and itemx@index entry between item and itemx}%
b
\index[fn]{b@\texttt{b}}%
}}]
\begin{Texinfopreformatted}%
\ttfamily l{-}{-}ine
\end{Texinfopreformatted}
\end{description}
\begin{Texinfopreformatted}%
\ttfamily 
\end{Texinfopreformatted}

\noindent\begin{tabularx}{\linewidth}{@{}Xr}
\rightskip=5em plus 1 fill \hangindent=2em \hyphenpenalty=10000
\texttt{}& [fun]
\end{tabularx}


\noindent\begin{tabularx}{\linewidth}{@{}Xr}
\rightskip=5em plus 1 fill \hangindent=2em \hyphenpenalty=10000
\texttt{machin \EmbracOn{}\textnormal{\textsl{bidule chose and}}\EmbracOff{}}& [truc]
\end{tabularx}

\index[fn]{machin@\texttt{machin}}%

\noindent\begin{tabularx}{\linewidth}{@{}Xr}
\rightskip=5em plus 1 fill \hangindent=2em \hyphenpenalty=10000
\texttt{machin \EmbracOn{}\textnormal{\textsl{bidule chose and  after}}\EmbracOff{}}& [truc]
\end{tabularx}

\index[fn]{machin@\texttt{machin}}%

\noindent\begin{tabularx}{\linewidth}{@{}Xr}
\rightskip=5em plus 1 fill \hangindent=2em \hyphenpenalty=10000
\texttt{machin \EmbracOn{}\textnormal{\textsl{bidule chose and }}\EmbracOff{}}& [truc]
\end{tabularx}

\index[fn]{machin@\texttt{machin}}%

\noindent\begin{tabularx}{\linewidth}{@{}Xr}
\rightskip=5em plus 1 fill \hangindent=2em \hyphenpenalty=10000
\texttt{machin \EmbracOn{}\textnormal{\textsl{bidule chose and and after}}\EmbracOff{}}& [truc]
\end{tabularx}

\index[fn]{machin@\texttt{machin}}%

\noindent\begin{tabularx}{\linewidth}{@{}Xr}
\rightskip=5em plus 1 fill \hangindent=2em \hyphenpenalty=10000
\texttt{followed \EmbracOn{}\textnormal{\textsl{by a comment}}\EmbracOff{}}& [truc]
\end{tabularx}

\index[fn]{followed@\texttt{followed}}%
\begin{Texinfopreformatted}%
\ttfamily Various deff lines
\end{Texinfopreformatted}

\noindent\begin{tabularx}{\linewidth}{@{}Xr}
\rightskip=5em plus 1 fill \hangindent=2em \hyphenpenalty=10000
\texttt{after \EmbracOn{}\textnormal{\textsl{a deff item}}\EmbracOff{}}& [truc]
\end{tabularx}

\index[fn]{after@\texttt{after}}%
\begin{Texinfopreformatted}%
\ttfamily 
\end{Texinfopreformatted}

\noindent\begin{tabularx}{\linewidth}{@{}Xr}
\rightskip=5em plus 1 fill \hangindent=2em \hyphenpenalty=10000
\texttt{\Texinfocommandstyletextvar{invalid} \EmbracOn{}\textnormal{\textsl{a g}}\EmbracOff{}}& [fsetinv]
\end{tabularx}

\index[fn]{invalid@\texttt{\Texinfocommandstyletextvar{invalid}}}%

\noindent\begin{tabularx}{\linewidth}{@{}Xr}
\rightskip=5em plus 1 fill \hangindent=2em \hyphenpenalty=10000
\texttt{}& [\textbf{id `\texttt{i}' ule}]
\end{tabularx}



\noindent\begin{tabularx}{\linewidth}{@{}Xr}
\rightskip=5em plus 1 fill \hangindent=2em \hyphenpenalty=10000
\texttt{}& [aaa]
\end{tabularx}


\noindent\begin{tabularx}{\linewidth}{@{}Xr}
\rightskip=5em plus 1 fill \hangindent=2em \hyphenpenalty=10000
\texttt{}& []
\end{tabularx}


\noindent\begin{tabularx}{\linewidth}{@{}Xr}
\rightskip=5em plus 1 fill \hangindent=2em \hyphenpenalty=10000
\texttt{}& [truc]
\end{tabularx}

\begin{Texinfopreformatted}%
\ttfamily 
\end{Texinfopreformatted}
\begin{Texinfopreformatted}%
\ttfamily g{-}{-}roupe
\end{Texinfopreformatted}
\begin{Texinfopreformatted}%
\ttfamily 
\texttt{@ref\{node\}}\ node

\texttt{@ref\{,cross ref name\}}\ 
\texttt{@ref\{{,}{,}title\}}\ title
\texttt{@ref\{{,}{,},file name\}}\ \texttt{file name}
\texttt{@ref\{{,}{,}{,}{,}manual\}}\ \textsl{manual}
\texttt{@ref\{node,cross ref name\}}\ node
\texttt{@ref\{node{,}{,}title\}}\ title
\texttt{@ref\{node{,}{,},file name\}}\ Section ``node'' in \texttt{file name}
\texttt{@ref\{node{,}{,}{,}{,}manual\}}\ Section ``node'' in \textsl{manual}
\texttt{@ref\{node,cross ref name,title,\}}\ title
\texttt{@ref\{node,cross ref name{,}{,}file name\}}\ Section ``node'' in \texttt{file name}
\texttt{@ref\{node,cross ref name{,}{,},manual\}}\ Section ``node'' in \textsl{manual}
\texttt{@ref\{node,cross ref name,title,file name\}}\ Section ``title'' in \texttt{file name}
\texttt{@ref\{node,cross ref name,title{,}{,}manual\}}\ Section ``title'' in \textsl{manual}
\texttt{@ref\{node,cross ref name,title,\ file name,\ manual\}}\ Section ``title'' in \textsl{manual}
\texttt{@ref\{node{,}{,}title,file name\}}\ Section ``title'' in \texttt{file name}
\texttt{@ref\{node{,}{,}title{,}{,}manual\}}\ Section ``title'' in \textsl{manual}
\texttt{@ref\{chapter{,}{,}title,\ file name,\ manual\}}\ Section ``title'' in \textsl{manual}
\texttt{@ref\{node{,}{,}title,\ file name,\ manual\}}\ Section ``title'' in \textsl{manual}
\texttt{@ref\{node{,}{,},file name,manual\}}\ Section ``node'' in \textsl{manual}
\texttt{@ref\{,cross ref name,title,\}}\ title
\texttt{@ref\{,cross ref name{,}{,}file name\}}\ \texttt{file name}
\texttt{@ref\{,cross ref name{,}{,},manual\}}\ \textsl{manual}
\texttt{@ref\{,cross ref name,title,file name\}}\ Section ``title'' in \texttt{file name}
\texttt{@ref\{,cross ref name,title{,}{,}manual\}}\ Section ``title'' in \textsl{manual}
\texttt{@ref\{,cross ref name,title,\ file name,\ manual\}}\ Section ``title'' in \textsl{manual}
\texttt{@ref\{{,}{,}title,file name\}}\ Section ``title'' in \texttt{file name}
\texttt{@ref\{{,}{,}title{,}{,}manual\}}\ Section ``title'' in \textsl{manual}
\texttt{@ref\{{,}{,}title,\ file name,\ manual\}}\ Section ``title'' in \textsl{manual}
\texttt{@ref\{{,}{,},file name,manual\}}\ \textsl{manual}

\texttt{@inforef\{,cross ref name \}}\ 
\texttt{@inforef\{{,}{,}file name\}}\ \texttt{file name}
\texttt{@inforef\{,cross ref name,\ file name\}}\ \texttt{file name}
\texttt{@inforef\{\}}\ 


\end{Texinfopreformatted}
\end{Texinfoindented}

\index[cp]{t-ruc@t--ruc}%
\index[cp]{T-ruc@T--ruc}%
\index[cp]{.@.}%
\index[cp]{?@?}%
\index[cp]{a@a}%
\index[fn]{t--ruc@\texttt{t{-}{-}ruc}}%
\index[fn]{T--ruc@\texttt{T{-}{-}ruc}}%
\index[fn]{.@\texttt{.}}%
\index[fn]{?@\texttt{?}}%
\index[fn]{a@\texttt{a}}%

\index[cp]{a--a@a---a}%
\index[cp]{b--b@b---b!c--c@c---c}%
\index[cp]{d--dd@d---dd!e--ee@e---ee!f--ff@f---ff}%

\index[fn]{f---aa@\texttt{f{-}{-}{-}aa}}%
\index[fn]{f---bb@\texttt{f{-}{-}{-}bb}!f---cc@\texttt{f{-}{-}{-}cc}}%
\index[fn]{f---ddd@\texttt{f{-}{-}{-}ddd}!f---eee@\texttt{f{-}{-}{-}eee}!ffff@\texttt{ffff}}%

\index[cp]{aaa@aaa|see{bbb}}%
\index[cp]{ddd@ddd|seealso{ccc}}%

\index[fn]{f---aaa@\texttt{f{-}{-}{-}aaa}|see{f---bbb}}%
\index[fn]{f---ddd@\texttt{f{-}{-}{-}ddd}|seealso{f---ccc}}%

\index[cp]{A---S@aaa!B---S1@bbb}%

\index[fn]{X---S@\texttt{xxx}!X---S1@\texttt{zzz}}%

\index[fn]{@\texttt{\hbox{}}}%

\index[codeidx]{a index---entry te\~{} --- i\^{}@\texttt{a \Texinfocommandstyletextvar{index---entry}\ t\~{e}\ {-}{-}{-} \^{\i{}}}}%

\index[truc]{truc@truc}%

\index[cp]{g--gg@g---gg!h--hh jjj@h---hh jjj!k--kk@k---kk!l--ll@l---ll}%

\index[fn]{f---ggg@\texttt{f{-}{-}{-}ggg}!f---hhh fjjj@\texttt{f{-}{-}{-}hhh fjjj}!f---kkk@\texttt{f{-}{-}{-}kkk}!f---lll@\texttt{f{-}{-}{-}lll}}%

\index[cp]{ddd@ddd|seealso{ccc}}%

\index[fn]{f---ddd@\texttt{f{-}{-}{-}ddd}|seealso{f---ccc}}%


Text\footnote{in footnote
\index[cp]{index entry in footnote@index entry in footnote}%

Blah

Blih

\index[cp]{counting entry@counting entry}%
}

truc

\printindex[truc]

codeidx

\printindex[codeidx]

cp
\printindex[cp]

fn
\printindex[fn]

vr

ky

pg

tp
\printindex[tp]



\footnote{in footnote}


\section{{A section}}
\label{anchor:s_002d_002dect_002cion}%


\subsection{{subsection}}
\label{anchor:subsection}%


\label{anchor:anchor}%

\subsubsection{{subsubsection ``simple-double--}}
\label{anchor:subsubsection-_0060_0060simple_002ddouble_002d_002d}%

\subsubsection{{three---four----''}}
\label{anchor:subsubsection-three_002d_002d_002dfour_002d_002d_002d_002d_0027_0027}%

\chapter*{{\centering chapter 2}}
\label{anchor:chapter2}%

\printindex[cp]
\printindex[fn]

\end{document}
