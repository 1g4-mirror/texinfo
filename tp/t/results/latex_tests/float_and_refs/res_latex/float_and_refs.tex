\documentclass{book}
\usepackage{makeidx}\makeindex
\usepackage{amsfonts}
\usepackage{amsmath}
\usepackage[gen]{eurosym}
\usepackage[T1]{fontenc}
\usepackage{textcomp}
\usepackage{graphicx}
\usepackage{needspace}
\usepackage{etoolbox}
\usepackage{fontsize}
\usepackage{geometry}
\usepackage{fancyhdr}
\usepackage{float}
% use hidelinks to remove boxes around links to be similar with Texinfo TeX
\usepackage[hidelinks]{hyperref}
\usepackage[utf8]{inputenc}
% new float for type `'
\newfloat{TexinfoFloat}{htb}{tfl}[chapter]
\floatname{TexinfoFloat}{}
% new float for type `Thing'
\newfloat{TexinfoFloatThing}{htb}{tfl}[chapter]
\floatname{TexinfoFloatThing}{}

% redefine the \mainmatter command such that it does not clear page
% as if in double page
\makeatletter
\renewcommand\mainmatter{\clearpage\@mainmattertrue\pagenumbering{arabic}}
\makeatother

% command that does nothing used to help with substitutions in commands
\newcommand{\GNUTexinfoplaceholder}[1]{}

% called when setting single headers
% use \nouppercase to match with Texinfo TeX style
\newcommand{\GNUTexinfosetsingleheader}{\pagestyle{fancy}
\fancyhf{}
\lhead{\nouppercase{\leftmark}}
\rhead{\thepage}
}

% called when setting double headers
\newcommand{\GNUTexinfosetdoubleheader}[1]{\pagestyle{fancy}
\fancyhf{}
\fancyhead[LE,RO]{\thepage}
\fancyhead[RE]{#1}
\fancyhead[LO]{\nouppercase{\leftmark}}
}

% for part and chapter, which call \thispagestyle{plain}
\fancypagestyle{plain}{ %
 \fancyhf{}
 \fancyhead[LE,RO]{\thepage}
}

% match Texinfo TeX style
\renewcommand{\headrulewidth}{0pt}%

% avoid pagebreak and headings setting for a sectionning command
\newcommand{\GNUTexinfonopagebreakheading}[2]{\let\clearpage\relax \let\cleardoublepage\relax \let\thispagestyle\GNUTexinfoplaceholder #1{#2}}


\renewcommand{\includegraphics}[1]{\fbox{FIG #1}}

% set default for @setchapternewpage
\makeatletter
\patchcmd{\chapter}{\if@openright\cleardoublepage\else\clearpage\fi}{\GNUTexinfoplaceholder{setchapternewpage placeholder}\clearpage}{}{}
\makeatother
\GNUTexinfosetsingleheader{}%


\begin{document}
\chapter{chapter}
\label{anchor:chapter}%

\begin{TexinfoFloat}
no type
\caption[short no type float]{no type float}


\label{anchor:no-type}%
\end{TexinfoFloat}

\begin{TexinfoFloatThing}
Something with
\caption[short with type float]{with type float}


\label{anchor:with-type}%
\end{TexinfoFloatThing}

\begin{figure}
In figure
\caption[short caption for figure]{caption for figure}


\label{anchor:my-figure}%
\end{figure}

\chapter{refs}

See \hyperref[anchor:no-type]{\ref*{anchor:no-type}}.
See \hyperref[anchor:no-type]{\ref*{anchor:no-type}}.
See \hyperref[anchor:no-type]{three}.
See Section ``three'' in \texttt{four}.
See Section ``three'' in \textit{five}.

See \hyperref[anchor:with-type]{Thing~\ref*{anchor:with-type}}.
See \hyperref[anchor:with-type]{Thing~\ref*{anchor:with-type}}.
See \hyperref[anchor:with-type]{three}.
See Section ``three'' in \texttt{four}.
See Section ``three'' in \textit{five}.

See \hyperref[anchor:my-figure]{Figure~\ref*{anchor:my-figure}}.
See \hyperref[anchor:my-figure]{Figure~\ref*{anchor:my-figure}}.
See \hyperref[anchor:my-figure]{three}.
See Section ``three'' in \texttt{four}.
See Section ``three'' in \textit{five}.

\chapter{lists}

\listof{TexinfoFloat}{}
\listoffigures
\listof{TexinfoFloatThing}{}

\end{document}
