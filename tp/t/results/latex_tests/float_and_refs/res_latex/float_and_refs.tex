\documentclass{book}
\usepackage{amsfonts}
\usepackage{amsmath}
\usepackage[gen]{eurosym}
\usepackage[T1]{fontenc}
\usepackage{textcomp}
\usepackage{graphicx}
\usepackage{etoolbox}
\usepackage{titleps}
\usepackage{float}
% use hidelinks to remove boxes around links to be similar to Texinfo TeX
\usepackage[hidelinks]{hyperref}
\usepackage[utf8]{inputenc}

\makeatletter
\newcommand{\GNUTexinfosettitle}{No Title}%

% new float for type `'
\newfloat{TexinfoFloat}{htb}{tfl}[chapter]
\floatname{TexinfoFloat}{}
% new float for type `Thing'
\newfloat{TexinfoFloatThing}{htb}{tfl}[chapter]
\floatname{TexinfoFloatThing}{}
% redefine the \mainmatter command such that it does not clear page
% as if in double page
\renewcommand\mainmatter{\clearpage\@mainmattertrue\pagenumbering{arabic}}
\newenvironment{GNUTexinfopreformatted}{%
  \par\GNUTobeylines\obeyspaces\frenchspacing\parskip=\z@\parindent=\z@}{}
{\catcode`\^^M=13 \gdef\GNUTobeylines{\catcode`\^^M=13 \def^^M{\null\par}}}
\newenvironment{GNUTexinfoindented}{\begin{list}{}{}\item\relax}{\end{list}}

% used for substitutions in commands
\newcommand{\GNUTexinfoplaceholder}[1]{}

\newpagestyle{single}{\sethead[\chaptername{} \thechapter{} \chaptertitle{}][][\thepage]
                              {\chaptername{} \thechapter{} \chaptertitle{}}{}{\thepage}}

% allow line breaking at underscore
\let\GNUTexinfounderscore\_
\renewcommand{\_}{\GNUTexinfounderscore\discretionary{}{}{}}
\renewcommand{\includegraphics}[1]{\fbox{FIG \detokenize{#1}}}

\makeatother
% set default for @setchapternewpage
\makeatletter
\patchcmd{\chapter}{\if@openright\cleardoublepage\else\clearpage\fi}{\GNUTexinfoplaceholder{setchapternewpage placeholder}\clearpage}{}{}
\makeatother
\pagestyle{single}%

\begin{document}
\label{anchor:Top}%
\chapter{{chapter}}
\label{anchor:chapter}%

\begin{TexinfoFloat}
no type
\caption[short no type float]{no type float}


\label{anchor:no-type}%
\end{TexinfoFloat}

\begin{TexinfoFloatThing}
Something with
\caption[short with type float]{with type float}


\label{anchor:with-type}%
\end{TexinfoFloatThing}

\begin{figure}
In figure
\caption[short caption for figure]{caption for figure}


\label{anchor:my-figure}%
\end{figure}

\chapter{{refs}}

See \hyperref[anchor:no-type]{\ref*{anchor:no-type}}.
See \hyperref[anchor:no-type]{\ref*{anchor:no-type}}.
See \hyperref[anchor:no-type]{three}.
See Section ``three'' in \texttt{four}.
See Section ``three'' in \textsl{five}.

See \hyperref[anchor:with-type]{Thing~\ref*{anchor:with-type}}.
See \hyperref[anchor:with-type]{Thing~\ref*{anchor:with-type}}.
See \hyperref[anchor:with-type]{three}.
See Section ``three'' in \texttt{four}.
See Section ``three'' in \textsl{five}.

See \hyperref[anchor:my-figure]{Figure~\ref*{anchor:my-figure}}.
See \hyperref[anchor:my-figure]{Figure~\ref*{anchor:my-figure}}.
See \hyperref[anchor:my-figure]{three}.
See Section ``three'' in \texttt{four}.
See Section ``three'' in \textsl{five}.

\chapter{{lists}}

\listof{TexinfoFloat}{}
\listoffigures
\listof{TexinfoFloatThing}{}

\end{document}
