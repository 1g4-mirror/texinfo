\documentclass{book}
\usepackage{amsfonts}
\usepackage{amsmath}
\usepackage[gen]{eurosym}
\usepackage[T1]{fontenc}
\usepackage{textcomp}
\usepackage{graphicx}
\usepackage{etoolbox}
\usepackage{titleps}
\usepackage{float}
% use hidelinks to remove boxes around links to be similar to Texinfo TeX
\usepackage[hidelinks]{hyperref}
\usepackage[utf8]{inputenc}

\makeatletter
\newcommand{\GNUTexinfosettitle}{No Title}%

% redefine the \mainmatter command such that it does not clear page
% as if in double page
\renewcommand\mainmatter{\clearpage\@mainmattertrue\pagenumbering{arabic}}
\newenvironment{GNUTexinfopreformatted}{%
  \par\GNUTobeylines\obeyspaces\frenchspacing
  \parskip=\z@\parindent=\z@}{}
{\catcode`\^^M=13 \gdef\GNUTobeylines{\catcode`\^^M=13 \def^^M{\null\par}}}
\newenvironment{GNUTexinfoindented}
  {\begin{list}{}{}
  \item\relax}
  {\end{list}}
% command that does nothing used to help with substitutions in commands
\newcommand{\GNUTexinfoplaceholder}[1]{}

% plain page style for part and chapter, which call \thispagestyle{plain}
\renewpagestyle{plain}{\sethead[\thepage{}][][]
                             {}{}{\thepage{}}}

\newpagestyle{single}{\sethead[\chaptername{} \thechapter{} \chaptertitle{}][][\thepage]
                              {\chaptername{} \thechapter{} \chaptertitle{}}{}{\thepage}}

% avoid pagebreak and headings setting for a sectionning command
\newcommand{\GNUTexinfonopagebreakheading}[2]{{\let\clearpage\relax \let\cleardoublepage\relax \let\thispagestyle\GNUTexinfoplaceholder #1{#2}}}

\renewcommand{\includegraphics}[1]{\fbox{FIG \detokenize{#1}}}

\makeatother
% set default for @setchapternewpage
\makeatletter
\patchcmd{\chapter}{\if@openright\cleardoublepage\else\clearpage\fi}{\GNUTexinfoplaceholder{setchapternewpage placeholder}\clearpage}{}{}
\makeatother
\pagestyle{single}%

\begin{document}
\label{anchor:node-before}%

In node before
\label{anchor:anch_003a-in-node-before}%

\label{anchor:Top}%
\label{anchor:anch_003a-in-node-top}%
\label{anchor:after}%

in node after
\label{anchor:anch_003a-in-node-after}%

\chapter{{chap}}
\label{anchor:chap}%

in chap
\label{anchor:anch_003a-in-chap}%

See \hyperref[anchor:anch_003a-in-node-before]{[top sectionning], page~\pageref*{anchor:anch_003a-in-node-before}}.
See \hyperref[anchor:anch_003a-in-node-top]{[top sectionning], page~\pageref*{anchor:anch_003a-in-node-top}}.
See \hyperref[anchor:anch_003a-in-node-after]{[top sectionning], page~\pageref*{anchor:anch_003a-in-node-after}}.
See \hyperref[anchor:anch_003a-in-chap]{[chap], page~\pageref*{anchor:anch_003a-in-chap}}.

See \hyperref[anchor:anch_003a-in-node-before]{[anch: in node before], page~\pageref*{anchor:anch_003a-in-node-before}}.
See \hyperref[anchor:anch_003a-in-node-top]{[anch: in node top], page~\pageref*{anchor:anch_003a-in-node-top}}.
See \hyperref[anchor:anch_003a-in-node-after]{[anch: in node after], page~\pageref*{anchor:anch_003a-in-node-after}}.
See \hyperref[anchor:anch_003a-in-chap]{[anch: in chap], page~\pageref*{anchor:anch_003a-in-chap}}.
\end{document}
