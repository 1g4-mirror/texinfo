\documentclass{book}
\usepackage{imakeidx}
\usepackage{amsfonts}
\usepackage{amsmath}
\usepackage[gen]{eurosym}
\usepackage[T1]{fontenc}
\usepackage{textcomp}
\usepackage{graphicx}
\usepackage{etoolbox}
\usepackage{embrac}
\usepackage{expl3}
\usepackage{tabularx}
\usepackage{titleps}
\usepackage{float}
% use hidelinks to remove boxes around links to be similar with Texinfo TeX
\usepackage[hidelinks]{hyperref}
\usepackage[utf8]{inputenc}

\makeatletter
\newcommand{\GNUTexinfosettitle}{No Title}%

\makeindex[name=fn]%
\makeindex[name=vr]%

% style command for var in 'cmd_text' formatting context
\newcommand\GNUTexinfocommandstyletextvar[1]{{\normalfont{}\textsl{#1}}}%

% redefine the \mainmatter command such that it does not clear page
% as if in double page
\renewcommand\mainmatter{\clearpage\@mainmattertrue\pagenumbering{arabic}}
% add command aliases to use the same command in book and report
\newcommand\GNUTexinfomainmatter{\mainmatter}
\newcommand\GNUTexinfofrontmatter{\frontmatter}
\newenvironment{GNUTexinfopreformatted}{%
  \par\obeylines\obeyspaces\frenchspacing
  \parskip=\z@\parindent=\z@}{}
% command that does nothing used to help with substitutions in commands
\newcommand{\GNUTexinfoplaceholder}[1]{}

% plain page style, for part and chapter, which call \thispagestyle{plain}
\renewpagestyle{plain}{\sethead[\thepage{}][][]
                             {}{}{\thepage{}}}

% single header
\newpagestyle{single}{\sethead[\chaptername{} \thechapter{} \chaptertitle{}][][\thepage]
                              {\chaptername{} \thechapter{} \chaptertitle{}}{}{\thepage}}

% called when setting single headers
\newcommand{\GNUTexinfosetsingleheader}{\pagestyle{single}}

% double header
\newpagestyle{double}{\sethead[\thepage{}][][\GNUTexinfosettitle]
                              {\chaptername{} \thechapter{} \chaptertitle{}}{}{\thepage}}

% called when setting double headers
\newcommand{\GNUTexinfosetdoubleheader}{\pagestyle{double}}


% avoid pagebreak and headings setting for a sectionning command
\newcommand{\GNUTexinfonopagebreakheading}[2]{\let\clearpage\relax \let\cleardoublepage\relax \let\thispagestyle\GNUTexinfoplaceholder #1{#2}}

% braces are upright in italic and slanted only in @def*
% so it is turned off here, and turned on @def* lines
\EmbracOff{}%

\renewcommand{\includegraphics}[1]{\fbox{FIG \detokenize{#1}}}

\makeatother
% set default for @setchapternewpage
\makeatletter
\patchcmd{\chapter}{\if@openright\cleardoublepage\else\clearpage\fi}{\GNUTexinfoplaceholder{setchapternewpage placeholder}\clearpage}{}{}
\makeatother
\GNUTexinfosetsingleheader{}%


\begin{document}
\label{anchor:Top}%
\chapter{{chapter}}
\label{anchor:chapter}%


\noindent\begin{tabularx}{\linewidth}{@{}Xr}
\rightskip=5em plus 1 fill
\hangindent=2em
\noindent\texttt{fname \EmbracOn{}\textnormal{\textsl{a---rg1 a--rg2}}\EmbracOff{}}& [Func]
\end{tabularx}

\index[fn]{fname@\texttt{fname}}%
\begin{quote}
\unskip{\parskip=0pt\noindent}%
deffn no var for \GNUTexinfocommandstyletextvar{a---rg1} and \GNUTexinfocommandstyletextvar{a--rg2}
\end{quote}


\noindent\begin{tabularx}{\linewidth}{@{}Xr}
\rightskip=5em plus 1 fill
\hangindent=2em
\noindent\texttt{fname \EmbracOn{}\textnormal{\textsl{\GNUTexinfocommandstyletextvar{a---rg1} \GNUTexinfocommandstyletextvar{a--rg2}}}\EmbracOff{}}& [Func]
\end{tabularx}

\index[fn]{fname@\texttt{fname}}%
\begin{quote}
\unskip{\parskip=0pt\noindent}%
deffn explict var for \GNUTexinfocommandstyletextvar{a---rg1} and \GNUTexinfocommandstyletextvar{a--rg2}
\end{quote}


\noindent\begin{tabularx}{\linewidth}{@{}Xr}
\rightskip=5em plus 1 fill
\hangindent=2em
\noindent\texttt{fname \EmbracOn{}\textnormal{\textsl{\EmbracOff{}\textnormal{\textsl{a---rg1}}\EmbracOn{} \EmbracOff{}\textnormal{\textsl{a--rg2}}\EmbracOn{}}}\EmbracOff{}}& [Func]
\end{tabularx}

\index[fn]{fname@\texttt{fname}}%
\begin{quote}
\unskip{\parskip=0pt\noindent}%
deffn r slanted for \GNUTexinfocommandstyletextvar{a---rg1} and \GNUTexinfocommandstyletextvar{a--rg2}
\end{quote}


\noindent\begin{tabularx}{\linewidth}{@{}Xr}
\rightskip=5em plus 1 fill
\hangindent=2em
\noindent\texttt{foobar \EmbracOn{}\textnormal{\textsl{(var [from to [inc]]) default}}\EmbracOff{}}& [Special Form]
\end{tabularx}

\index[fn]{foobar@\texttt{foobar}}%

\noindent\begin{tabularx}{\linewidth}{@{}Xr}
\rightskip=5em plus 1 fill
\hangindent=2em
\noindent\texttt{foobar \EmbracOn{}\textnormal{\textsl{(var \EmbracOff{}\textnormal{[}\EmbracOn{}from to \EmbracOff{}\textnormal{[}\EmbracOn{}inc\EmbracOff{}\textnormal{]]}\EmbracOn{}) r}}\EmbracOff{}}& [Special Form]
\end{tabularx}

\index[fn]{foobar@\texttt{foobar}}%

\noindent\begin{tabularx}{\linewidth}{@{}Xr}
\rightskip=5em plus 1 fill
\hangindent=2em
\noindent\texttt{foobar \EmbracOn{}\textnormal{\textsl{(var \GNUTexinfocommandstyletextvar{[}from to \GNUTexinfocommandstyletextvar{[}inc\GNUTexinfocommandstyletextvar{]]}) var}}\EmbracOff{}}& [Special Form]
\end{tabularx}

\index[fn]{foobar@\texttt{foobar}}%

\noindent\begin{tabularx}{\linewidth}{@{}Xr}
\rightskip=5em plus 1 fill
\hangindent=2em
\noindent\texttt{foobar \EmbracOn{}\textnormal{\textsl{(var \textsl{[}from to \textsl{[}inc\textsl{]]}) slanted}}\EmbracOff{}}& [Special Form]
\end{tabularx}

\index[fn]{foobar@\texttt{foobar}}%

\EmbracMakeKnown{texttt}%
\noindent\begin{tabularx}{\linewidth}{@{}Xr}
\rightskip=5em plus 1 fill
\hangindent=2em
\noindent\texttt{foobar \EmbracOn{}\textnormal{\textsl{(var \texttt{[}from to \texttt{[}inc\texttt{]]}) code}}\EmbracOff{}}& [Special Form]
\end{tabularx}
\ExplSyntaxOn%
\cs_undefine:N{\embrac_texttt:nn}\cs_undefine:N{\embrac_orig_texttt:n}\cs_undefine:N{\__embrac_texttt:n}%
\ExplSyntaxOff%

\index[fn]{foobar@\texttt{foobar}}%

\EmbracMakeKnown{texttt}%
\noindent\begin{tabularx}{\linewidth}{@{}Xr}
\rightskip=5em plus 1 fill
\hangindent=2em
\noindent\texttt{foobar \EmbracOn{}\textnormal{\textsl{(var \texttt{[}from to \texttt{[}inc\texttt{]]}) t}}\EmbracOff{}}& [Special Form]
\end{tabularx}
\ExplSyntaxOn%
\cs_undefine:N{\embrac_texttt:nn}\cs_undefine:N{\embrac_orig_texttt:n}\cs_undefine:N{\__embrac_texttt:n}%
\ExplSyntaxOff%

\index[fn]{foobar@\texttt{foobar}}%

\EmbracMakeKnown{texttt}%
\EmbracMakeKnown{textbf}%
\noindent\begin{tabularx}{\linewidth}{@{}Xr}
\rightskip=5em plus 1 fill
\hangindent=2em
\noindent\texttt{foobar \EmbracOn{}\textnormal{\textsl{(var \texttt{\textbf{[}}from to \texttt{\textbf{[}}inc\texttt{\textbf{]]}}) t:b}}\EmbracOff{}}& [Special Form]
\end{tabularx}
\ExplSyntaxOn%
\cs_undefine:N{\embrac_textbf:nn}\cs_undefine:N{\embrac_orig_textbf:n}\cs_undefine:N{\__embrac_textbf:n}%
\cs_undefine:N{\embrac_texttt:nn}\cs_undefine:N{\embrac_orig_texttt:n}\cs_undefine:N{\__embrac_texttt:n}%
\ExplSyntaxOff%

\index[fn]{foobar@\texttt{foobar}}%

\noindent\begin{tabularx}{\linewidth}{@{}Xr}
\rightskip=5em plus 1 fill
\hangindent=2em
\noindent\texttt{foobar \EmbracOn{}\textnormal{\textsl{(var \EmbracOff{}\textnormal{\GNUTexinfocommandstyletextvar{[}}\EmbracOn{}from to \EmbracOff{}\textnormal{\GNUTexinfocommandstyletextvar{[}}\EmbracOn{}inc\EmbracOff{}\textnormal{\GNUTexinfocommandstyletextvar{]]}}\EmbracOn{}) r:var}}\EmbracOff{}}& [Special Form]
\end{tabularx}

\index[fn]{foobar@\texttt{foobar}}%

\noindent\begin{tabularx}{\linewidth}{@{}Xr}
\rightskip=5em plus 1 fill
\hangindent=2em
\noindent\texttt{foobar \EmbracOn{}\textnormal{\textsl{(var \EmbracOff{}\textnormal{\textsl{[}}\EmbracOn{}from to \EmbracOff{}\textnormal{\textsl{[}}\EmbracOn{}inc\EmbracOff{}\textnormal{\textsl{]]}}\EmbracOn{}) r:slanted}}\EmbracOff{}}& [Special Form]
\end{tabularx}

\index[fn]{foobar@\texttt{foobar}}%

\noindent\begin{tabularx}{\linewidth}{@{}Xr}
\rightskip=5em plus 1 fill
\hangindent=2em
\noindent\texttt{foobar \EmbracOn{}\textnormal{\textsl{(var \EmbracOff{}\textnormal{\texttt{[}}\EmbracOn{}from to \EmbracOff{}\textnormal{\texttt{[}}\EmbracOn{}inc\EmbracOff{}\textnormal{\texttt{]]}}\EmbracOn{}) r:code}}\EmbracOff{}}& [Special Form]
\end{tabularx}

\index[fn]{foobar@\texttt{foobar}}%

\noindent\begin{tabularx}{\linewidth}{@{}Xr}
\rightskip=5em plus 1 fill
\hangindent=2em
\noindent\texttt{foobar \EmbracOn{}\textnormal{\textsl{(var \EmbracOff{}\textnormal{\texttt{[}}\EmbracOn{}from to \EmbracOff{}\textnormal{\texttt{[}}\EmbracOn{}inc\EmbracOff{}\textnormal{\texttt{]]}}\EmbracOn{}) r:t}}\EmbracOff{}}& [Special Form]
\end{tabularx}

\index[fn]{foobar@\texttt{foobar}}%
\begin{quote}
\unskip{\parskip=0pt\noindent}%
separators
\end{quote}


\noindent\begin{tabularx}{\linewidth}{@{}Xr}
\rightskip=5em plus 1 fill
\hangindent=2em
\noindent\texttt{foobar \EmbracOn{}\textnormal{\textsl{va---riable default}}\EmbracOff{}}& [Special Form]
\end{tabularx}

\index[fn]{foobar@\texttt{foobar}}%

\noindent\begin{tabularx}{\linewidth}{@{}Xr}
\rightskip=5em plus 1 fill
\hangindent=2em
\noindent\texttt{foobar \EmbracOn{}\textnormal{\textsl{\GNUTexinfocommandstyletextvar{va---riable} var}}\EmbracOff{}}& [Special Form]
\end{tabularx}

\index[fn]{foobar@\texttt{foobar}}%

\noindent\begin{tabularx}{\linewidth}{@{}Xr}
\rightskip=5em plus 1 fill
\hangindent=2em
\noindent\texttt{foobar \EmbracOn{}\textnormal{\textsl{\EmbracOff{}\textnormal{va---riable}\EmbracOn{} r}}\EmbracOff{}}& [Special Form]
\end{tabularx}

\index[fn]{foobar@\texttt{foobar}}%

\noindent\begin{tabularx}{\linewidth}{@{}Xr}
\rightskip=5em plus 1 fill
\hangindent=2em
\noindent\texttt{foobar \EmbracOn{}\textnormal{\textsl{\textsl{va---riable} slanted}}\EmbracOff{}}& [Special Form]
\end{tabularx}

\index[fn]{foobar@\texttt{foobar}}%

\EmbracMakeKnown{texttt}%
\noindent\begin{tabularx}{\linewidth}{@{}Xr}
\rightskip=5em plus 1 fill
\hangindent=2em
\noindent\texttt{foobar \EmbracOn{}\textnormal{\textsl{\texttt{va{-}{-}{-}riable} code}}\EmbracOff{}}& [Special Form]
\end{tabularx}
\ExplSyntaxOn%
\cs_undefine:N{\embrac_texttt:nn}\cs_undefine:N{\embrac_orig_texttt:n}\cs_undefine:N{\__embrac_texttt:n}%
\ExplSyntaxOff%

\index[fn]{foobar@\texttt{foobar}}%

\EmbracMakeKnown{texttt}%
\noindent\begin{tabularx}{\linewidth}{@{}Xr}
\rightskip=5em plus 1 fill
\hangindent=2em
\noindent\texttt{foobar \EmbracOn{}\textnormal{\textsl{\texttt{va{-}{-}{-}riable} t}}\EmbracOff{}}& [Special Form]
\end{tabularx}
\ExplSyntaxOn%
\cs_undefine:N{\embrac_texttt:nn}\cs_undefine:N{\embrac_orig_texttt:n}\cs_undefine:N{\__embrac_texttt:n}%
\ExplSyntaxOff%

\index[fn]{foobar@\texttt{foobar}}%

\EmbracMakeKnown{texttt}%
\EmbracMakeKnown{textbf}%
\noindent\begin{tabularx}{\linewidth}{@{}Xr}
\rightskip=5em plus 1 fill
\hangindent=2em
\noindent\texttt{foobar \EmbracOn{}\textnormal{\textsl{\texttt{\textbf{va{-}{-}{-}riable}} t:b}}\EmbracOff{}}& [Special Form]
\end{tabularx}
\ExplSyntaxOn%
\cs_undefine:N{\embrac_textbf:nn}\cs_undefine:N{\embrac_orig_textbf:n}\cs_undefine:N{\__embrac_textbf:n}%
\cs_undefine:N{\embrac_texttt:nn}\cs_undefine:N{\embrac_orig_texttt:n}\cs_undefine:N{\__embrac_texttt:n}%
\ExplSyntaxOff%

\index[fn]{foobar@\texttt{foobar}}%

\noindent\begin{tabularx}{\linewidth}{@{}Xr}
\rightskip=5em plus 1 fill
\hangindent=2em
\noindent\texttt{foobar \EmbracOn{}\textnormal{\textsl{\EmbracOff{}\textnormal{\GNUTexinfocommandstyletextvar{va---riable}}\EmbracOn{} r:var}}\EmbracOff{}}& [Special Form]
\end{tabularx}

\index[fn]{foobar@\texttt{foobar}}%

\noindent\begin{tabularx}{\linewidth}{@{}Xr}
\rightskip=5em plus 1 fill
\hangindent=2em
\noindent\texttt{foobar \EmbracOn{}\textnormal{\textsl{\EmbracOff{}\textnormal{\textsl{va---riable}}\EmbracOn{} r:slanted}}\EmbracOff{}}& [Special Form]
\end{tabularx}

\index[fn]{foobar@\texttt{foobar}}%

\noindent\begin{tabularx}{\linewidth}{@{}Xr}
\rightskip=5em plus 1 fill
\hangindent=2em
\noindent\texttt{foobar \EmbracOn{}\textnormal{\textsl{\EmbracOff{}\textnormal{\texttt{va{-}{-}{-}riable}}\EmbracOn{} r:code}}\EmbracOff{}}& [Special Form]
\end{tabularx}

\index[fn]{foobar@\texttt{foobar}}%

\noindent\begin{tabularx}{\linewidth}{@{}Xr}
\rightskip=5em plus 1 fill
\hangindent=2em
\noindent\texttt{foobar \EmbracOn{}\textnormal{\textsl{\EmbracOff{}\textnormal{\texttt{va{-}{-}{-}riable}}\EmbracOn{} r:t}}\EmbracOff{}}& [Special Form]
\end{tabularx}

\index[fn]{foobar@\texttt{foobar}}%
\begin{quote}
\unskip{\parskip=0pt\noindent}%
name
\end{quote}


\noindent\begin{tabularx}{\linewidth}{@{}Xr}
\rightskip=5em plus 1 fill
\hangindent=2em
\noindent\texttt{\texttt{.ft} \EmbracOn{}\textnormal{\textsl{[\EmbracOff{}\textnormal{\textsl{font}}\EmbracOn{}]}}\EmbracOff{}}& [Request]
\end{tabularx}

\index[fn]{.ft@\texttt{\texttt{.ft}}}%

\noindent\begin{tabularx}{\linewidth}{@{}Xr}
\rightskip=5em plus 1 fill
\hangindent=2em
\noindent\texttt{\texttt{\textbackslash{}f}\textnormal{\textsl{f}}\texttt{}}& [Escape~sequence]
\end{tabularx}

\index[fn]{\textbackslash{}ff@\texttt{\texttt{\textbackslash{}f}\textnormal{\textsl{f}}\texttt{}}}%

\noindent\begin{tabularx}{\linewidth}{@{}Xr}
\rightskip=5em plus 1 fill
\hangindent=2em
\noindent\texttt{\texttt{\textbackslash{}f(}\textnormal{\textsl{fn}}\texttt{}}& [Escape~sequence]
\end{tabularx}

\index[fn]{\textbackslash{}f(fn@\texttt{\texttt{\textbackslash{}f(}\textnormal{\textsl{fn}}\texttt{}}}%

\EmbracMakeKnown{texttt}%
\noindent\begin{tabularx}{\linewidth}{@{}Xr}
\rightskip=5em plus 1 fill
\hangindent=2em
\noindent\texttt{\texttt{\textbackslash{}f[}\textnormal{\textsl{font}}\texttt{]} \EmbracOn{}\textnormal{\textsl{\texttt{\textbackslash{}f[}\EmbracOff{}\textnormal{\textsl{font}}\EmbracOn{}\texttt{]}}}\EmbracOff{}}& [Escape~sequence]
\end{tabularx}
\ExplSyntaxOn%
\cs_undefine:N{\embrac_texttt:nn}\cs_undefine:N{\embrac_orig_texttt:n}\cs_undefine:N{\__embrac_texttt:n}%
\ExplSyntaxOff%

\index[fn]{\textbackslash{}f[font]@\texttt{\texttt{\textbackslash{}f[}\textnormal{\textsl{font}}\texttt{]}}}%

\noindent\begin{tabularx}{\linewidth}{@{}Xr}
\rightskip=5em plus 1 fill
\hangindent=2em
\noindent\texttt{\texttt{\textbackslash{}n[.sty]}}& [Register]
\end{tabularx}

\index[fn]{\textbackslash{}n[.sty]@\texttt{\texttt{\textbackslash{}n[.sty]}}}%
\begin{quote}
\unskip{\parskip=0pt\noindent}%
The \texttt{ft} request and the \texttt{\textbackslash{}f} escape change the current font
to \GNUTexinfocommandstyletextvar{font} (one-character name~\GNUTexinfocommandstyletextvar{f}, two-character name
\GNUTexinfocommandstyletextvar{fn}).
\end{quote}


\EmbracMakeKnown{texttt}%
\noindent\begin{tabularx}{\linewidth}{@{}Xr}
\rightskip=5em plus 1 fill
\hangindent=2em
\noindent\texttt{foobar \EmbracOn{}\textnormal{\textsl{[ \EmbracOff{}\textnormal{[}\EmbracOn{} \textsl{[} \texttt{[} \texttt{[} \EmbracOff{}\textnormal{\textsl{[}}\EmbracOn{} \EmbracOff{}\textnormal{\texttt{[}}\EmbracOn{} \EmbracOff{}\textnormal{\texttt{\textsl{[}}}\EmbracOn{} \EmbracOff{}\textnormal{\texttt{[}}\EmbracOn{} , \EmbracOff{}\textnormal{,}\EmbracOn{} \textsl{,} \texttt{,} \texttt{,} \EmbracOff{}\textnormal{\textsl{,}}\EmbracOn{} \EmbracOff{}\textnormal{\texttt{,}}\EmbracOn{} \EmbracOff{}\textnormal{\texttt{\textsl{,}}}\EmbracOn{} \EmbracOff{}\textnormal{\texttt{,}}\EmbracOn{} ] \EmbracOff{}\textnormal{]}\EmbracOn{} \textsl{]} \texttt{]} \texttt{]} \EmbracOff{}\textnormal{\textsl{]}}\EmbracOn{} \EmbracOff{}\textnormal{\texttt{]}}\EmbracOn{} \EmbracOff{}\textnormal{\texttt{\textsl{]}}}\EmbracOn{} \EmbracOff{}\textnormal{\texttt{]}}\EmbracOn{}}}\EmbracOff{}}& [Special Form]
\end{tabularx}
\ExplSyntaxOn%
\cs_undefine:N{\embrac_texttt:nn}\cs_undefine:N{\embrac_orig_texttt:n}\cs_undefine:N{\__embrac_texttt:n}%
\ExplSyntaxOff%

\index[fn]{foobar@\texttt{foobar}}%

\EmbracMakeKnown{texttt}%
\noindent\begin{tabularx}{\linewidth}{@{}Xr}
\rightskip=5em plus 1 fill
\hangindent=2em
\noindent\texttt{foobar \EmbracOn{}\textnormal{\textsl{[] \EmbracOff{}\textnormal{[]}\EmbracOn{} \textsl{[]} \texttt{[]} \texttt{[]} \EmbracOff{}\textnormal{\textsl{[]}}\EmbracOn{} \EmbracOff{}\textnormal{\texttt{[]}}\EmbracOn{} \EmbracOff{}\textnormal{\texttt{\textsl{[]}}}\EmbracOn{}}}\EmbracOff{}}& [Special Form]
\end{tabularx}
\ExplSyntaxOn%
\cs_undefine:N{\embrac_texttt:nn}\cs_undefine:N{\embrac_orig_texttt:n}\cs_undefine:N{\__embrac_texttt:n}%
\ExplSyntaxOff%

\index[fn]{foobar@\texttt{foobar}}%
\begin{quote}
\unskip{\parskip=0pt\noindent}%
test formatting of separators
\end{quote}


\noindent\begin{tabularx}{\linewidth}{@{}Xr}
\rightskip=5em plus 1 fill
\hangindent=2em
\noindent\texttt{int foobar (int\ \GNUTexinfocommandstyletextvar{f---oo},\ float\ \GNUTexinfocommandstyletextvar{b--ar})}& [Library Function]
\end{tabularx}

\index[fn]{foobar@\texttt{foobar}}%
\begin{quote}
\unskip{\parskip=0pt\noindent}%
\dots{}\@ with var for \GNUTexinfocommandstyletextvar{f---oo} and \GNUTexinfocommandstyletextvar{b--ar}
\end{quote}


\noindent\begin{tabularx}{\linewidth}{@{}Xr}
\rightskip=5em plus 1 fill
\hangindent=2em
\noindent\texttt{int foobar (int\ \textnormal{\textsl{f---oo}},\ float\ \textnormal{\textsl{b--ar}})}& [Library Function]
\end{tabularx}

\index[fn]{foobar@\texttt{foobar}}%
\begin{quote}
\unskip{\parskip=0pt\noindent}%
\dots{}\@ with r slanted for \GNUTexinfocommandstyletextvar{f---oo} and \GNUTexinfocommandstyletextvar{b--ar}
\end{quote}

\noindent{}produces:

\noindent\begin{tabularx}{\linewidth}{@{}Xr}
\rightskip=5em plus 1 fill
\hangindent=2em
\noindent\texttt{border-pattern}& [Class Option of \texttt{Window}]
\end{tabularx}

\index[vr]{border-pattern@\texttt{border-pattern}}%
\begin{quote}
\unskip{\parskip=0pt\noindent}%
\dots{}\@
\end{quote}


\noindent\begin{tabularx}{\linewidth}{@{}Xr}
\rightskip=5em plus 1 fill
\hangindent=2em
\noindent\texttt{\texttt{int} border-pattern}& [Class Option of \texttt{Window}]
\end{tabularx}

\index[vr]{border-pattern of Window@\texttt{border-pattern\ of Window}}%
\begin{quote}
\unskip{\parskip=0pt\noindent}%
\dots{}\@
\end{quote}

\begin{quote}

\noindent\begin{tabularx}{\linewidth}{@{}Xr}
\rightskip=5em plus 1 fill
\hangindent=2em
\noindent\texttt{int foobar (int\ \GNUTexinfocommandstyletextvar{foo},\ float\ \GNUTexinfocommandstyletextvar{bar})}& [Library Function]
\end{tabularx}

\index[fn]{foobar@\texttt{foobar}}%
\begin{quote}
\unskip{\parskip=0pt\noindent}%
\dots{}\@ for \GNUTexinfocommandstyletextvar{foo} and \GNUTexinfocommandstyletextvar{bar}
\end{quote}
\end{quote}


\noindent\begin{tabularx}{\linewidth}{@{}Xr}
\rightskip=5em plus 1 fill
\hangindent=2em
\noindent\texttt{apply \EmbracOn{}\textnormal{\textsl{function \&rest arguments}}\EmbracOff{}}& [Function]
\end{tabularx}

\index[fn]{apply@\texttt{apply}}%
\begin{quote}
\unskip{\parskip=0pt\noindent}%
\texttt{apply} calls no var \GNUTexinfocommandstyletextvar{function} with \GNUTexinfocommandstyletextvar{arguments}
\end{quote}


\noindent\begin{tabularx}{\linewidth}{@{}Xr}
\rightskip=5em plus 1 fill
\hangindent=2em
\noindent\texttt{apply \EmbracOn{}\textnormal{\textsl{function \EmbracOff{}\textnormal{\textbf{\&rest}}\EmbracOn{} argument}}\EmbracOff{}}& [Function]
\end{tabularx}

\index[fn]{apply@\texttt{apply}}%
\begin{quote}
\unskip{\parskip=0pt\noindent}%
explicit keyword marking, no var \GNUTexinfocommandstyletextvar{function} with \GNUTexinfocommandstyletextvar{arguments}
\end{quote}


\EmbracMakeKnown{texttt}%
\noindent\begin{tabularx}{\linewidth}{@{}Xr}
\rightskip=5em plus 1 fill
\hangindent=2em
\noindent\texttt{name \EmbracOn{}\textnormal{\textsl{argument \texttt{int} \texttt{a{-}{-}b} \GNUTexinfocommandstyletextvar{v--ar1}, word \texttt{{-}{-}} (\texttt{type o{-}{-}ther}, \GNUTexinfocommandstyletextvar{v---ar2}  [\texttt{float} [\GNUTexinfocommandstyletextvar{var4}]])}}\EmbracOff{}}& [Category]
\end{tabularx}
\ExplSyntaxOn%
\cs_undefine:N{\embrac_texttt:nn}\cs_undefine:N{\embrac_orig_texttt:n}\cs_undefine:N{\__embrac_texttt:n}%
\ExplSyntaxOff%

\index[fn]{name@\texttt{name}}%
\begin{quote}
\unskip{\parskip=0pt\noindent}%
In deffn with code and var used
\end{quote}


\noindent\begin{tabularx}{\linewidth}{@{}Xr}
\rightskip=5em plus 1 fill
\hangindent=2em
\noindent\texttt{int foobar (int\ \GNUTexinfocommandstyletextvar{f---oo}[,\ float\ \GNUTexinfocommandstyletextvar{b--ar}])\ default}& [Library Function]
\end{tabularx}

\index[fn]{foobar@\texttt{foobar}}%

\noindent\begin{tabularx}{\linewidth}{@{}Xr}
\rightskip=5em plus 1 fill
\hangindent=2em
\noindent\texttt{int foobar (int\ \GNUTexinfocommandstyletextvar{f---oo}\textnormal{[},\ float\ \GNUTexinfocommandstyletextvar{b--ar}\textnormal{]})\ r}& [Library Function]
\end{tabularx}

\index[fn]{foobar@\texttt{foobar}}%

\noindent\begin{tabularx}{\linewidth}{@{}Xr}
\rightskip=5em plus 1 fill
\hangindent=2em
\noindent\texttt{int foobar (int\ \GNUTexinfocommandstyletextvar{f---oo}\GNUTexinfocommandstyletextvar{[},\ float\ \GNUTexinfocommandstyletextvar{b--ar}\GNUTexinfocommandstyletextvar{]})\ var}& [Library Function]
\end{tabularx}

\index[fn]{foobar@\texttt{foobar}}%

\noindent\begin{tabularx}{\linewidth}{@{}Xr}
\rightskip=5em plus 1 fill
\hangindent=2em
\noindent\texttt{int foobar (int\ \GNUTexinfocommandstyletextvar{f---oo}\textsl{[},\ float\ \GNUTexinfocommandstyletextvar{b--ar}\textsl{]})\ slanted}& [Library Function]
\end{tabularx}

\index[fn]{foobar@\texttt{foobar}}%

\noindent\begin{tabularx}{\linewidth}{@{}Xr}
\rightskip=5em plus 1 fill
\hangindent=2em
\noindent\texttt{int foobar (int\ \GNUTexinfocommandstyletextvar{f---oo}\texttt{[},\ float\ \GNUTexinfocommandstyletextvar{b--ar}\texttt{]})\ code}& [Library Function]
\end{tabularx}

\index[fn]{foobar@\texttt{foobar}}%

\noindent\begin{tabularx}{\linewidth}{@{}Xr}
\rightskip=5em plus 1 fill
\hangindent=2em
\noindent\texttt{int foobar (int\ \GNUTexinfocommandstyletextvar{f---oo}\texttt{[},\ float\ \GNUTexinfocommandstyletextvar{b--ar}\texttt{]})\ t}& [Library Function]
\end{tabularx}

\index[fn]{foobar@\texttt{foobar}}%

\noindent\begin{tabularx}{\linewidth}{@{}Xr}
\rightskip=5em plus 1 fill
\hangindent=2em
\noindent\texttt{int foobar (int\ \GNUTexinfocommandstyletextvar{f---oo}\texttt{\textbf{[}},\ float\ \GNUTexinfocommandstyletextvar{b--ar}\texttt{\textbf{]}})\ t:b}& [Library Function]
\end{tabularx}

\index[fn]{foobar@\texttt{foobar}}%

\noindent\begin{tabularx}{\linewidth}{@{}Xr}
\rightskip=5em plus 1 fill
\hangindent=2em
\noindent\texttt{int foobar (int\ \GNUTexinfocommandstyletextvar{f---oo}\textnormal{\GNUTexinfocommandstyletextvar{[}},\ float\ \GNUTexinfocommandstyletextvar{b--ar}\textnormal{\GNUTexinfocommandstyletextvar{]}})\ r:var}& [Library Function]
\end{tabularx}

\index[fn]{foobar@\texttt{foobar}}%

\noindent\begin{tabularx}{\linewidth}{@{}Xr}
\rightskip=5em plus 1 fill
\hangindent=2em
\noindent\texttt{int foobar (int\ \GNUTexinfocommandstyletextvar{f---oo}\textnormal{\textsl{[}},\ float\ \GNUTexinfocommandstyletextvar{b--ar}\textnormal{\textsl{]}})\ r:slanted}& [Library Function]
\end{tabularx}

\index[fn]{foobar@\texttt{foobar}}%

\noindent\begin{tabularx}{\linewidth}{@{}Xr}
\rightskip=5em plus 1 fill
\hangindent=2em
\noindent\texttt{int foobar (int\ \GNUTexinfocommandstyletextvar{f---oo}\textnormal{\texttt{[}},\ float\ \GNUTexinfocommandstyletextvar{b--ar}\textnormal{\texttt{]}})\ r:code}& [Library Function]
\end{tabularx}

\index[fn]{foobar@\texttt{foobar}}%

\noindent\begin{tabularx}{\linewidth}{@{}Xr}
\rightskip=5em plus 1 fill
\hangindent=2em
\noindent\texttt{int foobar (int\ \GNUTexinfocommandstyletextvar{f---oo}\textnormal{\texttt{[}},\ float\ \GNUTexinfocommandstyletextvar{b--ar}\textnormal{\texttt{]}})\ r:t}& [Library Function]
\end{tabularx}

\index[fn]{foobar@\texttt{foobar}}%
\begin{quote}
\unskip{\parskip=0pt\noindent}%
separators
\end{quote}


\noindent\begin{tabularx}{\linewidth}{@{}Xr}
\rightskip=5em plus 1 fill
\hangindent=2em
\noindent\texttt{int foobar (i{-}{-}nt\ \GNUTexinfocommandstyletextvar{f---oo}[,\ float\ \GNUTexinfocommandstyletextvar{b--ar}])\ default}& [Library Function]
\end{tabularx}

\index[fn]{foobar@\texttt{foobar}}%

\noindent\begin{tabularx}{\linewidth}{@{}Xr}
\rightskip=5em plus 1 fill
\hangindent=2em
\noindent\texttt{int foobar (\textnormal{i--nt}\ \GNUTexinfocommandstyletextvar{f---oo}[,\ float\ \GNUTexinfocommandstyletextvar{b--ar}])\ r}& [Library Function]
\end{tabularx}

\index[fn]{foobar@\texttt{foobar}}%

\noindent\begin{tabularx}{\linewidth}{@{}Xr}
\rightskip=5em plus 1 fill
\hangindent=2em
\noindent\texttt{int foobar (\GNUTexinfocommandstyletextvar{i--nt}\ \GNUTexinfocommandstyletextvar{f---oo}[,\ float\ \GNUTexinfocommandstyletextvar{b--ar}])\ var}& [Library Function]
\end{tabularx}

\index[fn]{foobar@\texttt{foobar}}%

\noindent\begin{tabularx}{\linewidth}{@{}Xr}
\rightskip=5em plus 1 fill
\hangindent=2em
\noindent\texttt{int foobar (\textsl{i{-}{-}nt}\ \GNUTexinfocommandstyletextvar{f---oo}[,\ float\ \GNUTexinfocommandstyletextvar{b--ar}])\ slanted}& [Library Function]
\end{tabularx}

\index[fn]{foobar@\texttt{foobar}}%

\noindent\begin{tabularx}{\linewidth}{@{}Xr}
\rightskip=5em plus 1 fill
\hangindent=2em
\noindent\texttt{int foobar (\texttt{i{-}{-}nt}\ \GNUTexinfocommandstyletextvar{f---oo}[,\ float\ \GNUTexinfocommandstyletextvar{b--ar}])\ code}& [Library Function]
\end{tabularx}

\index[fn]{foobar@\texttt{foobar}}%

\noindent\begin{tabularx}{\linewidth}{@{}Xr}
\rightskip=5em plus 1 fill
\hangindent=2em
\noindent\texttt{int foobar (\texttt{i{-}{-}nt}\ \GNUTexinfocommandstyletextvar{f---oo}[,\ float\ \GNUTexinfocommandstyletextvar{b--ar}])\ t}& [Library Function]
\end{tabularx}

\index[fn]{foobar@\texttt{foobar}}%

\noindent\begin{tabularx}{\linewidth}{@{}Xr}
\rightskip=5em plus 1 fill
\hangindent=2em
\noindent\texttt{int foobar (\texttt{\textbf{i{-}{-}nt}}\ \GNUTexinfocommandstyletextvar{f---oo}[,\ float\ \GNUTexinfocommandstyletextvar{b--ar}])\ t:b}& [Library Function]
\end{tabularx}

\index[fn]{foobar@\texttt{foobar}}%

\noindent\begin{tabularx}{\linewidth}{@{}Xr}
\rightskip=5em plus 1 fill
\hangindent=2em
\noindent\texttt{int foobar (\textnormal{\GNUTexinfocommandstyletextvar{i--nt}}\ \GNUTexinfocommandstyletextvar{f---oo}[,\ float\ \GNUTexinfocommandstyletextvar{b--ar}])\ r:var}& [Library Function]
\end{tabularx}

\index[fn]{foobar@\texttt{foobar}}%

\noindent\begin{tabularx}{\linewidth}{@{}Xr}
\rightskip=5em plus 1 fill
\hangindent=2em
\noindent\texttt{int foobar (\textnormal{\textsl{i--nt}}\ \GNUTexinfocommandstyletextvar{f---oo}[,\ float\ \GNUTexinfocommandstyletextvar{b--ar}])\ r:slanted}& [Library Function]
\end{tabularx}

\index[fn]{foobar@\texttt{foobar}}%

\noindent\begin{tabularx}{\linewidth}{@{}Xr}
\rightskip=5em plus 1 fill
\hangindent=2em
\noindent\texttt{int foobar (\textnormal{\texttt{i{-}{-}nt}}\ \GNUTexinfocommandstyletextvar{f---oo}[,\ float\ \GNUTexinfocommandstyletextvar{b--ar}])\ r:code}& [Library Function]
\end{tabularx}

\index[fn]{foobar@\texttt{foobar}}%

\noindent\begin{tabularx}{\linewidth}{@{}Xr}
\rightskip=5em plus 1 fill
\hangindent=2em
\noindent\texttt{int foobar (\textnormal{\texttt{i{-}{-}nt}}\ \GNUTexinfocommandstyletextvar{f---oo}[,\ float\ \GNUTexinfocommandstyletextvar{b--ar}])\ r:t}& [Library Function]
\end{tabularx}

\index[fn]{foobar@\texttt{foobar}}%
\begin{quote}
\unskip{\parskip=0pt\noindent}%
name
\end{quote}

\end{document}
