@c ---content LibInfo---
@comment This file was generated by doc2tex.pl from d2t_singular/latex_lib.doc
@comment DO NOT EDIT DIRECTLY, BUT EDIT d2t_singular/latex_lib.doc INSTEAD
@c library version: (1.19.2.1,2002/02/20)
@c library file: ../Singular/LIB/latex.lib
@cindex latex.lib
@cindex latex_lib
@table @asis
@item @strong{Library:}
latex.lib
@item @strong{Purpose:}
    Typesetting of Singular-Objects in LaTeX2e
@item @strong{Author:}
Christian Gorzel, gorzelc@@math.uni-muenster.de

@end table

@strong{Procedures:}
@menu
* closetex:: writes closing line for LaTeX-document
* opentex:: writes header for LaTeX-file fnm
* tex:: calls LaTeX2e for LaTeX-file fnm
* texdemo:: produces a file explaining the features of this lib
* texfactorize:: creates string in LaTeX-format for factors of poly f
* texmap:: creates string in LaTeX-format for map m:r1->r2
* texname:: creates string in LaTeX-format for identifier
* texobj:: creates string in LaTeX-format for any (basic) type
* texpoly:: creates string in LaTeX-format for poly
* texproc:: creates string in LaTeX-format of text from proc p
* texring:: creates string in LaTeX-format for ring/qring
* rmx:: removes .aux and .log files of LaTeX-files
* xdvi:: calls xdvi for dvi-files
@end menu
@table @asis
@item @strong{Global variables:}
TeXwidth, TeXnofrac, TeXbrack, TeXproj, TeXaligned, TeXreplace, NoDollars
are used to control the typesetting.
Call @code{texdemo();} to obtain a LaTeX2e file @code{texlibdemo.tex}
explaining the features of @code{latex.lib} and its global variables.
@format
  @code{TeXwidth} (int) -1, 0, 1..9, >9:  controls breaking of long polynomials
  @code{TeXnofrac} (int) flag:  write 1/2 instead of \frac@{1@}@{2@}
  @code{TeXbrack} (string) "@{", "(", "<", "|", empty string: 
                                   controls brackets around ideals and matrices
  @code{TeXproj} (int) flag:  write ":" instead of "," in vectors
  @code{TeXaligned} (int) flag:  write maps (and ideals) aligned
  @code{TeXreplace} (list) list entries = 2 strings:  replacing symbols
  @code{NoDollars} (int) flag:  suppresses surrounding $ signs
@end format

@end table
@c ---end content LibInfo---

@c ------------------- closetex -------------
@node closetex, opentex,, latex_lib
@subsubsection closetex
@cindex closetex
@c ---content closetex---
Procedure from library @code{latex.lib} (@pxref{latex_lib}).

@table @asis
@item @strong{Usage:}
closetex(fname); fname string

@item @strong{Return:}
nothing; writes a LaTeX2e closing line into file @code{<fname>}.

@item @strong{Note:}
preceding ">>" are deleted and suffix ".tex" (if not given)
is added to @code{fname}.

@end table
@strong{Example:}
@smallexample
@c computed example closetex d2t_singular/latex_lib.doc:73 
LIB "latex.lib";
opentex("exmpl");
texobj("exmpl","@{\\large \\bf hello@}");
closetex("exmpl");
@c end example closetex d2t_singular/latex_lib.doc:73
@end smallexample
@c ---end content closetex---

@c ------------------- opentex -------------
@node opentex, tex, closetex, latex_lib
@subsubsection opentex
@cindex opentex
@c ---content opentex---
Procedure from library @code{latex.lib} (@pxref{latex_lib}).

@table @asis
@item @strong{Usage:}
opentex(fname); fname string

@item @strong{Return:}
nothing; writes a LaTeX2e header into a new file @code{<fname>}.

@item @strong{Note:}
preceding ">>" are deleted and suffix ".tex" (if not given)
is added to @code{fname}.

@end table
@strong{Example:}
@smallexample
@c computed example opentex d2t_singular/latex_lib.doc:103 
LIB "latex.lib";
opentex("exmpl");
texobj("exmpl","hello");
closetex("exmpl");
@c end example opentex d2t_singular/latex_lib.doc:103
@end smallexample
@c ---end content opentex---

@c ------------------- tex -------------
@node tex, texdemo, opentex, latex_lib
@subsubsection tex
@cindex tex
@c ---content tex---
Procedure from library @code{latex.lib} (@pxref{latex_lib}).

@table @asis
@item @strong{Usage:}
tex(fname); fname string

@item @strong{Return:}
nothing; calls latex (LaTeX2e) for compiling the file fname

@item @strong{Note:}
preceding ">>" are deleted and suffix ".tex" (if not given)
is added to @code{fname}.

@end table
@strong{Example:}
@smallexample
@c computed example tex d2t_singular/latex_lib.doc:133 
LIB "latex.lib";
ring r;
ideal I = maxideal(7);
opentex("exp001");              // open latex2e document
texobj("exp001","An ideal ",I);
closetex("exp001");
tex("exp001"); 
@expansion{} calling  latex2e  for : exp001.tex 
@expansion{} 
@expansion{} This is TeX, Version 3.14159 (Web2C 7.3.1)
@expansion{} (exp001.tex
@expansion{} LaTeX2e <1998/12/01> patch level 1
@expansion{} Babel <v3.6x> and hyphenation patterns for american, french, german, nger\
   man, i
@expansion{} talian, nohyphenation, loaded.
@expansion{} (/usr/share/texmf/tex/latex/base/article.cls
@expansion{} Document Class: article 1999/01/07 v1.4a Standard LaTeX document class
@expansion{} (/usr/share/texmf/tex/latex/base/size10.clo))
@expansion{} (/usr/share/texmf/tex/latex/amslatex/amsmath.sty
@expansion{} (/usr/share/texmf/tex/latex/amslatex/amstext.sty
@expansion{} (/usr/share/texmf/tex/latex/amslatex/amsgen.sty))
@expansion{} (/usr/share/texmf/tex/latex/amslatex/amsbsy.sty)
@expansion{} (/usr/share/texmf/tex/latex/amslatex/amsopn.sty))
@expansion{} (/usr/share/texmf/tex/latex/amsfonts/amssymb.sty
@expansion{} (/usr/share/texmf/tex/latex/amsfonts/amsfonts.sty))
@expansion{} No file exp001.aux.
@expansion{} (/usr/share/texmf/tex/latex/amsfonts/umsa.fd)
@expansion{} (/usr/share/texmf/tex/latex/amsfonts/umsb.fd) [1] (exp001.aux) )
@expansion{} Output written on exp001.dvi (1 page, 2912 bytes).
@expansion{} Transcript written on exp001.log.
system("sh","rm exp001.*");
@expansion{} 0
@c end example tex d2t_singular/latex_lib.doc:133
@end smallexample
@c ---end content tex---

@c ------------------- texdemo -------------
@node texdemo, texfactorize, tex, latex_lib
@subsubsection texdemo
@cindex texdemo
@c ---content texdemo---
Procedure from library @code{latex.lib} (@pxref{latex_lib}).

@table @asis
@item @strong{Usage:}
texdemo();

@item @strong{Return:}
nothing; generates a LaTeX2e file called @code{texlibdemo.tex}
explaining the features of @code{latex.lib} and its global variables.

@item @strong{Note:}
this proc may take some time.

@end table
@c ---end content texdemo---

@c ------------------- texfactorize -------------
@node texfactorize, texmap, texdemo, latex_lib
@subsubsection texfactorize
@cindex texfactorize
@c ---content texfactorize---
Procedure from library @code{latex.lib} (@pxref{latex_lib}).

@table @asis
@item @strong{Usage:}
texfactorize(fname,f); fname string, f poly

@item @strong{Return:}
if @code{fname=""}: string, f as a product of its irreducible
factors@*
otherwise: append this string to the file @code{<fname>}, and
return nothing.

@item @strong{Note:}
preceding ">>" are deleted and suffix ".tex" (if not given)
is added to @code{fname}.

@end table
@strong{Example:}
@smallexample
@c computed example texfactorize d2t_singular/latex_lib.doc:191 
LIB "latex.lib";
ring r2 = 13,(x,y),dp;
poly f = (x+1+y)^2*x3y*(2x-2y)*y12;
texfactorize("",f);
@expansion{} $-2\cdot x^@{3@}\cdot y^@{13@}\cdot (-x+y)\cdot (x+y+1)^@{2@}$
ring R49 = (7,a),x,dp;
minpoly = a2+a+3;
poly f = (a24x5+x3)*a2x6*(x+1)^2;
f;
@expansion{} (a+3)*x13+(2a-1)*x12+(-2a+1)*x10+(-a-3)*x9
texfactorize("",f);
@expansion{} $(a+3)\cdot (x-1)\cdot (x+1)^@{3@}\cdot x^@{9@}$
@c end example texfactorize d2t_singular/latex_lib.doc:191
@end smallexample
@c ---end content texfactorize---

@c ------------------- texmap -------------
@node texmap, texname, texfactorize, latex_lib
@subsubsection texmap
@cindex texmap
@c ---content texmap---
Procedure from library @code{latex.lib} (@pxref{latex_lib}).

@table @asis
@item @strong{Usage:}
texmap(fname,m,@@r1,@@r2); fname string, m string/map, @@r1,@@r2 rings

@item @strong{Return:}
if @code{fname=""}: string, the map m from @@r1 to @@r2 (preceded
by its name if m = string) in TeX-typesetting;@*
otherwise: append this string to the file @code{<fname>}, and
return nothing.

@item @strong{Note:}
preceding ">>" are deleted in @code{fname}, and suffix ".tex"
(if not given) is added to @code{fname}.
If m is a string then it has to be the name of an existing map
from @@r1 to @@r2.

@end table
@strong{Example:}
@smallexample
@c computed example texmap d2t_singular/latex_lib.doc:231 
LIB "latex.lib";
// -------- prepare for example ---------
if (defined(TeXaligned)) @{int Teali=TeXaligned; kill TeXaligned;@}
if (defined(TeXreplace)) @{list Terep=TeXreplace; kill TeXreplace;@}
// -------- the example starts here ---------
//
string fname = "tldemo";
ring @@r1=0,(x,y,z),dp;
if(system("with","Namespaces")) @{ exportto(Current, @@r1); @}
else @{ export @@r1; @}
@expansion{} // ** `@@r1` is already global
ring r2=0,(u,v),dp;
map @@phi =(@@r1,u2,uv -v,v2); export @@phi;
@expansion{} // ** `@@phi` is already global
list TeXreplace;
TeXreplace[1] = list("@@phi","\\phi");    // @@phi --> \phi
export TeXreplace;
@expansion{} // ** `TeXreplace` is already global
texmap("","@@phi",@@r1,r2);                // standard form
@expansion{} $$
@expansion{} \begin@{array@}@{rcc@}
@expansion{} \phi:\Q[x,y,z] & \longrightarrow & \Q[u,v]\\[2mm]
@expansion{} \left(x,y,z\right) & \longmapsto & 
@expansion{}  \left(
@expansion{} \begin@{array@}@{c@}
@expansion{} u^@{2@}\\
@expansion{} uv-v\\
@expansion{} v^@{2@}
@expansion{} \end@{array@}
@expansion{} \right)
@expansion{} \end@{array@}
@expansion{} $$
//
int TeXaligned; export TeXaligned;       // map in one line
@expansion{} // ** `TeXaligned` is already global
texmap("",@@phi,@@r1,r2);
@expansion{} $\Q[x,y,z]\longrightarrow\Q[u,v], \ \left(x,y,z\right)\longmapsto \left(u\
   ^@{2@},uv-v,v^@{2@}\right)$
//
kill @@r1,TeXreplace,TeXaligned;
//
// --- restore global variables if previously defined ---
if (defined(Teali)) @{int TeXaligned=Teali; export TeXaligned; kill Teali;@}
if (defined(Terep)) @{list TeXreplace=Terep; export TeXreplace; kill Terep;@}
@c end example texmap d2t_singular/latex_lib.doc:231
@end smallexample
@c ---end content texmap---

@c ------------------- texname -------------
@node texname, texobj, texmap, latex_lib
@subsubsection texname
@cindex texname
@c ---content texname---
Procedure from library @code{latex.lib} (@pxref{latex_lib}).

@table @asis
@item @strong{Usage:}
texname(fname,s); fname,s strings

@item @strong{Return:}
if @code{fname=""}: string, the transformed string s, where the
following rules apply:
@smallexample
      s' + "~"             -->  "\\tilde@{"+ s' +"@}"
     "_" + int             -->       "_@{" + int +"@}" 
  "[" + s' + "]"           -->      "_@{" + s' + "@}"
   "A..Z" + int            --> "A..Z" + "^@{" + int + "@}"    
   "a..z" + int            --> "a..z" + "_@{" + int + "@}"
"(" + int + "," + s' + ")" --> "_@{"+ int +"@}" + "^@{" + s'+"@}"
@end smallexample
Anyhow, strings which begin with a @code{"@{"} are only changed
by deleting the first and last character (intended to remove the
surrounding curly brackets).

if @code{fname!=""}: append the transformed string s to the file
@code{<fname>}, and return nothing.

@item @strong{Note:}
preceding ">>" are deleted in @code{fname}, and suffix ".tex"
(if not given) is added to @code{fname}.

@end table
@strong{Example:}
@smallexample
@c computed example texname d2t_singular/latex_lib.doc:297 
LIB "latex.lib";
ring r = 0,(x,y),lp;
poly f = 3xy4 + 2xy2 + x5y3 + x + y6;
texname("","@{f(10)@}");
@expansion{} f(10)
texname("","f(10) =");
@expansion{} f_@{10@} =
texname("","n1");
@expansion{} n_@{1@}
texname("","T1_12");
@expansion{} T^@{1@}_@{12@}
texname("","g'_11");
@expansion{} g'_@{11@}
texname("","f23");
@expansion{} f_@{23@}
texname("","M[2,3]");
@expansion{} M_@{2,3@}
texname("","A(0,3);");
@expansion{} A_@{0@}^@{3@};
texname("","E~(3)");
@expansion{} \tilde@{E@}_@{3@}
@c end example texname d2t_singular/latex_lib.doc:297
@end smallexample
@c ---end content texname---

@c ------------------- texobj -------------
@node texobj, texpoly, texname, latex_lib
@subsubsection texobj
@cindex texobj
@c ---content texobj---
Procedure from library @code{latex.lib} (@pxref{latex_lib}).

@table @asis
@item @strong{Usage:}
texobj(fname,l); fname string, l list

@item @strong{Return:}
if @code{fname=""}: string, the entries of l in LaTeX-typesetting;@*
otherwise: append this string to the file @code{<fname>}, and
return nothing.

@item @strong{Note:}
preceding ">>" are deleted in @code{fname}, and suffix ".tex"
(if not given) is added to @code{fname}.

@end table
@strong{Example:}
@smallexample
@c computed example texobj d2t_singular/latex_lib.doc:337 
LIB "latex.lib";
// -------- prepare for example ---------
if (defined(TeXaligned)) @{int Teali=TeXaligned; kill TeXaligned;@}
if (defined(TeXbrack))@{string Tebra=TeXbrack; kill TeXbrack;@}
//
//  --------------  typesetting for polynomials ----------
ring r = 0,(x,y),lp;
poly f = x5y3 + 3xy4 + 2xy2 + y6;
f;
@expansion{} x5y3+3xy4+2xy2+y6
texobj("",f);
@expansion{} $$\begin@{array@}@{rl@}
@expansion{} & x^@{5@}y^@{3@}+3xy^@{4@}+2xy^@{2@}+y^@{6@}\\
@expansion{} \end@{array@}
@expansion{} $$
@expansion{} 
//  --------------  typesetting for ideals ----------
ideal G = jacob(f);
G;
@expansion{} G[1]=5x4y3+3y4+2y2
@expansion{} G[2]=3x5y2+12xy3+4xy+6y5
texobj("",G);
@expansion{} $$\left(
@expansion{} \begin@{array@}@{c@}
@expansion{} 5x^@{4@}y^@{3@}+3y^@{4@}+2y^@{2@}, \\
@expansion{} 3x^@{5@}y^@{2@}+12xy^@{3@}+4xy+6y^@{5@}
@expansion{} \end@{array@}
@expansion{} \right)$$
@expansion{} 
//  --------------  variation of typesetting for ideals ----------
int TeXaligned = 1; export TeXaligned;
@expansion{} // ** `TeXaligned` is already global
string TeXbrack = "<"; export TeXbrack;
@expansion{} // ** `TeXbrack` is already global
texobj("",G);
@expansion{} $\left<5x^@{4@}y^@{3@}+3y^@{4@}+2y^@{2@},3x^@{5@}y^@{2@}+12xy^@{3@}+4xy+6y^@{5@}\right>$
@expansion{} 
kill TeXaligned, TeXbrack;
//  --------------  typesetting for matrices ----------
matrix J = jacob(G);
texobj("",J);
@expansion{} $$\left(
@expansion{} \begin@{array@}@{*@{2@}@{c@}@}
@expansion{} 20x^@{3@}y^@{3@} & 15x^@{4@}y^@{2@}+12y^@{3@}+4y \\
@expansion{} 15x^@{4@}y^@{2@}+12y^@{3@}+4y & 6x^@{5@}y+36xy^@{2@}+4x+30y^@{4@}
@expansion{} \end@{array@}
@expansion{} \right)
@expansion{} $$
@expansion{} 
//  --------------  typesetting for intmats ----------
intmat m[3][4] = 9,2,4,5,2,5,-2,4,-6,10,-1,2,7;
texobj("",m);
@expansion{} $$\left(
@expansion{} \begin@{array@}@{*@{4@}@{r@}@}
@expansion{} 9 & 2 & 4 & 5\\
@expansion{} 2 & 5 & -2 & 4\\
@expansion{} -6 & 10 & -1 & 2
@expansion{} \end@{array@}
@expansion{} \right)
@expansion{} $$
@expansion{} 
//
// --- restore global variables if previously defined ---
if (defined(Teali))@{int TeXaligned=Teali; export TeXaligned; kill Teali;@}
if (defined(Tebra))@{string TeXbrack=Tebra; export TeXbrack; kill Tebra;@}
@c end example texobj d2t_singular/latex_lib.doc:337
@end smallexample
@c ---end content texobj---

@c ------------------- texpoly -------------
@node texpoly, texproc, texobj, latex_lib
@subsubsection texpoly
@cindex texpoly
@c ---content texpoly---
Procedure from library @code{latex.lib} (@pxref{latex_lib}).

@table @asis
@item @strong{Usage:}
texpoly(fname,p); fname string, p poly

@item @strong{Return:}
if @code{fname=""}: string, the poly p in LaTeX-typesetting;@*
otherwise: append this string to the file @code{<fname>}, and
return nothing.

@item @strong{Note:}
preceding ">>" are deleted in @code{fname}, and suffix ".tex"
(if not given) is added to @code{fname}.

@end table
@strong{Example:}
@smallexample
@c computed example texpoly d2t_singular/latex_lib.doc:394 
LIB "latex.lib";
ring r0=0,(x,y,z),dp;
poly f = -1x^2 + 2;
texpoly("",f);
@expansion{} $-x^@{2@}+2$
ring rr= real,(x,y,z),dp;
texpoly("",2x2y23z);
@expansion{} $2.000x^@{2@}y^@{23@}z$
ring r7= 7,(x,y,z),dp;
poly f = 2x2y23z;
texpoly("",f);
@expansion{} $2x^@{2@}y^@{23@}z$
ring rab =(0,a,b),(x,y,z),dp;
poly f = (-2a2 +b3 -2)/a * x2y4z5 + (a2+1)*x + a+1;
f;
@expansion{} (-2a2+b3-2)/(a)*x2y4z5+(a2+1)*x+(a+1)
texpoly("",f);
@expansion{} $-\frac@{2a^@{2@}-b^@{3@}+2@}@{a@}x^@{2@}y^@{4@}z^@{5@}+(a^@{2@}+1)x+(a+1)$
@c end example texpoly d2t_singular/latex_lib.doc:394
@end smallexample
@c ---end content texpoly---

@c ------------------- texproc -------------
@node texproc, texring, texpoly, latex_lib
@subsubsection texproc
@cindex texproc
@c ---content texproc---
Procedure from library @code{latex.lib} (@pxref{latex_lib}).

@table @asis
@item @strong{Usage:}
texproc(fname,pname); fname,pname strings

@item @strong{Assume:}
@code{`pname`} is a procedure.

@item @strong{Return:}
if @code{fname=""}: string, the proc @code{`pname`} in a verbatim
environment in LaTeX-typesetting;@*
otherwise: append this string to the file @code{<fname>}, and
return nothing.

@item @strong{Note:}
preceding ">>" are deleted in @code{fname}, and suffix ".tex"
(if not given) is added to @code{fname}.@*
@code{texproc} cannot be applied to itself correctly.

@end table
@strong{Example:}
@smallexample
@c computed example texproc d2t_singular/latex_lib.doc:440 
LIB "latex.lib";
proc exp(int i,int j,list #)
@{ string s;
if (size(#))
@{
for(i;i<=j;i++)
@{ s = s + string(j) + string(#); @}
@}
return(s);
@}
export exp;
@expansion{} // ** `exp` is already global
texproc("","exp");
@expansion{} \begin@{verbatim@}
@expansion{} proc exp(int i,int j,list #)
@expansion{} @{ 
@expansion{}  string s;
@expansion{} if (size(#))
@expansion{} @{
@expansion{} for(i;i<=j;i++)
@expansion{} @{ s = s + string(j) + string(#); @}
@expansion{} @}
@expansion{} return(s);
@expansion{} 
@expansion{} @}
@expansion{} \end@{verbatim@}
@expansion{} 
kill exp;
@c end example texproc d2t_singular/latex_lib.doc:440
@end smallexample
@c ---end content texproc---

@c ------------------- texring -------------
@node texring, rmx, texproc, latex_lib
@subsubsection texring
@cindex texring
@c ---content texring---
Procedure from library @code{latex.lib} (@pxref{latex_lib}).

@table @asis
@item @strong{Usage:}
texring(fname, r[,L]); fname string, r ring, L list

@item @strong{Return:}
if @code{fname=""}: string, the ring in TeX-typesetting;@*
otherwise: append this string to the file @code{<fname>} and
return nothing.

@item @strong{Note:}
preceding ">>" are deleted and suffix ".tex" (if not given)
is added to @code{fname}.@*
The optional list L is assumed to be a list of strings which control,
e.g., the symbol for the field of coefficients.@*
For more details call @code{texdemo();} (generates a LaTeX2e
file called @code{texlibdemo.tex} which explains all features of
@code{texring}).

@end table
@strong{Example:}
@smallexample
@c computed example texring d2t_singular/latex_lib.doc:486 
LIB "latex.lib";
ring r0 = 0,(x,y),dp;                // char = 0, polynomial ordering
texring("",r0);
@expansion{} $\Q[x,y]$
//
ring r7 =7,(x(0..2)),ds;             // char = 7, local ordering
texring("",r7);
@expansion{} $\Z_@{7@}[[x_@{0@},x_@{1@},x_@{2@}]]$
//
ring r1 = 0,(x1,x2,y1,y2),wp(1,2,3,4);
texring("",r1);
@expansion{} $\Q[x_@{1@},x_@{2@},y_@{1@},y_@{2@}]$
//
ring rr = real,(x),dp;               // real numbers
texring("",rr);
@expansion{} $\R[x]$
//
ring rabc =(0,t1,t2,t3),(x,y),dp;    // ring with parameters
texring("",rabc);
@expansion{} $\Q(t_@{1@},t_@{2@},t_@{3@})[x,y]$
//
ring ralg = (7,a),(x1,x2),ds;        // algebraic extension
minpoly = a2-a+3;
texring("",ralg);
@expansion{} $\Z_@{7@}(a)[[x_@{1@},x_@{2@}]]$
texring("",ralg,"mipo");
@expansion{} $\Z_@{7@}(a)/(a^@{2@}-a+3)[[x_@{1@},x_@{2@}]]$
//
ring r49=(49,a),x,dp;                // Galois field  
texring("",r49);
@expansion{} $\F_@{49@}[x]$
//
setring r0;                          // quotient ring
ideal i = x2-y3;
qring q = std(i);
texring("",q);
@expansion{} $\Q[x,y]/\left(y^@{3@}-x^@{2@}\right)
@expansion{} $
//
// ------------------ additional features -------------------
ring r9 =0,(x(0..9)),ds;
texring("",r9,1);
@expansion{} $\Q[[x_@{0@},\ldots,x_@{9@}]]$
texring("",r9,"C","@{","^G");
@expansion{} $\C\@{x_@{0@},x_@{1@},x_@{2@},x_@{3@},x_@{4@},x_@{5@},x_@{6@},x_@{7@},x_@{8@},x_@{9@}\@}^G$
//
ring rxy = 0,(x(1..5),y(1..6)),ds;
intvec v = 5,6;
texring("",rxy,v);
@expansion{} $\Q[[x_@{1@},\ldots,x_@{5@},y_@{1@},\ldots,y_@{6@}]]$
@c end example texring d2t_singular/latex_lib.doc:486
@end smallexample
@c ---end content texring---

@c ------------------- rmx -------------
@node rmx, xdvi, texring, latex_lib
@subsubsection rmx
@cindex rmx
@c ---content rmx---
Procedure from library @code{latex.lib} (@pxref{latex_lib}).

@table @asis
@item @strong{Usage:}
rmx(fname); fname string

@item @strong{Return:}
nothing; removes the @code{.log} and @code{.aux} files associated to
the LaTeX file <fname>.@*

@item @strong{Note:}
If @code{fname} ends by @code{".dvi"} or @code{".tex"}, the
@code{.dvi} or @code{.tex} file will be deleted, too.

@end table
@strong{Example:}
@smallexample
@c computed example rmx d2t_singular/latex_lib.doc:550 
LIB "latex.lib";
ring r;
poly f = x+y+z;
opentex("exp001");              // defaulted latex2e document
texobj("exp001","A polynom",f);
closetex("exp001");
tex("exp001");
@expansion{} calling  latex2e  for : exp001.tex 
@expansion{} 
@expansion{} This is TeX, Version 3.14159 (Web2C 7.3.1)
@expansion{} (exp001.tex
@expansion{} LaTeX2e <1998/12/01> patch level 1
@expansion{} Babel <v3.6x> and hyphenation patterns for american, french, german, nger\
   man, i
@expansion{} talian, nohyphenation, loaded.
@expansion{} (/usr/share/texmf/tex/latex/base/article.cls
@expansion{} Document Class: article 1999/01/07 v1.4a Standard LaTeX document class
@expansion{} (/usr/share/texmf/tex/latex/base/size10.clo))
@expansion{} (/usr/share/texmf/tex/latex/amslatex/amsmath.sty
@expansion{} (/usr/share/texmf/tex/latex/amslatex/amstext.sty
@expansion{} (/usr/share/texmf/tex/latex/amslatex/amsgen.sty))
@expansion{} (/usr/share/texmf/tex/latex/amslatex/amsbsy.sty)
@expansion{} (/usr/share/texmf/tex/latex/amslatex/amsopn.sty))
@expansion{} (/usr/share/texmf/tex/latex/amsfonts/amssymb.sty
@expansion{} (/usr/share/texmf/tex/latex/amsfonts/amsfonts.sty))
@expansion{} No file exp001.aux.
@expansion{} (/usr/share/texmf/tex/latex/amsfonts/umsa.fd)
@expansion{} (/usr/share/texmf/tex/latex/amsfonts/umsb.fd) [1] (exp001.aux) )
@expansion{} Output written on exp001.dvi (1 page, 308 bytes).
@expansion{} Transcript written on exp001.log.
rmx("exp001");   // removes aux and log file of exp001
system("sh","rm exp001.*");
@expansion{} 0
@c end example rmx d2t_singular/latex_lib.doc:550
@end smallexample
@c ---end content rmx---

@c ------------------- xdvi -------------
@node xdvi,, rmx, latex_lib
@subsubsection xdvi
@cindex xdvi
@c ---content xdvi---
Procedure from library @code{latex.lib} (@pxref{latex_lib}).

@table @asis
@item @strong{Usage:}
xdvi(fname[,style]); fname,style = string

@item @strong{Return:}
nothing; displays dvi-file fname.dvi with previewer xdvi

@item @strong{Note:}
ending .dvi may miss in fname
@*style overwrites the default setting xdvi

@end table
@strong{Example:}
@smallexample
@c computed example xdvi d2t_singular/latex_lib.doc:585 
LIB "latex.lib";
intmat m[3][4] = 9,2,4,5,2,5,-2,4,-6,10,-1,2,7;
opentex("exp001"); 
texobj("exp001","An intmat:  ",m);
closetex("exp001");
tex("exp001");
@expansion{} calling  latex2e  for : exp001.tex 
@expansion{} 
@expansion{} This is TeX, Version 3.14159 (Web2C 7.3.1)
@expansion{} (exp001.tex
@expansion{} LaTeX2e <1998/12/01> patch level 1
@expansion{} Babel <v3.6x> and hyphenation patterns for american, french, german, nger\
   man, i
@expansion{} talian, nohyphenation, loaded.
@expansion{} (/usr/share/texmf/tex/latex/base/article.cls
@expansion{} Document Class: article 1999/01/07 v1.4a Standard LaTeX document class
@expansion{} (/usr/share/texmf/tex/latex/base/size10.clo))
@expansion{} (/usr/share/texmf/tex/latex/amslatex/amsmath.sty
@expansion{} (/usr/share/texmf/tex/latex/amslatex/amstext.sty
@expansion{} (/usr/share/texmf/tex/latex/amslatex/amsgen.sty))
@expansion{} (/usr/share/texmf/tex/latex/amslatex/amsbsy.sty)
@expansion{} (/usr/share/texmf/tex/latex/amslatex/amsopn.sty))
@expansion{} (/usr/share/texmf/tex/latex/amsfonts/amssymb.sty
@expansion{} (/usr/share/texmf/tex/latex/amsfonts/amsfonts.sty))
@expansion{} No file exp001.aux.
@expansion{} (/usr/share/texmf/tex/latex/amsfonts/umsa.fd)
@expansion{} (/usr/share/texmf/tex/latex/amsfonts/umsb.fd) [1] (exp001.aux) )
@expansion{} Output written on exp001.dvi (1 page, 524 bytes).
@expansion{} Transcript written on exp001.log.
xdvi("exp001");
@expansion{} calling  xdvi  for : exp001 
@expansion{} 
system("sh","rm exp001.*");
@expansion{} 0
@c end example xdvi d2t_singular/latex_lib.doc:585
@end smallexample
@c ---end content xdvi---
