@c ---content LibInfo---
@comment This file was generated by doc2tex.pl from d2t_singular/template_lib.doc
@comment DO NOT EDIT DIRECTLY, BUT EDIT d2t_singular/template_lib.doc INSTEAD
@c library version: (1.10,2001/01/16)
@c library file: ../Singular/LIB/template.lib
@cindex template.lib
@cindex template_lib
@table @asis
@item @strong{Library:}
template.lib
@item @strong{Purpose:}
  A Template for a Singular Library
@item @strong{Author:}
Olaf Bachmann, email: obachman@@mathematik.uni-kl.de

@cindex library, template.lib
@cindex template.lib
@cindex library, info string
@end table

@strong{Procedures:}
@menu
* mdouble:: return double of int argument
* mtripple:: return three times int argument
* msum:: sum of int arguments
@end menu
@c inserted refs from d2t_singular/template_lib.doc:25
@ifinfo
@menu
See also:
* Guidelines for writing a library::
* Typesetting of help strings::
* standard_lib::
@end menu
@end ifinfo
@iftex
@strong{See also:}
@ref{Guidelines for writing a library};
@ref{Typesetting of help strings};
@ref{standard_lib}.
@end iftex
@c end inserted refs from d2t_singular/template_lib.doc:25

@c ---end content LibInfo---

@c ------------------- mdouble -------------
@node mdouble, mtripple,, template_lib
@subsubsection mdouble
@cindex mdouble
@c ---content mdouble---
Procedure from library @code{template.lib} (@pxref{template_lib}).

@table @asis
@item @strong{Usage:}
mdouble(i); i int

@item @strong{Return:}
int: i+i

@item @strong{Note:}
Help string is in pure ASCII
@*this line starts on a new line since previous line is short
mdouble(i): no new line

@cindex procedure, ASCII help
@end table
@strong{Example:}
@smallexample
@c computed example mdouble d2t_singular/template_lib.doc:54 
LIB "template.lib";
mdouble(0);
@expansion{} 0
mdouble(-1);
@expansion{} -2
@c end example mdouble d2t_singular/template_lib.doc:54
@end smallexample
@c inserted refs from d2t_singular/template_lib.doc:60
@ifinfo
@menu
See also:
* Typesetting of help strings::
* msum::
* mtripple::
@end menu
@end ifinfo
@iftex
@strong{See also:}
@ref{Typesetting of help strings};
@ref{msum};
@ref{mtripple}.
@end iftex
@c end inserted refs from d2t_singular/template_lib.doc:60

@c ---end content mdouble---

@c ------------------- mtripple -------------
@node mtripple, msum, mdouble, template_lib
@subsubsection mtripple
@cindex mtripple
@c ---content mtripple---
Procedure from library @code{template.lib} (@pxref{template_lib}).

@c we do texinfo here
@table @asis
@item @strong{Usage:}
@code{mtripple(i)}; @code{i} int

@item @strong{Return:}
int: 
@ifinfo
@math{i+i+i}
@end ifinfo
@tex
$i+i+i$
@end tex

@item @strong{Note:}
Help is in pure Texinfo
@*This help string is written in texinfo, which enables you to use,
among others, the @@math command for mathematical typesetting (like

@ifinfo
@math{\alpha, \beta}
@end ifinfo
@tex
$\alpha, \beta$
@end tex
).
@*It also gives more control over the layout, but is, admittingly,
more cumbersome to write.
@end table
@c use @c ref contstuct for references
@cindex procedure, texinfo help
@c inserted refs from d2t_singular/template_lib.doc:90
@ifinfo
@menu
@strong{See also:}
* Typesetting of help strings::
* mdouble::
* msum::
@end menu
@end ifinfo
@iftex
@strong{See also:}
@ref{Typesetting of help strings};
@ref{mdouble};
@ref{msum}.
@end iftex
@c end inserted refs from d2t_singular/template_lib.doc:90

@strong{Example:}
@smallexample
@c computed example mtripple d2t_singular/template_lib.doc:97 
LIB "template.lib";
mtripple(0);
@expansion{} 0
mtripple(-1);
@expansion{} -3
@c end example mtripple d2t_singular/template_lib.doc:97
@end smallexample
@c ---end content mtripple---

@c ------------------- msum -------------
@node msum,, mtripple, template_lib
@subsubsection msum
@cindex msum
@c ---content msum---
Procedure from library @code{template.lib} (@pxref{template_lib}).

@table @asis
@item @strong{Usage:}
msum([i_1,..,i_n]); @code{i_1,..,i_n} def

@item @strong{Return:}
Sum of int arguments

@item @strong{Note:}
This help string is written in a mixture of ASCII and texinfo
@* Use a @@ref constructs for references (like @pxref{mtripple})
@* Use @@code for typewriter font (like @code{i_1})
@* Use @@math for simple math mode typesetting (like 
@ifinfo
@math{i_1}
@end ifinfo
@tex
$i_1$
@end tex
).
@* Note: No parenthesis like @} are allowed inside @@math and @@code
@* Use @@example for indented preformatted text typeset in typewriter
font like
@smallexample
         this  --> that
@end smallexample
Use @@format for preformatted text typeset in normal font
@format
         this --> that
@end format
Use @@texinfo for text in pure texinfo

@expansion{}
@tex
$i_{1,1}$
@end tex


Notice that
@*automatic linebreaking is still in affect (like on this line).

@cindex procedure, ASCII/Texinfo help
@end table
@strong{Example:}
@smallexample
@c computed example msum d2t_singular/template_lib.doc:149 
LIB "template.lib";
msum();
@expansion{} 0
msum(4);
@expansion{} 4
msum(1,2,3,4);
@expansion{} 10
@c end example msum d2t_singular/template_lib.doc:149
@end smallexample
@c inserted refs from d2t_singular/template_lib.doc:156
@ifinfo
@menu
See also:
* Typesetting of help strings::
* mdouble::
* mtripple::
@end menu
@end ifinfo
@iftex
@strong{See also:}
@ref{Typesetting of help strings};
@ref{mdouble};
@ref{mtripple}.
@end iftex
@c end inserted refs from d2t_singular/template_lib.doc:156

@c ---end content msum---
