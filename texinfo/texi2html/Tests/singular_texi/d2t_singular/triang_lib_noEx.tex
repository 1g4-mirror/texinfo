@c ---content LibInfo---
@comment This file was generated by doc2tex.pl from d2t_singular/triang_lib.doc
@comment DO NOT EDIT DIRECTLY, BUT EDIT d2t_singular/triang_lib.doc INSTEAD
@c library version: (1.7,2001/02/19)
@c library file: ../Singular/LIB/triang.lib
@cindex triang.lib
@cindex triang_lib
@table @asis
@item @strong{Library:}
triang.lib
@item @strong{Purpose:}
   Decompose Zero-dimensional Ideals into Triangular Sets
@item @strong{Author:}
D. Hillebrand

@end table

@strong{Procedures:}
@menu
* triangL:: Decomposition of (G) into triangular systems (Lazard).
* triangLfak:: Decomp. of (G) into tri. systems plus factorization.
* triangM:: Decomposition of (G) into triangular systems (Moeller).
* triangMH:: Decomp. of (G) into tri. syst. with disjoint varieties.
@end menu
@c ---end content LibInfo---

@c ------------------- triangL -------------
@node triangL, triangLfak,, triang_lib
@subsubsection triangL
@cindex triangL
@c ---content triangL---
Procedure from library @code{triang.lib} (@pxref{triang_lib}).

@table @asis
@item @strong{Usage:}
triangL(G); G=ideal

@item @strong{Assume:}
G is the reduced lexicographical Groebner bases of the
zero-dimensional ideal (G), sorted by increasing leading terms.

@item @strong{Return:}
a list of finitely many triangular systems, such that
the union of their varieties equals the variety of (G).

@item @strong{Note:}
Algorithm of Lazard (see: Lazard, D.: Solving zero-dimensional
algebraic systems, J. Symb. Comp. 13, 117 - 132, 1992).

@end table
@strong{Example:}
@smallexample
@c skipped computation of example triangL d2t_singular/triang_lib.doc:51 
LIB "triang.lib";
ring rC5 = 0,(e,d,c,b,a),lp;
triangL(stdfglm(cyclic(5)));
@end smallexample
@c ---end content triangL---

@c ------------------- triangLfak -------------
@node triangLfak, triangM, triangL, triang_lib
@subsubsection triangLfak
@cindex triangLfak
@c ---content triangLfak---
Procedure from library @code{triang.lib} (@pxref{triang_lib}).

@table @asis
@item @strong{Usage:}
triangLfak(G); G=ideal

@item @strong{Assume:}
G is the reduced lexicographical Groebner bases of the
zero-dimensional ideal (G), sorted by increasing leading terms.

@item @strong{Return:}
a list of finitely many triangular systems, such that
the union of their varieties equals the variety of (G).

@item @strong{Note:}
Algorithm of Lazard with factorization (see: Lazard, D.: Solving
zero-dimensional algebraic systems, J. Symb. Comp. 13, 117 - 132, 1992).

@item @strong{Remark:}
each polynomial of the triangular systems is factorized.

@end table
@strong{Example:}
@smallexample
@c skipped computation of example triangLfak d2t_singular/triang_lib.doc:88 
LIB "triang.lib";
ring rC5 = 0,(e,d,c,b,a),lp;
triangLfak(stdfglm(cyclic(5)));
@end smallexample
@c ---end content triangLfak---

@c ------------------- triangM -------------
@node triangM, triangMH, triangLfak, triang_lib
@subsubsection triangM
@cindex triangM
@c ---content triangM---
Procedure from library @code{triang.lib} (@pxref{triang_lib}).

@table @asis
@item @strong{Usage:}
triangM(G[,i]); G=ideal, i=integer,@*

@item @strong{Assume:}
G is the reduced lexicographical Groebner bases of the
zero-dimensional ideal (G), sorted by increasing leading terms.

@item @strong{Return:}
a list of finitely many triangular systems, such that
the union of their varieties equals the variety of (G).
If i = 2, then each polynomial of the triangular systems
is factorized.

@item @strong{Note:}
Algorithm of Moeller (see: Moeller, H.M.:
@*On decomposing systems of polynomial equations with
@*finitely many solutions, Appl. Algebra Eng. Commun. Comput. 4,
217 - 230, 1993).

@end table
@strong{Example:}
@smallexample
@c skipped computation of example triangM d2t_singular/triang_lib.doc:126 
LIB "triang.lib";
ring rC5 = 0,(e,d,c,b,a),lp;
triangM(stdfglm(cyclic(5))); //oder: triangM(stdfglm(cyclic(5)),2);
@end smallexample
@c ---end content triangM---

@c ------------------- triangMH -------------
@node triangMH,, triangM, triang_lib
@subsubsection triangMH
@cindex triangMH
@c ---content triangMH---
Procedure from library @code{triang.lib} (@pxref{triang_lib}).

@table @asis
@item @strong{Usage:}
triangMH(G[,i]); G=ideal, i=integer

@item @strong{Assume:}
G is the reduced lexicographical Groebner bases of the
zero-dimensional ideal (G), sorted by increasing leading terms.

@item @strong{Return:}
a list of finitely many triangular systems, such that
the disjoint union of their varieties equals the variety of (G).
If i = 2, then each polynomial of the triangular systems is factorized.

@item @strong{Note:}
Algorithm of Moeller and Hillebrand (see: Moeller, H.M.:
On decomposing systems of polynomial equations with finitely many
solutions, Appl. Algebra Eng. Commun. Comput. 4, 217 - 230, 1993 and
Hillebrand, D.: Triangulierung nulldimensionaler Ideale -
Implementierung und Vergleich zweier Algorithmen, master thesis,
Universitaet Dortmund, Fachbereich Mathematik, Prof. Dr. H.M. Moeller,
1999).

@end table
@strong{Example:}
@smallexample
@c skipped computation of example triangMH d2t_singular/triang_lib.doc:166 
LIB "triang.lib";
ring rC5 = 0,(e,d,c,b,a),lp;
triangMH(stdfglm(cyclic(5)));
@end smallexample
@c ---end content triangMH---
